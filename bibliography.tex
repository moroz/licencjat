\onecolumn
\chapter*{Bibliografia}
\addcontentsline{toc}{chapter}{Bibliografia}

Jing 靜, Yun 筠 i Mo Yansheng 莫言生. \textit{Zutang ji} [Antologia gmachu patriarchów]. [online:] <\url{http://www.lianhua33.com/c/c47-1.htm}> [Dostęp 19 kwietnia 2016]

Dumoulin, Heinrich. 1963. \textit{A History of Zen Buddhism}. New York: Pantheon Books.

Daoyuan i Chang, Chung-Yuan. 1971. \textit{Original Teachings of Ch'an Buddhism.} New York: Vintage Books.

Huineng, \textit{Sutra Szóstego Patriarchy Zen}, tłumacz nieznany, [online:] <\url{http://www.zen.ite.pl/teksty/sutra6.html}> [Dostęp 29 maja 2016]

McRae, John R. 1986. \textit{The Northern School and the Formation of Early Ch'an Buddhism}. Honolulu: University of Hawaii Press.

Shi Shengyan 釋聖嚴 1990. ``Liuzu tanjing de sixiang'' 六祖壇經的思想 [Idee Sutry Platformy Szóstego Patriarchy]. \textit{Zhonghua foxue xuebao} 中華佛學學報. [Chung-Hwa Buddhist Journal]. Taipei: The Chung-Hwa Institute of Buddhist Studies.

Fauré, Bernard. 1997. \textit{The Will to Orthodoxy: A Critical Genealogy of Northern Chan Buddhism.} Stanford, CA: Stanford University Press.

Bodhidharma i Broughton, Jeffrey L. 1999. \textit{The Bodhidharma Anthology: The Earliest Records of Zen}. Berkeley, CA: University of California Press.

Huineng i McRae, John R. 2000. \textit{The Platform Sutra of the Sixth Patriarch: Translated from the Chinese of Tsung-pao.} Berkeley, CA: Numata Center for Buddhist Translation and Research.

Li Weiying 李偉穎. 2001. ``Lüe tan Caodong zong de Chan feng'' 略談曹洞宗的禪風 [Omówienie stylu buddyzmu Chan w szkole Caodong]. \textit{Faguang zazhi} 法光雜誌 [Dharma Light Monthly], [online:] <\url{http://enlight.lib.ntu.edu.tw/FULLTEXT/JR-BJ013/bj013122209.pdf}> [Dostęp 30 maja 2016]

Buswell, Robert E. 2004. \textit{Encyclopedia of Buddhism.} New York: Macmillan Reference.

McRae, John R. 2004. \textit{Seeing Through Zen: Encounter, Transformation, and Genealogy in Chinese Chan Buddhism.} Berkeley, CA: University of California Press.

Fairbank, John King, i Goldman, Merle. 2006. \textit{China: A New History.} Cambridge, MA: Belknap Press of Harvard University Press.

Anonim 佚名 2007. ``Fojiao de futian'' 佛教的福田 [Pola błogosławieństw w buddyzmie]. \textit{Zhongguo minzu bao} 中國民族報, [online:] <\url{http://www.wuys.com/news/Article_Show.asp?ArticleID=12791}>. [Dostęp 29 maja 2016]

Poceski, Mario. 2007. \textit{Ordinary Mind as the Way: The Hongzhou School and the Growth of Chan Buddhism.} New York: Oxford University Press.

Shi Yinshun 釋印順 2008. \textit{Zhongguo Chanzong shi} 中國禪宗史 [Historia buddyzmu chińskiego tradycji Chan]. Yangzhou: Guangling shushe.

Chinese Buddhist Electronic Text Association (CBETA) 中華電子佛典協會. 2009. \textit{Nanzong dunjiao zuishang dacheng mohe bore boluomi jing liuzu Huineng dashi yu Shaozhou Dafan Si shifa tanjing yi juan} 南宗頓教最上大乘摩訶般若波羅蜜經六祖惠能大師於韶州大梵寺施法壇經一卷 [Doktryna nagłego oświecenia Szkoły Południowej, Najwyższa Doskonałość Mądrości Mahajany: Sutra Platformy, przekazana przez Szóstego Patriarchę Huineng w świątyni Dafan, w prefekturze Shao, w jednym zwoju] [w:] \textit{Taishō Tripi\d{t}aka} 大正新脩大藏經, tom 48, nr 2007. [online:] <\url{http://www.cbeta.org/result/normal/T48/2007_001.htm}> [Dostęp 29 maja 2016]

Nydahl, Ole. 2010. ``Sześć wyzwalających działań''. \textit{Diamentowa Droga} 34. [online:] <\url{http://diamentowadroga.pl/dd34/szesc_wyzwalajacych_dzialan}> [Dostęp 29 maja 2016]

Ferguson, Andrew. 2011. \textit{Zen's Chinese Heritage: The Masters and Their Teachings.} Boston: Wisdom Publications.

Huineng, Morten Schlütter i Teiser, Stephen F. 2012. \textit{Readings of the Platform sūtra.} New York: Columbia University Press.

Huineng i Yampolsky, Philip B. 2012. \textit{The Platform Sutra of the Sixth Patriarch. The Text of the Tun-huang Manuscript.} New York: Columbia University Press.

Schuman, Michael. 2015. \textit{Confucius: and the World He Created.} New York: Basic Books.
