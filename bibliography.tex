\onecolumn
\chapter*{Bibliografia}
\addcontentsline{toc}{chapter}{Bibliografia}

Jing 靜, Yun 筠 i Mo Yansheng 莫言生. \textit{Zutang ji} [Antologia gmachu patriarchów]. \url{http://www.lianhua33.com/c/c47-1.htm}

% WSTAWIĆ SUTRĘ Platformy Z CBETA
Dumoulin, Heinrich. 1963. \textit{A history of Zen Buddhism}. New York: Pantheon Books.

Daoyuan i Chung-Yuan Chang. 1971. \textit{Original teachings of Ch'an Buddhism.} New York: Vintage Books.

Huineng, Mou-lam Wong i Christmas Humphreys. 1973. \textit{The sutra of Wei Lang (or Hui Neng)}. Westport, Conn: Hyperion Press. \url{http://www.sinc.sunysb.edu/Clubs/buddhism/huineng/content.html}

Huineng, \textit{Sutra Szóstego Patriarchy Zen}, tłumacz nieznany, \url{http://www.zen.ite.pl/teksty/sutra6.html}

McRae, John R. 1986. \textit{The Northern School and the formation of early Ch'an Buddhism}. Honolulu: University of Hawaii Press.

Shi Shengyan 釋聖嚴 1990. ``Liuzu tanjing de sixiang'' 六祖壇經的思想 [Idee Sutry Platformy Szóstego Patriarchy]. \textit{Zhonghua foxue xuebao} 中華佛學學報. [Chung-Hwa Buddhist Journal]. Taipei: The Chung-Hwa Institute of Buddhist Studies.

Fauré, Bernard. 1997. \textit{The Will to Orthodoxy: A Critical Genealogy of Northern Chan Buddhism.} Stanford, CA: Stanford University Press.

Bodhidharma, and Jeffrey L. Broughton. 1999. \textit{The Bodhidharma anthology: the earliest records of Zen}. Berkeley, CA: University of California Press.

Huineng i John R. McRae. 2000. \textit{The Platform Sutra of the Sixth Patriarch: translated from the Chinese of Tsung-pao.} Berkeley, CA: Numata Center for Buddhist Translation and Research.

William E. Soothill i Lewis Hodous. 2003. \textit{A Dictionary of Chinese Buddhist Terms.} RoutledgeCurzon.% \url{http://buddhistinformatics.ddbc.edu.tw/glossaries/files/soothill-hodous.ddbc.pdf}

Buswell, Robert E. 2004. \textit{Encyclopedia of Buddhism.} New York: Macmillan Reference.

McRae, John R. 2004. \textit{Seeing Through Zen: Encounter, Transformation, and Genealogy in Chinese Chan Buddhism.} Berkeley, Calif: University of California Press.

Anonim 佚名 2007. ``Fojiao de futian'' 佛教的福田 [Pola błogosławieństw w buddyzmie]. \textit{Zhongguo minzu bao} 中國民族報, za: \url{http://www.wuys.com/news/Article_Show.asp?ArticleID=12791}

Shi Yinshun 釋印順 2008. \textit{Zhongguo Chanzong shi} 中國禪宗史 [Historia buddyzmu chińskiego tradycji Chan]. Yangzhou: Guangling shushe.

Nydahl, Ole. 2010. ``Sześć wyzwalających działań''. \textit{Diamentowa Droga} 34. \url{http://diamentowadroga.pl/dd34/szesc_wyzwalajacych_dzialan}

Ferguson, Andrew. 2011. \textit{Zen's Chinese Heritage: The Masters and Their Teachings.} Boston: Wisdom Publications.

Huineng, Morten Schlütter i Stephen F. Teiser. 2012. \textit{Readings of the Platform sūtra.} New York: Columbia University Press.% \url{http://site.ebrary.com/id/10538320}.

Huineng i Philip B. Yampolsky. 2012. \textit{The Platform sutra of the Sixth Patriarch. The Text of the Tun-huang manuscript.} New York: Columbia University Press.% \url{http://public.eblib.com/choice/publicfullrecord.aspx?p=909420.}

Schuman, Michael. 2015. \textit{Confucius: and the world he created.} New York: Basic Books.
