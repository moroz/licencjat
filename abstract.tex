\makeatletter
\@openrightfalse
\makeatother
\chapter*{Streszczenie}
\addcontentsline{toc}{chapter}{Streszczenia}
\markboth{STRESZCZENIE}{STRESZCZENIE}
\textit{Sutra Platformy Szóstego Patriarchy} jest jednym z najważniejszych tekstów buddyzmu Chan.
Najstarsza zachowana wersja tego dzieła powstała w VIII wieku w Chinach.
Jej autorstwo przypisuje się Dajian Huinengowi (638-713), legendarnemu Szóstemu Patriarsze Chan.
Niniejsza praca opisuje tło historyczne i okoliczności powstania tekstu, a także jego wpływ na dalszy rozwój buddyzmu chińskiego.

W drugim rozdziale pracy omówiono biografię Szóstego Patriachy.
W \textit{Sutrze Platformy} został on przedstawiony jako ubogi, niepiśmienny człowiek z południa Chin.
Przebywając w klasztorze Piątego Patriarchy, Daman Hongrena, pokonał w konkursie poezji Yuquan Shenxiu, a następnie został mianowany Szóstym Patriarchą.

Trzeci rozdział pracy poświęcony jest szczegółowej analizie \textit{Sutry Platformy} oraz zawartych w niej nauk.

W czwartym rozdziale pracy opisano wpływ sutry na dalszy rozwój buddyzmu Chan, ataki Heze Shenhui na Szkołę Północną, oraz schizmę w szkole Chan.
% przeniesienie Chan do Japonii

\chapter*{摘要}
\markboth{摘要}{摘要}
《六祖壇經》為中國禪宗佛教最重要經典之一。其最早現存的版本是在中國甘肅省敦煌市附近的石窟發覺的。

\chapter*{Abstract}
\markboth{ABSTRACT}{ABSTRACT}
\textit{The Platform Sutra of the Sixth Patriarch} is among the most important text of Chinese Chan Buddhism.
Its earliest extant version was compiled in the 8th century CE in China.
This thesis describes the historical background and the circumstances of its creation, as well as the impact it had on the further development of Chinese Buddhism.
