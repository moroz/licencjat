\makeatletter
\@openrightfalse
\makeatother
\chapter*{Streszczenie}
\addcontentsline{toc}{chapter}{Streszczenia}
\markboth{Streszczenie}{Streszczenie}
\textit{Sutra Platformy Szóstego Patriarchy} jest jednym z najważniejszych tekstów buddyzmu chan.
Najstarsza zachowana wersja tego dzieła powstała w VIII wieku w Chinach.
Jej autorstwo przypisuje się Dajian Huinengowi (638-713), legendarnemu Szóstemu Patriarsze Chan.
Niniejsza praca opisuje tło historyczne i okoliczności powstania tekstu, a także jego wpływ na dalszy rozwój buddyzmu chińskiego.

W pierwszym rozdziale pracy przedstawiono historię powstania tradycji chan. Za jej założyciela uważa się mistrza o imieniu Bodhidharma, który przeniósł nauki o \textit{dhjanie} z Indii.

W drugim rozdziale pracy omówiono biografię Szóstego Patriachy.
W \textit{Sutrze Platformy} został on przedstawiony jako ubogi, niepiśmienny człowiek z południa Chin.
Przebywając w klasztorze Piątego Patriarchy, Daman Hongrena (601-674), miał pokonać w konkursie poezji Yuquan Shenxiu (606-706), a następnie zostać mianowany Szóstym Patriarchą.

Trzeci rozdział pracy poświęcony jest szczegółowej analizie \textit{Sutry Platformy} oraz zawartych w niej nauk.
Najważniejsze z idei w tekście to identyfikacja medytacji (\textit{ding}) z mądrością (\textit{hui}), ,,oświecenie przez postrzeganie własnej natury'' (\textit{jian xing cheng fo}), a także tak zwane ,,bezforemne nauki'' (\textit{wu xiang jie}).
Chociaż \textit{Sutrę Platformy} uważa się za dzieło z kanonu \textit{Mahapradżniaparamity}, tekst łączy w sobie nauki pochodzące z różnych nurtów filozofii buddyjskiej, szczególnie \textit{Tathāgathagarbha}.

W czwartym rozdziale pracy opisano wpływ sutry na dalszy rozwój buddyzmu chan, ataki Heze Shenhui (684-758) na Szkołę Północną, oraz wynikłą z nich schizmę w szkole chan.
Tylko dwie spośród szkół wywodzących się od Huinenga przetrwały do dnia dzisiejszego. Są nimi szkoła Linji, założona przez Linji Yixuana (?-866), oraz szkoła Caodong, zapoczątkowana przez mistrzów Dongshan Liangjie (807-869) i Caoshan Benji (840-901).
Zostały one następnie przeniesione do Japonii, gdzie znane są jako szkoły Rinzai i Sōtō.

\chapter*{摘要}
\markboth{摘要}{摘要}
《六祖壇經》為中國禪宗佛教最重要的經典之一。其最早現存的版本是編於西元八世紀的敦煌本。按照傳統說法,《壇經》為六祖大鑒惠能(638-713)所說,其弟子法海所紀。本論文論述了《壇經》撰寫的情況及其對禪宗後來發展的影響。

第一章簡述了中國禪宗的來源。傳說禪宗建立者菩提達磨自天竺傳法,後來被尊為一祖。

第二章論述了惠能事跡。根據《壇經》中的傳記,惠能是一位不識字的中國南方貧民。他曾去黃梅見五祖弘忍

第三章裡,筆者仔細地分析了《壇經》及其所法門。雖然《壇經》通常被視為摩訶般若波羅密多類的經典,其文本是很多佛教哲學流派,尤其是如來藏。

第四章敘述了《壇經》對禪宗佛教後來的發展,荷澤神會對北宗的打擊與禪宗內的分裂。

\chapter*{Abstract}
\markboth{Abstract}{Abstract}
\textit{The Platform Sutra of the Sixth Patriarch} is among the most important texts of Chinese Chan Buddhism.
Its earliest extant version was compiled in the 8th century CE in China.
It is traditionally attributed to Dajian Huineng (638-713), the legendary Sixth Patriarch of Chan.
This thesis describes the historical background and circumstances of its creation, as well as the impact it had on the further development of Chinese Buddhism.

The first chapter of this paper presents the origins of Chan Buddhism, which is said to have been brought to China by a \textit{dhyāna} master called Bodhidharma.

The second chapter of this thesis deals with the biography of the Sixth Patriarch.
In the \textit{Platform Sutra}, he is presented as a poor, illiterate man from Southern China.
During his stay at the Huangmei Monastery with the Fifth Patriarch, Daman Hongren (601-674), he is said to have won a verse competiton against Yuquan Shenxiu (606-706), and to have been declared the Sixth Patriarch.

The third chapter is dedicated to a thorough analysis of the \textit{Platform Sutra} and the teachings it conveys.
The most important ideas found in the text are the identification of meditation (\textit{ding}) with wisdom (\textit{hui}), ``seeing into the self-nature and becoming a Buddha'' (\textit{jian xing cheng fo}), as well as the so-called ``formless precepts''.
Although generally considered to belong to the Mahaprajñaparamita canon, the text combines teachings originating from different doctrines of Buddhist philosophy, especially Tathāgathagarbha.

The fourth chapter describes the influence of the \textit{Platform Sutra} on the further development of Chan Buddhism, Heze Shenhui's (684-758) attacks on the Northern School, and the resulting schism in the Chan school.
Only two schools that trace their lineages back to Huineng survived to this day: the Linji school, founded by Linji Yixuan (?-866), and the Caodong school, started by the masters Dongshan Liangjie (807-869) and Caoshan Benji (840-901).
These were later transmitted to Japan, where they are known as Rinzai and Sōtō schools, respectively.
