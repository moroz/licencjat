\makeatletter
\@openrightfalse
\makeatother
\chapter*{Streszczenie}
\addcontentsline{toc}{chapter}{Streszczenia}
\markboth{STRESZCZENIE}{STRESZCZENIE}
\textit{Sutra Platformy Szóstego Patriarchy} jest jednym z najważniejszych tekstów buddyzmu chan.
Najstarsza zachowana wersja tego dzieła powstała w VIII wieku w Chinach.
Jej autorstwo przypisuje się Dajian Huinengowi (638-713), legendarnemu Szóstemu Patriarsze Chan.
Niniejsza praca opisuje tło historyczne i okoliczności powstania tekstu, a także jego wpływ na dalszy rozwój buddyzmu chińskiego.

W drugim rozdziale pracy omówiono biografię Szóstego Patriachy.
W \textit{Sutrze Platformy} został on przedstawiony jako ubogi, niepiśmienny człowiek z południa Chin.
Przebywając w klasztorze Piątego Patriarchy, Daman Hongrena (601-674), pokonał w konkursie poezji Yuquan Shenxiu (606-706), a następnie został mianowany Szóstym Patriarchą.

Trzeci rozdział pracy poświęcony jest szczegółowej analizie \textit{Sutry Platformy} oraz zawartych w niej nauk.
Chociaż \textit{Sutrę Platformy} uważa się za dzieło z kanonu Mahapradżniaparamity, tekst łączy w sobie nauki pochodzące z różnych nurtów filozofii buddyjskiej, szczególnie Tathagathagarbha.

W czwartym rozdziale pracy opisano wpływ sutry na dalszy rozwój buddyzmu chan, ataki Heze Shenhui na Szkołę Północną, oraz schizmę w szkole chan.
% przeniesienie Chan do Japonii

\chapter*{摘要}
\markboth{摘要}{摘要}
《六祖壇經》為中國禪宗佛教最重要經典之一。其最早現存的版本是編於西元八世紀的敦煌本。按照傳統說法,《壇經》為六祖大鑒惠能(638-713)所說,其弟子法海所寫。本論文論述了《壇經》撰寫的情況與它對禪宗後來的發展的影響。

本論文的第二章論述了惠能事跡。根據《壇經》裡面的傳記,惠能為中國南方貧民,不識字。

第三章裡,筆者仔細地分析了《壇經》及其所法門。雖然《壇經》通常被視為摩訶般若波羅密多類的經典,其文本是很多佛教哲學流派,尤其是如來藏。

第四章描敘述了《壇經》對禪宗佛教後來的發展,荷澤神會對北宗的打擊與禪宗內的分裂。

\chapter*{Abstract}
\markboth{ABSTRACT}{ABSTRACT}
\textit{The Platform Sutra of the Sixth Patriarch} is among the most important texts of Chinese Chan Buddhism.
Its earliest extant version was compiled in the 8th century CE in China.
It is traditionally attributed to Dajian Huineng (638-713), the legendary Sixth Patriarch of Chan.
This thesis describes the historical background and circumstances of its creation, as well as the impact it had on the further development of Chinese Buddhism.

The first chapter of this paper presents the origins of Chan Buddhism, which is said to have been brought to China by a \textit{dhyāna} master called Bodhidharma.

The second chapter of this thesis deals with the biography of the Sixth Patriarch.
In the \textit{Platform Sutra}, he is presented as a poor, illiterate man from Southern China.
During his stay at the Huangmei Monastery with the Fifth Patriarch, Daman Hongren (601-674), he is said to have won a verse competiton against Yuquan Shenxiu (606-706), and to have been declared the Sixth Patriarch.

The third chapter is dedicated to a thorough analysis of the \textit{Platform Sutra} and the teachings it conveys.
The most important ideas found in the text are the identification of meditation (\textit{ding}) with wisdom (\textit{hui}), ``seeing into the self-nature and becoming a Buddha'' (\textit{jian xing cheng fo}), as well as the so-called ``formless precepts''.
Although generally considered to belong to the Mahaprajñaparamita canon, the text combines teachings originating from different doctrines of Buddhist philosophy, especially Tathagathagarbha.

The fourth chapter describes the influence of the \textit{Platform Sutra} on the further development of Chan Buddhism, Heze Shenhui's attacks on the Northern School, and the resulting schism in the Chan school.
Only two schools that trace their lineages back to Huineng survived to this day: the Linji school, founded by Linji Yixuan, and the Caodong school, started by the masters Dongshan Liangjie and Caoshan Benji.
These were later transmitted to Japan, where they are known as Rinzai and Sōtō schools, respectively.
