%\vspace*{-15pt}
\maketitle
\vspace*{-45pt}
\section{Wprowadzenie}
\vspace*{-20pt}
\subsection{Historia tradycji Chan}
\textit{Chan} (禪宗 \pinyin{Chán zōng}) jest, obok Szkoły Czystej Krainy%
\footnote{Szkoła Czystej Krainy, zwana również Szkołą Czystej Ziemi lub amidyzmem --- tradycja buddyzmu chińskiego, w której za najważniejszą postać przyjmuje się \textit{Buddę Amitabhę} (阿彌陀佛 \pinyin{Āmìtuófó}). Celem praktyki tej tradycji jest odrodzenie po śmierci w Czystej Krainie tego buddy, \textit{Sukhavati} (極樂 \pinyin{Jílè}, 安樂 \pinyin{Ānlè}, 西天 \pinyin{Xītiān}).} % Ā U+100
(淨土宗 \pinyin{Jìngtǔ zōng}) jedną z najważniejszych tradycji buddyzmu w Chinach. Samo słowo \textit{chan} jest chińskim wariantem %odwzorowaniem
sanskryckiego słowa Dhyāna, które oznacza medytację. %Wyjaśnić
Początkowo termin ten tłumaczono jako 禪那 \pinyin{chánnà}, pierwotnie wymawiane \textit{dianna}, co było fonetycznym odwzorowaniem oryginalnego terminu. W późniejszym okresie upowszechniła się skrócona forma \textit{chan}. W Japonii buddyzm Chan znany jest pod nazwą Zen (禅 \textit{zen}), a w Korei Seon ({\Korean 선} \textit{seon}).

Buddyzm Chan został zapoczątkowany przez indyjskiego mistrza Bodhidharmę (菩提達摩 Pútídámó, od skt. \textit{bodhi} `oświecenie' i \textit{dharma} `zjawiska, porządek wszechświata, nauki Buddy'), dwudziestego ósmego patriarchę Indii. Dokładne daty jego narodzin i śmierci nie są znane, jednak współcześnie przyjmuje się, że żył on w V w. n.e. Przywędrował on z Indii do Chin, by nauczać buddyzmu Mahajany
\footnote{Buddyzm Mahajany (大乘佛教 \pinyin{Dàshèng fójiào} lub \pinyin{Dàchéng fójiào}, `buddyzm Wielkiego Wozu', od skt. \textit{Mahāyāna}, `wielki wóz', nazywany również buddyzmem Wielkiej Drogi) --- jeden z trzech głównych odłamów buddyzmu (dwa pozostałe to Hinajana, tzw. Mała Droga lub Mały Wóz, oraz Wadżrajana, Diamentowa Droga lub Diamentowy Wóz). Filarami Mahajany są wyzwalająca mądrość i współczucie dla wszystkich czujących istot, rozwijane w równowadze. Do buddyzmu Mahajany zalicza się m.in. buddyzm Chan i Zen, Szkołę Czystej Krainy, a także szkołę Gelugpa buddyzmu tybetańskiego.}
w tym kraju. Według chińskich przekazów, Bodhidharma był brodatym mężczyzną o niechińskiej fizjonomii. Z powodu bariery językowej nazywano go ,,błękitnookim barbarzyńcą'' (碧眼胡 \pinyin{Bìyǎnhú}). (Huineng and Hsüan Hua, 1998: bez nru strony; Soothill and Hodous, 2003: 1004; Buswell 2004: 57).

\subsection{Pochodzenie Sutry Platformy}

Pełen tytuł \textit{Sutry Szóstego Patriarchy} brzmi 南宗頓教最上大乘摩訶般若波羅蜜經六祖惠能大師於韶州大梵寺施法壇經 \pinyin{Nánzōng dùnjiào zuìshàng dàshèng móhēbānnuò bōluómì jīng liùzǔ Huìnéng Dàshī yú Shāozhōu Dàfán Sì shīfǎ Tánjīng}, `Doktryna nagłego oświecenia Szkoły Południowej, Najwyższa Doskonałość Mądrości Mahajany: Sutra Platformy, przekazana przez Szóstego Patriarchę Huineng w świątyni Dafan, w prefekturze Shao'. W języku chińskim zwykle nazywana jest w skrócie 壇經 \pinyin{Tánjīng} `Sutra platformy', 六祖壇經 \pinyin{Liùzǔ Tánjīng} `Sutra platformy Szóstego Patriarchy', bądź 六祖大師法寶壇經 \pinyin{Liùzǔ Dàshī Fǎbǎo Tánjīng}. Jej autorstwo przypisuje się półlegendarnemu patriarsze buddyzmu Chan, Huineng (惠能 \pinyin{Huìnéng}, zapisywane również jako 慧能). % źródło

Sutra Platformy jest uważana za jedno z najważniejszych dzieł buddyzmu Chan, ponieważ wprowadziła ona nauki o nagłym oświeceniu (頓教 \pinyin{dùnjiào}, `nagła szkoła, subityzm'), stojące w opozycji do nauk tzw. stopniowej szkoły, i wywołała podział szkoły Chan na odłam północny i południowy. % reference
(Buswell, 2004: 347-348).%; McRae, 2004: ).

Ważnym aspektem tekstu są nauki o tym, że każda czująca istota ma naturę buddy, i że zarówno ludzie świeccy, jak i mnisi mogą z powodzeniem praktykować jego nauki. Tekst opisuje również specjalny rytuał przekazywania mnichom i świeckim praktykującym ,,bezforemnych zasad''. % formless precepts
Były to niektóre z powodów, dla których w roku 796 Huineng został oficjalnie obwołany szóstym patriarchą Chan przez cesarską komisję, a jego dzieło stworzyło podwaliny pod dalszy rozwój szkoły Chan (Huineng, Schlütter and Teiser, 2012: 2).

\if 0
\subsection{Budowa Sutry Platformy}
Oryginalny tekst Sutry Platformy ma postać ciągłego tekstu, jednak w przekładzie została ona podzielona na dziesięć rozdziałów. W pierwszym rozdziale, \textit{Autobiografia}, Huineng opisuje historię swojego życia. W rozdziałach 2-9 zawarte są nauki w formie przemów wygłoszonych dla zgromadzenia słuchaczy. 
\fi

\section{Życie Szóstego Patriarchy}
Życie Szóstego Patriarchy Huineng jest owiane tajemnicą. Jego imię pojawia się bowiem w zapisach historycznych, tzw. \textit{Księdze przekazu lampy z okresu Jingde} (景德傳燈錄 \pinyin{Jǐngdé chuán dēng lù}) jako jednego z dziesięciu głównych uczniów piątego patriarchy Hongren, z tekstu nie wynika jednak, by był postacią szczególnie ważną dla rozwoju całej szkoły Chan. W tekście tym wspomniano, że Huineng żył i nauczał w miejscowości Caoxi (曹溪 \pinyin{Cáoxī}, również: \pinyin{Cáoqī}). Imię Huineng pojawia się również w pewnym tekście z Dunhuang, upamiętniającym Piątego Patriarchę Hongren, jednak tekst ów nie mówi nic o przypisywanych Huinengowi doktrynach.

Informacje o życiu legendarnego Huinenga zaczerpnięte są w dużej mierze z przypisywanej mu \textit{Sutry Platformy}. Postać ta jest przedstawiana jako ubogi, niepiśmienny człowiek świecki z południa Chin (McRae, 2004: 68).

Jak podaje tekst sutry, Huineng urodził się w miejscowości Xinxing w regionie Nanhai (南海新興 \pinyin{Nánhǎi Xīnxīng}, obecnie prowincja Guangdong). Za ramy czasowe jego życia przyjmuje się lata 638-713.
Jak podaje tekst, jego ojciec był urzędnikiem z regionu Fanyang (范陽 \pinyin{Fànyáng}, obecnie miasto Zhuozhou w prowincji Hebei 河北省涿州市 \pinyin{Héběi shěng Zhuōzhōu shì}), lecz został odwołany ze stanowiska i skazany na banicję. W związku z tym musiał przenieść się z całą rodziną do Xinxing, gdzie niedługo później zmarł. Po jego śmierci, Huineng trudnił się zbieraniem i sprzedażą drewna na opał.

%\subsection{Pierwsze zetknięcie z Dharmą}
Pewnego dnia, gdy dwudziestodwuletni Huineng sprzedawał drewno na targowisku, pewien klient zażyczył sobie, żeby drewno zostało przyniesione do jego sklepu. Huineng dostarczył drewno i dostał za nie pieniądze, a kiedy wyszedł ze sklepu, spotkał mężczyznę, który recytował na ulicy Sutrę Diamentową (金剛經 Jīngāng jīng, skt. \textit{Vajracchedikā Prajñāpāramitā Sūtra}
). Usłyszawszy ów tekst, Huineng uzyskał wgląd w naturę swego umysłu i osiągnął oświecenie. Następnie spytał mężczyznę, skąd przybył. Ten odpowiedział, że przybył z klasztoru Dongshan, na górze Fengmushan, w powiecie Huangmei, w Qizhou (蘄州黃梅懸東憑墓山
\pinyin{Qízhōu Huángméi xiàn Dōng Píngmù shān}), którego opatem był Piąty Patriarcha, Hong Ren (弘忍 \pinyin{Hóngrěn}), i gdzie przebywało około tysiąca mnichów. Patriarcha zalecał mnichom, aby recytowali ową sutrę, ponieważ dzięki tej praktyce można szybko osiągnąć oświecenie (Huineng and Hsüan Hua 1977: bez nru strony; Huineng and Yampolsky, 2012: bez nru strony).

\subsection{Podróż}
Pewien człowiek poradził Huinengowi, aby udał się do klasztoru Dongshan, aby poprosić Patriarchę o nauki, i dał mu pieniądze, aby mógł zaaranżować opiekę dla swej matki.

Kiedy Huineng przybył do klasztoru, spytano go:

--- Skąd przybyłeś i czego chcesz od patriarchy?

--- Wywodzę się z prostego ludu Xinzhou, w Kantonie --- odparł Huineng. --- Przybyłem z daleka, aby oddać cześć patriarsze. Nie proszę o nic prócz nauk Buddy.

--- Pochodzisz z Kantonu. Jesteś więc barbarzyńcą. Jak możesz stać się buddą?

--- Ludzie dzielą się na tych z południa i tych z północy, ale takie podziały nie mają wpływu na ich naturę buddy (Huineng, Wong and Humphreys, 1998: rozdział 1).

Patriarcha uznał, że Huineng dobrze rozumiał nauki Buddy, lecz w obawie, że inni uczniowie mogliby zrobić mu krzywdę, kazał mu iść pracować w stajni. Tam, przez następne osiem miesięcy, Huineng rąbał drewno i młócił zboże.

\subsection{Dwa wiersze}

Pewnego dnia Patriarcha zwołał zebranie wszystkich uczniów i ogłosił: ,,Uwarunkowana egzystencja jest kwestią doniosłą. Dzień po dniu zasiewacie tylko nasiona ponownego odrodzenia, zamiast starać się wyzwolić z oceanu samsary.\footnote{[Huineng and Yampolsky, 2012] podaje w tym miejscu: ,,Całymi dniami składacie ofiary i szukacie tylko pola błogosławieństw, (\ldots)''. Idea ,,pola błogosławieństw'' (福田 \pinyin{fútián}) *** jest związana z buddyjską koncepcją karmy jako prawa przyczyny i skutku. Oznacza stan, w którym dana osoba zgromadziła bardzo wiele dobrej karmy w rezultacie praktyki szczodrości (skt. \textit{dānā}), pierwszej z tzw. Sześciu Paramit lub Sześciu Wyzwalających Działań (pozostałe pięć to właściwe działanie, cierpliwość, radosny wysiłek, medytacja i mądrość). Słowo \textit{dānā} pojawia się m.in. w \textit{Dānādhikāramahāyānasūtra} (佛说布施经 \pinyin{Fóshuō bùshī jīng}): “若求勝妙福報而行施時,慈心不殺離諸嫉妒,正見相應遠於不善,堅持禁戒親近善友,閉惡趣門開生天路,自利利他其心平等。若如是施,是真布施,是大福田。” Piąty Patriarcha Hongren krytykował tu swoich uczniów, ponieważ praktyka szczodrości jest wprawdzie w buddyzmie postrzegana jako pozytywne działanie, jednak nie wystarcza ona do osiągnięcia ostatecznego wyzwolenia z samsary. (Anonim, 2007; Nydahl, 2010)} Te działania w niczym wam nie pomogą, jeżeli esencja waszego umysłu jest przysłoniona. Szukajcie pradżni (mądrości) w swoim umyśle i napiszcie wiersz na ten temat. Ten z was, który rozpozna esencję umysłu, otrzyma ode mnie szatę Patriarchy i przekaz nauk. (\ldots) Człowiek, który urzeczywistnił esencję umysłu, potrafi mówić o niej od razu, kiedy tylko zostanie o nią zapytany; nigdy też nie jest w stanie jej utracić, nawet podczas bitwy.'' (Huineng, Wong and Humphreys, 1998: rozdział 1).

Mnisi stwierdzili zgodnie: „Nie ma sensu oczyszczać umysłu i zadawać sobie trudu układania wiersza dla patriarchy. Shenxiu (神秀 Shénxiù), przewodniczący kongregacji, jest naszym nauczycielem. Kiedy on zostanie patriarchą, możemy liczyć na jego wsparcie. Dlatego nie będziemy układać wierszy.” Żaden z nich nie podjął się więc tego zadania (Huineng and Yampolsky, 2012: bez nru strony).

Shenxiu zaś dręczyły wątpliwości. Z jednej strony nie uważał, żeby jego zrozumienie nauk i urzeczywistnienie były wystarczające do przyjęcia stanowiska patriarchy, z drugiej zaś pragnął otrzymać przekaz Dharmy. Przyjęcie Dharmy dla pożytku istot byłoby bowiem pożądane i chwalebne, jednak dążenie do objęcia stanowiska patriarchy byłoby niewłaściwe.

Wychodził z założenia, że jeżeli Piąty Patriarcha Hongren uzna, że jego urzeczywistnienie natury umysłu jest niewystarczające, to będzie musiał porzucić dążenia do zostania dzierżawcą linii przekazu. Wreszcie skomponował wiersz i wymknął się w nocy ze swojej celi, by napisać go na ścianie:

\vspace*{6pt}
\begin{minipage}[t]{0.4\textwidth}
\begin{verse}
身是菩提樹\\
心如明鏡臺\\
時時勤佛拭\\
莫使有塵埃
\end{verse}
\end{minipage}
\begin{minipage}[t]{0.6\textwidth}
\itshape
\begin{verse}
Ciało jest drzewem Bodhi,\\
Umysł --- jasną lustrzaną podstawą.\\
Czyść ją stale i gorliwie,\\
Nie pozwalając aby przylgnął kurz. 
\end{verse}
\end{minipage}
\vspace*{6pt}

O poranku, Piąty Patriarcha Hongren ujrzał wiersz, napisany przez Shenxiu na ścianie i uznał, że wiersz ten mógł przynieść ludziom wiele pożytku. Zwołał całe zgromadzenie i nakazał mnichom recytować go. Patriarcha spytał Shenxiu, czy to on jest autorem wiersza, oznaczałoby to bowiem, że jest on właściwym spadkobiercą Dharmy i jego następcą na stanowisku patriarchy. Powiedział też jednak, że wiersz nie wskazuje na to, aby Shenxiu rozpoznał już naturę swojego umysłu. Wiersz nadawał się do recytacji przez zwykłych ludzi, jednak takie niepełne zrozumienie nie wystarczało do rozpoznania prawdziwej natury umysłu. Powiedział, że przekaże mu nauki i szatę patriarchy, jeżeli temu w ciągu dwóch dni uda się osiągnąć ostateczne urzeczywistnienie. Shenxiu rozmyślał przez wiele dni, ale nie udało mu się skomponować nic lepszego. (Huineng and Yampolsky, 2012: bez nru strony).

Pewnego dnia pewien młody mnich przechodził koło stajni, w której pracował akurat Huineng, recytując wiersz Shenxiu. Huineng zrozumiał, że autor wiersza nie rozpoznał jeszcze natury swojego umysłu. Spytał mnicha, co ten recytował.

--- Czy nie słyszałeś? --- odpowiedział pytaniem mnich. --- Patriarcha kazał wszystkim swoim uczniom skomponować wiersz i przynieść mu go, aby określić, kto otrzyma przekaz Dharmy i zostanie patriarchą. Pewien mnich o imieniu Shenxiu skomponował ten wiersz o pustości. Piąty Patriarcha nakazał wszystkim swoim uczniom recytować go, mówiąc, że ci, którzy go urzeczywistnią, zobaczą swoją prawdziwą naturę, a ci, którzy praktykują zgodnie z nim, osiągną wyzwolenie.

Huineng poprosił mnicha, aby ten zaprowadził go do miejsca, gdzie na ścianie wymalowany był wiersz Shenxiu. Ponieważ nie umiał czytać, poprosił kogoś, by przeczytał mu ten wiersz na głos. Usłyszawszy go, osiągnął ostateczne urzeczywistnienie. Huineng ułożył własny wiersz i poprosił kogoś o napisanie go na ścianie (Huineng and Yampolsky, 2012: bez nru strony).

%\vspace*{6pt}
\begin{minipage}[t]{0.4\textwidth}
\begin{verse}
菩提本無樹\\
明鏡亦無臺\\
佛性常清淨\\
何處有塵埃
\end{verse}
\end{minipage}
\begin{minipage}[t]{0.6\textwidth}
\itshape
\begin{verse}
Sama istota Bodhi nie ma drzewa,\\
Nie ma też jasnej lustrzanej podstawy.\\
W rzeczywistości nie ma niczego,\\
Cóż miałoby przyciągać jakikolwiek kurz?
\end{verse}
\end{minipage}
\vspace*{6pt}

Mnisi ze zgromadzenia byli pod wielkim wrażeniem wiersza Huinenga, a Piąty Patriarcha Hongren stwierdził na jego podstawie, że Huineng miał już wówczas ponadprzeciętne zrozumienie natury zjawisk, ale dla bezpieczeństwa Huinenga oznajmił zgromadzeniu, że wciąż nie było to pełne urzeczywistnienie.

Patriarcha przywołał go do siebie w nocy i udzielił mu wyjaśnień do Sutry Diamentowej, dzięki której Huineng natychmiast rozpoznał naturę umysłu. Hongren przekazał mu również nauki o spontanicznym oświeceniu oraz szatę, insygnium patriarchatu. Przekazał mu również ostatnie pouczenia: ,,Mianuję cię Szóstym Patriarchą. Szata jest tego dowodem i jest przekazywana z pokolenia na pokolenie. Moje nauki powinny być przekazywane z umysłu na umysł. Spraw, by ludzie rozpoznali swoją prawdziwą naturę. Od czasów starożytnych przekaz Dharmy był równie słaby, jak zwisający sznurek. Jeżeli pozostaniesz tutaj, inni ludzie zrobią ci krzywdę. Musisz więc niezwłocznie odejść.'' (Huineng and Yampolsky, 2012: bez nru strony)

Huineng udał się na południe. Jego śladem podążyło kilkuset ludzi, pragnących zabić Huinenga i siłą odebrać od niego szatę i Dharmę. W związku z prześladowaniami, Huineng schronił się w miejscu zwanym Caoxi (曹溪, także: Caoqi), gdzie przez piętnaście lat ukrywał się wśród prostego ludu --- myśliwych. Dopiero potem opuścił miejsce odosobnienia i zaczął nauczać Dharmy (Huineng, Wong and Humphreys, 1998: rozdział 5).

\section{Nauki Szóstego Patriarchy}

Sutra Platformy wprowadziła nauki o nagłym oświeceniu, jednak podział na pojęcia ,,ścieżki nagłego oświecenia'' i ,,stopniowej ścieżki'' są raczej pozorne, gdyż tak naprawdę chodzi tu o indywidualne zdolności uczniów --- inteligentniejsi, z otwartymi umysłami, są w stanie pojąć nauki o pustości i naturze buddy, i osiągnąć oświecenie w jednej chwili, podczas gdy inni muszą ćwiczyć się na owej ścieżce stopniowo.

W czwartym rozdziale sutry, Huineng nauczał, że za obiekt praktyki duchowej należy przyjąć ,,brak idei'', za jej podstawę przyjąć ,,brak obiektu'', zaś jej fundamentalną zasadą należy uczynić ,,brak przywiązania''. Są to trzy zbliżone i nierozerwalnie związane koncepcje.

,,Brak idei'' jest rozumiany jako wolność od rozproszenia --- pilnowanie, by umysł nie podążał za myślami i aby nic, co pojawia się w umyśle nie odwodziło go od praktyki. W przeciwnym razie, jeżeli praktykujący poświęca czas i energię swoim myślom o teraźniejszości, przeszłości i przyszłości, zaczną one pojawiać się, jedna po drugiej, i ograniczać przejrzystość umysłu. Błędem jest również próba całkowitego pozbycia się myśli; taka praktyka nie umożliwia rozpoznania natury umysłu i nie prowadzi do wyzwolenia. Właściwą praktyką jest koncentracja na prawdziwej naturze takości (真如 zhēnrú, skt. \textit{tathātā}), gdyż „takość jest esencją idei, a idea jest wynikiem aktywności Takości”.

,,Brak obiektu'' oznacza tu unikanie rozproszenia pod wpływem zewnętrznych obiektów. ,,Brak przywiązania'' zaś oznacza traktowanie wszystkich istot, zarówno wrogów, jak też przyjaciół, w taki sam sposób. Praktykujący powinien porzucić myślenie o przeszłości i chęć odwetu za dawne krzywdy (Huineng, Wong and Humphreys, 1998: rozdział 4).

Rozdział piąty Sutry Platformy traktuje o \textit{dhyāna}, medytacji. W medytacji Chan nie należy koncentrować się ani na umyśle, ani na czystości. Umysł jako taki jest zwodniczy, jest jedynie iluzją i jako taki nie powinien być obiektem medytacji. Koncentracja na czystości zaś prowadzi do fiksacji na koncepcji czystości. Właściwa medytacja oznacza urzeczywistnienie niewzruszonej esencji umysłu.

Taka praktyka powinna być zrównoważona na poziomie ciała, mowy i umysłu. Praktykujący, który pragnie rozwinąć nieporuszoność, powinien być obojętny na wady innych ludzi. Niewzruszony umysł nie działa w dualistycznych kategoriach, takich jak dobro i zło albo słabość i siła. Analogicznie, praktykujący nie powinien mówić krytykować innych ludzi przy pomocy tych kategorii myślowych (Huineng, Wong and Humphreys, 1998: rozdział 5).
%dwell on -- skupiać się na

\vspace*{-15pt}
\section*{Bibliografia}

Huineng, Mou-lam Wong, and Christmas Humphreys. 1973. \textit{The sutra of Wei Lang (or Hui Neng)}. Westport, Conn: Hyperion Press. \url{http://www.sinc.sunysb.edu/Clubs/buddhism/huineng/content.html}

Huineng, and Hsüan Hua. 1977. \textit{The Sixth Patriarch's Dharma jewel platform sutra, with the commentary of Tripitaka Master Hua} [translated from the Chinese by the Buddhist Text Translation Society]. San Francisco: Sino-American Buddhist Association. \url{http://www.cttbusa.org/6patriarch/6patriarch_contents.asp}

Huineng, \textit{Sutra Szóstego Patriarchy Zen}, tłumacz nieznany, \url{http://www.zen.ite.pl/teksty/sutra6.html}

William E. Soothill, and Lewis Hodous. 2003. \textit{A Dictionary of Chinese Buddhist Terms.} RoutledgeCurzon. \url{http://buddhistinformatics.ddbc.edu.tw/glossaries/files/soothill-hodous.ddbc.pdf}

Buswell, Robert E. 2004. \textit{Encyclopedia of Buddhism.} New York: Macmillan Reference, USA.

McRae, John R. 2004. \textit{Seeing through Zen encounter, transformation, and genealogy in Chinese Chan Buddhism.} Berkeley, Calif: University of California Press.

Anonim 佚名 2007. ``Fojiao de futian'' 佛教的福田 [Fields of blessings in Buddhism]. \textit{Zhongguo minzu bao} 中國民族報, za: \url{http://www.wuys.com/news/Article_Show.asp?ArticleID=12791}

Nydahl, Ole. 2010. ``Sześć wyzwalających działań''. \textit{Diamentowa Droga} 34. \url{http://diamentowadroga.pl/dd34/szesc_wyzwalajacych_dzialan}

Huineng, Morten Schlütter, and Stephen F. Teiser. 2012. \textit{Readings of the Platform sūtra.} New York: Columbia University Press. \url{http://site.ebrary.com/id/10538320}.

Huineng, Philip B. Yampolsky, and Huineng. 2012. \textit{The Platform sutra of the Sixth Patriarch the text of the Tun-huang manuscript.} New York: Columbia University Press. \url{http://public.eblib.com/choice/publicfullrecord.aspx?p=909420.}
