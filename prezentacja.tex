\maketitle

\section{Rys historyczny}

Sutra Szóstego Patriarchy Zen (jej pełen tytuł brzmi 南宗頓教最上大乘摩訶般若波羅蜜經六祖惠能大師於韶州大梵寺施法壇經 \pinyin{Nan2zong1 dun4jiao4 zui4shang4 da4sheng4 mo2he1ban1nuo4 bo1luo2mi4 jing1 liu4zu4 hui4neng2 da4shi1 yu2 Shao1zhou1 Da4fan2 Si4 shi1fa3 tan2jing1}, `Doktryna nagłego oświecenia Szkoły Południowej, Najwyższa Doskonałość Mądrości Mahajany: Sutra Platformy, przekazana przez Szóstego Patriarchę Huineng w świątyni Dafan, w prefekturze Shao', zwana również \emph{Sutrą platformy}), jest jednym z najważniejszych dzieł buddyzmu Chan. Jej twórcą jest szósty, ostatni patriarcha buddyzmu Chan, Huineng (惠能 \pinyin{Hui4neng2}, zapisywane również jako 慧能).

Southern School's Sudden Doctrine, Supreme Mahayana Great Perfection of Wisdom: The Platform Sutra as Delivered by the Sixth Patriarch Huineng at the Dafan Temple in Shao Prefecture

Sutra Spoken by the Sixth Patriarch on the High Seat of the Treasure of the Law

\section{Życie Szóstego Patriarchy}
Huineng urodził się w okolicach Guangzhou
Pewnego dnia na ulicy usłyszał człowieka, który recytował sutrę

Medytacja

[Reperkusje historyczne]

% Bibliografia

Hui Neng, \emph{Sutra Spoken by the Sixth Patriarch on the High Seat of the Treasure of the Law} http://www.sinc.sunysb.edu/Clubs/buddhism/huineng/content.html
