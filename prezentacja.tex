\maketitle
\section{Wprowadzenie}
\subsection{Historia tradycji Chan}
\emph{Chan} (禪宗 \pinyin{Chan2 zong1}) jest, obok Szkoły Czystej Krainy (淨土宗 \pinyin{Jing4tu3 zong1}, zwanej również Szkołą Czystej Ziemi) jedną z najważniejszych tradycji buddyzmu w Chinach. Samo słowo \emph{chan} jest chińskim wariantem %odwzorowaniem
sanskryckiego słowa Dhyāna, które oznacza medytację. %Wyjaśnić
Początkowo termin ten tłumaczono jako 禪那 \pinyin{chan2na4}, pierwotnie wymawiane \emph{dianna}, co było fonetycznym odwzorowaniem oryginalnego terminu. W późniejszym okresie upowszechniła się skrócona forma \emph{chan}.

Buddyzm Chan został zapoczątkowany *** przez indyjskiego mistrza Bodhidharmę (菩提達摩 \pinyin{Pu2ti2da2mo2}, od skt. \emph{bodhi} `oświecenie' i \emph{dharma} `zjawiska, porządek wszechświata, nauki Buddy'), dwudziestego ósmego patriarchę Indii. Dokładne daty jego narodzin i śmierci nie są znane, jednak współcześnie przyjmuje się, że żył on w V w. n.e. Przywędrował on z Indii do Chin, by nauczać buddyzmu Mahajany\footnote{Buddyzm Mahajany} w tym kraju. Według chińskich przekazów, Bodhidharma był brodatym mężczyzną o niechińskiej fizjonomii. Z powodu bariery językowej nazywano go ,,błękitnookim barbarzyńcą'' (碧眼胡 \pinyin{Bi4yan3hu2}). (Huineng and Hsüan Hua, 1998: bez nru strony; Soothill and Hodous, 2003: 1004; Buswell 2004: 57).

\subsection{Pochodzenie Sutry Platformy}

Pełen tytuł \emph{Sutry Szóstego Patriarchy} brzmi 南宗頓教最上大乘摩訶般若波羅蜜經六祖惠能大師於韶州大梵寺施法壇經 \pinyin{Nan2zong1 dun4jiao4 zui4shang4 da4sheng4 mo2he1ban1nuo4 bo1luo2mi4 jing1 liu4zu4 hui4neng2 da4shi1 yu2 Shao1zhou1 Da4fan2 Si4 shi1fa3 tan2jing1}, `Doktryna nagłego oświecenia Szkoły Południowej, Najwyższa Doskonałość Mądrości Mahajany: Sutra Platformy, przekazana przez Szóstego Patriarchę Huineng w świątyni Dafan, w prefekturze Shao'. W języku chińskim zwykle nazywana jest w skrócie 壇經 \pinyin{Tan2jing1} `Sutra platformy', 六祖壇經 \pinyin{Liu4zu3 tan2jing1} `Sutra platformy Szóstego Patriarchy', bądź 六祖大師法寶壇經 \pinyin{Liu4zu3 da4shi1 fa3bao3 tan2jing1}. Jej twórcą jest szósty, ostatni patriarcha buddyzmu Chan, Huineng (惠能 \pinyin{Hui4neng2}, zapisywane również jako 慧能). % źródło

Sutra Platformy jest jednym z najważniejszych dzieł buddyzmu Chan, ponieważ wprowadziła ona nauki tzw. szkoły nagłego oświecenia i w efekcie podział szkoły Chan na odłam północny i południowy. % reference

\subsection{Budowa Sutry Platformy}
Oryginalny tekst Sutry Platformy ma postać ciągłego tekstu, jednak w przekładzie została ona podzielona na dziesięć rozdziałów. W pierwszym rozdziale, \emph{Autobiografia}, Huineng opisuje historię swojego życia. W rozdziałach 2-9 zawarte są nauki w formie przemów wygłoszonych dla zgromadzenia słuchaczy. 

\section{Życie Szóstego Patriarchy}
Huineng urodził się w roku 638 n.e. w miejscowości Xinxing w regionie Nanhai (南海新興 \pinyin{Nan2hai3 Xin1xing1}, obecnie prowincja Guangdong 廣東 \pinyin{Guang3dong1}). W pierwszym rozdziale Sutry Platformy, Huineng opowiada historię swojego życia. Jak podaje, jego ojciec był urzędnikiem z regionu Fanyang (范陽 \pinyin{Fan4yang2}, obecnie miasto Zhuozhou w prowincji Hebei 河北省涿州市 \pinyin{He2bei3 sheng3 Zhuo1zhou1 shi4}), lecz został odwołany ze stanowiska i skazany na banicję. W związku z tym musiał przenieść się z całą rodziną do Xinxing, gdzie niedługo później zmarł. Po jego śmierci, rodzina trudniła się sprzedażą drewna na opał.

\subsection{Pierwsze zetknięcie z Dharmą}
Pewnego dnia, gdy dwudziestodwuletni Huineng sprzedawał drewno na targowisku, pewien klient zażyczył sobie, żeby drewno zostało przyniesione do jego sklepu. Huineng dostarczył drewno i dostał za nie pieniądze, a kiedy wyszedł ze sklepu, spotkał mężczyznę, który recytował na ulicy Sutrę Diamentową (金剛經 \pinyin{Jin1gang1 jing1}, skt. Vajracchedikā Prajñāpāramitā Sūtra % 金剛般若波羅蜜多經
). Usłyszawszy ów tekst, Huineng uzyskał wgląd w naturę swego umysłu i osiągnął oświecenie. Następnie spytał mężczyznę, skąd przybył. Ten odpowiedział, że przybył z klasztoru Dongshan (wpisać znaki i pinyin), którego opatem był Piąty Patriarcha, Hong Ren (弘忍 \pinyin{Hong2ren3}), i gdzie przebywało około tysiąca mnichów. Patriarcha zalecał mnichom, aby recytowali ową sutrę, ponieważ dzięki tej praktyce można szybko osiągnąć oświeceni (Huineng and Hsüan Hua 1977: bez nru strony; Huineng, Wong and Humphreys, 1998: bez nru strony).

Pewien człowiek poradził Huinengowi, aby udał się do klasztoru Dongshan, aby poprosić Patriarchę o nauki, i dał mu pieniądze, aby mógł zaaranżować opiekę dla swej matki.

Kiedy Huineng przybył do klasztoru, spytano go:

--- Skąd przybyłeś i czego chcesz od patriarchy?

--- Wywodzę się z prostego ludu Xinzhou, w Kantonie --- odparł Huineng. --- Przybyłem z daleka, aby oddać cześć patriarsze. Nie proszę o nic prócz nauk Buddy.

--- Pochodzisz z Kantonu. Jesteś więc barbarzyńcą. Jak możesz stać się buddą?

--- Ludzie dzielą się na tych z południa i tych z północy, ale takie podziały nie mają wpływu na ich naturę buddy (Huineng, Wong and Humphreys, 1998: rozdział 1).

Patriarcha uznał, że Huineng dobrze rozumiał nauki Buddy, lecz w obawie, że inni uczniowie mogliby zrobić mu krzywdę, kazał mu iść pracować w stajni. Tam, przez następne osiem miesięcy, Huineng rąbał drewno i młócił zboże.

\subsection{Dwa wiersze}

Pewnego dnia Patriarcha zwołał zebranie wszystkich uczniów i ogłosił: ,,Uwarunkowana egzystencja jest kwestią doniosłą. Dzień po dniu zasiewacie tylko nasiona ponownego odrodzenia, zamiast starać się wyzwolić z oceanu samsary. Te działania w niczym wam nie pomogą, jeżeli esencja waszego umysłu jest przysłoniona. Szukajcie pradżni (mądrości) w swoim umyśle i napiszcie wiersz na ten temat. Ten z was, który rozpozna esencję umysłu, otrzyma ode mnie szatę Patriarchy i przekaz nauk. (\ldots) Człowiek, który urzeczywistnił esencję umysłu, potrafi mówić o niej od razu, kiedy tylko zostanie o nią zapytany; nigdy też nie jest w stanie jej utracić, nawet podczas bitwy.'' (Huineng, Wong and Humphreys, 1998: rozdział 1).

\begin{verse}
身是菩提樹\\
心如明鏡臺\\
時時勤佛拭\\
莫使有塵埃
\end{verse}

\begin{verse}
菩提本無樹\\
明鏡亦無臺\\
佛性常清淨\\
何處有塵埃

心是菩提樹\\
身為明鏡臺\\
明鏡本清淨\\
何處染塵埃
\end{verse}

\section{Medytacja}

Pojęcia ,,ścieżka nagłego oświecenia'' i ,,stopniowa ścieżka'' są raczej pozorne, gdyż tak naprawdę chodzi tu o indywidualne zdolności uczniów --- inteligentniejsi, z otwartymi umysłami, są w stanie pojąć nauki o pustości i naturze buddy, i osiągnąć oświecenie w jednej chwili, podczas gdy inni muszą ćwiczyć się na owej ścieżce stopniowo.

% People under delusion believe obstinately in Dharmalaksana (things and form) and so they are stubborn in having their own way of interpreting the 'Samadhi of Specific Mode', which they define as 'sitting quietly and continuously without letting any idea arise in the mind'. Such an interpretation would rank us with inanimate objects, and is a stumbling block to the right Path which must be kept open. Should we free our mind from attachment to all 'things', the Path becomes clear; otherwise, we put ourselves under restraint.[5] If that interpretation 'sitting quietly and continuously, etc.' be correct, why on one occasion was Sariputra reprimanded by Vimalakirti for sitting quietly in the wood? [6]

W czwartym rozdziale sutry, Huineng nauczał, że za obiekt praktyki duchowej należy przyjąć ,,brak idei'', za jej podstawę przyjąć ,,brak obiektu'', zaś jej fundamentalną zasadą należy uczynić ,,brak przywiązania''. Są to trzy zbliżone i nierozerwalnie związane koncepcje.

,,Brak idei'' jest rozumiany jako wolność od rozproszenia --- pilnowanie, by umysł nie podążał za myślami i aby nic, co pojawia się w umyśle nie odwodziło go od praktyki. W przeciwnym razie, jeżeli praktykujący poświęca czas i energię swoim myślom o teraźniejszości, przeszłości i przyszłości, zaczną one pojawiać się, jedna po drugiej, i ograniczać przejrzystość umysłu. Błędem jest również próba całkowitego pozbycia się myśli; taka praktyka nie umożliwia rozpoznania natury umysłu i nie prowadzi do wyzwolenia. Właściwą praktyką jest koncentracja na prawdziwej naturze takości (真如 \pinyin{zhen1ru2}, skt. \emph{tathātā}), gdyż „takość jest esencją idei, a idea jest wynikiem aktywności Takości”.

,,Brak obiektu'' oznacza tu unikanie rozproszenia pod wpływem zewnętrznych obiektów. ,,Brak przywiązania'' zaś oznacza traktowanie wszystkich istot, zarówno wrogów, jak też przyjaciół, w taki sam sposób. Praktykujący powinien porzucić myślenie o przeszłości i chęć odwetu za dawne krzywdy (Huineng, Wong and Humphreys, 1998: rozdział 4).

Rozdział piąty Sutry Platformy traktuje o \emph{dhyana}, medytacji. W medytacji Chan nie należy koncentrować się ani na umyśle, ani na czystości. Umysł jako taki jest zwodniczy, jest jedynie iluzją i jako taki nie powinien być obiektem medytacji. Koncentracja na czystości zaś prowadzi do fiksacji na koncepcji czystości. Właściwa medytacja oznacza urzeczywistnienie niewzruszonej esencji umysłu.

Taka praktyka powinna być zrównoważona na poziomie ciała, mowy i umysłu. Praktykujący, który pragnie rozwinąć nieporuszoność, powinien być obojętny na wady innych ludzi. Niewzruszony umysł nie działa w dualistycznych kategoriach, takich jak dobro i zło albo słabość i siła. Analogicznie, praktykujący nie powinien mówić krytykować innych ludzi przy pomocy tych kategorii myślowych (Huineng, Wong and Humphreys, 1998: rozdział 5).
%dwell on -- skupiać się na

\section*{Bibliografia}

Huineng, Mou-lam Wong, and Christmas Humphreys. \emph{The Sutra of Wei Lang (or Hui Neng), Translated from the Chinese by Wong Mou-Lam. New Edition by C. Humphreys.} Westport, Conn: Hyperion Press, 1998 
\url{http://www.sinc.sunysb.edu/Clubs/buddhism/huineng/content.html}

Huineng, and Hsüan Hua. \emph{The Sixth Patriarch's Dharma Jewel Platform Sutra, with the Commentary of Tripitaka Master Hua} [Translated from the Chinese by the Buddhist Text Translation Society]. San Francisco: Sino-American Buddhist Association, 1977.
\url{http://www.cttbusa.org/6patriarch/6patriarch_contents.asp}

Buswell, Robert E. \emph{Encyclopedia of Buddhism.} New York: Macmillan Reference, USA, 2004. 

William E. Soothill, and Lewis Hodous. A Dictionary of Chinese Buddhist Terms. RoutledgeCurzon, 2003. \url{http://buddhistinformatics.ddbc.edu.tw/glossaries/files/soothill-hodous.ddbc.pdf}

Hui Neng, 南宗頓教最上大乘摩訶般若波羅蜜經六祖惠能大師於韶州大梵寺施法壇經

Heinrich Dumoulin, \emph{A History of Zen Buddhism}, New York, Pantheon Books 1963