\chapter{Analiza tekstu \textit{Sutry Platformy}}

\section{Budowa Sutry}
Tekst \textit{Sutry Platformy} w przekładzie Philipa B. Yampolsky'ego został podzielony na 57 sekcji o różnej długości.
Narratorem większej części tekstu jest Fahai, uczeń Huinenga, wyjątek stanowią fragmenty 2-11, spisane w formie autobiograficznego monologu Szóstego Patriarchy.
Pierwsze 37 sekcji zawiera nauki wygłoszone przez Huinenga w świątyni Dafan.
Jak podaje sutra, wysłuchało ich zgromadzenie ponad dziesięciu tysięcy praktykujących, w tym mnichów, mniszek i ludzi świeckich, a także prefekt Shaozhou, Wei Qu.
W sekcjach 12-37 przekazano ,,bezforemne nauki''.

Sekcje 39-44 zawierają anegdoty dotyczące najważniejszych spadkobierców Huinenga, takich jak Fahai, Zhichang%
\footnote{智常 \pinyin{Zhìcháng}. Nie mylić z innym uczniem Huinenga o imieniu Zhicheng (志誠 \nazwisko{Zhìchéng}).\label{ZhiChangChengDisambiguation}}, Fada (法達 \pinyin{Fǎdá}) i Shenhui.
Sekcje 45-47 obejmują nauki udzielone dziesięciu bliskim uczniom.
Sekcje 48-57 opisują okoliczności śmierci Szóstego Patriarchy.
We fragmencie tym zawarto również nauki, które Huineng przekazał swym uczniom w formie wierszy bezpośrednio przed śmiercią oraz informacje o jego następcach, dalszych losach \textit{Sutry Platformy} i przekazie nauk szkoły chan.

\section{Nauki o medytacji}
W sekcjach 13-19 przekazano nauki na temat medytacji.
Najważniejsza z przedstawionych w tym fragmencie nauk dotyczy jedności medytacji (惠 \pinyin{huì}) i mądrości (定 \pinyin{dìng}).\label{DingHui}
Według Szóstego Patriarchy stanowią one nierozdzielne, współzależne części tej samej całości i stwarzają siebie nawzajem.
Nie sposób stwierdzić, które z nich pojawiło się jako pierwsze.
Medytacja i madrość zostały w tekście porównane do światła i lampy.
Podobne wyjaśnienia pojawiają się w ``Cytatach Shenhui''
(Huineng i Yampolsky 2012: 137).

Według tych nauk, główną doktryną subityzmu jest ,,brak myśli'' lub ,,brak idei'' (無念 \pinyin{wúniàn} lub 無心 \pinyin{wúxīn}, `brak umysłu'), jego istotą --- ,,brak formy'' (無相 \pinyin{wúxiàng}), a jego podstawą --- ,,brak przywiązania'' (無住 \pinyin{wúzhù})\ibid
% istotą/substancją; subityzm był dodany; wúzhù --- no abiding

,,Brak idei'' oznacza wolność od rozproszenia, pilnowanie, by umysł nie podążał za myślami i aby nic, co się w nim pojawia, nie prowadziło do powstania negatywnych emocji i błędnych poglądów.
Oznacza też brak przywiązania do dualistycznego postrzegania zewnętrznych zjawisk i świadomości, która je przeżywa.
Według tej doktryny wszystkie myśli powstają w esencji umysłu i wyrażają jego potencjał, dlatego próby całkowitego ich wyparcia są błędne.
Właściwa praktyka polega na odcięciu pomieszanych, dualistycznych koncepcji i koncentracji na prawdziwej naturze Takości%
\footnote{Takość (skt. \textit{Tathātā}, chiń. 真如 \pinyin{zhēnrú}) --- termin stosowany w buddyzmie mahajany na określenie prawdziwej natury zjawisk. Ponieważ natura ta jest poza wszelkim wyobrażeniem i nie da się jej opisać,  termin ,,Takość'' również nie posiada ścisłej definicji, aby praktykujący nie mógł myśleć o niej w konceptualny sposób.}
(Huineng i Yampolsky 2012: 137-138).

,,Brak obiektu'' oznacza unikanie rozproszenia pod wpływem zewnętrznych zjawisk. Nauki te nie oznaczają, że należy fizycznie odciąć się od myśli i form, lecz ,,być oddzielonym od formy nawet wtedy, gdy jest się z nią związanym''. Człowiek praktykujący tę ścieżkę, nawet bedąc związany z formą i myślami, nie traktuje ich jako prawdziwie istniejących, a jedynie jako przejawienie się potencjału przestrzeni. ,,Brak przywiązania'' oznacza niemyślenie o przeszłości ani przeszłości\ibid

Według Huinenga, praktykujący chan nie powinien koncentrować się ani na umyśle, ani na czystości, nie powinien też mówić o niewzruszoności. Twierdzi, że umysł sam w sobie jest zwodniczy, jest jedynie iluzją i jako taki nie powinien być obiektem medytacji. Również sama koncentracja na czystości nie przynosi spodziewanych rezultatów. W sytuacji, gdy praktykujący nie zrozumie, że natura umysłu jest inherentnie doskonała, taka praktyka stwarza jedynie kolejne złudzenia i sztywne koncepcje. Jedynym powodem, dla którego istoty nie są w stanie postrzegać swojej natury w ten sposób, są błędne poglądy i zaciemnienia (Huineng i Yampolsky 2012: 139-140).

Prawidłowa praktyka medytacji siedzącej (坐禪 \pinyin{zuòchán}) została przez Huinenga zdefiniowana jako niepodążanie za myślami oraz postrzeganie własnej prawdziwej natury bez rozproszenia. Natomiast ,,medytacja chan'' (禪定 \pinyin{chándìng}) oznacza według niego niezależność od zewnętrznych zjawisk oraz unikanie pomieszania w umyśle (Huineng i Yampolsky 2012: 140-141).

\section{Bezforemne nauki}
W sekcjach 20-37 \textit{Sutry Platformy} podano tzw. ,,bezforemne nauki'', jeden z najważniejszych aspektów tekstu. W sekcji 20. Szósty Patriarcha naucza o obecności Trzech Ciał Buddy%
\footnote{Trzy Ciała Buddy (skt. \textit{trikāya}, chiń. 三身 \pinyin{sānshēn}), w języku polskim nazywane również trzema stanami buddy --- w buddyzmie mahajany termin ten oznacza różne sposoby wyrażania się oświecenia: stan prawdy (lub: ciało prawdy, skt. \textit{dharmakāya}, 法身 \pinyin{fǎshēn}), który oznacza ponadczasowe, wszechobecne oświecenie; stan radości (lub: ciało radości, skt. \textit{sambhogakāya}, chiń. 報身 \pinyin{bàoshēn}), oznaczający przejawienie się buddy w postaci formy z energii i światła; oraz stan wypromieniowania (lub: ciało wypromieniowania, skt. \textit{nirmā\d{n}akāya}, chiń. 應身 \pinyin{yīngshēn}), materialne ciało, które przejawia się w określonym miejscu w czasie i przestrzeni.}
w fizycznym ciele praktykującego.
Według niego stan prawdy, stan radości i stan wypromieniowania są nierozerwalnie związane z naturą buddy wszystkich istot.
Ponieważ jednak przesłaniają je błędne poglądy, wielu praktykujących poszukuje ich na zewnątrz
(Huineng i Yampolsky 2012: 141-143).

W sekcji 21. przekazane są cztery ślubowania (四弘大願 \pinyin{sì hóngdà yuàn}), dotyczące kolejno: wyzwolenia wszystkich czujących istot, odcięcia wszelkich negatywnych emocji, zgłębienia wszystkich buddyjskich nauk i urzeczywistnienia nieprzewyższonej ścieżki Buddy, tzn. osiągnięcia oświecenia.
Dalej wyjaśniono, że pierwsze ślubowanie nie powinno być traktowane dosłownie (jest jedynie rozwinięciem altruistycznej motywacji).
W istocie wszystkie istoty muszą same osiągnąć wyzwolenie przez rozpoznanie prawdziwej natury własnego umysłu.
Odcięcie negatywnych emocji oznacza w tym kontekście odrzucenie w umyśle tego, co nieprawdziwe i błędne.
Ostatnie ślubowanie nakazuje praktykującemu zachowywać w każdej sytuacji skromność, szacunek dla wszystkich istot, unikać przywiązania i, ostatecznie, przebudzić się w mądrości pradżni
(Huineng i Yampolsky 2012: 143-144).

Sekcja 22. zawiera nauki dotyczące ,,bezforemnej skruchy'' (無相懺悔 \pinyin{wúxiàng chànhuǐ}), które pozwalają praktykującemu uwolnić się od skutków wcześniejszych negatywnych działań.
Huineng uważa, że werbalne wyznawanie grzechów buddom nie ma sensu, zamiast tego zaleca ,,w każdej sytuacji praktykować nie-działanie'' (永斷不作 \pinyin{yǒng duàn bù zuò}). Można to zrozumieć jako pozbycie się negatywnych emocji%
\footnote{Przeszkadzające, negatywne emocje, zaciemnienia, trucizny umysłu (skt. \textit{kleśa}, chiń. 煩惱 \pinyin{fánnǎo}, zwane również 塵勞 \pinyin{chénláo}) --- uczucia i stany umysłu, które powodują cierpienie i motywują istoty do popełniania krzywdzących działań.}
i złudzeń oraz permanentne zaprzestanie przynoszących cierpienie działań
(Huineng i Yampolsky 2012: 144-145).

W sekcji 23. Huineng przekazał ,,bezforemne nauki o Trzech Schronieniach\fnm'' (無相三歸依戒 \pinyin{wúxiàng sān guīyī jiè}).
Szósty Patriarcha utożsamia Buddę z oświeceniem (覺~\pinyin{jué}), Dharmę z prawdą (正 \pinyin{zhèng}), a Sanghę z czystością (淨 \pinyin{jìng}).
Poleca swym uczniom przyjąć schronienie we własnej oświeconej naturze umysłu, trzymając się z daleka od innych buddów i nauk.
Obiecuje, że jeśli tak uczynią, ich umysł nie będzie splamiony złudzeniami i fizycznym pożądaniem.
Przyjęcie schronienia w Dharmie ma zabezpieczać przed błędnymi poglądami i przywiązaniem.
Polegając zaś na Sandze, praktykujący nie będzie ulegał szkodliwym emocjom i złudzeniom % negatywnym, przeszkadzającym
(Huineng i Yampolsky 2012: 145-146).
\footnotetext{Schronienie (歸依 \pinyin{guīyī}, zapisywane również w postaci 皈依)--- w większości tradycji buddyzmu jest to rytualna deklaracja, że Budda osiągnął najwyższe oświecenie i przekazał społeczności praktykujących (sandze) nauki, prowadzące do osiągnięcia przebudzenia.}
% Encyclopedia 714

W sekcjach 24-26 Szósty Patriarcha udziela wyjaśnień na temat Mahapradżniaparamity.
Huineng podkreśla tu, że Dharmę należy praktykować w swoim umyśle, a nie jedynie bezmyślnie powtarzać.
W tradycji chan recytacja sutr z gatunku \textit{Mahapradżniaparamity}, takich jak \textit{Sutra Diamentowa} lub \textit{Sutra Serca}%
\footnote{\textit{Sutra Serca}\label{HeartSutra} (skt. \textit{Prajñāpāramitāh\d{r}daya}, chiń. 般若波羅密多心經 \pinyin{Bōrě bōluómìduō xīnjīng} `Sutra Serca Doskonałości Mądrości' lub w skrócie 心經 \pinyin{xīnjīng} `Sutra Serca') --- sutra z gatunku Pradżniaparamity, jeden z najpopularniejszych i najbardziej znanych na świecie tekstów buddyjskich. Prawdopodobnie chiński apokryf, stworzony na podstawie chińskiego tłumaczenia ``Wielkiej Sutry Doskonałości Mądrości'' (skt. \textit{Mahāprajñāpāramitā Sūtra}, chiń. 摩訶般若波羅密多經 \pinyin{Móhē bōrě bōluómìduō jīng}), przypisywanej indyjskiemu filozofowi Nagardżunie (skt. \textit{Nāgārjuna}, chiń. 龍樹 \nazwisko{Lóngshù}).},
jest bowiem jednym z rodzajów praktyki medytacyjnej.
Wyjaśniając znaczenie nazwy \textit{Mahapradżniaparamita}, Huineng interpretuje pierwszy jej człon, sanskryckie słowo \textit{mahā} (`wielki', chiń. 摩訶 \pinyin{móhē}), jako odniesienie do nieograniczonej przestrzeni umysłu, która będąc pustą, zawiera w sobie wszystkie zjawiska.
Człowiek, który praktykuje zasadę \textit{mahā} nie powinien ani lgnąć do zjawisk, ani odpychać ich, lecz traktować je dokładnie tak, jak przestrzeń nieba.
\textit{Pradżnia} oznacza, że w żadnej myśli praktykującego nie może pojawić się niewiedza.
\textit{Paramita} (skt. \textit{pāramitā}, chiń. 波羅密多 \pinyin{bōluómìduō}) oznacza ,,osiągnięcie drugiego brzegu''.
Huineng mówi, że ten, kto zrozumie znaczenie słowa \textit{paramita}, będzie poza narodzinami i śmiercią
(Huineng i Yampolsky 2012: 146-148; Buswell 2004: 666).

Sekcja 27. mówi o czystej naturze zaciemnień.
Huineng naucza tu, że gdy medytujący pozbędzie się przeszkadzających emocji, zostanie tylko \textit{pradżnia}, która jest zawsze obecna i nieoddzielna od umysłu.
Ponieważ esencją niewiedzy, złudzeń i błędnych myśli jest natura buddy, praktykujący nie powinen próbować się ich pozbyć, a jedynie rozpoznać ich prawdziwą istotę
(Huineng i Yampolsky 2012: 148-149).

W sekcjach 28-30 Szósty Patriarcha mówi o naturalnych predyspozycjach słuchaczy i praktykujących oraz o mądrości pradżni, nieoddzielnej od prawdziwej natury umysłu.
Na wiele sposobów parafrazuje informację o tym, że zwykły człowiek różni się od Buddy tylko zrozumieniem nauk i poziomem urzeczywistnienia.
W sekcji 28. zachwala on ponadto \textit{Sutrę Diamentową} jako tekst, który sam w sobie wystarcza do osiągnięcia stanu umysłu zwanego \textit{prajñāsamādhi} (z skt. `wchłonięcie medytacyjne pradżni', chiń. 般若三昧 \pinyin{bōrě sānmèi}).
Stosuje przy tym parabolę o królu smoków, który zbiera wodę z oceanu i zrzuca ją na ziemię.
Jeśli wielki deszcz spadnie na stały ląd, wówczas miasta i wioski spłyną jak trawa i liście.
Jeżeli jednak woda spadnie na taflę oceanu, nic się nie wydarzy.
Podobnie człowiek nieposiadający odpowiednich zdolności i predyspozycji nie będzie w stanie zrozumieć tych nauk.
Jednak gdy człowiek praktykujący mahajanę usłyszy tekst sutry, jego umysł otworzy się i będzie mógł się przebudzić.
Mimo że natura umysłu tych dwóch typów ludzi jest taka sama, słuchacz nieposiadający głębokiego zrozumienia nie jest w stanie osiągnąć przebudzenia bezpośrednio po usłyszeniu nauk.
Powodem tego są błędne poglądy oraz przeszkadzające emocje.
Kiedy jednak rozpozna on pradżnię w swoim umyśle, nie będzie już musiał polegać na intelektualnym pojęciu nauk
(Huineng i Yampolsky 2012: 149-150).

W sekcji 31. Huineng naucza o roli nauczyciela duchowego na ścieżce.
Praktykujący, który nie jest w stanie w jednej chwili rozpoznać natury swojego umysłu patrząc w głąb siebie i obserwując swój umysł, powinien znaleźć mistrza, zdolnego nim pokierować.
Następnie podaje, że dobry nauczyciel rozumie, iż właściwą ścieżką jest Dharma Najwyższego Pojazdu%
\footnote{Dharma Najwyższego Pojazdu (最上乘法 \pinyin{zuì shàng chéng fǎ}) --- patrz: analiza sekcji 43. na stronie \pageref{SiChengFa}.}.
Ten jednak, kto potrafi sam rozpoznać naturę swego umysłu, nie powinien szukać nauczyciela na zewnątrz, lecz wewnątrz siebie, i urzeczywistnić ,,brak myśli''
(Huineng i Yampolsky 2012: 151-153).

Sekcja 32. zawiera zalecenia Huinenga na temat przekazu doktryny Nagłego Oświecenia w następnych pokoleniach.
Patriarcha poleca swym uczniom praktykować te nauki wspólnie i chronić je, obiecując, że ten, kto poświęca temu całe swoje życie, bez wątpienia ,,wstąpi w szeregi świętych'' (入聖位 \pinyin{rù shèngwèi}).
Wraz z naukami subityzmu przekazuje również ostrzeżenie, żeby nie wyjaśniać tych nauk ludziom, którzy nie są na nie gotowi, tzn. nieposiadającym właściwego zrozumienia i determinacji.
Na pewno nie przyniesie im to bowiem pożytku, a może nawet im zaszkodzić
(Huineng i Yampolsky 2012: 153-154).

W sekcji 33. Huineng przekazał swoim uczniom ,,wiersz usuwający negatywną karmę'' (滅罪頌 \pinyin{miè zuì sòng}).
Według tego wiersza, praktyka szczodrości (布施 \pinyin{bùshī}) i zasiewania nasion przyszłego szczęścia (修福 \pinyin{xiūfú}) nie są same w sobie wystarczające do osiągnięcia ostatecznego wyzwolenia i oświecenia.
Nawet jeśli praktyka ta przyniesie pożądane rezulaty, nie usunie ona przeszkadzających emocji.
Ten, kto pragnie je wykorzenić, musi w swoim umyśle usunąć ich przyczyny, wyrazić skruchę i pilnie praktykować nauki mahajany
(Huineng i Yampolsky 2012: 154-155).

Sekcja 34. zawiera dialog prefekta Wei Qu z Szóstym Patriarchą na temat nauk, udzielonych cesarzowi Wu z Liang przez Pierwszego Patriarchę (patrz: strona \pageref{LiangWuDi}).
Według Huinenga, zasługa powstaje z umysłu i w umyśle, a więc jedynym sposobem na jej zgromadzenie jest spoglądanie we własną naturę buddy, w \textit{dharmakāya}, i szacunek dla Trzech Klejnotów%
\footnote{Trzy Klejnoty (skt. \textit{triratna}, chiń. 三寶 \pinyin{sānbǎo}) --- zbiorcze określenie trzech obiektów, w których przyjmują schronienie buddyści większości tradycji. Są nimi: Budda (佛陀 \pinyin{Fótuó} lub 佛 \pinyin{fó}), rozumiany jako historyczny Budda Siakjamuni lub natura buddy; Dharma (skt., chiń. 法 \pinyin{fǎ}) --- nauki Buddy, prowadzące do osiągnięcia oświecenia; a także Sangha (skt. \textit{Sa\d{m}gha}, chiń. 僧伽 \pinyin{Sēngjiā} lub 僧 \pinyin{sēng}) --- społeczność praktykujących Dharmę, zarówno tych, którzy już osiągnęli oświecenie, jak i będących jeszcze na ścieżce.}
oraz innych ludzi
(Huineng i Yampolsky 2012: 155-156).

W sekcji 35. zawarto nauki Huinenga dotyczące praktyk Czystej Krainy, związanych z Buddą Amitābhą, już w jego czasach szeroko rozpowszechnionych w Chinach.
Na pytanie prefekta Wei Qu, czy możliwe jest odrodzenie się w Czystej Krainie, Szósty Patriarcha odrzekł, że w istocie nauki, które Budda Siakjamuni przekazał na ten temat w mieście Śrawasti, służyły wyłącznie nawróceniu ludzi.
Powiedział też, że dla osoby oświeconej nie ma różnicy między Zachodem (Czysta Kraina Buddy Amitābhy jest utożsamiana z tym kierunkiem geograficznym) a wschodem (Chinami).
Gdy dobry człowiek ze Wschodu oczyści swój umysł, będzie zawsze w Czystej Krainie.
Natomiast jeżeli zły człowiek odrodził się na Zachodzie, ale ma błędne poglądy, nie będzie wyzwolony
% analogia o czystej krainie w ludzkim ciele
(Huineng i Yampolsky 2012: 156-159).

Sekcja 36. odwołuje się do nauk z poprzedniego fragmentu, tłumacząc, że aby praktykować Dharmę, wcale nie trzeba udawać się do klasztoru.
Huineng porównuje tu mnicha, który nie żyje w zgodzie z naukami, do złego człowieka z Zachodu.
Człowieka świeckiego, praktykującego Dharmę, przyrównuje natomiast do dobrego człowieka ze Wschodu.
W czasach, gdy powstawała \textit{Sutra Platformy}, większość mnichów wywodziła się z bogatych rodzin.
Pouczenia zawarte w tej sekcji czynią jednak z Chanu uniwersalną ścieżkę, którą może podążać każdy, niezależnie od statusu społecznego.
W tym fragmencie Szósty Patriarcha przekazuje również „bezforemny wiersz” (無相頌 \pinyin{wúxiàng sòng}) dotyczący praktyki w życiu codziennym.
Mówi w nim, że biegłość w nauczaniu i w medytacji (說通及心通 \pinyin{shuō tōng jí xīn tōng}) są niczym słońce i pustka.
Następnie zaleca swym uczniom, by przekazując wyłącznie nauki o postrzeganiu natury umysłu, wykroczyli poza ograniczenia zwyczajnego życia i pozbyli się niewłaściwych poglądów. % (惟傳見性法 出世破邪宗)
Mówi, że dzięki zgłębianiu tych nauk, nawet ignorant nie zbłądzi na ścieżce % 若學頓教法 愚人不可迷
(Huineng i Yampolsky 2012: 159-161).

Następnie, w sekcji 37., Szósty Patriarcha przekazał dalsze pouczenia dotyczące wspomnianego ,,bezforemnego wiersza''.
Mówi, że nawet ktoś, kto przebywa  1000 \textit{li}%
\footnote{\textit{Li} (里 \pinyin{lǐ}) --- tradycyjna chińska jednostka odległości, współcześnie równa 500 m.} od niego, praktykując w zgodzie z tymi naukami, zawsze będzie blisko mistrza.
Natomiast jeżeli ktoś siedzi z nim twarzą w twarz i nie podąża właściwą ścieżką, zawsze będzie odległy o 1000 \textit{li}.
Po wygłoszeniu tych nauk Huineng udał się do Caoqi, zaznaczając, że jeśli ktokolwiek będzie miał wątpliwości lub pytania dotyczące Dharmy, powinien odwiedzić go na tej górze, a wtedy mistrz będzie mógł je rozwiać % 同見佛世
(Huineng i Yampolsky 2012: 161-162).

W sekcji 38. opowiedziano dalsze dzieje Szóstego Patriarchy.
Według tego zapisu Huineng przez następne 40 lat nauczał ludzi z Shaozhou i Guangzhou.
Tekst podaje, że mistrz miał wiele tysięcy uczniów, zarówno mnichów, jak i świeckich praktykujących.
Osią jego nauk miał być przekaz \textit{Sutra Platformy}, a w późniejszych pokoleniach jedynie ci, którzy otrzymali ten tekst, mieli być ortodoksyjnymi dzierżawcami Południowej Szkoły.
Sekcja podaje także sposób, w jaki tekst ten miał być przekazywany.
Każdy uczeń powinien dostać kopię sutry z dołączonym do niej zapisanym imieniem odbiorcy oraz miejscem i datą wręczenia.
Ten fragment powstał prawdopodobnie pod wpływem uczniów mistrza Fahai, podawanego jako autor \textit{Sutry Platformy} % źródło
(Huineng i Yampolsky 2012: 162).

Sekcja 39. mówi o pochodzeniu nazw ,,Szkoła Północna'' i ,,Szkoła Południowa''.
Według tego fragmentu wzięły się one stąd, że mistrz Shenxiu nauczał w świątyni Yuquan\index{Yuquan Si 玉泉寺} w pobliżu miasta Dangyang w obecnej prowincji Hubei, a więc na północy Chin.
Huineng zaś nauczał w Caoqi, w pobliżu Shaozhou, a więc w południowej części kraju.
Chociaż w Dharmie nie ma podziałów, ludzie z Południa i Północy różnią się od siebie; stąd podział na dwie szkoły chan.
Dalej podano, że nauki o nagłym i stopniowym oświeceniu są w swojej istocie jedynie różnymi aspektami tych samych nauk.
O ile człowiek o wybitnych zdolnościach jest w stanie urzeczywistnić trudniejszą ścieżkę subityzmu, o tyle ludzie mniej inteligentni muszą podążać ścieżką stopniową
(Huineng i Yampolsky 2012: 162-163).

\section{Opowieści o uczniach Huinenga}
W sekcjach 40-44 \textit{Sutry Platformy} zawarto informacje o spadkobiercach Huinenga, takich jak Shenhui, Fada czy Zhichang. Sekcje 40. i 41. opowiadają o przybyciu do Szóstego Patriarchy mnicha imieniem Zhicheng\footnote{Patrz: przypis na str. \pageref{ZhiChangChengDisambiguation}.}.
Anegdota ta ma dowodzić wyższości Szkoły Południowej nad Szkołą Północną.
Jak podaje ten fragment, pewnego razu Shenxiu usłyszał o Huinengu i jego naukach, które według wielu ludzi przynosiły szybkie rezultaty i bezpośrednio kierowały uczniów na właściwą ścieżkę. % direct pointing of the Way
Shenxiu postanowił sprawdzić, który z nich jest lepszym nauczycielem.
W tym celu wysłał Zhichenga, który był wówczas jego uczniem, do Caoqi, aby ten zebrał informacje o Południowej Szkole.
Kiedy ten przybył do świątyni Huinenga i wysłuchał jego wykładu, w jednej chwili rozpoznał naturę swojego umysłu i osiągnął oświecenie.
Następnie powiedział mu, skąd przyszedł i poprosił go o przyjęcie na ucznia.
Przyznał, że z początku szpiegował dla swojego mistrza Shenxiu, ale zrezygnował z tego zadania, gdy usłyszał nauki Huinenga.
Wówczas Szósty Patriarcha spytał Zhichenga o szczegóły doktryny Shenxiu, a dokładniej, jakie były jego nauki na temat moralności (właściwego postępowania), medytacji i mądrości (戒定惠 \pinyin{jiè dìng huì}, dosł. `ślubowania, medytacja i mądrość').
Zhicheng powiedział, że według wyjaśnień jego mistrza, właściwe postępowanie oznacza niepopełnianie różnego rodzaju negatywnych działań; mądrością jest wykonywanie pozytywnych działań, zaś medytacją oczyszczanie umysłu.
Huineng powiedział mu, że w istocie są to dobre nauki, ale reprezentują powolną ścieżkę, przeznaczoną dla mniej zdolnych słuchaczy, w odróżnieniu od jego własnej, szybkiej ścieżki, którą mogli praktykować jedynie wybitni uczniowie.
Według Huinenga ,,ślubowaniem natury umysłu'' (自性戒 \pinyin{zìxìng jiè})\index{ślubowanie natury umysłu 自性戒} jest ,,podstawa umysłu''%
\footnote{Podstawa umysłu (心地 \pinyin{xīndì}) została zdefiniowana w tekście \textit{Zutang ji} (祖堂集 \pinyin{Zǔtáng jí}), w biografii Nanyue Huairanga (南嶽懷讓 \pinyin{Nányuè Huáiràng}): “汝學心地法門,猶如下種。我說法要,譬彼天澤。汝緣合故,當見於道。”
`Powinieneś zrozumieć doktrynę o podstawie umysłu, która naucza, że w umyśle jak gdyby były zasiane nasiona. Gdy wyjaśnię ci podstawy Dharmy, to tak jakby deszcz spadł na tę ziemię. Ponieważ połączą się odpowiednie warunki i deszcz, będziesz w stanie zobaczyć ścieżkę.' (Tekst chiński za \textit{Zutang ji}, opublikowanym na stronie internetowej \url{www.lianhua33.com/c/c47-1.htm}, przekład polski na podstawie przypisu w \textit{The Platform sutra of the Sixth Patriarch. The Text of the Tun-huang manuscript} Philipa B. Yampolsky'ego, strona 164). Termin ten jest trudno oddać w języku polskim, gdyż znak \textit{di} (地) może oznaczać m.in. ziemię, glebę lub pole. Natomiast stosowane w języku angielskim tłumaczenie \textit{mind-ground} można zinterpretować zarówno jako ,,glebę umysłu'', jak też ,,podstawę umysłu''. Ze względów estetycznych autor zdecydował się na drugie tłumaczenie.}
wolna od błędu (心地無疑 \pinyin{xīndì wú yí}).
,,Podstawa umysłu'' wolna od przeszkadzających emocji (心地無亂 \pinyin{xīndì wú luàn}) stanowi ,,medytację natury umysłu'' (自性定 \pinyin{zìxìng dìng})\index{medytacja natury umysłu 自性定}, zaś wolna od niewiedzy (心地無癡 \pinyin{xīndì wú chī}) --- ,,mądrość natury umysłu'' (自性惠 \pinyin{zìxìng huì})\index{mądrość natury umysłu 自性惠}.
Dodał też, że wyjaśnienia na temat ślubowań, medytacji i mądrości potrzebne są jedynie ludziom o niewielkich zdolnościach.
Natomiast wybitni praktykujący w ogóle nie potrzebują takich nauk, gdyż esencją jego doktryny jest urzeczywistnienie własnej natury, w której nie ma błędów, przeszkadzających emocji ani niewiedzy.
Po wysłuchaniu tych nauk Zhicheng stał się uczniem Huinenga i postanowił pozostać w Caoqi
(Huineng i Yampolsky 2012: 163-165).

W sekcji 42. opowiedziano historię mnicha o imieniu Fada, który recytował \textit{Sutrę Lotosu}%
\footnote{Sutra Lotosu (skt. \textit{Saddharmapu\d{n}\d{d}arīka Sūtra}, chiń. 妙法蓮華經 \pinyin{Miàofǎ liánhuā jīng} `Sutra o białym lotosie mistycznej Dharmy', w skrócie 法華經 \pinyin{Fǎhuá jīng}) --- sutra buddyjska, jeden z najpopularniejszych i najlepiej znanych na świecie tekstów mahajany.} przez siedem lat,
lecz mimo usilnych starań nie był w stanie odnaleźć właściwej drogi.
W związku z tym udał się do Caoqi, by poprosić o wyjaśnienia  Huinenga.
Mistrz odparł, że powodem takiego stanu rzeczy były złudzenia, które Fada przez cały czas utrzymywał w umyśle.
Powiedział też, że wystarczy usunąć te błędne poglądy, a wówczas będzie mógł dogłębnie zrozumieć tekst, który z takim zapałem recytował.
Następnie poprosił, by Fada przeczytał mu \textit{Sutrę Lotosu}.
Wysłuchawszy tekstu, wyjaśnił, że wszystkie siedem jego części pełne jest parabol i przypowieści na temat przyczynowości.
Powiedział też, że nauki Buddy o Trzech Pojazdach%
\footnote{Trzy Pojazdy\label{Triyana} (skt. \textit{triyāna}, chiń. 三乘 \pinyin{sānchéng}) --- w buddyzmie mahajany odnosi się do trzech ścieżek praktyki duchowej, obieranych przez praktykujących o różnych skłonnościach: Ścieżki Słuchaczy (skt. \textit{śrāvakayāna}, chiń. 聲聞乘\pinyin{shēngwén chéng}), Ścieżki Pratjekabuddów (skt. \textit{pratyekabuddhayāna}, chiń. 緣覺乘 \pinyin{yuánjué chéng}) oraz Ścieżki Bodhisattwów (skt. \textit{boddhisattvayāna}, chiń. 菩薩乘 \pinyin{púsà chéng}).}
przeznaczone były wyłącznie dla przeciętnych uczniów, którzy nie byli w stanie zrozumieć, że istnieje tylko jeden pojazd nauk Buddy.
By wyjaśnić te nauki, Szósty Patriarcha podał cytat z \textit{Sutry Lotosu}, który mówi, że wszyscy buddowie, w tym historyczny Budda Siakjamuni (nazwany w tekście ,,czczonym przez świat'', chiń. 世尊 \pinyin{shìzūn}) pojawili się na świecie z ,,jednej, bardzo ważnej przyczyny'' (大事因緣 \pinyin{dà shì yīnyuán}).
W tekście \textit{Sutry Lotosu} zawarto także wyjaśnienie Buddy, iż przyczyną tą jest życzenie, by otworzyć oczy wszystkich czujących istot i ukazać im Dharmę.
Huineng powiedział natomiast, że prawdziwym znaczeniem tego określenia jest odrzucenie fałszywych, dualistycznych poglądów, takich jak lgnięcie do formy lub pustości.
Taka według niego jest prawidłowa interpretacja tej sutry, a zarazem jedyny prawdziwy pojazd nauk Buddy.
Dopiero w dalszej części tekstu pojawiają się nauki o Trzech Pojazdach.
Szósty Patriarcha wyraził życzenie, by wszyscy ludzie, na całym świecie w swojej własnej ,,podstawie umysłu''  otworzyli się na mądrość Buddy i trzymali z daleka od mądrości zwykłych ludzi % Cztery bramy
(Huineng i Yampolsky 2012: 165-168).

We wspomnianej sekcji pojawiła się też metafora ,,obracania lotosem'' (傳法華 \pinyin{chuán fǎhuá}) i ,,bycia obracanym przez lotos'' (法華傳 \pinyin{fǎhuá chuán}).
Huineng porównywał wszystkie pożyteczne dla pożytku istot działania (praktykę Dharmy w umyśle, utrzymywanie prawidłowych poglądów, otwarcie się na mądrość Buddy) do obracania lotosem, a ich przeciwieństwa (niepraktykowanie Dharmy w umyśle, utrzymywanie błędnych poglądów, otwarcie się na mądrość zwykłych ludzi) do bycia obracanym przez lotos.
Wysłuchawszy tych nauk, Fada miał w jednej chwili rozpoznać naturę swego umysłu.
Powiedział z płaczem, że przez siedem lat swojego życia praktykował w niewłaściwy sposób, ale od tej chwili sytuacja się zmieni\ibid

Sekcja 43. opowiada historię mnicha o imieniu Zhichang.
Przybywszy do Caoqi, zapytał on Szóstego Patriarchy o znaczenie nauk o czterech pojazdach Dharmy (四乘法 \pinyin{sì chéng fǎ}).\label{SiChengFa}
Chociaż Budda nauczał jedynie o trzech pojazdach, Huineng mówił również o Najwyższym Pojeździe (最上乘 \pinyin{zuì shàng chéng}).\index{Dharma Najwyższego Pojazdu 最上乘法}
Mistrz wyjaśnił, że początkowo taki podział nauk nie istniał, został wprowadzony tylko ze względu na indywidualne zdolności uczniów.
Następnie wytłumaczył, że widzenie, słuchanie i recytacja nauk jest Małym Pojazdem (Ścieżką Słuchaczy; patrz: przypis na str. \pageref{Triyana});
przebudzenie się na Dharmę i zrozumienie jej zasad --- Średnim Pojazdem (Ścieżką Pratjekabuddów; patrz: przypis na str. \pageref{Triyana});
Wielki Pojazd, mahajanę, stanowi praktykowanie zgodnie z Dharmą; zaś Najwyższym Pojazdem jest ,,całkowite przejście przez dziesięć tysięcy zjawisk, będąc wyposażonym w dziesięć tysięcy nauk, nieoddzielonym od rzeczy, a jedynie od ich cech, i nieosiąganie niczego w żadnym działaniu''.
Dodał też, że ponieważ ,,pojazd'' oznacza w istocie praktykę, nie są to nauki, o których należy rozmawiać, lecz je praktykować
(Huineng i Yampolsky 2012: 168-169).

W sekcji 44. opowiedziano historię Heze Shenhui, który miał przybyć do Caoqi z miasta Nanyang (南陽 \toponim{Nányáng}), obecnie położonym w granicach prowincji Henan.
Shenhui zapytał Szóstego Patriarchę, czy widzi coś, gdy siedzi w medytacji.
W odpowiedzi mistrz trzykrotnie uderzył Shenhui i spytał go, czy poczuł ból.
Nowo przybyły mnich odparł: ,,Czułem ból, a także nie czułem bólu''.
Huineng powiedział, że on sam, gdy siedzi w medytacji, widzi swoje błędy, a jednocześnie nie widzi błędów innych ludzi.
Zaznaczył jednak, że zadawanie pytań w takich dualistycznych kategoriach jest błędem, i polecił Shenhui, by ten najpierw medytował, a potem spytał go jeszcze raz.
\textit{Sutra Platformy} podaje, że po tym spotkaniu Shenhui został uczniem Huinenga i zamieszkał w Caoqi
(Huineng i Yampolsky 2012: 169-170).

\section{Nauki przekazane dziesięciu głównym uczniom}
W sekcjach 45-47 \textit{Sutry Platformy} zapisano nauki przekazane przez Szóstego Patriarchę  dziesięciu najbliższym uczniom: Fahai, Zhicheng, Fada, Zhichang, Zhitong (志通 \nazwisko{Zhìtōng}), Zhiche (志徹 \nazwisko{Zhìchè}), Zhidao (志道 \nazwisko{Zhìdào}), Fazhen (法珍 \nazwisko{Fǎzhēn}), Faru i Shenhui.

W sekcjach 45. i 46. Szósty Patriarcha wygłosił nauki dotyczące buddyjskiej teorii poznania, o trzech kategoriach (三科 \pinyin{sān kē}) i trzydziestu sześciu konfrontacjach aktywności (動三十六對 \pinyin{dòng sānshíliù duì}). % 三科法門
Przekazując te wyjaśnienia, Huineng miał ogłosić, że po jego śmierci to na nich, jako jego uczniach, spoczywać będzie obowiązek nauczania przyszłych pokoleń.
Powiedział, że kiedy ktoś zada im pytanie dotyczące Dharmy, powinni  udzielić temu komuś odpowiedzi w sposób symetryczny, za pomocą analogii i konfrontacji.
Ponieważ przeciwieństwa wynikają z siebie nawzajem, ostatecznie dualistyczne kategorie myślowe stają się zbędne i możliwe jest wyjście poza nie
(Huineng i Yampolsky 2012: 170-172).

W sekcji 45. Szósty Patriarcha wyjaśnił określenie ,,trzy kategorie''.
Składają się na nie: pięć skandh (skt. \textit{skandha}, 五蘊 \pinyin{wǔ yùn}, w \textit{Sutrze Platformy} zapisywane 五蔭 \pinyin{wǔ yìn}; czasem tłumaczone jako ,,skupiska'') oraz osiemnaście dhātu (skt., chiń. 十八界 \pinyin{shíbā jiè} `osiemnaście sfer'), w których z kolei zawiera się dwanaście podstaw zmysłowych (skt. \nohyphens{\itshape āyatana}, chiń. 十二入 \pinyin{shíèr rù}).
Następnie wylicza po kolei pięć skandh, są to: forma materialna (skt. \textit{rūpa}, chiń. 色 \pinyin{sè}), uczucia (skt. \textit{vedanā}, chiń. 受 \pinyin{shòu}), percepcja (skt. \textit{sa\d{m}jñā}, chiń. 想 \pinyin{xiǎng}), formacje mentalne (skt. \textit{sa\d{m}skāra}, chiń. 行 \pinyin{xíng}) oraz świadomość (skt. \textit{vijñāna}, chiń. 識 \pinyin{shí}).
Na osiemnaście dhātu składają się: sześć zewnętrznych podstaw zmysłowych (skt. \textit{bāhya-āyatana} chiń. 六塵 `sześć nieczystości')
--- formy (skt. \textit{rūpa-āyatana}, chiń.色塵 \pinyin{sè chén}), słuch (skt. \textit{śabda-āyatana}, chiń. 聲塵 \pinyin{shēng chén}), zapach (skt. \textit{gandha-āyatana}, chiń. 香塵 \pinyin{xiāng chén}), smak (skt. \textit{rasa-āyatana}, chiń. 味塵 \pinyin{wèi chén}), dotyk (skt. \textit{spra\d{s}\d{t}avya-āyatana}, chiń. 觸塵 \pinyin{chù chén}) i idee (skt. \textit{dharma-āyatana}, chiń. 法塵 \pinyin{fǎ chén});
odpowiadające im sześć wewnętrznych podstaw zmysłowych (skt. \textit{adhyātma-āyatana}, chiń. 六門 \pinyin{liù mén} `sześć bram') --- oczy (skt. \textit{cak\d{s}ur-indriya-āyatana}, chiń. 眼門 \pinyin{yǎn mén}), uszy (skt. \textit{śrota-indriya-āyatana}, chiń. 耳門 \pinyin{ěr mén}), nos (skt. \textit{ghrā\d{n}a-indriya-āyatana}, chiń. 鼻門 \pinyin{bí mén}), język (skt. \textit{jihvā-indriya-āyatana}, chiń. 舌門 \pinyin{shé mén}), ciało (skt. \textit{kaya-indriya-āyatana}, chiń. 身門 \pinyin{shēn mén}) i umysł (skt. \textit{mano-indriya-āyatana}, chiń. 意門 \pinyin{yì mén});
a także powstałe z nich sześć świadomości zmysłów (skt. \textit{vijñāna}, chiń. 六識 \pinyin{liù shí}) --- widzenie (skt. \textit{cak\d{s}ur-vijñāna}, chiń. 眼識 \pinyin{yǎn shí}), słyszenie (skt. \textit{śrota--vijñāna}, chiń. 耳識 \pinyin{ěr shí}), węch (skt. \textit{ghrā\d{n}a-vijñāna}, chiń. 鼻識 \pinyin{bí shí}), smak (skt. \textit{jihvā-vijñāna}, chiń. 舌識 \pinyin{shé shí}), dotyk (skt. \textit{kaya-vijñāna}, chiń. 身識 \pinyin{shēn shí}) i myślenie (skt. \textit{mano-vijñāna}, chiń. 意識 \pinyin{yì shí}).
Huineng wyjaśnił również, że wszystkie zjawiska, w tym dwanaście podstaw zmysłowych i sześć świadomości, powstają w przestrzeni umysłu.
Myślenie wprawia w ruch świadomość, a w rezultacie zewnętrzne obiekty percepcji są postrzegane przez narządy zmysłów.
Ta właściwość umysłu, która daje początek wszystkim zjawiskom, nazywana jest ,,świadomością magazynującą'' (skt. \textit{ālaya-vijñāna}, chiń. 含藏識 \pinyin{hán cáng shí})\ibid

Sekcja 46. zawiera szczegółowe wyjaśnienia dotyczące ,,trzydziestu sześciu konfrontacji aktywności''. W tym fragmencie tekstu Huineng wyliczył pięć konfrontacji zewnętrznych zjawisk, dwanaście konfrontacji języka i cech przedmiotów oraz dziewiętnaście konfrontacji aktywności powstałych z natury umysłu.
Następnie powiedział swym uczniom, że gdy wprowadzą nauki o trzech kategoriach i trzydziestu sześciu konfrontacjach w życie, będą mogli zastosować je w kontekście wszystkich sutr i wyjść poza dualizm.
Zalecił, żeby udzielając innym tych wyjaśnień, na poziomie zewnętrznym być oddzielonym od formy, a na poziomie wewnętrznym --- oddzielonym od pustości, nie lgnąc do żadnej z nich.
Wyjaśnił też, że spośród wymienionych wcześniej konfrontacji, żadna z nich nie jest pełna sama w sobie.
Użył przy tym analogii do ciemności i światła; ciemność może być ciemnością tylko dzięki światłu, wynikają one z siebie nawzajem\ibid

W sekcji 47. zawarto kolejną informację o przekazie nauk chan za pomocą \textit{Sutry Platformy}. % Fahai sekcja 38
Według tego zapisu ten, kto nie otrzymał przekazu tekstu, nie posiadł podstaw nauk Huinenga.
Wreszcie Huineng miał powiedzieć, że zetknąć się z tą sutrą to tak, jakby osobiście usłyszeć jego nauki.
Fragment kończy się stwierdzeniem, iż dziesięciu głównych uczniów Szóstego Patriarchy kopiowało tekst i przekazywało go późniejszym pokoleniom, a ci, którzy go otrzymali, uzyskali wgląd w swoją naturę
(Huineng i Yampolsky 2012: 173-174).

\section{Śmierć Huinenga i jej następstwa}
Sekcje 48-54 \textit{Sutry Platformy} opisują okoliczności śmierci Huinenga, nauki wygłoszone przez niego bezpośrednio przed odejściem, oraz przekaz nauk w dalszych pokoleniach.

Sekcja 48. podaje dokładną datę śmierci Szóstego Patriarchy jako trzeci dzień ósmego miesiąca drugiego roku ery Xiantian (先天 \pinyin{Xiāntiān}), czyli 28 sierpnia 713 r.
Wcześniej, w pierwszym roku ery Xiantian, miał wybudować pagodę w świątyni Guo'en Si (國恩寺 \pinyin{Guó'ēn sì}) w Xinzhou.
Huineng pożegnał się z uczniami na miesiąc przed odejściem.
Kiedy poinformował ich o swej zbliżającej się śmierci, wszyscy mnisi, z wyjątkiem Shenhui, zaczęli płakać.
Według tekstu, Szósty Patriarcha pochwalił go wówczas, mówiąc, że mimo młodego wieku osiągnął już stan, w którym nie ma różnicy między dobrem i złem, pochwałą i naganą.
Pozostałym uczniom powiedział, że ich smutek wynika wyłącznie z niewiedzy dotyczącej życia po śmierci, gdyż w naturze umysłu nie ma przychodzenia, odchodzenia, narodzin ani śmierci.
Następnie przekazał ,,wiersz o prawdzie, fałszu, ruchu i spokoju'' (真假動淨偈 \pinyin{zhēn jiǎ dòng jìng jì}), przekonując, że za jego pomocą można osiągnąć wysokie urzeczywistnienie
(Huineng i Yampolsky 2012: 174-175).

Sekcja 49. zawiera informacje o linii przekazu nauk chan.
Na pytanie mnicha Fahai, kto po śmierci mistrza odziedziczy szatę patriarchów i Dharmę, Szósty Patriarcha odrzekł, że Dharma została już przekazana.
Miał również przepowiedzieć, że dwadzieścia lat po jego odejściu zejdą się niesprzyjające warunki, które zakłócą dalszy przekaz jego nauk, oraz że zjawi się wówczas człowiek, który ryzykując życie, przywróci prawdę i fałsz w buddyzmie i sprawi, że Dharma będzie się rozwijać.
Wzmianka ta odnosi się do przemowy wygłoszonej przez Shenhui w roku 732 w świątyni Dayun (大雲寺) w miejscu zwanym Huatai (滑臺 \toponim{Huátái}) w obecnej prowincji Henan (patrz: str. \pageref{Huatai}).
Podobna przepowiednia pojawiła się w ``Cytatach Shenhui'' oraz ``Księdze przekazu lampy z okresu Jingde'', podając odpowiednio okres czterdziestu i siedemdziesięciu lat.
Szósty Patriarcha ogłosił również, że z jego śmiercią kończy się przekaz szaty patriarchów, co wyjaśnia, dlaczego nie została ona zachowana dla potomności.
Następnie tekst sutry podaje pięć krótkich wierszy, które pięciu patriarchów przed nim --- Bodhidharma, Huike, Sengcan, Daoxin i Hongren --- mieli przekazać wraz z szatą i Dharmą
(Huineng i Yampolsky 2012: 176-178).

W sekcji 50. Huineng wygłosił dwa wiersze, rzekomo zainspirowane naukami Bodhidharmy.
Traktują one o właściwych i niewłaściwych kwiatach, które kwitną w podstawie umysłu.
Pięć niewłaściwych kwiatów (prawdopodobnie metafora pięciu przeszkadzających emocji --- arogancji, gniewu, głupoty, zazdrości i przywiązania) zasiewa karmiczne nasiona niewiedzy i sprawia, że wiatry karmy kontrolują umysł.
Po wysłuchaniu tych nauk, zgromadzenie mnichów rozeszło się
(Huineng i Yampolsky 2012: 178).

Sekcja 51. opisuje linię przekazu szkoły nagłego oświecenia, prowadzącej od siedmiu buddów, przez dwudziestu ośmiu patriarchów Indii i pięciu Chin, do Huinenga.
% * uzupełnić o porównanie do innych opisów linii przekazu (np. z przedmowy do przekładu)
(Huineng i Yampolsky 2012: 179).

W sekcji 52. Szósty Patriarcha udziela nauk o czującej istocie i buddzie w naturze umysłu każdego człowieka, dodając, że zwyczajny człowiek nie będzie w stanie zobaczyć buddy, gdyż może to uczynić tylko przebudzony.
Dodaje, że natura buddy jest obecna w umysłach wszystkich czujących istot i nigdzie indziej.
Przekazuje również ,,Odę o dostrzeganiu prawdziwego buddy i osiągnięciu wyzwolenia'' (見真佛解脫頌 \pinyin{jiàn zhēn fó jiětuō sòng}), według którego wszystkie istoty są buddami, a od przebudzenia dzieli je tylko niewiedza.\index{Oda o dostrzeganiu prawdziwego buddy i osiągnięciu wyzwolenia 見真佛解脫頌}
Mówi, że wystarczy na jedną chwilę przyjąć bezstronny sposób postrzegania, by zobaczyć innych jako oświeconych
(Huineng i Yampolsky 2012: 179-180).

W sekcji 53. Huineng przekazał ,,Hymn o prawdziwym buddzie w naturze umysłu i osiągnięciu wyzwolenia'' (自性真佛解脫頌 \pinyin{zìxìng zhēn fó jiětuō sòng}).\index{Hymn o prawdziwym buddzie w naturze umysłu i osiągnięciu wyzwolenia 自性真佛解脫頌}
W tym fragmencie Huineng podkreśla, że istotą oświecenia jest prawdziwa natura umysłu oraz właściwe poglądy, a wszelkie negatywności pochodzą z błędnych koncepcji oraz Trzech Trucizn Umysłu%
\footnote{Trzy trucizny\label{SanDu} (skt. \textit{trivi\d{s}a}, chiń. 三毒 \pinyin{sān dú}) --- w buddyzmie mahajany zbiorcza nazwa przywiązania (skt. \textit{rāga}, chiń. 貪 \pinyin{tān}), niechęci (skt. \textit{dve\d{s}a}, chiń. 瞋 \pinyin{chēn}) i głupoty (skt. \textit{moha}, chiń. 癡 \textit{chī}).}.\index{Trzy Trucizny Umysłu 三毒}
Wreszcie Szósty Patriarcha oznajmił, że nadszedł czas pożegnania, i poprosił swych uczniów, by po jego odejściu wiedli życie tak, jakby on cały czas był przy nich, wspólnie medytując i praktykując Dharmę, oraz by nie nosili żałoby.
Powiedział, że istotą Wielkiej Drogi jest ,,pozostawać spokojnym, bez ruchu i bezruchu, narodzin i śmierci, przychodzenia i odchodzenia; bez osądzania o tym, co właściwe i błędne, bez pozostawania i odchodzenia''.
Miały to być jego ostatnie słowa, a o północy tegoż dnia (28 sierpnia 713 r.) zmarł
(Huineng i Yampolsky 2012: 180-182).

Sekcja 54. zawiera opis cudów, które miały mieć miejsce po odejściu Szóstego Patriarchy.
W tym fragmencie zapisano również, że w jedenastym miesiącu tego samego roku jego ciało miało zostać pochowane w Caoqi, a także, że prefekt Wei Qu wzniósł kamienną stelę ku jego czci.
Wzmianki o płycie z poświęconą Huinengowi inskrypcją pojawiają się również w ``Cytatach Shenhui'' oraz dziele zatytułowanym \textit{Lidai fabao ji} (歷代法寶記 \pinyin{Lìdài fǎbǎo jì}), przy czym drugi z wymienionych tekstów wysuwa również oskarżenie, jakoby napis został w roku 719 wymazany przez przedstawicieli Szkoły Północnej, a na jego miejscu wyryto inny, sporządzony przez urzędnika o imieniu Song Ding (宋鼎)
(Huineng i Yampolsky 2012: 182).

\section{Zapisy o przekazie \textit{Sutry Platformy}}
W ostatnich trzech sekcjach \textit{Sutry Platformy}, tzn. 55-57, podano informacje o dalszym przekazie tekstu.

W sekcji 55. zapisano, że sutra ta została ułożona przez mnicha Fahai, a po jego śmierci dzierżawcą przekazu stał się jego przyjaciel w Dharmie, mnich Daocan%
\footnote{Philip B. Yampolsky w swoim przekładzie podaje jego imię jako Daocan (道澯 \nazwisko{Dàocàn}), w innych źródłach zapisywane jest jako Daoji (道漈 \nazwisko{Dàojì}).}.

Sekcja 56. opisuje warunki, które musi spełnić przyszły dzierżawca nauk chan: powinien być obdarzony ponadprzeciętną mądrością i wielkim współczuciem oraz mieć zaufanie do nauk Buddy.
Otrzymawszy ten przekaz, powinien przekazać go dalej i dołożyć starań, by w późniejszych czasach linia nie została przerwana.

Sekcja 57. zawiera informacje o autorze sutry, mnichu Fahai. Według tego zapisu, pochodził on z powiatu Quxiang (曲江縣 \toponim{Qǔjiāng xiàn}).
Po tym następuje pochwała jego zasług: był wielkim bodhisattwą, nauczał prawdziwej doktryny i praktykował zgodnie z nią.
Dalej we fragmencie wymieniono kolejne warunki niezbędne do przyjęcia przekazu Dharmy: przyszły dzierżawca musi ślubować wyzwolenie wszystkich czujących istot i praktykować bez ustanku, nie poddając się w obliczu niepowodzeń i cierpienia.
Człowiek, który nie posiadł wspomnianych przymiotów, a jego uzdolnienia są niewystarczające, nie powinien dążyć do otrzymania tych nauk.
\textit{Sutra Platformy} kończy się zachętą skierowaną do wszystkich praktykujących, by ze wszech miar starali się zrozumieć i urzeczywistnić nauki
(Huineng i Yampolsky 2012: 183).
