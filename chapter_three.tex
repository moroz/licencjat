\chapter{Historyczne reperkusje \textit{Sutry Platformy}}
\textit{Sutra Platformy} pozostaje jednym z najbardziej popularnych i wpływowych tekstów buddyzmu Chan do dnia dzisiejszego. Jej publikacja stanowi punkt kulminacyjny w historii buddyzmu Chan, a analizie zawartych w niej nauk i opowieści poświęcono wiele prac naukowych.

\if 0
Twierdził, że Szkoła Północna, której przewodzili Shenxiu i Puji, nie była autentyczna, gdyż propagowała nauki stopniowej ścieżki. Prawdziwe, ponadczasowe nauki buddy, tzn. nauki o nagłym oświeceniu, znane również jako subityzm, miały być przekazywane w południowym Chan.

Jednym z najważniejszych dzieł Shenhui, które zachowały się do dzisiejszych czasów, jest odnaleziony w Dunhuang tekst, zatytułowany \pinyin{Nanyang heshang wenda za zheng yi} (南陽和尚問答雜徵義 \textit{Nányáng héshang wèndá zá zhēngyì}), obecnie znany lepiej pod nazwą ``Cytaty Shenhui'' (神會語錄 \pinyin{Shénhuì yǔlù}).
Jego ataki na Szkołę Północną zostały opisane przez Dugu Pei (獨孤沛) w dziele zwanym \textit{Putidamo Nanzong ding shifei lun} (菩提達摩南宗定是非論 \pinyin{Pútídámó Nánzōng dìng shìfēi lùn}).
Chiński pisarz, doktor filozofii Hu Shi (胡適, 1891-1962) zebrał odkryte w Dunhuang dzieła Shenhui i jego uczniów i opisał je w pracy pt. \textit{Shenhui heshang yiji} (神會和尚遺集 \pinyin{Shénhuì héshàng yíjí})
W dziełach Shenhui pojawiały się twierdzenia, jakoby Puji wysłał swojego ucznia, niejakiego Zhang Xingchang (張行昌), do Shaozhou, z poleceniem ucięcia głowy zwłokom Huinenga.
Twierdził też, że inny uczeń Puji, imieniem Wu Pingyi (武平一), wymazał inskrypcję na steli poświęconej Huinengowi i wstawił tam własną, podającą Shenxiu jako prawowitego Szóstego Patriarchę.

Kim był Shenxiu i czym zasłużył sobie na osobiste ataki Shenhui? Yampolsky (2012, str. 15-16) podaje, że na przełomie VII i VIII w. Shenxiu był uważany za jednego z najbardziej znaczących i najwybitniejszych mistrzów Chan.
Jego biografia jest szczególnie dobrze zachowana. Jej stosunkowo rzetelna wersja została zapisana w dziele ``Annały przekazu skarbu Dharmy'' (傳法寶紀 \pinyin{Chuán fǎbǎo jì}) w pozbawiony elementów fantastycznych sposób.
O ile we wszystkich innych dziełach z tego okresu jest wymieniony jako uczeń Hongrena, ``Annały'' podają, że był uczniem Faru, a ten --- Hongrena. Według tej biografii pochodził z miasta Daliang (大梁 \toponim{Dàliáng}), obecnie Kaifeng (開封 \toponim{Kāifēng}) w prowincji Henan (河南 \toponim{Hénán}) i był członkiem rodu Li.

Shenxiu już od najmłodszych lat wykazywał się ponadprzeciętnymi uzdolnieniami.
W wieku 13 lat, w związku z zawirowaniami historycznymi i związaną z nimi klęską głodu, postanowił porzucić dotychczasowe życie i zostać mnichem buddyjskim.
Później wędrował od świątyni do świątyni, by wreszcie jako dwudziestolatek otrzymać pełne ślubowania.
W wieku 46 lat udał się do Hongrena, a ten natychmiast poznał się na jego talencie.
Po wielu latach studiowania nauk osiągnął ostateczne oświecenie, a następnie udał się do Jingzhou (荊州 \toponim{Jīngzhōu}) w prowincji Hubei (湖北 \toponim{Húběi}).
Za panowania cesarza Tang Gaozonga (唐高宗 \nazwisko{Táng Gāozōng}), podczas ery Yifeng (儀鳳 \pinyin{Yífèng}) udał się do świątyni Yuquan Si (玉泉寺 \toponim{Yùquán sì}) w pobliżu obecnego miasta Dangyang (當陽 \toponim{Dāngyáng}) w prowincji Hubei.
Dopiero po śmierci swojego mistrza zaczął gromadzić wokół siebie uczniów, nauczając ich Dharmy. Przynosił pożytek wielu istotom, prowadząc je do wyzwolenia
(Yampolsky 15-16).
\fi

\section{Podział Chan na Szkołę Północną i Południową}
Shenhui zyskał wpływy dopiero po pojawieniu się negatywnego sentymentu do jego rywali, w następstwie jego publicznych ataków na Szkołę Północną.

Jak podaje \textit{Putidamo nanzong ding shifei lun}, w roku 732 Shenhui zorganizował otwartą konferencję buddyjską w świątyni Dayun w miejscu zwanym Huatai w obecnej prowincji Henan.\label{Huatai}
W tekście opisano dyskusje Shenhui z mistrzem Chongyuan (崇遠 \nazwisko{Chóngyuǎn}).
Chongyuan był znanym specjalistą od sutr buddyjskich, wygrywającym wszystkie debaty filozoficzne, jednak na konferencji w Huatai miał ponieść porażkę.
W późniejszym czasie Szkoła Południowa używała tego faktu na dowód swej wyższości nad Szkołą Północną, była to jednak manipulacja, gdyż mistrz Chongyuan nie reprezentował żadnej ze szkół
(Shi 2008: 200).

Na konferencji w Huatai Shenhui zarzucał swoim przeciwnikom, że uzurpowali sobie prawo do linii przekazu patriarchatu, oraz że przeinaczyli istotę buddyzmu, nauczając błędnej, stopniowej ścieżki.
W rzeczywistości, Shenxiu opierał swoje nauki na tych samych podstawowych ideach buddyzmu Mahajany, co Shenhui.
W obu szkołach ostatecznym celem praktyki duchowej było osiągnięcie oświecenia poprzez rozpoznanie prawdziwej natury umysłu; w obydwu przebudzenie było postrzegane jako krótki, natychmiastowy przebłysk wglądu.
Jednak w doktrynie Szkoły Północnej, opartej głównie na sutrze \textit{La\.nkā\-vatāra}, kładziono nacisk na praktyki przygotowawcze.
Ich zadaniem było oczyszczenie umysłu z zaciemnień, postrzeganych w tej tradycji jako istniejące i rzeczywiste.
Tymczasem Szkoła Południowa głosiła pustość wszystkich myśli i zjawisk.
Ideą praktyki w tej doktrynie było nie tyle pozbyć się trucizn umysłu, ile zrozumieć, że w rzeczywistości nigdy nie istniały.
W \textit{Sutrze Platformy} dychotomię tę ilustrują ostatnie dwa wersy wiersza Huinenga: ,,W rzeczywistości nie ma niczego, / Cóż miałoby przyciągać jakikolwiek kurz?''
Nauki Shenxiu obejmowały o wiele większą część tradycji medytacyjnej Mahajany, podczas gdy Shenhui bardziej radykalnie promował doktrynę nagłego oświecenia, traktując ją jako jedyne kryterium decydujące o tym, która ze szkół reprezentowała ortodoksyjny buddyzm.
(Dumoulin 1963: 81, 84-87). % sprawdzić

W roku 720 Shenhui przebywał w miejscowości Nanyang nieopodal Luoyangu, gdzie nauczał medytacji.

\section{Dalszy podział Szkoły Południowej}
Spośród uczniów Huinenga, najbardziej wpływowi byli Qingyuan Xingsi (青原行思 \nazwisko{Qīngyuán Xíngsī}) oraz Nanyue Huairang (南嶽懷讓 \nazwisko{Nányuè Huáiràng}).
Ich spadkobiercy podzielili się na szkoły: Caodong (曹洞宗 \pinyin{Cáodòng zōng}) (雲門宗 \pinyin{Yúnmén zōng}) (法眼宗 \pinyin{Fǎyǎn zōng}) od Xingsi, (臨濟宗 \pinyin{Línjì zōng}) (溈仰宗 \pinyin{Guīyǎng zōng}) od Huairanga.
Następnie od Linji oddzieliły się linie Huanglong (黃龍派 Huánglóng pài) oraz Yangqi (楊岐派 Yángqí pài). Do chwili obecnej zachowały się Linji i Caodong. (佛學課本)

\subsection{Nanyue Huairang i jego uczniowie}
Nanyue Huairang narodził się w drugim roku ery Yifeng (677) w miejscowości Ankang w obecnej prowincji Shaanxi (陝西 \toponim{Shǎnxī}).
% Shengji Si (聖跡寺 \pinyin{Shèngjì Sì}) w pobliżu miasta Nanping (南平 \toponim{Nánpíng}) w prowincji Fujian (福建 \toponim{Fújiàn}), % 福建建陽佛迹岭聖跡寺
% 还有江西抚州临川西山
% Baohua Si (寶華寺) na górze Gonggongshan (龔公山) w pobliżu miasta Qianzhou (虔州), obecnie Ganzhou (贛州) w prowincji Jiangxi
% Jego pierwszy mistrz nazywał się 玉泉弘景
W roku 699 udał się do Caoqi, gdzie przez dwanaście lat uczył się Dharmy od Szóstego Patriarchy.
Następnie w roku 713 przybył do świątyni Bore Si (般若寺 \pinyin{Bōrě Sì}), na górze Hengshan (衡山 \toponim{Héngshān}) w prowincji Hunan (湖南 \toponim{Húnán}).
Jego głównymi uczniami byli Daojun (道峻 \nazwisko{Dàojùn}), Shenzhao (神照 \nazwisko{Shénzhào}) oraz Mazu Daoyi (馬祖道一 \nazwisko{Mǎzǔ Dàoyī}).

Mazu Daoyi (709-788) odwiedził Hengshan w roku 735.
W ``Antologii gmachu patriarchów'' opisano
Huairang udzielił mu nauki: ,,Polerując cegłę, nie zrobisz z niej lustra; jak można osiągnąć oświecenie, praktykując medytację \textit{zuochan}?'' (磨磚既不成鏡,坐禪豈得成佛)
% Tu będzie o jego doktrynie
Za czasów cesarza Tang Daizong (唐代宗) zebrał uczniów w świątyni Kaiyuan Si (開元寺), obecnie (佑民寺) w mieście Nanchang (南昌), stolicy prowincji Jiangxi.
Założony przez niego odłam Chanu nazywa się szkołą Hongzhou
Daoyi miał bardzo wielu uczniów, którzy rozprzestrzenili jego nauki dalej, a następnie utworzyli szkoły Linji oraz Guiyang.


Szkoła Shenxiu była popularna na północy Chin, podupadła po śmierci głównych uczniów Shenxiu: Songshan Puji i Dazhi Yifu (大智義福), by w końcu wygasnąć. % sprawdź kiedy
\if 0
Doczytać o rebelii An Lushana!

McRae 1986: 3
W początkowej fazie istnienia Południowej Szkoły, była ona mało znana; sutra wyjaśnia to długim czasem, jaki Huineng spędził, ukrywając się u myśliwych
McRae 1986: 5
Szkoła Południowa twierdziła, że posiada nauki niedualne
Wg SS natura oświecenie, przeszkadzające emocje, cierpienie i iluzje są w istocie tym samym, co oświecenie, ale NS widziała je jako różne; wg 宗密 oznacza to, że wiele lat, lub nawet żywotów praktyki idzie na marne; wszystko, czego potrzebuje praktykujące, to całkowite odcięcie dualistyczn, ego myślenia
SS była lewicowa: uważała, że każdy powinien mieć prawo poznać Dharmę i osiągnąć oświecenie, a nie tylko ci, którzy włożyli w to wysiłek
Zongmi usystematyzował różne interpretacje Chan, Szkoła Północna była najniżej
wykładnia 宗密 NS na początku była popularna, ale potem została niemal całkowicie wyparta przez SS, bo prawowitym spadkobiercą był Huineng, a nie Shenxiu
Nie ma dobrych badań nt. NS

Dumoulin 1963: 70
W czasie, kiedy Bodhidharma przyniósł Chan do Chin, na północy Chin był Buddhabhadra, a na południu szkoła 三论
Sanlun interesowała się Pradżniaparamitą Nagardżuny

79: 弘忍 rozwijał własne metody medytacji w oparciu o sutry Avatamsaka (華嚴)

Dumoulin 81
Pierwsza schizma w tradycji Chan
SS wygrała walkę o dominację, bo NS nie rozwijało się po śmierci uczniów Shenxiu, a SS publikowało wiele koanów i kronik
w 700 Shenxiu polecił cesarzowi zaprosić Huinenga do stolicy
(89: w 705 Huineng to zaproszenie odrzucił)
Shenxiu cieszył się szacunkiem dworu, główni uczniowie 普寂、義福
83
Według niektórych podań Shenhui był przez kilka lat uczniem Shenxiu, ale jest to mało prawdopodobne
Shenhui na pewno przebywał razem z Huinengiem
91 Dualizm rozpuszcza się w pustości -- myśl z Diamentowej Sutry
91/92 Samadhi w którym nie ma myśli
Umysł ma skłonność do konceptualizacji absolutu, np. przywiązuje się do koncepcji nirwany albo pustki
Jeżeli zamiast tego pozbędzie się wszelkich koncepcji, to pozostanie samo lustro


中國禪宗史 115:
W sutrze jest napisane, że Huineng 先天二年八月三日滅度 oraz 春秋七十有六, ale są stele, które mówią co innego i przesuwają datę śmierci o 3 lata
《曹溪大師別傳》
117: Z legend 神會, zapisanych przez 王維 wynika, że Huineng ukrywał się przez 16 lat, dostał 衣法 od 弘忍 na jego łożu śmierci, i że po 16 latach ukrycia spotkał się z 印宗 i zaczął nauczać
神會語錄:能禪師過嶺至韶州,居曹溪山,來往四十年。 nie ma 印宗,隱遁
歷代法寶記:ukrycie, 印宗
Teoria ukrycia wzięła się z niespójności lat
\fi
