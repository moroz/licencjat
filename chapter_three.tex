\chapter{Historyczne reperkusje \textit{Sutry platformy}}
\textit{Sutra platformy} pozostaje jednym z najbardziej popularnych i wpływowych tekstów buddyzmu Chan do dnia dzisiejszego, a analizie zawartych w niej nauk i opowieści poświęcono wiele prac naukowych.

\section{Podział Chan na Szkołę Północną i Południową}
Shenhui zyskał wpływy dopiero po pojawieniu się negatywnego sentymentu do jego rywali, w następstwie jego publicznych ataków na Szkołę Północną.
W roku 720 Shenhui przebywał w miejscowości Nanyang nieopodal Luoyangu, gdzie nauczał medytacji.

\section{Dalszy podział Szkoły Południowej}
Spośród uczniów Huinenga, najbardziej wpływowi byli Qingyuan Xingsi (青原行思 \nazwisko{Qīngyuán Xíngsī}) oraz Nanyue Huairang (南嶽懷讓 \nazwisko{Nányuè Huáiràng}).
Ich spadkobiercy podzielili się na szkoły: Caodong (曹洞 \pinyin{Cáodòng zōng}) (雲門 \pinyin{Yúnmén zōng}) (法眼 \pinyin{Fǎyǎn zōng}) od Xingsi, (臨濟 \pinyin{Línjì zōng}) 溈仰 od Huairanga.
Następnie od Linji oddzieliły się linie 黃龍 楊岐. Do chwili obecnej zachowały się Linji i Caodong. (佛學課本)

\subsection{Uczniowie Nanyue Huairanga}
Nanyue Huairang narodził się w drugim roku ery Yifeng (677) w miejscowości Ankang w obecnej prowincji Shaanxi (陝西 \toponim{Shǎnxī}).
Jego uczniem był Mazu Daoyi (馬祖道一 \nazwisko{Mǎzǔ Dàoyī}).
Nauczał głównie w świątyniach: Shengji Si (福建建陽佛迹岭聖跡寺) w pobliżu miasta Nanping (南平 \toponim{Nánpíng}) w prowincji Fujian (福建 \toponim{Fújiàn})
还有江西抚州临川西山
Baohua Si (寶華寺) na górze Gonggongshan (龔公山) w pobliżu miasta Qianzhou (虔州), obecnie Ganzhou (贛州) w prowincji Jiangxi

Szkoła Shenxiu była popularna na północy Chin, podupadła po śmierci głównych uczniów Shenxiu: Songshan Puji i Dazhi Yifu (大智義福), by w końcu wygasnąć. % sprawdź kiedy
\if 0
McRae 1986: 3
Sutra platformy napisała historię Chan od nowa; wskazuje na to pominięcie roli Shenhui
W początkowej fazie istnienia Południowej Szkoły, była ona mało znana; sutra wyjaśnia to długim czasem, jaki Huineng spędził, ukrywając się u myśliwych
Lankavatara => Diamentówka
McRae 1986: 5
Szkoła Południowa twierdziła, że posiada nauki niedualne
Wg SS natura oświecenie, przeszkadzające emocje, cierpienie i iluzje są w istocie tym samym, co oświecenie, ale NS widziała je jako różne; wg 宗密 oznacza to, że wiele lat, lub nawet żywotów praktyki idzie na marne; wszystko, czego potrzebuje praktykujące, to całkowite odcięcie dualistyczn, ego myślenia
SS była lewicowa: uważała, że każdy powinien mieć prawo poznać Dharmę i osiągnąć oświecenie, a nie tylko ci, którzy włożyli w to wysiłek
Zongmi usystematyzował różne interpretacje Chan, Szkoła Północna była najniżej
wykładnia 宗密 NS na początku była popularna, ale potem została niemal całkowicie wyparta przez SS, bo prawowitym spadkobiercą był Huineng, a nie Shenxiu
Nie ma dobrych badań nt. NS

17 Pierwsze wzmianki o Bodhidharmie w 洛陽伽藍記, to był tekst, w którym ludzie z różnych stron świata przyjeżdżali do Luoyangu, by zachwycać się jego architekturą
(Wiki) 曇林 napisał przedmowę do 二入四行, który tradycyjnie przypisuje się Bodhidharmie, wspomina 道育 i 慧可, i że był z południowych Indii Broughton 1999, p. 53

續高僧傳:
南天竺婆羅門種 Brahmin z Południowych Indii
Przybył do Nanyue w okresie Liu Song, (a więc przed rokiem 479), nie wiadomo, gdzie zginął

Dumoulin 1963: 70
W czasie, kiedy Bodhidharma przyniósł Chan do Chin, na północy Chin był Buddhabhadra, a na południu szkoła 三论
Sanlun interesowała się Pradżniaparamitą Nagardżuny

79: 弘忍 rozwijał własne metody medytacji w oparciu o sutry Avatamsaka (華嚴)

Dumoulin 81
Pierwsza schizma w tradycji Chan
SS wygrała walkę o dominację, bo NS nie rozwijało się po śmierci uczniów Shenxiu, a SS publikowało wiele koanów i kronik
w 700 Shenxiu polecił cesarzowi zaprosić Huinenga do stolicy
Shenxiu cieszył się szacunkiem dworu, główni uczniowie 普寂、義福
83
Według niektórych podań Shenhui był przez kilka lat uczniem Shenxiu, ale jest to mało prawdopodobne
Shenhui na pewno przebywał razem z Huinengiem

中國禪宗史 115:
W sutrze jest napisane, że Huineng 先天二年八月三日滅度 oraz 春秋七十有六, ale są stele, które mówią co innego i przesuwają datę śmierci o 3 lata
《曹溪大師別傳》
117: Z legend 神會, zapisanych przez 王維 wynika, że Huineng ukrywał się przez 16 lat, dostał 衣法 od 弘忍 na jego łożu śmierci, i że po 16 latach ukrycia spotkał się z 印宗 i zaczął nauczać
神會語錄:能禪師過嶺至韶州,居曹溪山,來往四十年。 nie ma 印宗,隱遁
歷代法寶記:ukrycie, 印宗
Teoria ukrycia wzięła się z niespójności lat
\fi
