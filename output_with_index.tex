\chapter*{Wprowadzenie}
\addcontentsline{toc}{chapter}{Wprowadzenie}
\markright{WPROWADZENIE}
\textit{Sutra platformy Szóstego Patriarchy} (chiń. 六祖壇經, Pinyin: \pinyin{Liùzǔ Tánjīng}) jest apokryficznym tekstem buddyzmu Chan (patrz: sekcja \textit{Przedstawienie buddyzmu Chan}), przypisywanym legendarnemu Szóstemu Patriarsze Chan, Dajian Huinengowi\index{Dajian Huineng} (chiń. 大鑒惠能, Pinyin: \nazwisko{Dàjiàn Huìnéng}). Najstarsza zachowana wersja tego dzieła powstała w VIII w. w Chinach.

Celem niniejszej pracy jest przedstawienie historycznego tła powstania \textit{Sutry Platformy}, analiza jej treści oraz opisanie wpływu, jaki wywarła na buddyzm chiński.

W pierwszym rozdziale pracy omówiona jest hagiografia Huinenga\index{Dajian Huineng}, głównej postaci sutry.
Rozdział drugi stanowi analiza treści \textit{Sutry platformy}, trzeci natomiast poświęcony jest dalszemu rozwojowi szkoły Chan.

\section{Wstęp techniczny}
Terminy chińskie w niniejszej pracy podane są w nawiasach w znakach tradycyjnych oraz w transkrypcji \textit{Hanyu Pinyin} (漢語拼音 \pinyin{Hànyǔ Pīnyīn}) z oznaczonymi tonami. Wyjątkiem są chińskie nazwiska, do których nie podano transkrypcji.

Odwołania do tekstu \textit{Sutry Platformy} w niniejszej pracy odnoszą się do jej przekładu pt. \textit{The Platform Sutra of the Sixth Patriarch: The Text of the Tun-huang Manuscript} Philipa B. Yampolsky'ego, wydanego drukiem przez Columbia University Press w roku 1967 i wznowionego w roku 2012.
Fragmenty oryginału podano za wersją z \textit{Chinese Electronic Tripitaka}, opublikowaną w Internecie przez Chinese Buddhist Electronic Text Association.
%
% Słowo ,,Budda'', zapisywane w tekście wielką literą, odnosi się do historycznego Buddy Siakjamuniego, tj. Siddhārta Gautamy. Zapisywane małą literą ,,budda'' oznacza stan umysłu.

\section{Przedstawienie buddyzmu Chan}
Buddyzm Chan (禪宗 \pinyin{Chán zōng}) jest, obok Szkoły Czystej Krainy%
\footnote{Szkoła Czystej Krainy, zwana również Szkołą Czystej Ziemi lub amidyzmem --- tradycja buddyzmu chińskiego, w której za najważniejszą postać przyjmuje się Buddę Amitābhę\index{Budda Amitābha} (阿彌陀佛 \pinyin{Āmìtuófó}). Celem praktyki tej tradycji jest odrodzenie po śmierci w Czystej Krainie tego buddy, \textit{Sukhavati} (w języku chińskim nazywana 極樂 \pinyin{Jílè}, 安樂 \pinyin{Ānlè} lub 西天 \pinyin{Xītiān}).}
jedną z najważniejszych tradycji buddyzmu Mahajany\index{Buddyzm Mahajany}%
\footnote{Buddyzm Mahajany (大乘佛教 \pinyin{Dàshèng fójiào} lub \pinyin{Dàchéng fójiào}, `buddyzm Wielkiego Wozu', od skt. \textit{Mahāyāna}, `wielki wóz', nazywany również buddyzmem Wielkiej Drogi) --- jeden z trzech głównych odłamów buddyzmu (dwa pozostałe to Hinajana, tzw. Mała Droga lub Mały Wóz, oraz Wadżrajana, Diamentowa Droga lub Diamentowy Wóz). Filarami Mahajany są wyzwalająca mądrość i współczucie dla wszystkich czujących istot, rozwijane w równowadze. Do buddyzmu Mahajany zalicza się m.in. buddyzm Chan i Zen, Szkołę Czystej Krainy, a także szkołę Gelugpa buddyzmu tybetańskiego.}
w Chinach. Samo słowo \textit{chan} jest chińskim wariantem %odwzorowaniem
sanskryckiego słowa \textit{dhyāna}, które oznacza medytację.
Początkowo termin ten zapisywano w języku chińskim jako \textit{channa} (禪那), dawniej wymawiane \textit{dianna}, co było fonetycznym odwzorowaniem oryginalnego terminu.
W późniejszym okresie upowszechniła się skrócona forma \textit{chan}.
W Japonii buddyzm Chan nazywany jest Zen\footnote{Shinjitai Kanji: {\ipaexgothic 禅}, Rōmaji: \textit{zen}.}, i pod taką nazwą znany jest na Zachodzie, kojarząc się ze spokojem i prostotą.
Buddyzm Chan przeniknął również do Korei, gdzie nazywany jest Seon\footnote{Hangeul: {\Korean 선}, Revised Romanization of Korean: \textit{seon}}.

Tradycyjne chińskie zapisy dotyczące historii tradycji Chan przedstawiają ją jako nieprzerwaną linię przekazu nauk i doświadczenia z patriarchy (祖師 \pinyin{zǔshī}) na patriarchę, sięgającą aż do historycznego Buddy.
Obecnie uważa się jednak, że takie zapisy nie oddają stanu faktycznego.
Miały one nadać szkole Chan autentyczność i przedstawić go jako ortodoksyjną, lepszą od innych tradycję buddyzmu, w ramach rywalizacji z innymi odłamami o wsparcie warstwy rządzącej oraz osób świeckich.
Pierwsze wzmianki o linii przekazu patriarchów Chan pojawiają się na steli pogrzebowej poświęconej mnichowi Faru (法如 \nazwisko{Fǎrú}), który miał być uczniem Piątego Patriarchy, Daman Hongrena\index{Daman Hongren, 大滿弘忍} (大滿弘忍 \nazwisko{Dàmǎn Hóngrěn}).
Według tego zapisu linia przekazu Chan prowadziła od Bodhidharmy (菩提達摩 Pútídámó, od skt. \textit{bodhi} `oświecenie' i \textit{dharma} `zjawiska; nauki Buddy'), poprzez Dazu Huike  (大祖慧可 \nazwisko{Dàzǔ Huìkě}), Jianzhi Sengcana\index{Jianzhi Sengcan} (鑑智僧璨 \nazwisko{Jiànzhì Sēngcàn}), Dayi Daoxina (大醫道信 \nazwisko{Dàyī Dàoxìn}) i Hongrena, do Faru
(Huineng\index{Dajian Huineng}, Schlütter i Teiser 2012: 53-54, 56).

Za założyciela buddyzmu Chan uznaje się przybyłego z Indii mistrza Bodhidharmę.
Pierwsza wzmianka o nim pojawia się w tekście pt. ``Zapisy o klasztorach w Luoyangu'' (洛陽伽藍記 \pinyin{Luòyáng qiélán jì}) autorstwa Yang Xuanzhi (楊衒之), pochodzącym z ok. 547 roku.
Dzieło to opisuje świątynie i klasztory miasta Luoyang (洛陽 \toponim{Luòyáng}) w prowincji Henan\index{Prowincja Henan, 河南省} (河南 \toponim{Hénán}), a także podróżnych, którzy przybywali tam z dalekich stron, by podziwiać wyszukaną architekturę.
Bodhidharma został w tym tekście przedstawiony jako stupięćdziesięcioletni \textit{śrama\d{n}a} (`święty mąż; mnich' chiń. 沙門 \pinyin{shāmén}) z Persji.
Przybywszy do Luoyangu, Pierwszy Patriarcha miał się zachwycać pięknem tamtejszych świątyń, a zwłaszcza jednej, zwanej Yongning Si (永寧寺 \pinyin{Yǒngníng Sì} `Świątynia Wiecznego Spokoju').
Budowla ta została wzniesiona w roku 516 i zniszczona w wyniku działań wojennych oraz katastrof naturalnych w roku 526, a więc można wnioskować, że Bodhidharma przebywał w mieście w ciągu tych dziesięciu lat
(McRae 1986: 17).

Kanoniczna biografia Pierwszego Patriarchy oparta jest na przedmowie do przypisywanego mu ``Traktatu o dwóch wejściach i czterech praktykach'' (二入四行論 \pinyin{Èrrù sìxíng lùn}).
Została ona napisana przez uczonego imieniem Tanlin (曇林 \nazwisko{Tánlín}), specjalistę od tekstu zwanego ``Sutrą Ryku Lwa Królowej Śrimala'' (skt. \textit{Śrīmālādevī Si\d{m}hanāda Sūtra}, chiń. 勝鬘師子吼一乘大方便方廣經 \pinyin{Shèng mán shīzi hǒu yī chéng dà fāngbiàn fāng guǎng jīng}).
Tradycyjnie uważano go za ucznia Bodhidharmy, jednak bardziej prawdopodobne jest, że jego nauczycielem był Dazu Huike, uczeń Pierwszego Patriarchy.
Tanlin podaje, że Bodhidharma był trzecim synem pewnego króla z południowych Indii, i że ,,przeszedł przez morza i góry'', by nauczać buddyzmu na północy Chin.
Według tego zapisu jego najważniejszymi uczniami byli Daoyu (道育 \nazwisko{Dàoyù}) i Huike.
\footnotetext{Buddyzm Mahajany\index{Buddyzm Mahajany} (ej Krainy, a także szkołę Gelugpa buddyzmu tybetańskiego.}
Większa część ``Traktatu o dwóch wejściach i czterech praktykach'' jest niemal identyczna z sutrą pt. \textit{Vajrasamādhi-sūtra} (金剛三昧經 \pinyin{Jīngāng sānmèi jīng}).
Tekst ten nie przedstawiał sobą nic nowatorskiego, nie licząc koncepcji \textit{biguan} (壁觀 \pinyin{bìguān}), dosłownie ,,patrzenia na ścianę''.
W traktacie biograficznym ``Kontynuowane biografie wybitnych mnichów'' (續高僧傳 \pinyin{Xù gāosēng zhuàn}) autorstwa mistrza Daoxuan (道宣 \nazwisko{Dàoxuān}) z dynastii Tang\footnote{Dynastia Tang (唐朝 \pinyin{Táng Cháo}) --- dynastia panująca w Chinach w latach 618-907. Okres szybkiego rozwoju buddyzmu chińskiego.} zawarta została natomiast zmodyfikowana wersja biografii z ``Traktatu o dwóch wejściach i czterech praktykach''.
Daoxuan uściślił lakoniczny zapis o podróży Bodhidharmy, podając, że przybył on drogą morską do południowych Chin za czasów dynastii Liu Song\footnote{Dynastia Liu Song (劉宋朝 \pinyin{Liú Sòng Cháo}), zwana też Południową Song (南宋朝 \pinyin{Nán Sòng Cháo}) --- dynastia panująca w południowych Chinach w latach 420-479, pierwsza z czterech Południowych Dynastii (南朝 \pinyin{Nán Cháo}).} i przeprawił się przez rzekę Yangzi (揚子 \toponim{Yángzǐ}).
Bodhidharma miał też udzielić Huike przekazu \textit{Sutry Lankavatara} (skt. \textit{La\.nkāvatāra}, chiń. 楞伽經 \pinyin{Léngqié jīng}).
Z biografii tej wynika, że Bodhidharma musiał przybyć do Chin przed rokiem 479, kiedy dynastia Liu Song została podbita przez Południową Qi%
\footnote{Dynastia Południowa Qi (南齊朝 \pinyin{Nán Qí cháo}) --- dynastia panująca w południowych Chinach w latach 479-502, druga z czterech Południowych Dynastii.}
(Broughton 1999: 53-56; Buswell 2004: 57; Dumoulin 1963: 71).

W ``Kontynuowanych biografiach wybitnych mnichów'' zawarta jest również historia życia praktykującego imieniem Sengfu (僧副).
Był on uczniem mistrza dhjany, nazwanego w tekście imieniem Dharma. Sengfu pochodził z powiatu Qi (祁縣 \toponim{Qíxiàn}) w pobliżu miasta Jinzhong (晉中 \toponim{Jìnzhōng}) w prowincji Shanxi (山西 \toponim{Shānxī}).
Spotkał on swego nauczyciela w jaskini, w której ten mieszkał, a otrzymawszy od niego pouczenia na temat ,,zasad medytacji'' (定學宗 \pinyin{dìngxué zōng}), przyjął ślubowania mnisie. Pomiędzy rokiem 494 a 497 udał się do miasta Jiankang\footnote{Jiankang (建康 \toponim{Jiànkāng}), w różnych okresach znane również pod nazwami Jianye (建鄴 \toponim{Jiànyè}) oraz Jinling (金陵 \toponim{Jīnlíng}) --- miasto w delcie rzeki Yangzi, stolica sześciu różnych dynastii, m.in. czterech Południowych Dynastii. Ruiny Jiankangu znajdują się w granicach administracyjnych Nankinu (南京 \toponim{Nánjīng}) w prowincji Jiangsu.}, wówczas stolicy południowych Chin, i osiedlił w świątyni (定林下寺 \pinyin{Dìnglín xià sì}) w bezpośrednim sąsiedztwie miasta.
Jeżeli przyjąć, że mistrz Dharma był w istocie Bodhidharmą, tak jak czynią to niektórzy historycy buddyzmu, z tego zapisu wynikałoby, że Pierwszy Patriarcha przewędrował na północ najpóźniej w roku 495, a być może nawet około roku 480.
(McRae 1986: 18-21).

Wedle tradycji Bodhidharma miał otrzymać przekaz Dharmy, pochodzący w nieprzerwanej linii od indyjskiego mistrza Mahakaśjapy (skt. \textit{Mahākāśyapa}, chiń. 摩訶迦葉 \nazwisko{Móhējiāshè} lub \nazwisko{Móhējiāyè}), ucznia historycznego Buddy Siakjamuniego, Siddhārta Gautamy.
(Buswell 2004: 57).
\if 0
Stolica Jiankang 建康 tak jak zdrowie, tylko bez człowieka; obecnie ruiny w granicach administracyjnych Nankinu
\fi % Odniesienie do Mahajany\index{Buddyzm Mahajany} zostało usunięte z tekstu, dlatego tymczasowo to ukrywamy

\section{Pochodzenie \textit{Sutry platformy} oraz jej przekłady na język angielski}
\textit{Sutra platformy Szóstego Patriarchy} (六祖壇經 \pinyin{Liùzǔ Tánjīng}) jest apokryficznym tekstem buddyzmu Chan.
Jego najstarsza zachowana wersja powstała w VIII w. w Chinach. Tekst napisany został częściowo w formie monologu, a częściowo w formie dialogu nauczyciela z uczniami.
Nauki w niej zawarte miały zostać wygłoszone przez legendarnego Szóstego Patriarchę.
Huineng\index{Dajian Huineng} jest w \textit{Sutrze platformy} przedstawiony jako niepiśmienny, prosty człowiek z leżącego poza zasięgiem chińskiej cywilizacji południa, choć w istocie jego nauki bazują na powszechnie znacych w VIII w. tekstach Mahajany\index{Buddyzm Mahajany}
(Huineng, Schlütter i Teiser 2012: 78).

Pełen tytuł \textit{Sutry platformy} brzmi ``Doktryna nagłego oświecenia Szkoły Południowej, Najwyższa Doskonałość Mądrości Mahajany\index{Buddyzm Mahajany}: Sutra Platformy, przekazana przez Szóstego Patriarchę Huineng\index{Dajian Huineng} w świątyni Dafan, w prefekturze Shao'' (南宗頓教最上大乘摩訶般若波羅蜜經六祖惠能大師於韶州大梵寺施法壇經 \pinyin{Nánzōng dùnjiào zuìshàng dàshèng móhēbōrě bōluómì jīng liùzǔ Huìnéng Dàshī yú Shāozhōu Dàfán Sì shīfǎ Tánjīng}).
W języku chińskim zwykle nazywana jest w skrócie ``Sutrą platformy'' (壇經 \pinyin{Tánjīng}) ,``Sutra platformy Szóstego Patriarchy'' (六祖壇經 \pinyin{Liùzǔ Tánjīng}, bądź ``Skarb Dharmy, Sutra platformy Szóstego Patriarchy'' (六祖大師法寶壇經 \pinyin{Liùzǔ Dàshī Fǎbǎo Tánjīng}).

\textit{Sutrę platformy} zalicza się do korpusu dzieł \textit{Mahapradżniaparamity}%
\footnote{Mahapradżniaparamita (skt. \textit{Mahāprajñāpāramitā}, chiń. 摩訶般若波羅蜜多 \pinyin{Móhē Bōrě Bōluómìduō}, `Wielka Doskonałość Mądrości') --- zbiór tekstów, nauk i praktyk oraz związana z nimi tradycja filozoficzna Mahajany\index{Buddyzm Mahajany}, podkreślająca rolę mądrości jako najważniejszej z Sześciu Paramit (patrz: przyp. na str. \pageref{Paramitas}).
Według tych nauk, doskonałość tylko w tej jednej paramicie wystarczy do osiągnięcia perfekcji w pozostałych pięciu. Do tego gatunku zalicza się m.in. takie dzieła, jak \textit{Sutrę diamentową} (patrz: przyp. na str. \pageref{DiamondSutra}), \textit{Sutrę serca} (patrz: przyp. na str. \pageref{HeartSutra}), a także omawianą tu \textit{Sutrę platformy}.}.
Na przynależność sutry do tego gatunku wskazują, oprócz przekazanych w tekście nauk na temat Doskonałości Mądrości, jej pełen tytuł (patrz: poprzedni akapit), a także słowa, którymi Szósty Patriarcha miał rozpocząć swoją przemowę: ,,Drodzy przyjaciele, oczyśćcie swoje umysły i skoncentrujcie się na Dharmie Wielkiej Doskonałości Mądrości.''
Ponadto w \textit{Sutrze platformy} pojawiają się nauki z doktryny \textit{Tathāgatagarbha}%
\footnote{\textit{Tathāgatagarbha} (od skt. \textit{Tathāgata} `budda', dosł. `Ten, Który Przyszedł w Ten Sposób' + \textit{garbha} `płód; zarodek', chiń. 如來藏 \pinyin{Rúláizàng}) --- grupa sutr buddyzmu Mahajany. Ich głównym założeniem jest obecność w każdej czującej istocie zalążka oświecenia, tzw. ,,natury buddy'', dzięki któremu możliwe jest wykroczenie poza krąg uwarunkowanej egzystencji i osiągnięcie stanu buddy.} % powtórzenie
(Huineng\index{Dajian Huineng} i Yampolsky 2012: 127; Huineng, Schlütter i Teiser 2012: 78).

\textit{Sutrę platformy} uważa się za jedno z najważniejszych dzieł buddyzmu Chan, ponieważ wprowadziła nauki o nagłym oświeceniu (頓教 \pinyin{dùnjiào}, `nagła szkoła, subityzm'), stojące w opozycji do nauk tzw. stopniowej szkoły (漸教 \pinyin{jiànjiào}), i wywołała podział szkoły Chan na odłam północny i południowy. % reference
(Buswell 2004: 347-348).%; McRae, 2004: ).

Najważniejsze z koncepcji, które wprowadziła \textit{Sutra platformy}, to identyfikacja medytacji z mądrością (omówione szerzej w sekcji \textit{Nauki o medytacji} na str. \pageref{DingHui}), ***
Ważnym aspektem tekstu są nauki o tym, że każda czująca istota ma naturę buddy, i że zarówno ludzie świeccy, jak i mnisi mogą z powodzeniem praktykować jego nauki.
Tekst opisuje również specjalny rytuał przekazywania mnichom i świeckim praktykującym ,,bezforemnych zasad'' (無相戒 \pinyin{wúxiàng jiè}). % formless precepts
Były to niektóre z powodów, dla których w roku 796 Huineng\index{Dajian Huineng} został oficjalnie obwołany szóstym patriarchą Chan przez cesarską komisję, a jego dzieło stworzyło podwaliny pod dalszy rozwój szkoły Chan
(Huineng, Schlütter i Teiser 2012: 2).


\chapter{Biografie Huinenga}
Życie Szóstego Patriarchy jest owiane tajemnicą. Jego imię pojawia się w kronice pt. ``Księga przekazu lampy z okresu Jingde\index{Księga przekazu lampy z okresu Jingde, 景德傳燈錄}'' (景德傳燈錄 \pinyin{Jǐngdé chuándēng lù}) jako jednego z dziesięciu głównych uczniów Piątego Patriarchy Hongren.\index{Daman Hongren, 大滿弘忍}
Z owego tekstu nie wynika jednak, by był postacią szczególnie ważną dla rozwoju całej szkoły Chan, wspomniano w nim natomiast, że Huineng\index{Dajian Huineng} żył i nauczał w miejscowości Caoqi (曹溪 \toponim{Cáoqī}).

Imię Huineng\index{Dajian Huineng} pojawia się również w pewnym tekście z grot Dunhuang%
\footnote{Groty Dunhuang (敦煌石窟 \pinyin{Dūnhuáng shíkū}), --- zbiorcze określenie stanowisk archeologicznych w północno-zachodniej części prowincji Gansu\index{Prowincja Gansu, 甘肅省} (甘肅 \toponim{Gānsù}), nazwanych od pobliskiego miasta Dunhuang (敦煌 \toponim{Dūnhuáng}).},
upamiętniającym Piątego Patriarchę, Hongrena\index{Daman Hongren, 大滿弘忍}, jednak tekst ów nie mówi nic o związanych z Huinengiem doktrynach. Kanoniczna biografia Huinenga oparta jest na przypisywanej mu \textit{Sutrze Platformy}. Pierwszą osobą, która przedstawiła Huinenga jako świętego, był Heze Shenhui\index{Heze Shenhui, 荷澤神會} (菏澤神會 \nazwisko{Hézé Shénhuì}, 684-758). Biografia Szóstego Patriarchy w takiej wersji, jak opisana w tekście \textit{Sutry Platformy} z Dunhuang, jest najprawdopodobniej uzupełnioną i zmienioną wersją jego opowieści. (McRae 2004: 68).

\section{Biografia Huinenga\index{Dajian Huineng} według \textit{Sutry Platformy} w wersji z Dunhuang}
Według \textit{Sutra Platformy} Huineng urodził się w miejscowości Xinxing (新興 \toponim{Xīnxīng}) w regionie Nanhai (南海 \toponim{Nánhǎi}, obecnie prowincja Guangdong). Za ramy czasowe jego życia przyjmuje się lata 638-713. Szósty Patriarcha jest w tym tekście przedstawiany jako ubogi, niepiśmienny człowiek świecki.

Jak podaje tekst, ojciec Huinenga\index{Dajian Huineng} był urzędnikiem z regionu Fanyang (范陽 \toponim{Fànyáng}), obecnie miasto Zhuozhou (涿州 \toponim{Zhuōzhōu}) w prowincji Hebei (河北 \toponim{Héběi}), lecz został odwołany ze stanowiska i skazany na banicję. W związku z tym musiał przenieść się z całą rodziną do Xinxing, gdzie niedługo później zmarł. Po jego śmierci Huineng trudnił się zbieraniem i sprzedażą drewna na opał.

Pewnego dnia, gdy dwudziestodwuletni Huineng\index{Dajian Huineng} sprzedawał drewno na targowisku, pewien klient zamówił zapas opału z dostawą do jego sklepu.
Huineng dostarczył towar i otrzymał swoją zapłatę, a kiedy wyszedł na zewnątrz, spotkał człowieka, który recytował na ulicy Sutrę Diamentową%
\footnote{Sutra Diamentowa (金剛經 \pinyin{Jīngāng jīng}, skt. \textit{Vajracchedikā Prajñāpāramitā Sūtra}) --- jeden z najważniejszych tekstów buddyzmu Mahajany\index{Buddyzm Mahajany}. Ułożony prawdopodobnie około II lub IV w. n.e. w Indiach, przetłumaczony na język chiński na początku V w.}.\label{DiamondSutra}
Usłyszawszy ów tekst, Huineng uzyskał wgląd w naturę swego umysłu i osiągnął oświecenie. Następnie spytał tajemniczego mężczyznę, skąd pochodził.
Ten odpowiedział, że przybył z klasztoru Dongshan (東山寺 \pinyin{Dōngshān sì}) na górze Fengmushan (憑墓山 \toponim{Féngmù shān}) w powiecie Huangmei (黃梅懸 \toponim{Huángméi xiàn}) w Qizhou (蘄州 \toponim{Qízhōu}), którego opatem był Piąty Patriarcha, Hongren,\index{Daman Hongren, 大滿弘忍} i gdzie przebywało około tysiąca mnichów.
Patriarcha zalecił swoim uczniom recytować ową sutrę, mówiąc, że dzięki tej praktyce można szybko osiągnąć oświecenie.
Wkrótce potem Huineng spotkał kogoś, kto poradził mu udać się do klasztoru Dongshan, aby poprosić Patriarchę o nauki, i dał mu pieniądze na zaaranżowanie opieki dla starej matki.
(Huineng i Yampolsky 2012: 127).

Kiedy Huineng\index{Dajian Huineng} dotarł do klasztoru, Piąty Patriarcha Hongren \index{Daman Hongren, 大滿弘忍}zapytał go, skąd i w jakiej sprawie przyszedł do patriarchy.
Huineng odparł, że przybył z Kantonu\footnote{Nazwa ,,Kanton'' w języku polskim odnosi się zarówno do prowincji Guangdong (廣東 \toponim{Guǎngdōng}), jak i do jej stolicy --- miasta Guangzhou (廣州 \toponim{Guǎngzhōu}).}, by oddać cześć patriarsze, oraz że nie prosi o nic prócz Dharmy.
Patriarcha stwierdził wówczas, że Huineng, jako \textit{geliao} (獦獠 \pinyin{géliáo}, `barbarzyńca'), niegodny jest otrzymania nauk.
Obszar obecnego Kantonu był wówczas zamieszkany przez niechińskie ludy, mówiące własnymi językami, posiadające własną kulturę i nieżyjące zgodnie z naukami Buddy --- mieszkańcy południa polowali bowiem i jedli mięso.
Dla wielu ówczesnych buddystów nie do pomyślenia było, by człowiek z południa mógł otrzymać nauki od Patriarchy i osiągnąć oświecenie.
Huineng odparł wtedy, że chociaż w społeczeństwie istnieją takie podziały, nie mają one wpływu na naturę buddy, która jest taka sama we wszystkich ludziach.
Patriarcha uznał, że Huineng dobrze rozumiał nauki Buddy, lecz w obawie, że inni uczniowie mogliby zrobić mu krzywdę, wyznaczył mu prace gospodarcze.
Przez następne osiem miesięcy Huineng rąbał drewno i młócił zboże
(Huineng i Yampolsky: 127-128; Huineng, Schlütter i Teiser 2012: 27).

Pewnego dnia Patriarcha Hongren \index{Daman Hongren, 大滿弘忍}zwołał zebranie wszystkich uczniów i ogłosił:
,,Dla ludzi w tym świecie narodziny i śmierć są doniosłymi kwestiami. Całymi dniami składacie dary i poszukujecie pola błogosławieństw, ale nie staracie się wyzwolić z pełnego goryczy oceanu uwarunkowanej egzystencji%
\footnote{Idea ,,pola błogosławieństw'' (福田 \pinyin{fútián}, skt. \textit{pu\d{n}yak\d{s}etra}) jest związana z buddyjską koncepcją karmy jako prawa przyczyny i skutku. Oznacza stan, w którym dana osoba zgromadziła bardzo wiele dobrej karmy w rezultacie praktyki szczodrości (skt. \textit{dānā}, chiń. 布施 \pinyin{bùshī}), pierwszej z tzw. Sześciu Paramit lub Sześciu Wyzwalających Działań (pozostałe pięć to właściwe działanie, cierpliwość, radosny wysiłek, medytacja i mądrość).\label{Paramitas}
% Słowo \textit{dānā} pojawia się m.in. w \textit{Dānādhikāramahāyānasūtra} (skt. {Dānādhikāramahāyānasūtra} 佛说布施经 \pinyin{Fóshuō bùshī jīng}).
Piąty Patriarcha Hongren krytykował tu swoich uczniów, ponieważ praktyka szczodrości jest wprawdzie w buddyzmie postrzegana jako pozytywne działanie, jednak nie wystarcza ona do osiągnięcia ostatecznego wyzwolenia z samsary (Anonim 2007; Nydahl 2010).}.
Wasze własne ego stoi na drodze do błogosławieństw. Jak w takiej sytuacji możecie osiągnąć wyzwolenie? Powróćcie teraz do swoich cel i spójrzcie w swój umysł. Ludzie mądrzy samoistnie pojmą prawdziwą naturę \textit{pradżni}\fnm. Niech każdy z was napisze wiersz i przyniesie mi go. Przeczytam każdy z nich, a jeżeli jest wśród was ktoś, kto rozpoznał swoją prawdziwą naturę, przekażę mu swoją szatę i Dharmę, a także uczynię go Szóstym Patriarchą. Spieszcie się!''
(Huineng\index{Dajian Huineng} i Yampolsky 2012: 128).
\footnotetext{Pradżnia (skt. \textit{prajñā}, w języku chińskim nazywana 慧 \pinyin{huì}, 智 \pinyin{zhì} lub 智慧 \pinyin{zhìhuì} --- wszystkie trzy terminy oznaczają `mądrość' --- lub fonetycznie 般若 \pinyin{bōrě}), to, obok współczucia (悲 \pinyin{bēi} `litość', skt. \textit{karu\d{n}ā}), jedna z dwóch najważniejszych cnót buddyzmu Mahajany\index{Buddyzm Mahajany}. Termin ten można rozumieć na wiele sposobów, zależnie od tradycji, praktykowanej ścieżki i metody interpretacji. Tu odnosi się do prawidłowego, ponadintelektualnego zrozumienia prawdziwej natury zjawisk.}

Mnisi stwierdzili zgodnie: ,,Nie ma sensu oczyszczać umysłu i zadawać sobie trudu układania wiersza dla patriarchy. Shenxiu (神秀 \nazwisko{Shénxiù}), przewodniczący kongregacji, jest naszym nauczycielem.
Kiedy on zostanie patriarchą, możemy liczyć na jego wsparcie. Dlatego nie będziemy układać wierszy.'' Żaden z nich nie podjął się więc tego zadania
(Huineng\index{Dajian Huineng} i Yampolsky 2012: 127).

Shenxiu zaś dręczyły wątpliwości. Z jednej strony nie uważał się za godnego przyjęcia stanowiska patriarchy, z drugiej zaś pragnął otrzymać przekaz nauk.
Przyjęcie przekazu Dharmy dla pożytku istot byłoby bowiem pożądane i chwalebne, nie mógł jednak za wszelką cenę dążyć do zostania patriarchą.
Wychodził z założenia, że jeżeli jego mistrz uzna jego urzeczywistnienie natury umysłu za niewystarczające, to będzie musiał pogodzić się z faktem, iż kto inny zostanie dzierżawcą linii przekazu.
Wreszcie skomponował wiersz i wymknął się w nocy ze swojej celi, by napisać go na ścianie, w miejscu, gdzie miały zostać namalowane sceny z sutry \textit{La\.nkāvatāra} (楞伽經 \pinyin{Léngqié jīng}) (McRae 2004: 62):

\vspace*{6pt}
\begin{minipage}[t]{0.4\textwidth}
\begin{verse}
身是菩提樹\\
心如明鏡臺\\
時時勤佛拭\\
莫使有塵埃
\end{verse}
\end{minipage}
\begin{minipage}[t]{0.6\textwidth}
\itshape
\begin{verse}
Ciało jest drzewem Bodhi,\\
Umysł --- jasną lustrzaną podstawą.\\
Czyść ją stale i gorliwie,\\
Nie pozwalając aby przylgnął kurz.\fnm
\end{verse}
\end{minipage}
\label{ShenxiuVerse}
\vspace*{6pt}
\footnotetext{Tekst w języku polskim przytoczony za przekładem \textit{Sutry Szóstego Patriarchy Zen} nieznanego tłumacza, zamieszczonym w serwisie mahajana.net.}

O poranku, Piąty Patriarcha Hongren \index{Daman Hongren, 大滿弘忍}ujrzał wiersz napisany przez Shenxiu na ścianie i uznał, że wiersz ten mógł przynieść ludziom wiele pożytku. Zapłacił wówczas malarzowi, któremu zlecił namalowanie scen z sutry \textit{La\.nkāvatāra}, i odwołał zamówienie.
Zwołał całe zgromadzenie i nakazał mnichom recytować wiersz. Patriarcha spytał Shenxiu, czy to on jest jego autorem, oznaczałoby to bowiem, że jest on właściwym spadkobiercą Dharmy i jego następcą na stanowisku patriarchy.
Przynał jednak, że wiersz nie wskazuje na to, aby Shenxiu rozpoznał już naturę swojego umysłu. Wiersz nadawał się do recytacji przez zwykłych ludzi i dawał gwarancję, że praktykujący nie upadnie do niższych sfer egzystencji%
\footnote{Według kosmologii buddyjskiej, istoty krążące w samsarze, tj. uwarunkowanej egzystencji, od niemającego początku czasu odradzają się w jednej z sześciu sfer egzystencji, zależnie od swojej karmy i indywidualnych skłonności. Trzy z nich, sfera niebiańska (天道 \pinyin{tiāndào}, skt. \textit{devaloka}), którą zamieszkują bogowie, sfera półbogów lub asurów (阿修羅 \pinyin{Āxiūluódào}) i sfera ludzi (人道 \pinyin{réndào}), nazywa się trzema wyższymi sferami egzystencji (三善道 \pinyin{sān shàndào}), ponieważ życie w tych sferach jest relatywnie przyjemne. Trzy niższe sfery egzystencji, sfera zwierząt (畜牲道 \pinyin{chùshēngdào}), sfera głodnych duchów lub pretów (餓鬼道 \pinyin{èguǐdào}) oraz sfery piekielne (地獄道 \pinyin{dìyùdào}), w których życie pełne jest cierpienia, nazywane są niższymi sferami egzystencji (三惡道 \pinyin{sān èdào}).},
jednak takie niepełne zrozumienie nie wystarczało do rozpoznania prawdziwej natury umysłu.
Powiedział, że przekaże mu nauki i szatę patriarchy, jeżeli w ciągu dwóch dni uda mu się osiągnąć ostateczne urzeczywistnienie.
Shenxiu rozmyślał przez wiele dni, ale nie udało mu się skomponować nic lepszego.
(Huineng\index{Dajian Huineng} i Yampolsky 2012: 131).

Pewnego dnia młody mnich-akolita recytował wiersz Shenxiu przechodząc koło gumna, gdzie Huineng\index{Dajian Huineng} młócił zboże. Przyszły Szósty Patriarcha zrozumiał, że autor wiersza nie rozpoznał jeszcze natury swojego umysłu. Spytał więc akolity, co właśnie powtarzał. Mnich odparł, że wiersz o pustości, który powtarzał, został skomponowany przez mnicha o imieniu Shenxiu, wspomniał również o zaleceniach Piątego Patriarchy, dotyczących tego wiersza.

Huineng poprosił mnicha, aby ten zaprowadził go do miejsca, gdzie na ścianie wymalowano wiersz Shenxiu. Ponieważ nie umiał czytać, poprosił kogoś, by przeczytał mu te słowa na głos. Usłyszawszy je, Szósty Patriarcha osiągnął ostateczne urzeczywistnienie. Następnie ułożył własny wiersz i poprosił kogoś o napisanie go na ścianie (Huineng\index{Dajian Huineng} i Yampolsky 2012: 131).

\begin{minipage}[t]{0.4\textwidth}
\begin{verse}
菩提本無樹\\
明鏡亦無臺\\
佛性常清淨\\
何處有塵埃
\end{verse}
\end{minipage}
\begin{minipage}[t]{0.6\textwidth}
\itshape
\begin{verse}
Sama istota Bodhi nie ma drzewa,\\
Nie ma też jasnej lustrzanej podstawy.\\
W rzeczywistości nie ma niczego,\\
Cóż miałoby przyciągać jakikolwiek kurz?
\end{verse}
\end{minipage}
\label{HuinengVerse}
\vspace*{6pt}

Mnisi ze zgromadzenia byli pod wielkim wrażeniem tego wiersza, a Piąty Patriarcha Hongren \index{Daman Hongren, 大滿弘忍}stwierdził na jego podstawie, że Huineng\index{Dajian Huineng} miał już wówczas ponadprzeciętne zrozumienie natury zjawisk, ale dla jego bezpieczeństwa oznajmił zgromadzeniu, że wciąż nie było to pełne urzeczywistnienie
(Huineng i Yampolsky 2012: 132).

Patriarcha przywołał go do siebie w nocy i udzielił mu wyjaśnień do \textit{Sutry Diamentowej}, dzięki którym Huineng\index{Dajian Huineng} natychmiast rozpoznał naturę umysłu.
Hongren \index{Daman Hongren, 大滿弘忍}przekazał mu również nauki o spontanicznym oświeceniu oraz szatę, insygnium patriarchatu. Przekazał mu również ostatnie pouczenia: ,,Mianuję cię Szóstym Patriarchą. Szata jest tego dowodem, przechodzącym z pokolenia na pokolenie. Moja Dharma musi być przekazywana z umysłu na umysł. Spraw, by ludzie rozpoznali swoją prawdziwą naturę. (\ldots) Od czasów starożytnych przekaz Dharmy był równie słaby, jak zwisający sznurek. Jeżeli pozostaniesz tutaj, inni ludzie zrobią ci krzywdę. Musisz więc niezwłocznie odejść.'' (Huineng i Yampolsky 2012: 133).

Huineng udał się na południe. Jego śladem podążyło kilkuset ludzi, pragnących go zabić i siłą odebrać szatę oraz Dharmę. Po dwóch miesiącach miał dotrzeć do miejsca zwanego \textit{Dayu ling} (大庾嶺 \toponim{Dàyǔ líng}). Jest to pasmo górskie, znajdujące się pomiędzy południowym wschodem prowincji Jiangxi\index{Prowincja Jiangxi, 江西省} (江西 \toponim{Jiāngxī}) a prowincją Guangdong (廣東 \toponim{Guǎngdōng}). Tam doścignął go mnich imieniem Huiming (惠明 \pinyin{Huìmíng}) lub Huishun (惠順 \pinyin{Huìshùn}), były generał, człowiek szorstki i porywczy. Huiming groził Huinengowi\index{Dajian Huineng}, który bez wahania oddał mu szatę, lecz Huiming nie chciał jej przyjąć, wyjaśniając, że przybył wyłącznie po to, by otrzymać przekaz Dharmy. Huineng miał przekazać mu Dharmę na szczycie góry, a gdy Huiming usłyszał nauki, natychmiast osiągnął oświecenie.
Następnie Huineng polecił Huimingowi udać się na północ i nauczać tamtejszych ludzi, a sam udał się do Kantonu (Huineng i Yampolsky 2012: 134; Huineng, Schlütter i Teiser: 31).

Dalszy ciąg \textit{Sutry Platformy} podaje, że Huineng\index{Dajian Huineng} przebywał w Caoqi w sumie przez 40 lat.
Nauczał ludzi z Shaozhou i Kantonu w oparciu o \textit{Sutrę Diamentową}, a jako symbolu przekazu Dharmy używał \textit{Sutry Platformy}. Huineng miał wiele tysięcy uczniów, z których dziesięciu zostało mistrzami, mało znanymi poza obszarem ich działalności.
W 712 roku powrócił do Xinzhou, miejsca swych narodzin, a w 713 roku zmarł w wieku 76 lat.
Tuż przed jego śmiercią jego uczeń Fahai (法海 \nazwisko{Fǎhǎi}), uważany za autora sutry, zapytał, kto będzie jego następcą i co stanie się z szatą patriarchy.
Huineng odrzekł, że przekaz szaty dobiegł końca, i zasugerował, że w przyszłości pojawi się uczeń o imieniu Shenhui\index{Heze Shenhui, 荷澤神會}. W momencie jego śmierci pojawiło się wiele pomyślnych znaków. Huineng miał zostać pochowany w Caoqi, a Wei Qu (韋璩), prefekt, który wysłuchał nauk zawartych w dalszej części tekstu, napisał ku jego czci inskrypcję. Mieli ją następnie zniszczyć przedstawiciele Północnej Szkoły
(Huineng, Schlütter i Teiser 2012: 31, 34).

% Schlütter 31
% Wang Wei 王維 napisał epitafium dla Huinenga\index{Dajian Huineng}, w którym pojawia się postać mnicha Yinzonga, który miał ostrzyc głowę Huinengowi
% nie atakuje Północnego Chan
% Matka Wang Wei była wyznawczynią Puji
% Huineng nie był członkiem szlachetnego rodu
% romans: quest (podróż), konkurs, walka o życie, w której Huineng w końcu wygrywa, bo jest lepszy

\section{Analiza biografii Huinenga\index{Dajian Huineng} zawartej w tekście \textit{Sutry Platformy} z Dunhuang}
Obecnie uważa się, że autobiograficzny monolog Huinenga, przytoczony w \textit{Sutrze Platformy}, nie jest autentyczną historią jego życia, a jedynie hagiografią. Tekst wysuwa twierdzenia, jakoby Huineng miał być prawowitym spadkobiercą Piątego Patriarchy Hongrena\index{Daman Hongren, 大滿弘忍}, szóstym dzierżawcą przekazu Chan, pochodzącego w prostej, nieprzerwanej linii od samego historycznego Buddy. (Huineng, Schlütter i Teiser 2012: 25-26). % ***

Shenhui w 732 roku zaczął promować Huinenga\index{Dajian Huineng} jako Szóstego Patriarchę linii Chan i jednocześnie atakował uczniów i spadkobierców Shenxiu, szczególnie Songshan Puji (嵩山普寂 \pinyin{Sōngshān Pǔjì}, 651-739), roszczącego sobie prawa do tytułu Siódmego Patriarchy. Twierdził, że Szkoła Północna, której przewodzili Shenxiu i Puji, nie była autentyczna, gdyż propagowała nauki stopniowej ścieżki. Prawdziwe, ponadczasowe nauki buddy, tzn. nauki o nagłym oświeceniu, znane również jako subityzm, miały być przekazywane w południowym Chan. Znamienne jest to, że w pismach Shenhui\index{Heze Shenhui, 荷澤神會} nie było żadnej wzmianki o konkursie poezji ani o dwóch wierszach, co świadczy o tym, iż fragmenty te zostały dodane po jego śmierci. O ile w tekście \textit{Sutry platformy} zawarto nauki Shenhui i jego spadkobierców, jego wkład w powstanie sutry został w tekście przemilczany (Huineng, Schlütter i Teiser 2012: 32-33; Huineng i Yampolsky 2012: 26, 28; McRae 2004: 63).

Historyczny Huineng\index{Dajian Huineng} był postacią stosunkowo mało znaną i prawdopodobnie dlatego Shenhui\index{Heze Shenhui, 荷澤神會} wybrał go jako bohatera swoich historii.
Ponieważ niewiele było wiadomo o jego prawdziwych naukach, Shenhui mógł przypisać mu dowolne pouczenia niestojące w sprzeczności z doktryną subityzmu.
Prawdopodobnie był on mistrzem medytacji nauczającym o nagłym oświeceniu, ale mimo tego, co możemy przeczytać w \textit{Sutrze platformy}, już za czasów jego życia doktryna ta nie uchodziła za nic nadzwyczajnego.
Pochodził z południa, obszaru oddalonego od serca chińskiej cywilizacji.
Wbrew dramatycznej autobiografii zapisanej w tekście, raczej nie wchodził w konflikty z innymi mistrzami medytacji, a wręcz utrzymywał z nimi dobre relacje (Huineng i McRae 2000: xv).

W późniejszym okresie, tj. po śmierci Shenxiu, historia życia Huinenga\index{Dajian Huineng} została uzupełniona opowieścią o konkursie poezji, w którym Huineng miał pokonać Shenxiu, jednoznacznie dowodząc wyższości subityzmu Szkoły Południowej nad stopniową ścieżką Szkoły Północnej.
Biorąc pod uwagę, że Huineng został w tekście przedstawiony jako ubogi, niepiśmienny człowiek, jest niezwykle mało prawdopodobne, by był w stanie ułożyć przypisywany mu wiersz w klasycznym, literackim języku chińskim.
Z drugiej strony, ponieważ Huineng wywodził się z rodziny urzędnika (nawet popadłego w niełaskę władz i skazanego na banicję), wydaje się nieprawdopodobne, że mógłby nie otrzymać żadnego wykształcenia.
W dziełach Shenhui\index{Heze Shenhui, 荷澤神會} pojawiały się również twierdzenia, jakoby Puji wysłał swojego ucznia, niejakiego Zhang Xingchang (張行昌), do Shaozhou, z poleceniem ucięcia głowy zwłokom Huinenga.
Twierdził też, że inny uczeń Puji, imieniem Wu Pingyi (武平一), wymazał inskrypcję na steli poświęconej Huinengowi i wstawił tam własną, podającą Shenxiu jako prawowitego Szóstego Patriarchę.
Jednym z najważniejszych dzieł Shenhui, które zachowały się do dzisiejszych czasów, jest odnaleziony w Dunhuang tekst, zatytułowany \pinyin{Nanyang heshang wenda za zheng yi} (南陽和尚問答雜徵義 \textit{Nányáng héshang wèndá zá zhēngyì}), obecnie znany lepiej pod nazwą ``Cytaty Shenhui'' (神會語錄 \pinyin{Shénhuì yǔlù}).
Jego ataki na Szkołę Północną zostały opisane przez Dugu Pei (獨孤沛) w dziele zwanym \textit{Putidamo Nanzong ding shifei lun} (菩提達摩南宗定是非論 \pinyin{Pútídámó Nánzōng dìng shìfēi lùn}).
Chiński pisarz, doktor filozofii Hu Shi (胡適, 1891-1962) zebrał odkryte w Dunhuang dzieła Shenhui i jego uczniów i opisał je w pracy pt. \textit{Shenhui heshang yiji} (神會和尚遺集 \pinyin{Shénhuì héshàng yíjí})
(Huineng, Schlütter i Teiser 2012: 32-33; Huineng i Yampolsky 2012: 26, 28; Huineng i McRae 2000: xv).

Kim był Shenxiu i czym zasłużył sobie na osobiste ataki Shenhui\index{Heze Shenhui, 荷澤神會}? W przedmowie do przekładu \textit{Sutry platformy} Philipa B. Yampolsky'ego napisano, że na przełomie VII i VIII w. Shenxiu był uważany za jednego z najbardziej znaczących i najwybitniejszych mistrzów Chan. Jego biografia jest szczególnie dobrze zachowana. Jej stosunkowo rzetelna wersja została zapisana w dziele ``Annały przekazu skarbu Dharmy'' (傳法寶紀 \pinyin{Chuán fǎbǎo jì}) w pozbawiony elementów fantastycznych sposób. O ile we wszystkich innych dziełach z tego okresu jest wymieniony jako uczeń Hongrena\index{Daman Hongren, 大滿弘忍}, ``Annały'' podają, że był uczniem Faru, a ten --- Hongrena. Według tej biografii pochodził z miasta Daliang (大梁 \toponim{Dàliáng}), obecnie Kaifeng (開封 \toponim{Kāifēng}) w prowincji Henan\index{Prowincja Henan, 河南省} (河南 \toponim{Hénán}) i był członkiem rodu Li (Huineng\index{Dajian Huineng} i Yampolsky 15-16). % 李

Shenxiu już od najmłodszych lat wykazywał się ponadprzeciętnymi uzdolnieniami. W wieku 13 lat, w związku z zawirowaniami historycznymi i związaną z nimi klęską głodu, postanowił porzucić dotychczasowe życie i zostać mnichem buddyjskim. Później wędrował od jednej świątyni do drugiej, by wreszcie jako dwudziestolatek otrzymać pełne ślubowania. W wieku 46 lat udał się do Hongrena\index{Daman Hongren, 大滿弘忍}, a ten natychmiast poznał się na jego talencie. Po wielu latach studiowania nauk osiągnął ostateczne oświecenie, a następnie udał się do Jingzhou (荊州 \toponim{Jīngzhōu}) w prowincji Hubei (湖北 \toponim{Húběi}). Za panowania cesarza Tang Gaozonga, w okresie Yifeng (儀鳳 \pinyin{Yífèng}) udał się do świątyni Yuquan (玉泉寺 \toponim{Yùquán sì}) w pobliżu obecnego miasta Dangyang (當陽 \toponim{Dāngyáng}) w prowincji Hubei. Dopiero po śmierci swojego mistrza zaczął gromadzić wokół siebie uczniów, nauczając ich Dharmy. Przynosił pożytek wielu istotom, prowadząc je do wyzwolenia\ibid % (Yampolsky 15-16)

Między rokiem 730 a 750 Shenhui\index{Heze Shenhui, 荷澤神會} spisał historię życia Huinenga\index{Dajian Huineng} zbliżoną do wersji zawartej w \textit{Sutrze Platformy}, z tą różnicą, że nie było w niej jeszcze wzmianki o konkursie poezji między Shenxiu i Huinengiem.
Wedle obecnego stanu wiedzy Shenhui nie posiadał niemal żadnych wiarygodnych informacji na temat postaci Huinenga.
Wiedział o nim jedynie, że był uczniem Hongrena\index{Daman Hongren, 大滿弘忍}, nauczycielem Dharmy o regionalnym zasięgu działalności, i że mieszkał w Shaozhou.
Najwcześniejsze pisma przypisywane Shenhui podają jedynie, że Huineng był Szóstym Patriarchą i otrzymał od Hongrena, swojego poprzednika, szatę --- insygnium patriarchatu oraz przekaz nauk.
Wybitny poeta Wang Wei (王維) wykonał stelę, w której zapisano, że Huineng przebywał w ukryciu przez 16 lat, a następnie przyjął ślubowania mnisie od mistrza Yinzonga (印宗 \nazwisko{Yìnzōng}).
W \textit{Sutrze Platformy} Shenhui został wymienony jako ostatni z dziesięciu uczniów Huinenga, a także jedyny, który nie płakał, gdy mistrz poinformował ich o zbliżającej się śmierci.
Częste odniesienia do postaci Shenhui wskazują, że \textit{Sutra Platformy} powstała najprawdopodobniej wkrótce po jego śmierci
(Huineng, Schlütter i Teiser 2012: 33; Shi 2008: 117).

Współcześni badacze, tacy jak nieżyjący już John R. McRae, uważają, że niemożliwe jest, by wydarzenia przedstawione w biografii Huinenga\index{Dajian Huineng} faktycznie miały miejsce, i należy je traktować jako ciekawą anegdotę o wyraźnym podtekście duchowym. Za tą tezą przemawia fakt, że Shenxiu był uczniem Hongrena\index{Daman Hongren, 大滿弘忍} jedynie przez kilka lat w początkowym etapie działalności Piątego Patriarchy, a więc kiedy nie istniał jeszcze problem wyboru jego następcy. Shenxiu i Huineng nie przebywali w klasztorze Hongrena w tym samym czasie, a więc nie mogli współzawodniczyć w konkursie poezji. Po drugie, w owym okresie nie istniała jeszcze koncepcja jedynego prawowitego patriarchy --- pojawiła się ona dopiero w dziełach Shenhui\index{Heze Shenhui, 荷澤神會}. Ponadto historia życia Huinenga w zachowanej do dnia dzisiejszego wersji nie pojawia się w pismach Shenhui, a jako propagator Huinenga w roli Szóstego Patriarchy na pewno zapisałby tę historię, gdyby była mu znana. (McRae 2004: 67; Huineng i McRae 2000: xv).

Ważnym aspektem nauk Shenhui\index{Heze Shenhui, 荷澤神會}, przypisywanych Huinengowi\index{Dajian Huineng}, jest odejście od sutry \textit{La\.nkā\-vatāra} na rzecz \textit{Sutry Diamentowej}, która zyskiwała na popularności w VIII w.
O tym, jak wielką rolę odgrywała ona dla Shenhui, świadczą liczne odniesienia do niej w \textit{Sutrze Platformy}.
W tekście jest wysunięta również teza, jakoby to \textit{Sutra Diamentowa}, a nie \textit{La\.nkāvatāra}, była podstawą nauk przekazywanych przez patriarchów, od Bodhidharmy do Huinenga.
Twierdzeniom tym zaprzeczają jednak inne teksty na temat linii przekazu Chan, takie jak ``Kontynuowane biografie wybitnych mnichów'', ``Annały przekazu skarbu Dharmy'' i ``Zapisy mistrzów i uczniów w przekazie Sutry La\.nkāvatāra'' (楞伽師資記 \pinyin{Léngqié shīzī jì}).
Symboliczne odejście od sutry \textit{La\.nkāvatāra} jest też zaznaczone w biografii Huinenga w \textit{Sutrze Platformy} w miejscu, w którym Piąty Patriarcha Hongren \index{Daman Hongren, 大滿弘忍}zrezygnował z wykonania malowideł ze scenami z sutry \textit{La\.nkāvatāra} na rzecz wiersza Shenxiu (Huineng i Yampolsky 2012: 34; McRae 2004: ***).

Interesujące jest, jak w \textit{Sutrze platformy} połączono nauki, pochodzące z różnych systemów filozofii buddyjskiej.
Według filozofii Madhjamiki oraz związanego z nią kanonu sutr Pradżniaparamity, takich jak \textit{Sutra diamentowa} czy \textit{Sutra serca}, rozwinięcie mądrości pradżni jest ostatecznym celem praktyki duchowej.
Osiągnięcie go równa się zrozumieniu pustej natury wszystkich zjawisk, czyniąc zbędnymi nauki o naturze buddy.
Natomiast w systemie \textit{Tathagathagarbha}, z którym związana jest sutra \textit{La\.nkā\-vatāra}, pradżnia jest wyłącznie zdolnością umysłu, zaś głównym obiektem zainteresowania jest natura buddy, obecna w umyśle wszystkich istot.
W \textit{Sutrze platformy} obecne są idee Mahapradżniaparamity, ale fundamentem jej filozofii jest ewidentntie system Tathagathagarbha.
Chociaż w sutrze mówi się o odrzuceniu sutry \textit{La\.nkā\-vatāra} na rzecz \textit{Sutry diamentowej}, w rzeczywistości w tekście wygłoszono wiele nauk, bliższych pierwszemu z tekstów.
(Shi 1990: 150-152).

W dziełach Shenhui\index{Heze Shenhui, 荷澤神會} pojawiły się też dwie opowieści, powielone w późniejszych dziełach. Pierwsza z nich, zapisana zarówno w \textit{Sutrze Platformy}, jak i \textit{Putidamo Nanzong ding shifei lun}, dotyczyła Bodhidharmy i cesarza Liang Wudi (梁武帝 \nazwisko{Liáng Wǔdì}).
Według tej historii, kiedy Bodhidharma przybył do stolicy Liang, miasta Jiangling (江陵 \toponim{Jiānglíng}) przeprowadził dyskusję z cesarzem. Cesarz miał spytać Bodhidharmy, czy budując świątynie, dając ofiary mnichom i ludziom w potrzebie, zgromadził zasługę% ***
\footnote{Zasługa (功德 \pinyin{gōngdé}, skt. \textit{pu\d{n}ya}) w większości tradycji buddyzmu odnosi się do dobrych wrażeń karmicznych, zebranych w rezultacie właściwego postępowania i podążania ścieżką duchową. (Buswell 2004: 532).}.
Mistrz odparł: ,,Nie zgromadziłeś zasługi.'' Miał przez to na myśli, że cesarz, nie podążając za właściwą ścieżką, szukał jedynie błogosławieństw, a nie prawdziwej zasługi. Cesarz, nie rozumiejąc tej nauki, był nią rozczarowany i wygnał Bodhidharmę ze swego państwa. Następnie mistrz udał się do państwa Wei\label{LiangWuDi} (Huineng\index{Dajian Huineng} i Yampolsky 2012: 27, 155-156).

% T48n2007_p0341a24(10)║朕一生未來造寺布施供養有有功德否。達磨答言。並無功德。
% T48n2007_p0341a25(03)║武帝惆悵遂遣。達磨出境。未審此言。請和尚說。
% a w Wei spotkał Huike

Inna opowieść propagowana przez Shenhui\index{Heze Shenhui, 荷澤神會} dotyczy Bodhidharmy oraz jego ucznia i spadkobiercy, Huike. Według tej historii, kiedy mistrz i uczeń spotkali się po raz pierwszy, Huike był zdeterminowany zostać uczniem Bodhidharmy, lecz ten nie chciał go przyjąć. Mistrz miał ustąpić Huike dopiero wówczas, gdy ten w dowód swej determinacji dobył miecza i demonstracyjnie uciął swoje lewe ramię. Huike został następnie głównym uczniem Bodhidharmy i odziedziczył po nim szatę. Później tę samą szatę mieli otrzymać kolejni patriarchowie: Sengcan, Daoxin, Hongren,\index{Daman Hongren, 大滿弘忍} aż do Huinenga\index{Dajian Huineng}. Przy pomocy tej opowieści Shenhui osiągnął dwa cele: nie tylko ustanowił szatę Bodhidharmy insygnium prawowitego patriarchy Chan, lecz również podważył uznaną dotychczas linię przekazu, wiodącą od Bodhidharmy do Shenxiu. Pochodzenie tych legend nie jest znane. Hu Shi uważa, że zostały one wymyślone przez Shenhui, ale równie prawdopodobne jest, że krążyły wśród ludu, a Shenhui jedynie zapisał je i wykorzystał do swoich celów (Huineng i Yampolsky 2012: 27).

\if 0
Yampolsky
Nawet jeżeli istniał Bodhidharma, linia przekazu Chan raczej nie istniała w nieprzerwanej formie
Chan jako taki wymyślono po czasach Hongrena\index{Daman Hongren, 大滿弘忍}
biografia Huinenga\index{Dajian Huineng} ma poprzeć roszczenia
Schlütter 32
\fi

\subsection{Wiersze Shenxiu i Huinenga}
Tradycyjna interpretacja wierszy Shenxiu i Huinenga\index{Dajian Huineng}, zawartych w tekście \textit{Sutry Platformy} (patrz: strony \pageref{ShenxiuVerse} i \pageref{HuinengVerse}), jest prosta.
Według filozofa Guifeng Zongmi (圭峰宗密 \pinyin{Guīfēng Zōngmì}, 780-841) wiersz Shenxiu ma symbolizować stopniową ścieżkę, zaś wiersz Huinenga --- ścieżkę nagłego, ostatecznego oświecenienia, które dokonuje się w jednej chwili.
Tym samym, w rozumieniu Zongmi wiersze te reprezentują pogląd dwóch konkurencyjnych tradycji Chan, północnej i południowej.
W późniejszym okresie była to dominująca interpretacja szkoły Chan
(McRae 2004: 63).

Takie rozumienie jest jednak nadmiernym uproszczeniem. Wiersz przypisywany Shenxiu odnosi się nie tyle do stopniowej ścieżki, ile do ciągłej, bezustannej praktyki oczyszczania zwierciadła z kurzu.
Wiersz Huinenga\index{Dajian Huineng} nie opisuje natomiast poglądu nagłej ścieżki, a jedynie neguje twierdzenia, zawarte w wierszu Shenxiu.
Oprócz tego, wiersze przedstawiają dwa punkty widzenia na ten sam temat i nie mogą być interpretowane osobno
(McRae 2004: 63-64).

Opis konkursu poezji, zawarty w tekście \textit{Sutry platformy}, ma za zadanie udowodnić wyższość nauk Huinenga\index{Dajian Huineng} i jego spadkobierców nad naukami Shenxiu.
Jak zaznacza McRae (2004, str. 65), chociaż zwycięzcą w starciu był Huineng, to wiersz, przypisywany Shenxiu, wcale nie jest mierny, a wręcz przeciwnie --- głęboki i wyrafinowany.
W ten sposób wiersz Huinenga tylko zyskuje w oczach odbiorcy. Nie jest bowiem sztuką stworzyć dzieło lepsze od czegoś miernego.

\if 0
McRae xvi
Shenxiu i Huineng\index{Dajian Huineng} nie przebywali w Huangmei w tym samym czasie

McRae 62-64:
Wiersz Shenxiu:
Nie ma dowodu, żeby kiedykolwiek go napisał, albo żeby porównał umysł do jasnej lustrzanej podstawy, ale z innych jego dzieł wynika, że mógł napisać coś w tym stylu
constant and perfect teaching, the endless personal manifestation of the
bodhisattva ideal
McRae, Northern School, 235. The English “suchlike'' renders the word 如, as in the Chinese translation of Tathágata, 如來, when used as a modifier.
神秀觀心論
T85n2833_p1271c15(08)║眾生修伽藍鑄形像燒香散花然長明燈。
T85n2833_p1271c16(09)║晝夜六時遶塔行道持齋禮拜。種種功德皆成佛道。
T85n2833_p1271c17(06)║若唯觀心總攝諸行。如是事應妄也 答曰。
T85n2833_p1271c18(06)║佛所說無量方便。一切眾生鈍根狹劣。
T85n2833_p1271c19(08)║甚深所以假有為喻無為。若不內行唯只外求。希望獲福。
T85n2833_p1271c20(03)║無有是處。言伽藍者。西國梵音。
T85n2833_p1271c21(07)║此地翻為清淨處地。若永除三毒常淨六根。
T85n2833_p1271c22(07)║身心湛然內外清淨。是名為修伽藍也。又鑄形像者。
ciało jest drzewem bodhi, oba są fizyczne
ciało jest miejscem, w którym człowiek osiąga oświecenie
lustrzana podstawa

Wiersz Shenxiu był głęboki i wyrafinowany, bo tym sposobem wiersz Huinenga\index{Dajian Huineng} jest lepszy od czegoś wybitnego; być lepszym od czegoś miernego to żadne osiągnięcie
Wiersz Huinenga wcale nie wyraża poglądów szkoły nagłego oświecenia, a jedynie zaprzecza poglądom wiersza Shenxiu
Więc nie jest to wykładnia szkół nagłego i stopniowego oświecenia
Wiersz Huinenga pojawiał się w kilku wersjach w różnych tekstach
Wiersze oparte na pismach Szkoły Północnej (McRae 67)

Przejście od Lankavatara do Diamentowej Sutry, w VIII w. ta właśnie sutra nabierała popularności, na niekorzyść Lankavatary
\fi
\subsection{Analogie do biografii Konfucjusza}
Pisząc historię życia Huinenga\index{Dajian Huineng}, Shenhui\index{Heze Shenhui, 荷澤神會} w oczywisty sposób czerpał z legendy o Konfucjuszu, opisanej w \textit{Zapiskach historyka} (史記 \pinyin{Shǐjì}) autorstwa Sima Qian (司馬遷 \nazwisko{Sīmǎ Qiān}). Tekst ten miał wówczas ugruntowaną pozycję wśród chińskich elit, jako że Konfucjusz był twórcą głównego systemu filozoficznego w państwie i był znany niemal wszystkim jako uniwersalny wzorzec cnót (Huineng, Schlütter i Teiser 2012: 36).

Według wspomnianej legendy Konfucjusz urodził się jako Kong Qiu (孔丘 \nazwisko{Kǒng Qiū}), owoc mezaliansu Shuliang He (叔梁紇), oficera wojsk państwa Lu, i lokalnej kobiety, Yan Zhengzai (顏徵在). Jego ojciec zmarł, gdy ten był jeszcze dzieckiem, w związku z czym Konfucjusz, tak jak Huineng\index{Dajian Huineng}, dorastał w ubóstwie, wychowywany przez samotną matkę. Rodzina Shuliang He nie pomagała im po jego śmierci. Niewiele wiadomo o najwcześniejszych latach życia Kong Qiu, ale według \textit{Zapisków historyka}, już w dzieciństwie wykazywał wielkie zainteresowanie naczyniami rytualnymi z brązu. Według legendy, lubił bawić się nimi, starannie układając je tak, jak przy składaniu ofiar przodkom. W wieku 15 lat Kong Qiu zaczął zgłębiać teksty i rytuały z początków dynastii Zhou%
\footnote{Dynastia Zhou (周朝 \pinyin{Zhōu Cháo}) --- dynastia rządząca Chinami w latach ok. 1045-256 p.n.e. Dzieli się na tzw. Zachodnią Dynastię Zhou i Wschodnią Dynastię Zhou.}.
Uważał ów okres za złoty wiek cywilizacji chińskiej, czas zjednoczenia i pokoju, zaś jej władców, a także królów sprzed założenia tej dynastii --- za mędrców, rządzących swoimi państwami sprawiedliwie i moralnie.
Twierdził, że powodem, dla którego we współczesnych mu Chinach zapanował chaos, było odejście od ideałów Zhou i postawił sobie za cel przywrócenie w kraju dawnych wartości i rytuałów.
Konfucjusz nie cieszył się szczególną popularnością wśród lokalnej elity państwa Lu.
Jego idee nie zostały docenione za jego życia, tak jak idee Huinenga, które spopularyzowała dopiero \textit{Sutra platformy}.
Pierwszą pracą, którą podjął, było zarządzanie gospodarstwem rolnym lokalnej arystokracji.
Przez pewien czas pracował jako urzędnik niskiego szczebla, lecz wkrótce został z tej posady zwolniony, mimo że wykazywał ponadprzeciętne uzdolnienia.
(Schuman 2015: 45-50; Huineng, Schlütter i Teiser 2012: 36-37).

Shenhui był wykształcony w zakresie filozofii konfucjańskiej, a nawet został przez Zongmi porównany do Konfucjusza. Zan Ning (贊寧), autor ``Biografii wybitnych mnichów Song'' (宋高僧傳 \pinyin{Sòng gāosēng zhuàn}) porównał go do Yan Hui (顏回), ulubionego ucznia Konfucjusza.
Symboliczne znaczenie miało również liczba patriarchów w linii przekazu Chan. Według panującej wówczas interpretacji myśli konfucjańskiej jedynie cesarz mógł w świątyni swojej rodziny urządzić siedem izb. Linia przekazu w wersji Shenhui\index{Heze Shenhui, 荷澤神會} wiodła od Bodhidharmy, przez Hongrena\index{Daman Hongren, 大滿弘忍}, do Huinenga\index{Dajian Huineng}. Jako spadkobierca Szóstego Patriarchy, Shenhui stawał się Siódmym Patriarchą i tym samym ustanawiał doskonałą linię Chan.
\if 0
Southern Learning
siedem świątyń lub izb w świątyni przodków było zarezerwowane dla cesarzy
宗
Schlütter 37 \fi
(Huineng, Schlütter i Teiser 2012: 37).

Z perspektywy nauk Konfucjusza Huineng\index{Dajian Huineng} był postacią godną naśladowania.
Mimo iż wywodził się z nizin społecznych i wychował się w oddalonym od centrum cywilizacji chińskiej rejonie, to posiadał jedną z najwyższych cnót konfucjańskich --- \textit{nabożność synowską} (子孝 \pinyin{zǐxiào}).
Po śmierci ojca ciężko pracował, utrzymywał starą matkę i opiekował się nią.
Taki obraz Huinenga stał w opozycji do wykształconych, bogatych elit, z których wywodziła się w owym okresie większość mnichów buddyjskich.
Twórcy jego hagiografii mogli również inspirować się historią życia Hongrena\index{Daman Hongren, 大滿弘忍}, który według niektórych podań medytował za dnia, zaś w nocy zajmował się bydłem.
Sam Shenxiu, rzekomy konkurent Huinenga do pozycji patriachy, pochodził ze szlachetnego rodu i był wykształcony zarówno w literaturze buddyjskiej, jak i świeckiej, a niektórzy podejrzewają nawet, że mógł być związany z rodem cesarskim
(McRae 2004: 68).

Celem takiego przedstawienia Szóstego Patriarchy było pokazanie, że każdy, nawet osoba świecka, niezależnie od pozycji społecznej, miejsca pochodzenia i wykształcenia mógł rozpoznać naturę swojego umysłu i zostać zwierzchnikiem Chan.
Co za tym idzie, jeżeli oświecenie było jedynym warunkiem zostania patriachą, to czytelnik mógł oczekiwać, że wszyscy poprzednicy i następcy Huinenga\index{Dajian Huineng} również byli oświeceni.
W ten sposób szkoła Chan zyskiwała autorytet i wiarygodność (McRae 2004: 69).


\chapter{Analiza tekstu \textit{Sutry Platformy}}

\section{Budowa Sutry}
Tekst \textit{Sutry platformy} w przekładzie Philipa B. Yampolsky'ego został podzielony na 57 sekcji o różnej długości. Narratorem tekstu jest Fahai, uczeń Huinenga\index{Dajian Huineng}, nie licząc sekcji 2-11, będących autobiografią Szóstego Patriarchy w formie monologu. Pierwsze 37 sekcji zawiera nauki wygłoszone przez Huinenga w świątyni Dafan. Jak podaje tekst, wysłuchało ich zgromadzenie ponad dziesięciu tysięcy praktykujących, w tym mnichów, mniszek i ludzi świeckich, a także prefekt Shaozhou, Wei Qu (韋璩). W sekcjach 12-37 przekazane zostały ,,bezforemne nauki'' (無相戒 \pinyin{wúxiàng jiè}, ang. \textit{precepts of formlessness} lub \textit{formless precepts}).

W sekcjach 39-44 zamieszczono anegdoty dotyczące najważniejszych spadkobierców Huinenga\index{Dajian Huineng}, takich jak Fahai, Zhichang (智常 \pinyin{Zhìcháng}), Fada (法達 \pinyin{Fǎdá}) i Shenhui\index{Heze Shenhui, 荷澤神會}. Sekcje 45-47 to nauki udzielone dziesięciu bliskim uczniom. Sekcje 48-57 opisują okoliczności śmierci Szóstego Patriarchy, a także: nauki, które przekazał swym uczniom w formie wierszy bezpośrednio przed śmiercią; informacje o następcach Huinenga, dalszych losach \textit{Sutry platformy} i przekazie nauk szkoły Chan.

\section{Nauki o medytacji}
W sekcjach 13-19 przekazane są pouczenia o medytacji. Najważniejsza nauka, przedstawiona w tym fragmencie, dotyczy jedności medytacji (惠 \pinyin{huì}) i mądrości (定 \pinyin{dìng}).\label{DingHui}
Według Szóstego Patriarchy medytacja i mądrość są częściami tej samej całości i stwarzają siebie nawzajem.
Są od siebie współzależne, i nie można stwierdzić, które z nich pojawiło się jako pierwsze. W tekście zostały porównane do światła i lampy. Takie same nauki pojawiają się w ``Cytatach Shenhui\index{Heze Shenhui, 荷澤神會}''
(Huineng\index{Dajian Huineng} i Yampolsky 2012: 137).

Według tych nauk, główną doktryną subityzmu jest ,,brak myśli'' lub ,,brak idei'' (無念 \pinyin{wúniàn} lub 無心 \pinyin{wúxīn}, `brak umysłu'), jego istotą --- ,,brak formy'' (無相 \pinyin{wúxiàng}), a jego podstawą --- ,,brak przywiązania'' (無住 \pinyin{wúzhù})\ibid
% istotą/substancją; subityzm był dodany; wúzhù --- no abiding

,,Brak idei'' oznacza wolność od rozproszenia --- pilnowanie, by umysł nie podążał za myślami i aby nic, co pojawia się w umyśle, nie prowadziło do powstania negatywnych emocji i błędnych poglądów. Oznacza też brak przywiązania do dualistycznego postrzegania zewnętrznych zjawisk i świadomości, która je przeżywa. Według tej doktryny wszystkie myśli powstają w esencji umysłu i wyrażają jego potencjał, dlatego próby całkowitego wyparcia myśli są błędne. Właściwa praktyka polega na odcięciu pomieszanych, dualistycznych koncepcji i koncentracji na prawdziwej naturze Takości%
\footnote{Takość (skt. \textit{Tathātā}, chiń. 真如) --- stan }
(Huineng\index{Dajian Huineng} i Yampolsky 2012: 137-138).

,,Brak obiektu'' oznacza unikanie rozproszenia pod wpływem zewnętrznych zjawisk. Nauki te nie oznaczają, że należy fizycznie odciąć się od myśli i form, lecz ,,być oddzielonym od formy nawet wtedy, gdy jest się z nią związanym''. Człowiek praktykujący tę ścieżkę nawet bedąc związany z formą i myślami, nie traktuje ich jako prawdziwie istniejących, a jedynie jako przejawienie się potencjału przestrzeni. ,,Brak przywiązania'' oznacza niemyślenie o przeszłości ani przeszłości\ibid
% *** TUTAJ NIE WIEDZIAŁEM ZA BARDZO, O CO CHODZIŁO ***

Według Huinenga\index{Dajian Huineng}, praktykujący Chan nie powinien koncentrować się ani na umyśle, ani na czystości, nie powinien też mówić o niewzruszoności. Twierdzi, że umysł sam w sobie jest zwodniczy, jest jedynie iluzją i jako taki nie powinien być obiektem medytacji. Również sama koncentracja na czystości nie przynosi spodziewanych rezultatów. W sytuacji, gdy praktykujący nie zrozumie, że natura umysłu jest sama w sobie doskonała i czysta, taka praktyka stwarza jedynie kolejne złudzenia i sztywne koncepcje, jeśli praktykujący. Jedynym powodem, dla którego istoty nie są w stanie postrzegać swojej natury w ten sposób, są błędne poglądy i zaciemnienia (Huineng i Yampolsky 2012: 139-140).

Prawidłowa praktyka medytacji siedzącej (坐禪 \pinyin{zuòchán}) została przez Huinenga\index{Dajian Huineng} zdefiniowana jako niepodążanie za myślami oraz postrzeganie własnej prawdziwej natury bez rozproszenia. Natomiast ,,medytacja Chan'' (禪定 \pinyin{chándìng}) oznacza według niego niezależność od zewnętrznych zjawisk oraz unikanie pomieszania w umyśle (Huineng i Yampolsky 2012: 140-141).

\section{Bezforemne nauki}
W sekcjach 20-37 \textit{Sutry platformy} przekazane zostały tzw. ,,bezforemne nauki'', jeden z najważniejszych aspektów tekstu. W sekcji 20. Szósty Patriarcha naucza o obecności trzech ciał buddy%
\footnote{Trzy ciała buddy (skt. \textit{trikāya}, chiń. 三身 \pinyin{sānshēn}), w języku polskim nazywane również trzema stanami buddy --- w buddyzmie Mahajany\index{Buddyzm Mahajany} termin ten oznacza różne sposoby wyrażania się oświecenia: stan prawdy (lub: ciało prawdy, skt. \textit{dharmakāya} (法身 \pinyin{fǎshēn}), który oznacza ponadczasowe, wszechobecne oświecenie; stan radości (lub: ciało radości, skt. \textit{sambhogakāya}, chiń. 報身 \pinyin{bàoshēn}), oznaczający przejawienie się buddy w postaci formy z energii i światła; oraz stan wypromieniowania (lub: ciało wypromieniowania, skt. \textit{nirmā\d{n}akāya}, chiń. 應身 \pinyin{yīngshēn}), materialne ciało, które przejawia się w określonym miejscu w czasie i przestrzeni.}
w fizycznym ciele praktykującego. Według niego, stan prawdy, stan radości i stan wypromieniowania są nierozerwalnie związane z naturą buddy wszystkich istot. Ponieważ jednak przesłaniają je błędne poglądy, wielu praktykujących poszukuje ich na zewnątrz.
(Huineng\index{Dajian Huineng} i Yampolsky 2012: 141-143).

W sekcji 21. przekazane są cztery ślubowania (四弘大願 \pinyin{sì hóngdà yuàn}), dotyczące kolejno: wyzwolenia wszystkich czujących istot, odcięcia wszelkich negatywnych emocji, zgłębienia wszystkich buddyjskich nauk i urzeczywistnienia nieprzewyższonej ścieżki Buddy, tzn. osiągnięcia oświecenia.
Dalej wyjaśniono, że pierwsze ślubowanie nie powinno być traktowane dosłownie (jest jedynie rozwinięcia altruistycznej motywacji). W istocie wszystkie istoty muszą same osiągnąć wyzwolenie przez rozpoznanie prawdziwej natury własnego umysłu.
Odcięcie negatywnych emocji oznacza w tym kontekście odrzucenie w umyśle tego, co nieprawdziwe i błędne.
Ostatnie ślubowanie nakazuje praktykującemu zachowywać w każdej sytuacji skromność, szacunek dla wszystkich istot, unikać przywiązania i, ostatecznie, przebudzić się w mądrości pradżni
(Huineng\index{Dajian Huineng} i Yampolsky 2012: 143-144).

Sekcja 22. zawiera nauki dotyczące ,,bezforemnej skruchy'' (無相懺悔 \pinyin{wúxiàng chànhuǐ}), które pozwalają praktykującemu uwolnić się od skutków wcześniejszych negatywnych działań.
Huineng\index{Dajian Huineng} uważa, że werbalne wyznawanie grzechów buddom nie ma sensu, zamiast tego zaleca ,,w każdej sytuacji praktykować nie-działanie'' (永斷不作 \pinyin{yǒng duàn bù zuò}). Można to zrozumieć jako pozbycie się negatywnych emocji\footnote{Przeszkadzające emocje, negatywne emocje, zaciemnienia, trucizny umysłu (skt. \textit{kleśa}, chiń. 煩惱 \pinyin{fánnǎo}, zwane również 塵勞 \pinyin{chénláo}) --- uczucia i stany umysłu, które powodują cierpienie i motywują istoty do popełniania krzywdzących działań.} i złudzeń oraz niewykonywanie nigdy więcej przynoszących cierpienie działań (Huineng i Yampolsky 2012: 144-145).

W sekcji 23. Huineng\index{Dajian Huineng} przekazał ,,bezforemne nauki o Trzech Schronieniach\fnm'' (無相三歸依戒 \pinyin{wúxiàng sān guīyī jiè}).
Szósty Patriarcha utożsamia Buddę z oświeceniem (覺~\pinyin{jué}), Dharmę --- z prawdą (正 \pinyin{zhèng}), a Sanghę --- z czystością (淨 \pinyin{jìng}).
Poleca swym uczniom przyjąć schronienie we własnej oświeconej naturze umysłu, trzymając się z daleka od innych buddów i nauk.
Obiecuje, że jeśli tak uczynią, ich umysł nie będzie splamiony złudzeniami i fizycznym pożądaniem.
Przyjęcie schronienia w Dharmie ma zabezpieczać przed błędnymi poglądami i przywiązaniem.
Polegając zaś na Sandze, praktykujący nie będzie ulegał przeszkadzającym emocjom i złudzeniom
(Huineng i Yampolsky 2012: 145-146).
\footnotetext{Schronienie (歸依 \pinyin{guīyī}, zapisywane również 皈依)--- w większości tradycji buddyzmu jest to rytualna deklaracja, że Budda osiągnął najwyższe oświecenie i przekazał społeczności praktykujących (sandze) nauki, prowadzące do osiągnięcia przebudzenia.}
% Encyclopedia 714

W sekcjach 24-26 Szósty Patriarcha udziela wyjaśnień na temat Mahapradżniaparamity.
Huineng\index{Dajian Huineng} podkreśla tu, że Dharmę należy praktykować w swoim umyśle, a nie jedynie bezmyślnie powtarzać.
W tradycji Chan recytacja sutr z gatunku \textit{Mahapradżniaparamity}, takich jak \textit{Sutra Diamentowa} lub \textit{Sutra Serca}%
\footnote{\textit{Sutra Serca}\label{HeartSutra} (skt. \textit{Prajñāpāramitāh\d{r}daya}, chiń. 般若波羅密多心經 \pinyin{Bōrě bōluómìduō xīnjīng} `Sutra Serca Doskonałości Mądrości' lub w skrócie 心經 \pinyin{xīnjīng} `Sutra Serca') --- sutra z gatunku Pradżniaparamity, jeden z najpopularniejszych i najbardziej znanych na świecie tekstów buddyjskich. Prawdopodobnie chiński apokryf, stworzony na podstawie chińskiego tłumaczenia ``Wielkiej Sutry Doskonałości Mądrości'' (skt. \textit{Mahāprajñāpāramitā Sūtra}, chiń. 摩訶般若波羅密多經 \pinyin{Móhē bōrě bōluómìduō jīng}), przypisywanej indyjskiemu filozofowi Nagardżunie (skt. \textit{Nāgārjuna}, chiń. 龍樹 \nazwisko{Lóngshù}).},
jest bowiem jednym z rodzajów praktyki medytacyjnej.
Wyjaśniając znaczenie nazwy \textit{Mahapradżniaparamita}, Huineng interpretuje pierwszy jej człon, sanskryckie słowo \textit{mahā} (`wielki', chiń. 摩訶 \pinyin{móhē}), jako odniesienie do nieograniczonej przestrzeni umysłu, która będąc pusta, zawiera w sobie wszystkie zjawiska.
Człowiek, który praktykuje zasadę \textit{mahā} nie powinien ani lgnąć do zjawisk, ani odpychać ich, lecz traktować je dokładnie tak, jak przestrzeń nieba.
\textit{Pradżnia} oznacza, że w żadnej myśli praktykującego nie może pojawić się niewiedza.
\textit{Paramita} (skt. \textit{pāramitā}, chińskie odwzorowanie fonetyczne: 波羅密多 \pinyin{bōluómìduō}) oznacza ,,osiągnięcie drugiego brzegu''.
Huineng mówi, że ten, kto zrozumie znaczenie słowa \textit{paramita}, będzie poza narodzinami i śmiercią
(Huineng i Yampolsky 2012: 146-148; Buswell 2004: 666).

Sekcja 27. mówi o czystej naturze zaciemnień.
Huineng\index{Dajian Huineng} naucza tu, że gdy medytujący pozbędzie się przeszkadzających emocji, zostanie tylko \textit{pradżnia}, która jest zawsze obecna i nieoddzielna od umysłu.
Ponieważ esencją niewiedzy, złudzeń i błędnych myśli jest natura buddy, praktykujący nie powinen próbować się ich pozbyć, a jedynie rozpoznać ich prawdziwą istotę
(Huineng i Yampolsky 2012: 148-149).

W sekcjach 28-30 Szósty Patriarcha mówi o naturalnych predyspozycjach słuchaczy i praktykujących oraz o mądrości pradżni, nieoddzielnej od prawdziwej natury umysłu.
Na wiele sposobów parafrazuje informację o tym, że zwykły człowiek różni się od Buddy tylko zrozumieniem nauk i poziomem urzeczywistnienia.
W sekcji 28. zachwala on ponadto \textit{Sutrę Diamentową} jako tekst, który sam w sobie wystarcza do osiągnięcia stanu umysłu zwanego \textit{prajñā samādhi} (z skt. `wchłonięcie medytacyjne pradżni', chiń. 般若三昧 \pinyin{bōrě sānmèi}).
Stosuje przy tym parabolę o królu smoków, który zbiera wodę z oceanu i zrzuca ją na ziemię.
Jeśli wielki deszcz spadnie na stały ląd, wówczas miasta i wioski spłyną jak trawa i liście.
Jeżeli jednak woda spadnie na taflę oceanu, nic się nie wydarzy.
Podobnie człowiek nieposiadający odpowiednich zdolności i predyspozycji nie będzie w stanie zrozumieć tych nauk.
Jednak gdy człowiek praktykujący Mahajanę\index{Buddyzm Mahajany} usłyszy tekst sutry, jego umysł otworzy się i będzie mógł się przebudzić.
Mimo że natura umysłu tych dwóch typów ludzi jest taka sama, słuchacz nieposiadający głębokiego zrozumienia nie jest w stanie osiągnąć przebudzenia bezpośrednio po usłyszeniu nauk.
Powodem tego są błędne poglądy oraz przeszkadzające emocje.
Kiedy jednak rozpozna on pradżnię w swoim umyśle, nie będzie już musiał polegać na intelektualnym pojęciu nauk
% a smok symbolizuje pierwotną mądrość pradżni w umysłach wszystkich czujących istot?
(Huineng\index{Dajian Huineng} i Yampolsky 2012: 149-150).

W sekcji 31. Huineng\index{Dajian Huineng} naucza o roli nauczyciela duchowego na ścieżce.
Praktykujący, który nie jest w stanie w jednej chwili rozpoznać natury swojego umysłu patrząc w głąb siebie i obserwując swój umysł, powinien znaleźć mistrza, zdolnego nim pokierować.
Następnie podaje, że dobry nauczyciel rozumie, iż właściwą ścieżką jest Dharma Najwyższego Pojazdu%
\footnote{Dharma Najwyższego Pojazdu (最上乘法 \pinyin{zuì shàng chéng fǎ}) --- patrz: analiza sekcji 43. na stronie \pageref{SiChengFa}.}.% ***
Ten jednak, kto potrafi sam rozpoznać naturę swego umysłu, nie powinien szukać nauczyciela na zewnątrz, lecz wewnątrz siebie, i urzeczywistnić ,,brak myśli''
(Huineng i Yampolsky 2012: 151-153).

Sekcja 32. zawiera zalecenia Huinenga\index{Dajian Huineng} na temat przekazu doktryny Nagłego Oświecenia w następnych pokoleniach.
Patriarcha poleca swym uczniom praktykować te nauki wspólnie i chronić je, obiecując, że ten, kto poświęca temu całe swoje życie, bez wątpienia ,,wstąpi w szeregi świętych'' (入聖位 \pinyin{rù shèngwèi}).
Wraz z naukami subityzmu przekazuje również ostrzeżenie, żeby nie wyjaśniać tych nauk ludziom, którzy nie są na nie gotowi, tzn. nieposiadającym właściwego zrozumienia i determinacji.
Na pewno nie przyniesie im to bowiem pożytku, a może nawet im zaszkodzić
(Huineng i Yampolsky 2012: 153-154).

W sekcji 33. Huineng\index{Dajian Huineng} przekazał swoim uczniom ,,wiersz usuwający negatywną karmę'' (滅罪頌 \pinyin{miè zuì sòng}).
Według tego wiersza, praktyka szczodrości (布施 \pinyin{bùshī}) i zasiewania nasion przyszłego szczęścia (修福 \pinyin{xiūfú}) nie są same w sobie wystarczające do osiągnięcia ostatecznego wyzwolenia i oświecenia.
Nawet jeśli praktyka ta przyniesie pożądane rezulaty, nie usunie ona przeszkadzających emocji.
Ten, kto pragnie je wykorzenić, musi w swoim umyśle usunąć ich przyczyny, wyrazić skruchę i pilnie praktykować nauki Mahajany\index{Buddyzm Mahajany}
(Huineng i Yampolsky 2012: 154-155).

Sekcja 34. zawiera dialog prefekta Wei Qu z Szóstym Patriarchą na temat nauk, udzielonych cesarzowi Wu z Liang przez Pierwszego Patriarchę (patrz: strona \pageref{LiangWuDi}).
Według Huinenga\index{Dajian Huineng}, zasługa powstaje z umysłu i w umyśle, a więc jedynym sposobem na jej zgromadzenie jest spoglądanie we własną naturę buddy, w \textit{dharmakāya}, i szacunek dla Trzech Klejnotów%
\footnote{Trzy Klejnoty (skt. \textit{triratna}, chiń. 三寶 \pinyin{sānbǎo} --- zbiorcze określenie trzech obiektów, w których przyjmują schronienie buddyści większości tradycji. Są nimi: Budda (佛陀 \pinyin{Fótuó} lub 佛 \pinyin{fó}), rozumiany jako historyczny Budda Siakjamuni lub natura buddy; Dharma (skt., chiń. 法 \pinyin{fǎ}) --- nauki Buddy, prowadzące do osiągnięcia oświecenia; a także Sangha (skt. \textit{Sa\d{m}gha}, chiń. 僧伽 \pinyin{Sēngjiā} lub 僧 \pinyin{sēng}) --- społeczność praktykujących Dharmę, zarówno tych, którzy już osiągnęli oświecenie, jak i będących jeszcze na ścieżce.}
oraz innych ludzi. % ***
(Huineng i Yampolsky 2012: 155-156).

W sekcji 35. zawarto nauki Huinenga\index{Dajian Huineng} dotyczące praktyk Czystej Krainy, związanych z Buddy Amitābhą\index{Budda Amitābha}, które w jego czasach były już w Chinach szeroko rozpowszechnione.
Na pytanie prefekta Wei Qu, czy możliwe jest odrodzenie się w Czystej Krainie, Szósty Patriarcha odrzekł, że w istocie nauki, które Budda Siakjamuni przekazał na ten temat w mieście Śrawasti, służyły wyłącznie nawróceniu ludzi.
Mówi, że dla osoby oświeconej nie ma różnicy między Zachodem (Czysta Kraina Buddy Amitābhy jest utożsamiana z tym kierunkiem geograficznym) a wschodem (Chinami).
Gdy dobry człowiek ze Wschodu oczyści swój umysł, będzie zawsze w Czystej Krainie.
Natomiast jeżeli zły człowiek odrodził się na Zachodzie, ale ma błędne poglądy, nie będzie wyzwolony.
% analogia o czystej krainie w ludzkim ciele
(Huineng i Yampolsky 2012: 156-159).

Sekcja 36. odwołuje się do nauk z poprzedniego fragmentu, tłumacząc, że aby praktykować Dharmę, wcale nie trzeba udawać się do klasztoru.
Huineng\index{Dajian Huineng} porównuje tu mnicha, który nie żyje w zgodzie z naukami, do złego człowieka z Zachodu.
Człowieka świeckiego, praktykującego Dharmę, przyrównuje natomiast do dobrego człowieka ze Wschodu.
W czasach, gdy powstawała \textit{Sutra platformy}, większość mnichów wywodziła się z bogatych rodzin.
Pouczenia zawarte w tej sekcji czynią jednak z Chanu uniwersalną ścieżkę, którą może podążać każdy, niezależnie od statusu społecznego.
W tym fragmencie Szósty Patriarcha przekazuje również „bezforemny wiersz” (無相頌) dotyczący praktyki w życiu codziennym.
Mówi w nim, że biegłość w nauczaniu i w medytacji (說通及心通) są niczym słońce i pustka.
Następnie zaleca swym uczniom, by przekazując wyłącznie nauki o postrzeganiu natury umysłu, wykroczyli poza ograniczenia zwyczajnego życia i pozbyli się błędnych poglądów. % (惟傳見性法 出世破邪宗)
Mówi, że dzięki zgłębianiu tych nauk, nawet ignorant nie zbłądzi na ścieżce % 若學頓教法 愚人不可迷
(Huineng i Yampolsky 2012: 159-161).

Następnie, w sekcji 37., Szósty Patriarcha przekazał dalsze pouczenia dotyczące tego ,,bezforemnego wiersza''.
Mówi, że nawet ktoś, kto znajduje się bardzo daleko od niego (dosłownie 1000 \textit{li}\footnote{\textit{Li} (里 \pinyin{lǐ}) --- tradycyjna chińska jednostka odległości, współcześnie równa 500 m.}) jeżeli praktykuje w zgodzie z tymi naukami, zawsze będzie blisko niego.
Natomiast jeżeli ktoś siedzi z nim twarzą w twarz i nie podąża właściwą ścieżką, zawsze będzie 1000 \textit{li} od niego.
Po wygłoszeniu tych nauk Huineng\index{Dajian Huineng} udał się do Caoqi, zaznaczając, że jeśli ktokolwiek będzie miał wątpliwości lub pytania dotyczące , powinien odwiedzić go na tej górze, a wtedy mistrz będzie mógł je rozwiać % 同見佛世
(Huineng i Yampolsky 2012: 161-162).

W sekcji 38. opowiedziano dalsze dzieje życia Szóstego Patriarchy.
Według tego zapisu Huineng\index{Dajian Huineng} przez następne 40 lat nauczał ludzi z Shaozhou i Guangzhou.
Tekst podaje, że mistrz miał wiele tysięcy uczniów, zarówno mnichów, jak i świeckich praktykujących.
Według tego zapisu, osią jego nauk był przekaz \textit{Sutra platformy}, a w późniejszych pokoleniach jedynie ci, którzy otrzymali ten przekaz, mieli być ortodoksyjnymi dzierżawcami Południowej Szkoły.
Następnie podaje warunki, na jakich tekst ten miał być przekazywany.
Każdy uczeń miał dostać kopię sutry z dołączonym do niej zapisanym imieniem odbiorcy oraz miejscem i datą wręczenia.
Ten fragment powstał prawdopodobnie pod wpływem uczniów mistrza Fahai, podawanego jako autor \textit{Sutry platformy} % źródło
(Huineng i Yampolsky 2012: 162).

Sekcja 39. mówi o pochodzeniu nazw ,,Szkoła Północna'' i ,,Szkoła Południowa''.
Według tego zapisu, wzięły się one stąd, że mistrz Shenxiu nauczał w świątyni Yuquan w pobliżu miasta Dangyang w obecnej prowincji Hubei, a więc na północy Chin.
Huineng\index{Dajian Huineng} zaś nauczał w Caoqi, w pobliżu Shaozhou, a więc w południowej części kraju.
Chociaż w Dharmie nie ma podziałów, ludzie z Południa i Północy różnią się od siebie; stąd podział na dwie szkoły Chan.
Dalej podano, że nauki o nagłym i stopniowym oświeceniu są w swojej istocie jedynie różnymi aspektami tych samych nauk.
O ile człowiek o wybitnych zdolnościach jest w urzeczywistnić ścieżkę subityzmu, o tyle ludzie mniej inteligentni muszą podążać stopniową ścieżką
(Huineng i Yampolsky 2012: 162-163).

\section{Opowieści o uczniach Huinenga}
W sekcjach 40-44 \textit{Sutry platformy} zawarto informacje o spadkobiercach Huinenga\index{Dajian Huineng}, takich jak Shenhui\index{Heze Shenhui, 荷澤神會}, Fada czy Zhichang.

Sekcje 40. i 41. opowiadają o przybyciu do Szóstego Patriarchy mnicha imieniem Zhicheng (志誠 \pinyin{Zhìchéng}).
Anegdota ta ma dowodzić wyższości Szkoły Południowej nad Szkołą Północną.
Jak podaje ten fragment, pewnego razu Shenxiu usłyszał o Huinengu i rzekomej wyższości jego nauk.
Nauki Szóstego Patriarchy miały przynosić szybkie rezultaty i bezpośrednio kierować uczniów na właściwą ścieżkę. % direct pointing of the Way
Shenxiu postanowił sprawdzić, czyje nauki są lepsze.
W tym celu wysłał Zhichenga, który był wówczas jego uczniem, do Caoqi, celem zebrania informacji o Południowej Szkole.
Kiedy ten przybył do świątyni Huinenga\index{Dajian Huineng} i usłyszał jego nauki, w jednej chwili rozpoznał naturę swojego umysłu i osiągnął oświecenie.
Następnie powiedział mu, skąd przybył i poprosił go o przyjęcie na ucznia.
Przyznał, że z początku był szpiegiem mistrza Shenxiu, ale przestał nim być, gdy usłyszał nauki Huinenga
(Huineng i Yampolsky 2012: 163-164).

Wówczas Szósty Patriarcha spytał Zhichenga o szczegóły doktryny Shenxiu, a dokładniej, jakie były jego nauki na temat moralności (właściwego postępowania), medytacji i mądrości (戒定惠 \pinyin{jiè dìng huì}, dosł. `ślubowania, medytacja i mądrość').
Zhicheng powiedział, że według wyjaśnień jego mistrza, właściwe postępowanie oznaczało niepopełnianie różnego rodzaju negatywnych działań; mądrością jest wykonywanie pozytywnych działań, zaś medytacją --- oczyszczanie umysłu.
Huineng\index{Dajian Huineng} powiedział mu, że w istocie są to dobre nauki, ale reprezentują powolną ścieżkę, przeznaczoną dla mniej zdolnych słuchaczy, w odróżnieniu od jego własnej, szybkiej ścieżki, którą mogli praktykować jedynie wybitni uczniowie.
Według Huinenga, ,,ślubowaniem natury umysłu'' (自性戒 \pinyin{zìxìng jiè}) jest ,,podstawa umysłu''%
\footnote{Podstawa umysłu (心地 \pinyin{xīndì}) została zdefiniowana w tekście \textit{Zutang ji} (祖堂集 \pinyin{Zǔtáng jí}), w biografii Nanyue Huairanga (南嶽懷讓 \pinyin{Nányuè Huáiràng}): “汝學心地法門,猶如下種。我說法要,譬彼天澤。汝緣合故,當見於道。”
`Powinieneś zrozumieć doktrynę o podstawie umysłu, która naucza, że w umyśle jak gdyby były zasiane nasiona. Gdy wyjaśnię ci podstawy Dharmy, to tak jakby deszcz spadł na tę ziemię. Ponieważ połączą się odpowiednie warunki i deszcz, będziesz w stanie zobaczyć ścieżkę.' (Tekst chiński za \textit{Zutang ji}, opublikowanym na stronie internetowej \url{www.lianhua33.com/c/c47-1.htm}, przekład polski na podstawie przypisu w \textit{The Platform sutra of the Sixth Patriarch. The Text of the Tun-huang manuscript} Philipa B. Yampolsky'ego, strona 164). Termin ten jest trudno oddać w języku polskim, gdyż znak \textit{di} (地) może oznaczać m.in. ziemię, glebę lub pole. Natomiast stosowane w języku angielskim tłumaczenie \textit{mind-ground} można zinterpretować zarówno jako ,,glebę umysłu'', jak też ,,podstawę umysłu''. Ze względów estetycznych autor zdecydował się na drugie tłumaczenie.})
wolna od błędu (心地無疑 \pinyin{xīndì wú yí}).
,,Podstawa umysłu'' wolna od przeszkadzających emocji (心地無亂 \pinyin{xīndì wú luàn}) stanowi ,,medytację natury umysłu'' (自性定 \pinyin{zìxìng dìng}), zaś wolna od niewiedzy (心地無癡 \pinyin{xīndì wú chī}) --- ,,mądrość natury umysłu'' (自性惠 \pinyin{zìxìng huì}).
Dodał też, że wyjaśnienia na temat ślubowań, medytacji i mądrości potrzebne są jedynie ludziom o niewielkich zdolnościach.
Natomiast wybitni praktykujący w ogóle nie potrzebują takich nauk, gdyż esencją jego doktryny jest urzeczywistnienie własnej natury, w której nie ma błędów, przeszkadzających emocji ani niewiedzy.
Po wysłuchaniu tych nauk, Zhicheng stał się uczniem Huinenga i postanowił pozostać w Caoqi
(Huineng i Yampolsky 2012: 164-165).

W sekcji 42. opowiedziano historię mnicha o imieniu Fada, który recytował \textit{Sutrę lotosu}%
\footnote{Sutra lotosu (skt. \textit{Saddharmapu\d{n}\d{d}arīka Sūtra}, chiń. 妙法蓮華經 \pinyin{Miàofǎ liánhuā jīng} `Sutra o białym lotosie mistycznej Dharmy', w skrócie 法華經 \pinyin{Fǎhuá jīng}) --- sutra buddyjska, jeden z najpopularniejszych i najlepiej znanych na świecie tekstów Mahajany\index{Buddyzm Mahajany}.} przez siedem lat,
lecz mimo usilnych starań nie był w stanie odnaleźć właściwej drogi.
W związku z tym udał się do Caoqi, by poprosić o wyjaśnienia  Huinenga\index{Dajian Huineng}.
Mistrz odparł, że powodem takiego stanu rzeczy były złudzenia, które Fada przez cały czas utrzymywał w umyśle.
Powiedział też, że wystarczy usunąć te błędne poglądy, a wówczas będzie mógł dogłębnie zrozumieć tekst, który z takim zapałem recytował.
Następnie poprosił, by Fada przeczytał mu \textit{Sutrę lotosu}
(Huineng i Yampolsky 2012: 165-168).

Wysłuchawszy tekstu, wyjaśnił, że wszystkie siedem jego części pełne jest parabol i przypowieści na temat przyczynowości. % (譬喻內緣).
Powiedział też, że nauki Buddy o Trzech Pojazdach%
\footnote{Trzy Pojazdy\label{Triyana} (skt. \textit{triyāna}, chiń. 三乘 \pinyin{sānchéng}) --- w buddyzmie Mahajany\index{Buddyzm Mahajany} odnosi się do trzech ścieżek praktyki duchowej, obieranych przez praktykujących o różnych skłonnościach: Ścieżki Słuchaczy (skt. \textit{śrāvakayāna}, chiń. 聲聞乘\pinyin{shēngwén chéng}), Ścieżki Pratjekabuddów (skt. \textit{pratyekabuddhayāna}, chiń. 緣覺乘 \pinyin{yuánjué chéng}) oraz Ścieżki Bodhisattwów (skt. \textit{boddhisattvayāna}, chiń. 菩薩乘 \pinyin{púsà chéng}).}
przeznaczone były wyłącznie dla przeciętnych uczniów, którzy nie byli w stanie zrozumieć, że istnieje tylko jeden pojazd nauk Buddy, i żaden inny.
By wyjaśnić te nauki, Szósty Patriarcha podał cytat \textit{Sutry lotosu}, który mówi, że wszyscy buddowie, w tym historyczny Budda Siakjamuni (nazwany w tekście ,,czczonym przez świat'', chiń. 世尊 \pinyin{shìzūn}) pojawili się na świecie z ,,jednej, bardzo ważnej przyczyny'' (大事因緣 \pinyin{dà shì yīnyuán}).
W tekście \textit{Sutry lotosu} Budda tłumaczył następnie, że przyczyną tą jest życzenie, by otworzyć oczy wszystkich czujących istot i ukazać im Dharmę.
Huineng\index{Dajian Huineng} powiedział natomiast, że prawdziwym znaczeniem tego określenia jest odrzucenie fałszywych, dualistycznych poglądów, takich jak lgnięcie do formy lub pustości.
To miało być prawdziwe znaczenie tej sutry, jedyny prawdziwy pojazd nauk Buddy; dopiero później w jej tekście miały pojawić się nauki o trzech pojazdach
Wyraził życzenie, by wszyscy ludzie, na całym świecie w swojej własnej ,,podstawie umysłu''  otworzyli się na mądrość Buddy i trzymali z daleka od mądrości zwykłych ludzi.
% Cztery bramy

W tej sekcji pojawiła się też metafora ,,obracania lotosem'' (傳法華 \pinyin{chuán fǎhuá}) i ,,bycia obracanym przez lotos'' (法華傳 \pinyin{fǎhuá chuán}).
Huineng\index{Dajian Huineng} nazwał wszystkie pożyteczne dla pożytku istot działania (praktykę Dharmy w umyśle, utrzymywanie prawidłowych poglądów, otwarcie się na mądrość Buddy) do obracania lotosem, a ich przeciwieństwa (niepraktykowanie Dharmy w umyśle, utrzymywanie błędnych poglądów, otwarcie się na mądrość zwykłych ludzi) do bycia obracanym przez lotos.
Wysłuchawszy tych nauk, Fada miał w jednej chwili rozpoznać naturę swego umysłu.
Powiedział z płaczem, że przez siedem lat swojej praktykował w niewłaściwy sposób, ale od tej chwili sytuacja się zmieni\ibid

Sekcja 43. mówi o przybyciu do Caoqi ucznia o imieniu Zhichang.
Spytał on Szóstego Patriarchy o znaczenie nauk o czterech pojazdach Dharmy (四乘法 \pinyin{sì chéng fǎ}).\label{SiChengFa}
Chociaż Budda nauczał jedynie o trzech pojazdach, Huineng\index{Dajian Huineng} mówił również o Najwyższym Pojeździe (最上乘 \pinyin{zuì shàng chéng}).
Mistrz wyjaśnił, że chociaż początkowo taki podział nauk nie istniał, został wprowadzony ze względu na indywidualne zdolności uczniów.
Następnie wytłumaczył, że widzenie, słuchanie i recytacja nauk jest Małym Pojazdem (Ścieżką Słuchaczy; patrz: przypis na str. \pageref{Triyana});
przebudzenie się na Dharmę i zrozumienie jej zasad --- Średnim Pojazdem (Ścieżką Pratjekabuddów; patrz: przypis na str. \pageref{Triyana});
Wielki Pojazd, Mahajanę\index{Buddyzm Mahajany}, stanowi praktykowanie zgodnie z Dharmą; zaś Najwyższym Pojazdem jest ,,całkowite przejście przez dziesięć tysięcy zjawisk, będąc wyposażonym w dziesięć tysięcy nauk, nieoddzielonym od rzeczy, a jedynie od ich cech, i nieosiąganie niczego w żadnym działaniu''.
Dodał też, że ponieważ ,,pojazd'' oznacza w istocie praktykę, nie są to nauki, o których należy rozmawiać, lecz je praktykować
(Huineng i Yampolsky 2012: 168-169).

W sekcji 44. opowiedziano historię Heze Shenhui\index{Heze Shenhui, 荷澤神會}, który miał przybyć do Caoqi z miasta Nanyang (南陽 \toponim{Nányáng}), w obecnej prowincji Henan\index{Prowincja Henan, 河南省}.
Shenhui zapytał Szóstego Patriarchę, czy widzi coś, gdy siedzi w medytacji.
W odpowiedzi mistrz trzykrotnie uderzył Shenhui i spytał go, czy poczuł ból.
Nowo przybyły mnich odparł: ,,Czułem ból, a także nie czułem bólu''.
Huineng\index{Dajian Huineng} wyjaśnił, że tak samo on sam, gdy siedzi w medytacji, widzi swoje błędy, a jednocześnie nie widzi błędów innych ludzi.
Na końcu powiedział jednak, że zadawanie pytań w takich dualistycznych kategoriach jest błędem, i polecił Shenhui, by ten najpierw medytował, a potem spytał go o to jeszcze raz.
\textit{Sutra platformy} podaje, że po tym spotkaniu Shenhui został uczniem Huinenga i zamieszkał w Caoqi
(Huineng i Yampolsky 2012: 169-170).

\section{Nauki przekazane dziesięciu głównym uczniom}
W sekcjach 45-47 \textit{Sutry platformy} zapisano nauki przekazane przez Szóstego Patriarchę  dziesięciu najbliższym uczniom: Fahai, Zhicheng, Fada, Zhichang, Zhitong (志通 \nazwisko{Zhìtōng}), Zhiche (志徹 \nazwisko{Zhìchè}), Zhidao (志道 \nazwisko{Zhìdào}), Fazhen (法珍 \nazwisko{Fǎzhēn}), Faru i Shenhui\index{Heze Shenhui, 荷澤神會}.

W sekcjach 45. i 46. Szósty Patriarcha wygłosił nauki dotyczące buddyjskiej teorii poznania, o trzech kategoriach (三科 \pinyin{sān kē}) i trzydziestu sześciu konfrontacjach aktywności (動三十六對 \pinyin{dòng sānshíliù duì}). % 三科法門
Przekazując te wyjaśnienia, Huineng\index{Dajian Huineng} miał ogłosić, że po jego śmierci to na nich, jako jego uczniach, spoczywać będzie obowiązek nauczania przyszłych pokoleń.
Powiedział, że kiedy ktoś zada im pytanie dotyczące Dharmy, powinni  udzielić temu komuś odpowiedzi w sposób symetryczny, za pomocą analogii i konfrontacji.
Ponieważ przeciwieństwa wynikają z siebie nawzajem, ostatecznie nie będzie już miejsca na dualistyczne kategorie myślowe i możliwe będzie wyjście poza nie
(Huineng i Yampolsky 2012: 170-172).

W sekcji 45. Szósty Patriarcha wyjaśnił określenie ,,trzy kategorie''.
Składają się na nie: pięć skandh (skt. \textit{skandha}, 五蘊 \pinyin{wǔ yùn}, w \textit{Sutrze platformy} zapisywane 五蔭 \pinyin{wǔ yìn}; czasem tłumaczone jako ,,skupiska'') oraz osiemnaście dhātu (skt., chiń. 十八界 \pinyin{shíbā jiè} `osiemnaście sfer'), w których z kolei zawiera się dwanaście podstaw zmysłowych (skt. \nohyphens{\itshape āyatana}, chiń. \pinyin{shíèr rù}).
Następnie wylicza po kolei pięć skandh, są to: forma materialna (skt. \textit{rūpa}, chiń. 色 \pinyin{sè}), uczucia (skt. \textit{vedanā}, chiń. 受 \pinyin{shòu}), percepcja (skt. \textit{sa\d{m}jñā}, chiń. 想 \pinyin{xiǎng}), formacje mentalne (skt. \textit{sa\d{m}skāra}, chiń. 行 \pinyin{xíng}) oraz świadomość (skt. \textit{vijñāna}, chiń. 識 \pinyin{shí}).
Na osiemnaście dhātu składają się: sześć zewnętrznych podstaw zmysłowych (skt. \textit{bāhya-āyatana} chiń. 六塵 `sześć nieczystości') %***
--- formy (skt. \textit{rūpa-āyatana}, chiń.色塵 \pinyin{sè chén}), słuch (skt. \textit{śabda-āyatana}, chiń. 聲塵 \pinyin{shēng chén}), zapach (skt. \textit{gandha-āyatana}, chiń. 香塵 \pinyin{xiāng chén}), smak (skt. \textit{rasa-āyatana}, chiń. 味塵 \pinyin{wèi chén}), dotyk (skt. \textit{spra\d{s}\d{t}avya-āyatana}, chiń. 觸塵 \pinyin{chù chén}) i idee (skt. \textit{dharma-āyatana}, chiń. 法塵 \pinyin{fǎ chén});
odpowiadające im sześć wewnętrznych podstaw zmysłowych (skt. \textit{adhyātma-āyatana}, chiń. 六門 \pinyin{liù mén} `sześć bram') --- oczy (skt. \textit{cak\d{s}ur-indriya-āyatana}, chiń. 眼門 \pinyin{yǎn mén}), uszy (skt. \textit{śrota-indriya-āyatana}, chiń. 耳門 \pinyin{ěr mén}), nos (skt. \textit{ghrā\d{n}a-indriya-āyatana}, chiń. 鼻門 \pinyin{bí mén}), język (skt. \textit{jihvā-indriya-āyatana}, chiń. 舌門 \pinyin{shé mén}), ciało (skt. \textit{kaya-indriya-āyatana}, chiń. 身門 \pinyin{shēn mén}) i umysł (skt. \textit{mano-indriya-āyatana}, chiń. 意門 \pinyin{yì mén});
a także powstałe z nich sześć świadomości zmysłów (skt. \textit{vijñāna}, chiń. 六識 \pinyin{liù shí}) --- widzenie (skt. \textit{cak\d{s}ur-vijñāna}, chiń. 眼識 \pinyin{yǎn shí}), słyszenie (skt. \textit{śrota--vijñāna}, chiń. 耳識 \pinyin{ěr shí}), węch (skt. \textit{ghrā\d{n}a-vijñāna}, chiń. 鼻識 \pinyin{bí shí}), smak (skt. \textit{jihvā-vijñāna}, chiń. 舌識 \pinyin{shé shí}), dotyk (skt. \textit{kaya-vijñāna}, chiń. 身識 \pinyin{shēn shí}) i myślenie (skt. \textit{mano-vijñāna}, chiń. 意識 \pinyin{yì shí}).
Huineng\index{Dajian Huineng} wyjaśnił również, że wszystkie zjawiska, w tym dwanaście podstaw zmysłowych i sześć świadomości, powstają w przestrzeni umysłu.
Myślenie wprawia w ruch świadomość, a w rezultacie obiekty postrzegania są postrzegane przez narządy zmysłów.
Ta właściwość umysłu, która daje początek wszystkim zjawiskom, nazywana jest ,,świadomością magazynującą'' (skt. \textit{ālaya-vijñāna}, chiń. 含藏識 \pinyin{hán cáng shí})\ibid

Sekcja 46. zawiera szczegółowe wyjaśnienia, dotyczące ,,trzydziestu sześciu konfrontacji aktywności''. W tym fragmencie tekstu Huineng\index{Dajian Huineng} wyliczył: pięć konfrontacji zewnętrznych zjawisk, dwanaście konfrontacji języka i cech przedmiotów oraz dziewiętnaście konfrontacji aktywności powstałych z natury umysłu.
Następnie powiedział swym uczniom, że gdy wprowadzą nauki o trzech kategoriach i trzydziestu sześciu konfrontacjach w życie, będą mogli zastosować je w kontekście wszystkich sutr i wyjść poza dualizm.
Zalecił, żeby udzielając innym tych wyjaśnień, na poziomie zewnętrznym być oddzielonym od formy, a na poziomie wewnętrznym --- oddzielonym od pustości, nie lgnąc do żadnej z nich.
Wyjaśnił też, że spośród wymienionych wcześniej konfrontacji, żadna z nich nie jest pełna sama w sobie.
Użył przy tym analogii do ciemności i światła; ciemność może być ciemnością tylko dzięki światłu, wynikają one z siebie nawzajem\ibid

W sekcji 47. zawarto kolejną informację o przekazie nauk Chan za pomocą \textit{Sutry platformy}. % Fahai sekcja 38
Według tego zapisu ten, kto nie otrzymał przekazu tekstu, nie posiadł podstaw nauk Huinenga\index{Dajian Huineng}.
Wreszcie Huineng miał powiedzieć, że zetknąć się z tą sutrą to tak, jakby osobiście usłyszeć jego nauki.
Fragment kończy się stwierdzeniem, iż dziesięciu głównych uczniów Szóstego Patriarchy kopiowało tekst i przekazywało go późniejszym pokoleniom, a ci, którzy go otrzymali, uzyskali wgląd w swoją naturę
(Huineng i Yampolsky 2012: 173-174).

\section{Śmierć Huinenga\index{Dajian Huineng} i jej następstwa}
Sekcje 48-54 \textit{Sutry platformy} opisują okoliczności śmierci Huinenga, nauki wygłoszone przez niego bezpośrednio przed odejściem, oraz przekaz nauk w dalszych pokoleniach.

Sekcja 48. podaje dokładną datę śmierci Szóstego Patriarchy jako trzeci dzień ósmego miesiąca drugiego roku ery Xiantian (先天 \pinyin{Xiāntiān}), czyli 28 sierpnia 713 r.
Wcześniej, w pierwszym roku ery Xiantian, miał wybudować pagodę w świątyni Guo'en Si (國恩寺 \pinyin{Guó'ēn sì}) w Xinzhou.
Huineng\index{Dajian Huineng} miał pożegnać się z uczniami na miesiąc przed śmiercią.
Kiedy poinformował ich o swej zbliżającej się śmierci, wszyscy mnisi, z wyjątkiem Shenhui\index{Heze Shenhui, 荷澤神會}, zaczęli płakać.
Według tekstu, Szósty Patriarcha pochwalił go wówczas, mówiąc, że mimo młodego wieku osiągnął już stan, w którym nie ma różnicy między dobrem i złem, pochwałą i naganą.
Pozostałym uczniom powiedział, że ich smutek wynika wyłącznie z niewiedzy dotyczącej życia po śmierci, gdyż w naturze umysłu nie ma przychodzenia, odchodzenia, narodzin ani śmierci.
Następnie przekazał ,,wiersz o prawdzie, fałszu, ruchu i spokoju'' (真假動淨偈 \pinyin{zhēn jiǎ dòng jìng jì}), przekonując, że za jego pomocą można osiągnąć wysokie urzeczywistnienie
(Huineng i Yampolsky 2012: 174-175).

Sekcja 49. zawierają informacje o linii przekazu nauk Chan.
Na pytanie mnicha Fahai, kto po śmierci mistrza odziedziczy szatę patriarchów i Dharmę, Szósty Patriarcha odrzekł, że Dharma została już przekazana.
Miał również przepowiedzieć, że dwadzieścia lat po jego odejściu zbiorą się niesprzyjające warunki, które zakłócą dalszy przekaz jego nauk, oraz że zjawi się wówczas człowiek, który ryzykując życie, przywróci prawdę i fałsz w buddyzmie i sprawi, że Dharma będzie się rozwijać.
Wzmianka ta odnosi się do przemowy wygłoszonej przez Shenhui\index{Heze Shenhui, 荷澤神會} w roku 732, w miejscu zwanym Huatai (滑臺 \toponim{Huátái}) w obecnej prowincji Henan\index{Prowincja Henan, 河南省}, w której skrytykował on nauki Shenxiu jako stopniową ścieżkę oraz proklamował Szkołę Południową.
Podobna przepowiednia pojawiła się w ``Cytatach Shenhui'' oraz ``Księdze przekazu lampy z okresu Jingde\index{Księga przekazu lampy z okresu Jingde, 景德傳燈錄}'', podając odpowiednio okres czterdziestu i siedemdziesięciu lat.
Szósty Patriarcha miał również ogłosić, że z jego śmiercią kończy się przekaz szaty patriarchów, zapewniając tym samym wyjaśnienie, dlaczego nie została ona zachowana dla potomności.
Następnie tekst sutry podaje pięć krótkich wierszy, które pięciu patriarchów przed nim --- Bodhidharma, Huike, Sengcan, Daoxin i Hongren \index{Daman Hongren, 大滿弘忍}--- mieli przekazać wraz z szatą i Dharmą
(Huineng\index{Dajian Huineng} i Yampolsky 2012: 176-178).

W sekcji 50. Huineng\index{Dajian Huineng} wygłosił dwa wiersze, rzekomo zainspirowane naukami Bodhidharmy.
Traktują one o właściwych i niewłaściwych kwiatach, które kwitną w podstawie umysłu.
Pięć niewłaściwych kwiatów (prawdopodobnie metafora pięciu przeszkadzających emocji --- arogancji, gniewu, głupoty, zazdrości i przywiązania) zasiewa karmiczne nasiona niewiedzy i sprawia, że wiatry karmy kontrolują umysł.
Po wysłuchaniu tych nauk, zgromadzenie mnichów rozeszło się
(Huineng i Yampolsky 2012: 178).

Sekcja 51. opisuje linię przekazu szkoły nagłego oświecenia, prowadzącej od siedmiu buddów, przez dwudziestu ośmiu patriarchów Indii i pięciu Chin, do Huinenga\index{Dajian Huineng}.
* uzupełnić o porównanie do innych opisów linii przekazu (np. z przedmowy do przekładu)
(Huineng i Yampolsky 2012: 179).

W sekcji 52. Szósty Patriarcha udziela nauk o czującej istocie i buddzie w naturze umysłu każdego człowieka, dodając, że człowiek pomieszany nie będzie w stanie zobaczyć buddy, gdyż może to uczynić tylko przebudzony.
Dodaje, że natura buddy jest obecna w umysłach wszystkich czujących istot i nigdzie indziej.
Przekazuje również ,,Wiersz o dostrzeganiu prawdziwego buddy i osiągnięciu wyzwolenia'' (見真佛解脫頌 \pinyin{jiàn zhēn fó jiětuō sòng}), według którego wszystkie istoty są buddami, a od przebudzenia dzieli je tylko niewiedza.
Mówi, że wystarczy na jedną chwilę przyjąć bezstronny sposób postrzegania, by zobaczyć innych jako oświeconych
(Huineng\index{Dajian Huineng} i Yampolsky 2012: 179-180).

W sekcji 53. Huineng\index{Dajian Huineng} przekazał ,,Wiersz o prawdziwym buddzie w naturze umysłu i osiągnięciu wyzwolenia'' (自性真佛解脫頌 \pinyin{zìxìng zhēn fó jiětuō sòng}).
W tym fragmencie Huineng podkreśla, że istotą oświecenia jest prawdziwa natura umysłu oraz właściwe poglądy, a wszelkie negatywności pochodzą z błędnych poglądów oraz Trzech Trucizn Umysłu%
\footnote{Trzy trucizny\label{SanDu} (skt. \textit{trivi\d{s}a}, chiń. 三毒 \pinyin{sān dú}) --- w buddyzmie Mahajany\index{Buddyzm Mahajany} zbiorcza nazwa przywiązania (skt. \textit{rāga}, chiń. 貪 \pinyin{tān}), niechęci (skt. \textit{dve\d{s}a}, chiń. 瞋 \pinyin{chēn}) i głupoty (skt. \textit{moha}, chiń. 癡 \textit{chī}).}.
Wreszcie Szósty Patriarcha oznajmił, że nadszedł czas pożegnania, i poprosił swych uczniów, by po jego odejściu wiedli życie tak, jakby on cały czas był przy nich, wspólnie medytując i praktykując Dharmę, oraz by nie nosili żałoby.
Powiedział, że istotą Wielkiej Drogi jest ,,pozostawać spokojnym, bez ruchu i bezruchu, narodzin i śmierci, przychodzenia i odchodzenia; bez osądzania o tym, co właściwe i błędne, bez pozostawania i odchodzenia''.
Miały to być jego ostatnie słowa, a o północy tegoż dnia zmarł
(Huineng i Yampolsky 2012: 181-182).

Sekcja 54. zawiera opis cudów, które miały mieć miejsce po odejściu Szóstego Patriarchy.
W tym fragmencie zapisano również, że w jedenastym miesiącu tego samego roku jego ciało miało zostać pochowane w Caoqi, a prefekt Wei Qu wzniósł w tym miejscu kamienną stelę ku jego czci
(Huineng\index{Dajian Huineng} i Yampolsky 2012: 182).

\section{Zapisy o przekazie \textit{Sutry platformy}}
W ostatnich trzech sekcjach \textit{Sutry platformy}, tzn. 55-57, podano informacje o dalszym przekazie tekstu.

W sekcji 55. zapisano, że sutra ta została ułożona przez mnicha Fahai, a po jego śmierci dzierżawcą przekazu stał się jego przyjaciel w Dharmie, mnich Daocan%
\footnote{Philip B. Yampolsky w swoim przekładzie podaje jego imię jako Daocan (道澯 \nazwisko{Dàocàn}), w innych źródłach zapisywane jest jako Daoji (道漈 \nazwisko{Dàojì}).}.

Sekcja 56. opisuje warunki, które musi spełnić przyszły dzierżawca nauk Chan: powinien być obdarzony ponadprzeciętną mądrością i wielkim współczuciem oraz mieć zaufanie do nauk Buddy.
Otrzymawszy ten przekaz, powinien przekazać go dalej i dołożyć starań, by w późniejszych czasach linia nie została przerwana.

Sekcja 57. zawiera informacje o autorze sutry, mnichu Fahai. Według tego zapisu, pochodził on z powiatu Quxiang (曲江縣 \toponim{Qǔjiāng xiàn}).
Po tym następuje pochwała jego zasług: był wielkim bodhisattwą, nauczał prawdziwej doktryny i praktykował zgodnie z nią.
kolejny zapis o warunkach niezbędnych do przyjęcia przekazu Dharmy: dzierżawca musi ślubować wyzwolenie wszystkich czujących istot, praktykować bez ustanku, nie poddając się w obliczu niepowodzeń i cierpienia.
Człowiek, który nie posiadł wspomnianych przymiotów, a jego uzdolnienia są niewystarczające, nie powinien dążyć do otrzymania tych nauk.
\textit{Sutra platformy} kończy się zachętą skierowaną do wszystkich praktykujących, by ze wszech miar starali się zrozumieć i urzeczywistnić nauki
(Huineng\index{Dajian Huineng} i Yampolsky 2012: 183).


\chapter{Historyczne reperkusje \textit{Sutry platformy}}
\textit{Sutra platformy} pozostaje jednym z najbardziej popularnych i wpływowych tekstów buddyzmu Chan do dnia dzisiejszego, a analizie zawartych w niej nauk i opowieści poświęcono wiele prac naukowych.

\section{Podział Chan na Szkołę Północną i Południową}
Shenhui\index{Heze Shenhui, 荷澤神會} zyskał wpływy dopiero po pojawieniu się negatywnego sentymentu do jego rywali, w następstwie jego publicznych ataków na Szkołę Północną.
W roku 720 Shenhui przebywał w miejscowości Nanyang nieopodal Luoyangu, gdzie nauczał medytacji.

\section{Dalszy podział Szkoły Południowej}
Spośród uczniów Huinenga\index{Dajian Huineng}, najbardziej wpływowi byli Qingyuan Xingsi (青原行思 \nazwisko{Qīngyuán Xíngsī}) oraz Nanyue Huairang (南嶽懷讓 \nazwisko{Nányuè Huáiràng}).
Ich spadkobiercy podzielili się na szkoły: Caodong (曹洞 \pinyin{Cáodòng zōng}) (雲門 \pinyin{Yúnmén zōng}) (法眼 \pinyin{Fǎyǎn zōng}) od Xingsi, (臨濟 \pinyin{Línjì zōng}) 溈仰 od Huairanga.
Następnie od Linji oddzieliły się linie 黃龍 楊岐. Do chwili obecnej zachowały się Linji i Caodong. (佛學課本)

\subsection{Uczniowie Nanyue Huairanga}
Nanyue Huairang narodził się w drugim roku ery Yifeng (677) w miejscowości Ankang w obecnej prowincji Shaanxi\index{Prowincja Shaanxi, 陝西省} (陝西 \toponim{Shǎnxī}).
Jego uczniem był Mazu Daoyi\index{Mazu Daoyi, 馬祖道一} (馬祖道一 \nazwisko{Mǎzǔ Dàoyī}).
Nauczał głównie w świątyniach: Shengji Si (福建建陽佛迹岭聖跡寺) w pobliżu miasta Nanping (南平 \toponim{Nánpíng}) w prowincji Fujian (福建 \toponim{Fújiàn})
还有江西抚州临川西山
Baohua Si (寶華寺) na górze Gonggongshan (龔公山) w pobliżu miasta Qianzhou (虔州), obecnie Ganzhou (贛州) w prowincji Jiangxi\index{Prowincja Jiangxi, 江西省}

Szkoła Shenxiu była popularna na północy Chin, podupadła po śmierci głównych uczniów Shenxiu: Songshan Puji i Dazhi Yifu (大智義福), by w końcu wygasnąć. % sprawdź kiedy
\if 0
McRae 1986: 3
Sutra platformy napisała historię Chan od nowa; wskazuje na to pominięcie roli Shenhui\index{Heze Shenhui, 荷澤神會}
W początkowej fazie istnienia Południowej Szkoły, była ona mało znana; sutra wyjaśnia to długim czasem, jaki Huineng\index{Dajian Huineng} spędził, ukrywając się u myśliwych
Lankavatara => Diamentówka
McRae 1986: 5
Szkoła Południowa twierdziła, że posiada nauki niedualne
Wg SS natura oświecenie, przeszkadzające emocje, cierpienie i iluzje są w istocie tym samym, co oświecenie, ale NS widziała je jako różne; wg 宗密 oznacza to, że wiele lat, lub nawet żywotów praktyki idzie na marne; wszystko, czego potrzebuje praktykujące, to całkowite odcięcie dualistyczn, ego myślenia
SS była lewicowa: uważała, że każdy powinien mieć prawo poznać Dharmę i osiągnąć oświecenie, a nie tylko ci, którzy włożyli w to wysiłek
Zongmi usystematyzował różne interpretacje Chan, Szkoła Północna była najniżej
wykładnia 宗密 NS na początku była popularna, ale potem została niemal całkowicie wyparta przez SS, bo prawowitym spadkobiercą był Huineng, a nie Shenxiu
Nie ma dobrych badań nt. NS

17 Pierwsze wzmianki o Bodhidharmie w 洛陽伽藍記, to był tekst, w którym ludzie z różnych stron świata przyjeżdżali do Luoyangu, by zachwycać się jego architekturą
(Wiki) 曇林 napisał przedmowę do 二入四行, który tradycyjnie przypisuje się Bodhidharmie, wspomina 道育 i 慧可, i że był z południowych Indii Broughton 1999, p. 53

續高僧傳:
南天竺婆羅門種 Brahmin z Południowych Indii
Przybył do Nanyue w okresie Liu Song, (a więc przed rokiem 479), nie wiadomo, gdzie zginął

Dumoulin 1963: 70
W czasie, kiedy Bodhidharma przyniósł Chan do Chin, na północy Chin był Buddhabhadra, a na południu szkoła 三论
Sanlun interesowała się Pradżniaparamitą Nagardżuny

79: 弘忍 rozwijał własne metody medytacji w oparciu o sutry Avatamsaka (華嚴)

Dumoulin 81
Pierwsza schizma w tradycji Chan
SS wygrała walkę o dominację, bo NS nie rozwijało się po śmierci uczniów Shenxiu, a SS publikowało wiele koanów i kronik
w 700 Shenxiu polecił cesarzowi zaprosić Huinenga\index{Dajian Huineng} do stolicy
Shenxiu cieszył się szacunkiem dworu, główni uczniowie 普寂、義福
83
Według niektórych podań Shenhui\index{Heze Shenhui, 荷澤神會} był przez kilka lat uczniem Shenxiu, ale jest to mało prawdopodobne
Shenhui na pewno przebywał razem z Huinengiem

中國禪宗史 115:
W sutrze jest napisane, że Huineng\index{Dajian Huineng} 先天二年八月三日滅度 oraz 春秋七十有六, ale są stele, które mówią co innego i przesuwają datę śmierci o 3 lata
《曹溪大師別傳》
117: Z legend 神會, zapisanych przez 王維 wynika, że Huineng ukrywał się przez 16 lat, dostał 衣法 od 弘忍 na jego łożu śmierci, i że po 16 latach ukrycia spotkał się z 印宗 i zaczął nauczać
神會語錄:能禪師過嶺至韶州,居曹溪山,來往四十年。 nie ma 印宗,隱遁
歷代法寶記:ukrycie, 印宗
Teoria ukrycia wzięła się z niespójności lat
\fi
