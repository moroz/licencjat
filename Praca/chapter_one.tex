\chapter{Biografie Huinenga}
Życie Szóstego Patriarchy Huineng jest owiane tajemnicą. Jego imię pojawia się bowiem w kronice pt. \textit{Księga przekazu lampy z okresu Jingde} (景德傳燈錄 \pinyin{Jǐngdé chuán dēng lù}) jako jednego z dziesięciu głównych uczniów piątego patriarchy Hongren, z tekstu nie wynika jednak, by był postacią szczególnie ważną dla rozwoju całej szkoły Chan. W tekście tym wspomniano, że Huineng żył i nauczał w miejscowości Caoxi (曹溪 \toponim{Cáoxī}, również: \toponim{Cáoqī}). Imię Huineng pojawia się również w pewnym tekście z grot Dunhuang, upamiętniającym Piątego Patriarchę Hongren, jednak tekst ów nie mówi nic o przypisywanych Huinengowi doktrynach. Kanoniczna biografia Huinenga oparta jest na przypisywanej mu \textit{Sutrze Platformy}. Szósty Patriarcha jest w tym tekście przedstawiany jako ubogi, niepiśmienny człowiek świecki z południa Chin (McRae 2004: 68).

\section{Biografia Huinenga według \textit{Sutry Platformy}}
Jak podaje tekst \textit{Sutry Platformy}, Huineng urodził się w miejscowości Xinxing w regionie Nanhai (南海新興 \toponim{Nánhǎi Xīnxīng}, obecnie prowincja Guangdong). Za ramy czasowe jego życia przyjmuje się lata 638-713.

Jak podaje tekst, jego ojciec był urzędnikiem z regionu Fanyang (范陽 \toponim{Fànyáng}), obecnie miasto Zhuozhou (涿州 \toponim{Zhuōzhōu}) w prowincji Hebei), lecz został odwołany ze stanowiska i skazany na banicję. W związku z tym musiał przenieść się z całą rodziną do Xinxing, gdzie niedługo później zmarł. Po jego śmierci, Huineng trudnił się zbieraniem i sprzedażą drewna na opał.

% ===============
% TAJEMNICZY MĘŻCZYZNA I PODRÓŻ
% ===============
Pewnego dnia, gdy dwudziestodwuletni Huineng sprzedawał drewno na targowisku, pewien klient zażyczył sobie, żeby drewno zostało przyniesione do jego sklepu. Huineng dostarczył drewno i dostał za nie pieniądze, a kiedy wyszedł ze sklepu, spotkał mężczyznę, który recytował na ulicy Sutrę Diamentową (金剛經 \pinyin{Jīngāng jīng}, skt. \textit{Vajracchedikā Prajñāpāramitā Sūtra}). Usłyszawszy ów tekst, Huineng uzyskał wgląd w naturę swego umysłu i osiągnął oświecenie. Następnie spytał tajemniczego mężczyznę, skąd przybył. Ten odpowiedział, że przybył z klasztoru Dongshan (東山寺 \pinyin{Dōngshān sì}) na górze Fengmushan (憑墓山 \toponim{Féngmù shān}) w powiecie Huangmei (黃梅懸 \toponim{Huángméi xiàn}) w Qizhou (蘄州 \toponim{Qízhōu}), którego opatem był Piąty Patriarcha, Hong Ren (弘忍 \pinyin{Hóngrěn}), i gdzie przebywało około tysiąca mnichów. Patriarcha miał zalecić mnichom, aby recytowali ową sutrę, ponieważ dzięki tej praktyce można szybko osiągnąć oświecenie (Huineng i Hsüan Hua 1977: bez nru strony; Huineng i Yampolsky 2012: 127).

Według tej biografii, niedługo po spotkaniu tajemniczego mężczyzny, Huineng spotkał kogoś, kto poradził mu udać się do klasztoru Dongshan, aby poprosić Patriarchę o nauki, i dał mu pieniądze, aby mógł zaaranżować opiekę dla swej matki.

Kiedy Huineng przybył do klasztoru, spytano go, skąd przybył i w jakiej sprawie przybył do patriarchy. Odparł, że pochodzi z Kantonu i przyszedł oddać cześć patriarsze, i że nie prosi o nic prócz Dharmy. Mnisi stwierdzili wówczas, że Huineng, jako \textit{geliao} (獦獠 \pinyin{géliáo}, `barbarzyńca'), niegodny jest otrzymania nauk. Obszar obecnego Kantonu był wówczas zamieszkany przez niechińskie ludy, mówiące własnymi językami, posiadające własną, niechińską kulturę i nieżyjące zgodnie z naukami Buddy --- mieszkańcy południa polowali bowiem i jedli mięso. Nie do pomyślenia było dla mnichów, by człowiek z południa mógł otrzymać nauki od Patriarchy i osiągnąć oświecenie. Huineng odparł wtedy, że ludzie dzielą się na tych z południa i tych z północy, ale takie podziały nie mają wpływu na ich naturę buddy. Patriarcha uznał, że Huineng dobrze rozumiał nauki Buddy, lecz w obawie, że inni uczniowie mogliby zrobić mu krzywdę, kazał mu iść pracować w stajni. Tam, przez następne osiem miesięcy, Huineng rąbał drewno i młócił zboże (Huineng, Wong i Humphreys 1998: rozdział 1; Huineng, Schlütter i Teiser 2012: 27).

% ===============
% KONKURS POEZJI
% ===============
Według biografii zawartej w \textit{Sutrze Platformy}, pewnego dnia Patriarcha Hongren zwołał zebranie wszystkich uczniów i ogłosił: ,,Uwarunkowana egzystencja jest kwestią doniosłą. Dzień po dniu zasiewacie tylko nasiona ponownego odrodzenia, zamiast starać się wyzwolić z oceanu samsary.\footnote{[Huineng i Yampolsky 2012] podaje w tym miejscu: ,,Całymi dniami składacie ofiary i szukacie tylko pola błogosławieństw, (\ldots)''. Idea ,,pola błogosławieństw'' (福田 \pinyin{fútián}) jest związana z buddyjską koncepcją karmy jako prawa przyczyny i skutku. Oznacza stan, w którym dana osoba zgromadziła bardzo wiele dobrej karmy w rezultacie praktyki szczodrości (skt. \textit{dānā}), pierwszej z tzw. Sześciu Paramit lub Sześciu Wyzwalających Działań (pozostałe pięć to właściwe działanie, cierpliwość, radosny wysiłek, medytacja i mądrość). Słowo \textit{dānā} pojawia się m.in. w \textit{Dānādhikāramahāyānasūtra} (佛说布施经 \pinyin{Fóshuō bùshī jīng}): “若求勝妙福報而行施時,慈心不殺離諸嫉妒,正見相應遠於不善,堅持禁戒親近善友,閉惡趣門開生天路,自利利他其心平等。若如是施,是真布施,是大福田。” Piąty Patriarcha Hongren krytykował tu swoich uczniów, ponieważ praktyka szczodrości jest wprawdzie w buddyzmie postrzegana jako pozytywne działanie, jednak nie wystarcza ona do osiągnięcia ostatecznego wyzwolenia z samsary. (Anonim, 2007; Nydahl, 2010)} Te działania w niczym wam nie pomogą, jeżeli esencja waszego umysłu jest przysłoniona. Szukajcie pradżni (mądrości) w swoim umyśle i napiszcie wiersz na ten temat. Ten z was, który rozpozna esencję umysłu, otrzyma ode mnie szatę Patriarchy i przekaz nauk. (\ldots) Człowiek, który urzeczywistnił esencję umysłu, potrafi mówić o niej od razu, kiedy tylko zostanie o nią zapytany; nigdy też nie jest w stanie jej utracić, nawet podczas bitwy.'' (Huineng, Wong i Humphreys 1998: rozdział 1).

Mnisi stwierdzili zgodnie: „Nie ma sensu oczyszczać umysłu i zadawać sobie trudu układania wiersza dla patriarchy. Shenxiu (神秀 \pinyin{Shénxiù}), przewodniczący kongregacji, jest naszym nauczycielem. Kiedy on zostanie patriarchą, możemy liczyć na jego wsparcie. Dlatego nie będziemy układać wierszy.” Żaden z nich nie podjął się więc tego zadania (Huineng i Yampolsky 2012: 127).

Shenxiu zaś dręczyły wątpliwości. Z jednej strony nie uważał, żeby jego zrozumienie nauk i urzeczywistnienie były wystarczające do przyjęcia stanowiska patriarchy, z drugiej zaś pragnął otrzymać przekaz Dharmy. Przyjęcie Dharmy dla pożytku istot byłoby bowiem pożądane i chwalebne, jednak dążenie do objęcia stanowiska patriarchy byłoby niewłaściwe. Wychodził z założenia, że jeżeli Piąty Patriarcha Hongren uzna, że jego urzeczywistnienie natury umysłu jest niewystarczające, to będzie musiał porzucić dążenia do zostania dzierżawcą linii przekazu. Wreszcie skomponował wiersz i wymknął się w nocy ze swojej celi, by napisać go na ścianie:

\vspace*{6pt}
\begin{minipage}[t]{0.4\textwidth}
\begin{verse}
身是菩提樹\\
心如明鏡臺\\
時時勤佛拭\\
莫使有塵埃
\end{verse}
\end{minipage}
\begin{minipage}[t]{0.6\textwidth}
\itshape
\begin{verse}
Ciało jest drzewem Bodhi,\\
Umysł --- jasną lustrzaną podstawą.\\
Czyść ją stale i gorliwie,\\
Nie pozwalając aby przylgnął kurz.\fnm
\end{verse}
\end{minipage}
\vspace*{6pt}
\footnotetext{Niniejszy wiersz, jak również następny, został przytoczony w języku chińskim za \textit{Chinese Electronic Tripitaka} T48, no. 2007, w języku polskim za tekstem \textit{Sutry Szóstego Patriarchy Zen} nieznanego tłumacza.}

O poranku, Piąty Patriarcha Hongren ujrzał wiersz napisany przez Shenxiu na ścianie i uznał, że wiersz ten mógł przynieść ludziom wiele pożytku. Zwołał całe zgromadzenie i nakazał mnichom recytować go. Patriarcha spytał Shenxiu, czy to on jest autorem wiersza, oznaczałoby to bowiem, że jest on właściwym spadkobiercą Dharmy i jego następcą na stanowisku patriarchy. Powiedział też jednak, że wiersz nie wskazuje na to, aby Shenxiu rozpoznał już naturę swojego umysłu. Wiersz nadawał się do recytacji przez zwykłych ludzi, jednak takie niepełne zrozumienie nie wystarczało do rozpoznania prawdziwej natury umysłu. Powiedział, że przekaże mu nauki i szatę patriarchy, jeżeli temu w ciągu dwóch dni uda się osiągnąć ostateczne urzeczywistnienie. Shenxiu rozmyślał przez wiele dni, ale nie udało mu się skomponować nic lepszego. (Huineng i Yampolsky 2012: 131).

Pewnego dnia młody mnich-akolita przechodził koło stajni, w której pracował akurat Huineng, recytując wiersz Shenxiu. Huineng zrozumiał, że autor wiersza nie rozpoznał jeszcze natury swojego umysłu. Spytał mnicha, co ten recytował. Mnich odparł, że Patriarcha Hongren kazał wszystkim swoim uczniom skomponować wiersz i przynieść mu go, aby określić, kto otrzyma przekaz Dharmy i zostanie patriarchą. Wiersz o pustości, który właśnie recytował, został skomponowany przez mnicha o imieniu Shenxiu. Piąty Patriarcha nakazał wszystkim swoim uczniom recytować go, mówiąc, że ci, którzy go urzeczywistnią, zobaczą swoją prawdziwą naturę, a ci, którzy praktykują zgodnie z nim, osiągną wyzwolenie.

Huineng poprosił mnicha, aby ten zaprowadził go do miejsca, gdzie na ścianie wymalowany był wiersz Shenxiu. Ponieważ nie umiał czytać, poprosił kogoś, by przeczytał mu ten wiersz na głos. Usłyszawszy go, osiągnął ostateczne urzeczywistnienie. Huineng ułożył własny wiersz i poprosił kogoś o napisanie go na ścianie (Huineng i Yampolsky 2012: 131).

%\vspace*{6pt}
\begin{minipage}[t]{0.4\textwidth}
\begin{verse}
菩提本無樹\\
明鏡亦無臺\\
佛性常清淨\\
何處有塵埃
\end{verse}
\end{minipage}
\begin{minipage}[t]{0.6\textwidth}
\itshape
\begin{verse}
Sama istota Bodhi nie ma drzewa,\\
Nie ma też jasnej lustrzanej podstawy.\\
W rzeczywistości nie ma niczego,\\
Cóż miałoby przyciągać jakikolwiek kurz?
\end{verse}
\end{minipage}
\vspace*{6pt}

Mnisi ze zgromadzenia byli pod wielkim wrażeniem wiersza Huinenga, a Piąty Patriarcha Hongren stwierdził na jego podstawie, że Huineng miał już wówczas ponadprzeciętne zrozumienie natury zjawisk, ale dla bezpieczeństwa Huinenga oznajmił zgromadzeniu, że wciąż nie było to pełne urzeczywistnienie (Huineng i Yampolsky 2012: 132).

% ===============
% PRZEKAZ DHARMY I WYGNANIE
% ===============
Patriarcha przywołał go do siebie w nocy i udzielił mu wyjaśnień do Sutry Diamentowej, dzięki której Huineng natychmiast rozpoznał naturę umysłu. Hongren przekazał mu również nauki o spontanicznym oświeceniu oraz szatę, insygnium patriarchatu. Przekazał mu również ostatnie pouczenia: ,,Mianuję cię Szóstym Patriarchą. Szata jest tego dowodem i jest przekazywana z pokolenia na pokolenie. Moja Dharma musi być przekazywana z umysłu na umysł. Spraw, by ludzie rozpoznali swoją prawdziwą naturę. (\ldots) Od czasów starożytnych przekaz Dharmy był równie słaby, jak zwisający sznurek. Jeżeli pozostaniesz tutaj, inni ludzie zrobią ci krzywdę. Musisz więc niezwłocznie odejść.'' (Huineng i Yampolsky 2012: 133).

Huineng udał się na południe. Jego śladem podążyło kilkuset ludzi, pragnących go zabić i siłą odebrać od niego szatę i Dharmę. Po dwóch miesiącach miał dotrzeć do miejsca zwanego \textit{Dayu ling}\fnm (大庾嶺 \toponim{Dàyǔ líng}). Jest to pasmo górskie, znajdujące się pomiędzy południowym wschodem prowincji Jiangxi a prowincją Guangzhou. Tam doścignął go mnich imieniem Huiming (惠明 \pinyin{Huìmíng}***) lub Huishun (惠順 \pinyin{Huìshùn}), były generał, człowiek szorstki i porywczy. Huiming groził Huinengowi, który bez wahania oddał mu szatę, lecz Huiming nie chciał jej przyjąć, mówiąc, że przybył wyłącznie po to, by otrzymać przekaz Dharmy. Huineng miał przekazać mu Dharmę na szczycie góry, a gdy Huiming usłyszał nauki, natychmiast osiągnął oświecenie. Następnie Huineng polecił Huimingowi udać się na północ i nauczać tamtejszych ludzi (Huineng i Yampolsky 2012: 134).
%
\footnotetext{W tekście sutry w \textit{Chinese Electronic Tripitaka} w tym miejscu podana jest nazwa 大庚嶺 \pinyin{Dageng ling}. Jest to najprawdopodobniej błąd w tekście. ***}

 W związku z prześladowaniami, Huineng schronił się w miejscu zwanym Caoxi (曹溪, także: Caoqi), gdzie przez piętnaście lat ukrywał się wśród prostego ludu --- myśliwych. Dopiero potem opuścił miejsce odosobnienia i zaczął nauczać Dharmy (Huineng, Wong i Humphreys 1998: rozdział 5).
