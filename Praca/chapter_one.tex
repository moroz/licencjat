\chapter{Biografie Huinenga}
Życie Szóstego Patriarchy jest owiane tajemnicą. Jego imię pojawia się w kronice pt. \textit{Jingde chuandeng lu} (景德傳燈錄 \pinyin{Jǐngdé chuándēng lù}, `Księga przekazu lampy z okresu Jingde') jako jednego z dziesięciu głównych uczniów Piątego Patriarchy Hongren. Z owego tekstu nie wynika jednak, by był postacią szczególnie ważną dla rozwoju całej szkoły Chan, wspomniano w nim natomiast, że Huineng żył i nauczał w miejscowości Caoqi (曹溪 \toponim{Cáoqī}, również: \toponim{Cáoxī}).

Imię Huineng pojawia się również w pewnym tekście z grot Dunhuang (敦煌石窟 \pinyin{Dūnhuáng shíkū}), nazwanych od miasta Dunhuang (敦煌 \toponim{Dūnhuáng}) w prowincji Gansu\prowincja{甘肅 \toponim{Gānsù}}, upamiętniającym Piątego Patriarchę, Hongrena (大滿弘忍 \nazwisko{Dàmǎn Hóngrěn}), jednak tekst ów nie mówi nic o przypisywanych Huinengowi doktrynach. Kanoniczna biografia Huinenga oparta jest na przypisywanej mu \textit{Sutrze Platformy}. Pierwszą osobą, która przedstawiła Huinenga jako świętego, był Shenhui (菏澤神會 \nazwisko{Hézé Shénhuì}, 684-758). Biografia Szóstego Patriarchy w takiej wersji, jak opisana w tekście \textit{Sutry Platformy} z Dunhuang, jest najprawdopodobniej uzupełnioną i zmienioną wersją jego opowieści. (McRae 2004: 68).

\section{Biografia Huinenga według \textit{Sutry Platformy} w wersji z Dunhuang}
Jak podaje \textit{Sutra Platformy}, Huineng urodził się w miejscowości Xinxing (新興 \toponim{Xīnxīng}) w regionie Nanhai (南海 \toponim{Nánhǎi}, obecnie prowincja Guangdong). Za ramy czasowe jego życia przyjmuje się lata 638-713. Szósty Patriarcha jest w tym tekście przedstawiany jako ubogi, niepiśmienny człowiek świecki z południa Chin.

Jak podaje tekst, ojciec Huinenga był urzędnikiem z regionu Fanyang (范陽 \toponim{Fànyáng}), obecnie miasto Zhuozhou (涿州 \toponim{Zhuōzhōu}) w prowincji Hebei\prowincja{河北 \toponim{Héběi}}, lecz został odwołany ze stanowiska i skazany na banicję. W związku z tym musiał przenieść się z całą rodziną do Xinxing, gdzie niedługo później zmarł. Po jego śmierci Huineng trudnił się zbieraniem i sprzedażą drewna na opał.

% ===============
% TAJEMNICZY MĘŻCZYZNA I PODRÓŻ
% ===============
Pewnego dnia, gdy dwudziestodwuletni Huineng sprzedawał drewno na targowisku, pewien klient zamówił od niego opał z dostawą do jego sklepu. Huineng dostarczył towar i otrzymał swoją zapłatę, a kiedy wyszedł na zewnątrz, spotkał człowieka, który recytował na ulicy Sutrę Diamentową (金剛經 \pinyin{Jīngāng jīng}, skt. \textit{Vajracchedikā Prajñāpāramitā Sūtra}). Usłyszawszy ów tekst, Huineng uzyskał wgląd w naturę swego umysłu i osiągnął oświecenie. Następnie spytał tajemniczego mężczyznę, skąd pochodził. Ten odpowiedział, że przybył z klasztoru Dongshan (東山寺 \pinyin{Dōngshān sì}) na górze Fengmushan (憑墓山 \toponim{Féngmù shān}) w powiecie Huangmei (黃梅懸 \toponim{Huángméi xiàn}) w Qizhou (蘄州 \toponim{Qízhōu}), którego opatem był Piąty Patriarcha, Hongren, i gdzie przebywało około tysiąca mnichów. Patriarcha miał zalecić mnichom, aby recytowali ową sutrę, ponieważ dzięki tej praktyce można szybko osiągnąć oświecenie. Wkrótce potem Huineng spotkał kogoś, kto poradził mu udać się do klasztoru Dongshan, aby poprosić Patriarchę o nauki, i dał mu pieniądze, aby mógł zaaranżować opiekę dla swej matki. (Huineng i Yampolsky 2012: 127).

Kiedy Huineng dotarł do klasztoru, Piąty Patriarcha Hongren zapytał go, skąd i w jakiej sprawie przyszedł do patriarchy. Huineng odparł, że pochodzi z Kantonu\footnote{Nazwa ,,Kanton'' w języku polskim odnosi się zarówno do prowincji Guangdong\prowincja{廣東 \toponim{Guǎngdōng}}, jak i do jej stolicy --- miasta Guangzhou (廣州 \toponim{Guǎngzhōu}).} i przyszedł oddać cześć patriarsze, oraz że nie prosi o nic prócz Dharmy. Patriarcha stwierdził wówczas, że Huineng, jako \textit{geliao} (獦獠 \pinyin{géliáo}, `barbarzyńca'), niegodny jest otrzymania nauk. Obszar obecnego Kantonu był wówczas zamieszkany przez niechińskie ludy, mówiące własnymi językami, posiadające własną, niechińską kulturę i nieżyjące zgodnie z naukami Buddy --- mieszkańcy południa polowali bowiem i jedli mięso. Dla wielu ówczesnych buddystów nie do pomyślenia było, by człowiek z południa mógł otrzymać nauki od Patriarchy i osiągnąć oświecenie. Huineng odparł wtedy, że ludzie dzielą się na tych z południa i tych z północy, ale takie kategorie nie mają wpływu na ich naturę buddy. Patriarcha uznał, że Huineng dobrze rozumiał nauki Buddy, lecz w obawie, że inni uczniowie mogliby zrobić mu krzywdę, kazał mu iść pracować w stajni. Przez następne osiem miesięcy Huineng rąbał drewno i młócił zboże (Huineng i Yampolsky: 127-128; Huineng, Schlütter i Teiser 2012: 27).

% ===============
% KONKURS POEZJI
% ===============
Pewnego dnia Patriarcha Hongren zwołał zebranie wszystkich uczniów i ogłosił:
,,Dla ludzi w tym świecie narodziny i śmierć są doniosłymi kwestiami. Całymi dniami składacie podarowania i poszukujecie tylko pola błogosławieństw, ale nie staracie się wyzwolić z pełnego goryczy oceanu uwarunkowanej egzystencji\footnotetext{%
Idea ,,pola błogosławieństw'' (福田 \pinyin{fútián}, skt. \textit{pu\d{n}yak\d{s}etra}) jest związana z buddyjską koncepcją karmy jako prawa przyczyny i skutku. Oznacza stan, w którym dana osoba zgromadziła bardzo wiele dobrej karmy w rezultacie praktyki szczodrości (skt. \textit{dānā}), pierwszej z tzw. Sześciu Paramit lub Sześciu Wyzwalających Działań (pozostałe pięć to właściwe działanie, cierpliwość, radosny wysiłek, medytacja i mądrość).
Słowo \textit{dānā} (布施 \pinyin{bùshī}) pojawia się m.in. w \textit{Dānādhikāramahāyānasūtra} (佛说布施经 \pinyin{Fóshuō bùshī jīng}).
Piąty Patriarcha Hongren krytykował tu swoich uczniów, ponieważ praktyka szczodrości jest wprawdzie w buddyzmie postrzegana jako pozytywne działanie, jednak nie wystarcza ona do osiągnięcia ostatecznego wyzwolenia z samsary (Anonim 2007; Nydahl 2010).}.
%: “若求勝妙福報而行施時,慈心不殺離諸嫉妒,正見相應遠於不善,堅持禁戒親近善友,閉惡趣門開生天路,自利利他其心平等。
%: “若求勝妙福報而行施時,慈心不殺離諸嫉妒,正見相應遠於不善,堅持禁戒親近善友,閉惡趣門開生天路,自利利他其心平等。
%若如是施,是真布施,是大福田。''
Wasze własne ego stoi na drodze do błogosławieństw. Jak w takiej sytuacji możecie osiągnąć wyzwolenie? Powróćcie teraz do swoich cel i spójrzcie w swój umysł. Ludzie mądrzy samoistnie pojmą prawdziwą naturę \textit{pradżni}\fnm. Niech każdy z was napisze wiersz i przyniesie mi go. Przeczytam każdy z nich, a jeżeli jest wśród was ktoś, kto rozpoznał swoją prawdziwą naturę, przekażę mu swoją szatę i Dharmę, i uczynię go Szóstym Patriarchą. Spieszcie się!''
(Huineng i Yampolsky 2012: 128).
\footnotetext{Pradżnia (skt. \textit{prajñā}, w języku chińskim nazywana 慧 \pinyin{huì}, 智 \pinyin{zhì} lub 智慧 \pinyin{zhìhuì} --- wszystkie trzy terminy oznaczają `mądrość' --- lub fonetycznie 般若 \pinyin{bōrě}), to, obok współczucia (悲 \pinyin{bēi} `litość', skt. \textit{karu\d{n}ā}), jedna z dwóch najważniejszych cnót buddyzmu Mahajany. Termin ten można rozumieć na wiele sposobów, zależnie od tradycji, praktykowanej ścieżki i metody interpretacji. Tu odnosi się do prawidłowego, ponadintelektualnego zrozumienia prawdziwej natury zjawisk.}
% Pradżniaparamita! Tekst należy do tego trendu

Mnisi stwierdzili zgodnie: ,,Nie ma sensu oczyszczać umysłu i zadawać sobie trudu układania wiersza dla patriarchy. Shenxiu (神秀 \nazwisko{Shénxiù}), przewodniczący kongregacji, jest naszym nauczycielem. Kiedy on zostanie patriarchą, możemy liczyć na jego wsparcie. Dlatego nie będziemy układać wierszy.'' Żaden z nich nie podjął się więc tego zadania (Huineng i Yampolsky 2012: 127).

Shenxiu zaś dręczyły wątpliwości. Z jednej strony nie uważał się za godnego przyjęcia stanowiska patriarchy, z drugiej zaś pragnął otrzymać przekaz nauk. Otrzymanie przekazu Dharmy dla pożytku istot byłoby bowiem pożądane i chwalebne, jednak dążenie do zostania patriarchą byłoby niewłaściwe. Wychodził z założenia, że jeżeli jego mistrz uzna, że jego urzeczywistnienie natury umysłu jest niewystarczające, to będzie musiał pogodzić się z faktem, że kto inny zostanie dzierżawcą linii przekazu. Wreszcie skomponował wiersz i wymknął się w nocy ze swojej celi, by napisać go na ścianie, w miejscu, gdzie miały zostać namalowane sceny z sutry \textit{La\.nkāvatāra} (楞伽經 \pinyin{Léngqié jīng}) (McRae 2004: 62):

\vspace*{6pt}
\begin{minipage}[t]{0.4\textwidth}
\begin{verse}
身是菩提樹\\
心如明鏡臺\\
時時勤佛拭\\
莫使有塵埃
\end{verse}
\end{minipage}
\begin{minipage}[t]{0.6\textwidth}
\itshape
\begin{verse}
Ciało jest drzewem Bodhi,\\
Umysł --- jasną lustrzaną podstawą.\\
Czyść ją stale i gorliwie,\\
Nie pozwalając aby przylgnął kurz.\fnm
\end{verse}
\end{minipage}
\label{ShenxiuVerse}
\vspace*{6pt}
\footnotetext{Niniejszy wiersz, jak również następny, został przytoczony w języku chińskim za \textit{Chinese Electronic Tripitaka} T48, no. 2007, w języku polskim za tekstem \textit{Sutry Szóstego Patriarchy Zen} nieznanego tłumacza, zamieszczonym w serwisie mahajana.net.}

O poranku, Piąty Patriarcha Hongren ujrzał wiersz napisany przez Shenxiu na ścianie i uznał, że wiersz ten mógł przynieść ludziom wiele pożytku. Zapłacił wówczas malarzowi, któremu zlecił namalowanie scen z sutry \textit{La\.nkāvatāra}, i odwołał zamówienie. Zwołał całe zgromadzenie i nakazał mnichom recytować wiersz. Patriarcha spytał Shenxiu, czy to on jest jego autorem, oznaczałoby to bowiem, że jest on właściwym spadkobiercą Dharmy i jego następcą na stanowisku patriarchy. Powiedział też jednak, że wiersz nie wskazuje na to, aby Shenxiu rozpoznał już naturę swojego umysłu. Wiersz nadawał się do recytacji przez zwykłych ludzi i dawał gwarancję, że praktykujący nie upadnie do niższych sfer egzystencji\glosref{Niższe sfery egzystencji}{glosLowerRealms}, jednak takie niepełne zrozumienie nie wystarczało do rozpoznania prawdziwej natury umysłu. Powiedział, że przekaże mu nauki i szatę patriarchy, jeżeli w ciągu dwóch dni uda mu się osiągnąć ostateczne urzeczywistnienie. Shenxiu rozmyślał przez wiele dni, ale nie udało mu się skomponować nic lepszego. (Huineng i Yampolsky 2012: 131).

Pewnego dnia młody mnich-akolita recytował wiersz Shenxiu, przechodząc koło stajni, w której pracował Huineng. Szósty Patriarcha zrozumiał, że autor wiersza nie rozpoznał jeszcze natury swojego umysłu. Spytał więc akolity, co właśnie recytował. Mnich odparł, że wiersz o pustości, który powtarzał, został skomponowany przez mnicha o imieniu Shenxiu, wspomniał również o zaleceniach Piątego Patriarchy, dotyczących tego wiersza.

Huineng poprosił mnicha, aby ten zaprowadził go do miejsca, gdzie na ścianie wymalowany był wiersz Shenxiu. Ponieważ nie umiał czytać, poprosił kogoś, by przeczytał mu te słowa na głos. Usłyszawszy je, Szósty Patriarcha osiągnął ostateczne urzeczywistnienie. Następnie ułożył własny wiersz i poprosił kogoś o napisanie go na ścianie (Huineng i Yampolsky 2012: 131).

\begin{minipage}[t]{0.4\textwidth}
\begin{verse}
菩提本無樹\\
明鏡亦無臺\\
佛性常清淨\\
何處有塵埃
\end{verse}
\end{minipage}
\begin{minipage}[t]{0.6\textwidth}
\itshape
\begin{verse}
Sama istota Bodhi nie ma drzewa,\\
Nie ma też jasnej lustrzanej podstawy.\\
W rzeczywistości nie ma niczego,\\
Cóż miałoby przyciągać jakikolwiek kurz?
\end{verse}
\end{minipage}
\label{HuinengVerse}
\vspace*{6pt}

Mnisi ze zgromadzenia byli pod wielkim wrażeniem wiersza Huinenga, a Piąty Patriarcha Hongren stwierdził na jego podstawie, że Huineng miał już wówczas ponadprzeciętne zrozumienie natury zjawisk, ale dla bezpieczeństwa Huinenga oznajmił zgromadzeniu, że wciąż nie było to pełne urzeczywistnienie (Huineng i Yampolsky 2012: 132).

% ===============
% PRZEKAZ DHARMY I WYGNANIE
% ===============
Patriarcha przywołał go do siebie w nocy i udzielił mu wyjaśnień do \textit{Sutry Diamentowej}, dzięki której Huineng natychmiast rozpoznał naturę umysłu. Hongren przekazał mu również nauki o spontanicznym oświeceniu oraz szatę, insygnium patriarchatu. Przekazał mu również ostatnie pouczenia: ,,Mianuję cię Szóstym Patriarchą. Szata jest tego dowodem i jest przekazywana z pokolenia na pokolenie. Moja Dharma musi być przekazywana z umysłu na umysł. Spraw, by ludzie rozpoznali swoją prawdziwą naturę. (\ldots) Od czasów starożytnych przekaz Dharmy był równie słaby, jak zwisający sznurek. Jeżeli pozostaniesz tutaj, inni ludzie zrobią ci krzywdę. Musisz więc niezwłocznie odejść.'' (Huineng i Yampolsky 2012: 133).

Huineng udał się na południe. Jego śladem podążyło kilkuset ludzi, pragnących go zabić i siłą odebrać od niego szatę i Dharmę. Po dwóch miesiącach miał dotrzeć do miejsca zwanego \textit{Dayu ling} (大庾嶺 \toponim{Dàyǔ líng}). Jest to pasmo górskie, znajdujące się pomiędzy południowym wschodem prowincji Jiangxi \prowincja{江西 \toponim{Jiāngxī}} a prowincją Guangdong \prowincja{廣東 \toponim{Guǎngdōng}}. Tam doścignął go mnich imieniem Huiming (惠明 \pinyin{Huìmíng}) lub Huishun (惠順 \pinyin{Huìshùn}), były generał, człowiek szorstki i porywczy. Huiming groził Huinengowi, który bez wahania oddał mu szatę, lecz Huiming nie chciał jej przyjąć, mówiąc, że przybył wyłącznie po to, by otrzymać przekaz Dharmy. Huineng miał przekazać mu Dharmę na szczycie góry, a gdy Huiming usłyszał nauki, natychmiast osiągnął oświecenie. Następnie Huineng polecił Huimingowi udać się na północ i nauczać tamtejszych ludzi (Huineng i Yampolsky 2012: 134).

W związku z prześladowaniami, Huineng schronił się w Caoqi, gdzie przez piętnaście lat ukrywał się wśród myśliwych. Dopiero potem opuścił miejsce odosobnienia i zaczął nauczać Dharmy (Huineng, Wong i Humphreys 1998: rozdział 5).

\textit{Sutra Platformy} podaje, że Huineng przebywał w Caoqi w sumie przez 40 lat, nauczając ludzi z Shaozhou i Kantonu w oparciu o \textit{Sutrę Diamentową}, a jako symbolu przekazu nauk używał \textit{Sutry Platformy}. Huineng miał wiele tysięcy uczniów, z których dziesięciu zostało mistrzami, mało znanymi poza regionami, w których nauczali. W 712 roku powrócił do Xinzhou, miejsca swych narodzin, a w 713 roku zmarł w wieku 76 lat. Tuż przed jego śmiercią, jego uczeń Fahai, uważany za autora sutry, zapytał, kto będzie jego następcą i co stanie się z szatą patriarchy. Huineng odrzekł, że przekaz szaty dobiegł końca, i sugeruje, że w przyszłości pojawi się uczeń o imieniu Shenhui. W momencie jego śmierci pojawiło się wiele pomyślnych znaków. Huineng miał zostać pochowany w Caoqi, a Wei Qu (韋璩), prefekt, który wysłuchał nauk zawartych w dalszej części tekstu, napisał ku jego czci inskrypcję, którą następnie zniszczyli przedstawiciele Północnej Szkoły (Huineng, Schlütter i Teiser 2012: 31, 34).

\if 0
Schlütter 31
Wang Wei 王維 napisał epitafium dla Huinenga, w którym pojawia się postać mnicha Yinzonga, który miał ostrzyc głowę Huinengowi
nie atakuje Północnego Chan
Matka Wang Wei była wyznawczynią Puji
Huineng nie był członkiem szlachetnego rodu

romans: quest (podróż), konkurs, walka o życie, w której Huineng w końcu wygrywa, bo jest lepszy

\fi

\section{Analiza biografii Huinenga zawartej w tekście \textit{Sutry Platformy} z Dunhuang}
Obecnie uważa się, że autobiograficzny monolog Huinenga, przytoczony w \textit{Sutrze Platformy}, nie jest autentyczną historią jego życia, a jedynie hagiografią. Tekst wysuwa twierdzenia, jakoby Huineng miał być prawowitym spadkobiercą Piątego Patriarchy Hongrena, szóstym dzierżawcą przekazu Chan, pochodzącego w prostej, nieprzerwanej linii od samego historycznego Buddy. (Huineng, Schlütter i Teiser 2012: 25-26). % ***

Shenhui w 732 roku zaczął promować Huinenga jako Szóstego Patriarchę linii Chan, i jednocześnie atakował uczniów i spadkobierców Shenxiu, szczególnie Puji (嵩山普寂 \pinyin{Sōngshān Pǔjì}, 651-739), roszczącego sobie prawa do tytułu Siódmego Patriarchy. Twierdził, że Szkoła Północna, której przewodzili Shenxiu i Puji, nie była autentyczna, gdyż propagowała nauki stopniowej ścieżki. Prawdziwe, ponadczasowe nauki buddy, to jest nauki o nagłym oświeceniu, znane również jako subityzm, miały być przekazywane w południowym Chan. Znamienne jest to, że w pismach Shenhui nie było żadnej wzmianki o konkursie poezji ani o dwóch wierszach, co świadczy o tym, że fragmenty te zostały dodane po jego śmierci. O ile w tekście \textit{Sutry platformy} zawarto nauki Shenhui i jego spadkobierców, jego wkład w powstanie sutry został w tekście przemilczany (Huineng, Schlütter i Teiser 2012: 32-33; Huineng i Yampolsky 2012: 26, 28; McRae 2004: 63).

Historyczny Huineng był postacią stosunkowo mało znaną i prawdopodobnie dlatego Shenhui wybrał go jako bohatera swoich historii. Ponieważ niewiele było wiadomo o jego prawdziwych naukach, Shenhui mógł przypisać mu dowolne pouczenia niestojące w sprzeczności z doktryną subityzmu. Prawdopodobnie był on mistrzem medytacji nauczającym o nagłym oświeceniu, ale wbrew temu, co możemy przeczytać w \textit{Sutrze platformy}, w jego czasach nie było to w istocie niczym szczególnym. Pochodził z południa, obszaru oddalonego od serca chińskiej cywilizacji. Wbrew autobiografii zapisanej w tekście, raczej nie wchodził w konflikty z innymi mistrzami medytacji, a wręcz miał z nimi dobre relacje (Huineng i McRae 2000: xv).

W późniejszym okresie, tj. po śmierci Shenxiu, historia życia Huinenga została uzupełniona o opowieść o konkursie poezji, w którym Huineng miał pokonać Shenxiu, jednoznacznie dowodząc wyższości subityzmu Szkoły Południowej nad stopniową ścieżką Szkoły Północnej. Biorąc pod uwagę, że Huineng został w tekście przedstawiony jako ubogi, niepiśmienny człowiek, jest niezwykle mało prawdopodobne, by był w stanie ułożyć przypisywany mu wiersz w klasycznym, literackim języku chińskim. Z drugiej strony, ponieważ Huineng wywodził się z rodziny urzędnika (nawet popadłego w niełaskę władz i skazanego na banicję), wydaje się nieprawdopodobne, że mógłby nie otrzymać żadnego wykształcenia. W dziełach Shenhui pojawiały się również twierdzenia, jakoby Puji wysłał swojego ucznia, niejakiego Zhang Xingchang (張行昌), do Shaozhou, z poleceniem ucięcia głowy zwłokom Huinenga. Twierdził też, że inny uczeń Puji, imieniem Wu Pingyi (武平一), wymazał inskrypcję na steli poświęconej Huinengowi i wstawił tam własną, podającą Shenxiu jako prawowitego Szóstego Patriarchę. Ataki Shenhui na Szkołę Północną zostały spisane przez Dugu Pei (獨孤沛) w dziele zwanym \textit{Putidamo Nanzong ding shifei lun} (菩提達摩南宗定是非論 \pinyin{Pútídámó Nánzōng dìng shìfēi lùn}). Chiński pisarz, doktor filozofii Hu Shi (胡適, 1891-1962) zebrał odkryte w Dunhuang dzieła Shenhui i jego uczniów i opisał je w pracy pt. \textit{Shenhui heshang yiji} (社會和尚遺集 \pinyin{Shénhuì héshàng yíjí}) % (\textit{Ibidem}).
(Huineng, Schlütter i Teiser 2012: 32-33; Huineng i Yampolsky 2012: 26, 28; Huineng i McRae 2000: xv).

Kim był Shenxiu i czym zasłużył sobie na osobiste ataki Shenhui? W przedmowie do przekładu \textit{Sutry platformy} Philipa B. Yampolsky'ego napisano, że na przełomie VII i VIII w. Shenxiu był uważany za jednego z najbardziej znaznych i najwybitniejszych mistrzów Chan, a jego biografia jest szczególnie dobrze zachowana. Stosunkowo rzetelna historia jego życia została zapisana w dziele \textit{Chuan fabao ji} (傳法寶紀 \pinyin{Chuán fǎbǎo jì}, `Annały przekazu skarbu Dharmy') w pozbawiony elementów fantastycznych sposób. O ile we wszystkich innych dziełach z tego okresu jest wymieniony jako uczeń Hongrena, \textit{Chuan fabao ji} podaje, że był uczniem Faru, a ten --- Hongrena. Według tej biografii pochodził z miasta Daliang (大梁 \toponim{Dàliáng}), obecnie Kaifeng (開封 \toponim{Kāifēng}) w prowincji Henan\prowincja{河南 \toponim{Hénán}}, i był członkiem rodu Li (Huineng i Yampolsky 15-16). % 李

Shenxiu był ponadprzeciętnie uzdolnionym dzieckiem. W wieku 13 lat, w związku z zawirowaniami historycznymi i związaną z nimi klęską głodu, postanowił wyrzec się światowego życia i zostać mnichem buddyjskim. Później wędrował od jednej świątyni do drugiej, by wreszcie otrzymać pełne ślubowania mnisie jako dwudziestolatek. W wieku 46 lat udał się do Hongrena, a ten natychmiast poznał się na jego talencie. Po wielu latach studiowania nauk osiągnął ostateczne oświecenie, a następnie udał się do Jingzhou (荊州 \toponim{Jīngzhōu}) w prowincji Hubei\prowincja{湖北 \toponim{Húběi}}. Za panowania cesarza Tang Gaozonga w okresie Yifeng (儀鳳 \pinyin{Yífèng}) udał się do świątyni Yuquan (玉泉寺 \toponim{Yùquán sì}) w pobliżu obecnego miasta Dangyang (當陽 \toponim{Dāngyáng}) w prowincji Hubei. Dopiero po śmierci swojego mistrza zaczął gromadzić wokół siebie uczniów z dalekich stron, nauczając ich Dharmy. Przynosił pożytek wielu istotom, prowadząc je do wyzwolenia (\textit{Ibidem}). % (Yampolsky 15-16)

Między rokiem 730 a 750 Shenhui stworzył wersję historii życia Huinenga zbliżoną do zawartej w \textit{Sutrze Platformy}, z tą różnicą, że nie było w niej jeszcze wzmianki o konkursie poezji między Shenxiu i Huinengiem. Wedle obecnego stanu wiedzy Shenhui nie posiadał niemal żadnych wiarygodnych informacji na temat postaci Huinenga. Wiedział o nim jedynie, że był uczniem Hongrena, nauczycielem Dharmy o regionalnym zasięgu działalności, i że mieszkał w Shaozhou. Najwcześniejsze pisma przypisywane Shenhui podają jedynie, że Huineng był Szóstym Patriarchą i otrzymał szatę, insygnium patriarchatu oraz przekazu nauk, od Hongrena, Piątego Patriarchy. W \textit{Sutrze Platformy} Shenhui został wymienony jako ostatni z dziesięciu uczniów Huinenga, a także jedyny, który nie płakał, gdy Huineng poinformował ich o zbliżającej się śmierci. Częste odniesienia do jego postaci wskazują, że \textit{Sutra Platformy} powstała najprawdopodobniej wkrótce po jego śmierci
(Huineng, Schlütter i Teiser 2012: 33).

Współcześni badacze uważają, że niemożliwe jest, by wydarzenia przedstawione w biografii Huinenga faktycznie miały miejsce, i należy je traktować jako ciekawą anegdotę o wyraźnym podtekście duchowym. Za tą tezą przemawia fakt, że Shenxiu był uczniem Hongrena jedynie przez kilka lat w początkowym etapie działalności Piątego Patriarchy, a więc kiedy nie istniał jeszcze problem wyboru jego następcy. Shenxiu i Huineng nie przebywali w klasztorze Hongrena w tym samym czasie, a więc nie mogli współzawodniczyć w konkursie poezji. Po drugie, w owym okresie nie istniała jeszcze koncepcja jedynego prawowitego patriarchy --- pojawiła się ona dopiero w dziełach Shenhui. Ponadto historia życia Huinenga w zachowanej do dnia dzisiejszego wersji nie pojawia się w pismach Shenhui, a jako propagator Huinenga w roli Szóstego Patriarchy na pewno zapisałby tę historię, gdyby była mu znana. Jednak niedługo po roku 730 dzieje życia Huinenga były mu jeszcze w dużej mierze obce. (McRae 2004: 67; Huineng i McRae 2000: xv).

Ważnym aspektem nauk Shenhui, przypisywanych Huinengowi, jest odejście od sutry \textit{La\.nkā\-vatāra} na rzecz \textit{Sutry Diamentowej}, która zyskiwała na popularności w VIII w.
O tym, jak wielką rolę odgrywała ona dla Shenhui, świadczą liczne odniesienia do niej w \textit{Sutrze Platformy}.
W tekście jest wysunięta również teza, jakoby to \textit{Sutra Diamentowa}, a nie \textit{La\.nkāvatāra}, była podstawą nauk przekazywanych przez patriarchów, od Bodhidharmy do Huinenga.
Twierdzeniom tym zaprzeczają jednak inne teksty na temat linii przekazu Chan, takie jak \textit{Xu gaoseng zhuan}, \textit{Chuan fabao ji} i \textit{Lengqie shizi ji} (楞伽師資記 \pinyin{Léngqié shīzī jì}, `Zapisy mistrzów i uczniów w przekazie Sutry La\.nkāvatāra').
Symboliczne odejście od sutry \textit{La\.nkāvatāra} jest też zaznaczone w biografii Huinenga w \textit{Sutrze Platformy} w miejscu, w którym Piąty Patriarcha Hongren zrezygnował z wykonania malowideł ze scenami z sutry \textit{La\.nkāvatāra} na rzecz wiersza Shenxiu (Huineng i Yampolsky 2012: 34; McRae 2004: ***).

W dziełach Shenhui pojawiły się też dwie opowieści, powielone w późniejszych dziełach. Pierwsza z nich, zapisana zarówno w \textit{Sutrze Platformy}, jak i \textit{Putidamo Nanzong ding shifei lun}, dotyczyła Bodhidharmy i cesarza Liang Wudi (梁武帝 \nazwisko{Liáng Wǔdì}).
Według tej historii, kiedy Bodhidharma przybył do stolicy Liang (***), przeprowadził dyskusję z cesarzem. Cesarz miał spytać Bodhidharmy, czy budując świątynie, dając ofiary mnichom i ludziom w potrzebie, zgromadził zasługę\glosref{Zasługa}{glosMerit}.
Mistrz odparł: ,,Nie zgromadziłeś zasługi.'' Miał przez to na myśli, że cesarz, nie podążając za właściwą ścieżką, szukał jedynie błogosławieństw, a nie prawdziwej zasługi. Cesarz, nie rozumiejąc tej nauki, był nią rozczarowany i wygnał Bodhidharmę ze swego państwa. Następnie mistrz udał się do państwa Wei (Huineng i Yampolsky 2012: 27, 155-156).
\if 0
T48n2007_p0341a24(10)║朕一生未來造寺布施供養有有功德否。達磨答言。並無功德。
T48n2007_p0341a25(03)║武帝惆悵遂遣。達磨出境。未審此言。請和尚說。
a w Wei spotkał Huike
\fi

Inna, nieco drastyczna opowieść, propagowana przez Shenhui, dotyczy Bodhidharmy i jego ucznia i spadkobiercy, Huike. Według tej historii, kiedy mistrz i uczeń spotkali się po raz pierwszy, Huike był zdesperowany, by zostać uczniem Bodhidharmy, lecz ten nie chciał go przyjąć. Mistrz miał ustąpić Huike dopiero wówczas, gdy ten w dowód swej determinacji dobył miecza i demonstracyjnie uciął swoje lewe ramię. Huike został następnie głównym uczniem Bodhidharmy i odziedziczył po nim szatę, symbol przekazu Dharmy. Później tę samą szatę mieli otrzymać kolejni patriarchowie: Sengcan, Daoxin, Hongren, aż do Huinenga. Przy pomocy tej opowieści Shenhui osiągnął dwa cele: nie tylko ustanowił szatę Bodhidharmy insygnium prawowitego patriarchy Chan, lecz również podważył uznaną dotychczas linię przekazu, wiodącą od Bodhidharmy do Shenxiu. Pochodzenie tych legend nie jest znane. Hu Shi uważa, że zostały one wymyślone przez Shenhui, ale równie prawdopodobne jest, że krążyły wśród ludu, a Shenhui jedynie zapisał je i wykorzystał do swoich celów (Huineng i Yampolsky 2012: 27).

\if 0
Yampolsky
Nawet jeżeli istniał Bodhidharma, linia przekazu Chan raczej nie istniała w nieprzerwanej formie
Chan jako taki wymyślono po czasach Hongrena
biografia Huinenga ma poprzeć roszczenia
Schlütter 32
\fi

\subsection{Wiersze Shenxiu i Huinenga}
Tradycyjna interpretacja wierszy Shenxiu i Huinenga, zawartych w tekście \textit{Sutry Platformy} (patrz: strony \pageref{ShenxiuVerse} i \pageref{HuinengVerse}), jest prosta. Według filozofa Zongmi (圭峰宗密 \pinyin{Guīfēng Zōngmì}, 780-841) wiersz Shenxiu ma symbolizować stopniową ścieżkę, zaś wiersz Huinenga --- ścieżkę nagłego, ostatecznego oświecenienia, które dokonuje się w jednej chwili. Tym samym, w rozumieniu Zongmi wiersze te reprezentują pogląd dwóch konkurencyjnych tradycji Chan, północnej i południowej. W późniejszym okresie była to dominująca interpretacja w szkole Chan (McRae 2004: 63).

Takie rozumienie jest jednak nadmiernym uproszczeniem. Wiersz przypisywany Shenxiu odnosi się nie tyle do stopniowej ścieżki, ile do ciągłej, bezustannej praktyki oczyszczania zwierciadła z kurzu. Wiersz Huinenga nie opisuje natomiast poglądu nagłej ścieżki, a jedynie neguje twierdzenia, zawarte w wierszu Shenxiu. Oprócz tego, wiersze przedstawiają dwa punkty widzenia na ten sam temat i nie mogą być interpretowane osobno. W wierszu Huinenga pojawiają się odwołania do wiersza Shenxiu (McRae 2004: 63-64).

Opis konkursu poezji, zawarty w tekście \textit{Sutry platformy}, ma za zadanie udowodnić wyższość nauk Huinenga i jego spadkobierców nad naukami Shenxiu. Chociaż zwycięzcą w starciu był Huineng, to wiersz, przypisywany Shenxiu, wcale nie jest mierny, a wręcz przeciwnie --- jest głęboki i wyrafinowany. W ten sposób wiersz Huinenga tylko zyskuje w oczach odbiorcy. Nie jest bowiem sztuką stworzyć dzieła lepszego od czegoś miernego (McRae 2004: 65).

\if 0
McRae xvi
Shenxiu i Huineng nie przebywali w Huangmei w tym samym czasie

McRae 62-64:
Wiersz Shenxiu:
Nie ma dowodu, żeby kiedykolwiek go napisał, albo żeby porównał umysł do jasnej lustrzanej podstawy, ale z innych jego dzieł wynika, że mógł napisać coś w tym stylu
constant and perfect teaching, the endless personal manifestation of the
bodhisattva ideal
McRae, Northern School, 235. The English “suchlike'' renders the word 如, as in the Chinese translation of Tathágata, 如來, when used as a modifier.
神秀觀心論
T85n2833_p1271c15(08)║眾生修伽藍鑄形像燒香散花然長明燈。
T85n2833_p1271c16(09)║晝夜六時遶塔行道持齋禮拜。種種功德皆成佛道。
T85n2833_p1271c17(06)║若唯觀心總攝諸行。如是事應妄也 答曰。
T85n2833_p1271c18(06)║佛所說無量方便。一切眾生鈍根狹劣。
T85n2833_p1271c19(08)║甚深所以假有為喻無為。若不內行唯只外求。希望獲福。
T85n2833_p1271c20(03)║無有是處。言伽藍者。西國梵音。
T85n2833_p1271c21(07)║此地翻為清淨處地。若永除三毒常淨六根。
T85n2833_p1271c22(07)║身心湛然內外清淨。是名為修伽藍也。又鑄形像者。
ciało jest drzewem bodhi, oba są fizyczne
ciało jest miejscem, w którym człowiek osiąga oświecenie
lustrzana podstawa

Wiersz Shenxiu był głęboki i wyrafinowany, bo tym sposobem wiersz Huinenga jest lepszy od czegoś wybitnego; być lepszym od czegoś miernego to żadne osiągnięcie
Wiersz Huinenga wcale nie wyraża poglądów szkoły nagłego oświecenia, a jedynie zaprzecza poglądom wiersza Shenxiu
Więc nie jest to wykładnia szkół nagłego i stopniowego oświecenia
Wiersz Huinenga pojawiał się w kilku wersjach w różnych tekstach
Wiersze oparte na pismach Szkoły Północnej (McRae 67)

Przejście od Lankavatara do Diamentowej Sutry, w VIII w. ta właśnie sutra nabierała popularności, na niekorzyść Lankavatary
\fi
\subsection{Analogie do biografii Konfucjusza}
Pisząc historię życia Huinenga, Shenhui w oczywisty sposób czerpał z legendy o Konfucjuszu, opisanej w \textit{Zapiskach historyka} (史記 \pinyin{Shǐjì}) autorstwa Sima Qian (司馬遷 \nazwisko{Sīmǎ Qiān}) zwanego ,,Wielkim Historykiem''. Tekst ten miał wówczas ugruntowaną pozycję wśród chińskich elit, jako że Konfucjusz był twórcą głównego systemu filozoficznego w państwie, i był znany niemal wszystkim jako uniwersalny wzorzec cnót (Huineng, Schlütter i Teiser 2012: 36).

Według wspomnianej legendy Konfucjusz urodził się jako Kong Qiu (孔丘 \nazwisko{Kǒng Qiū}), owoc mezaliansu Shuliang He (叔梁紇 \nazwisko{Shūliáng Hé}), oficera wojsk państwa Lu, i lokalnej kobiety, Yan Zhengzai (顏徵在 \nazwisko{Yán Zhēngzài}). Jego ojciec zmarł, gdy ten był jeszcze dzieckiem, w związku z czym Konfucjusz, tak jak Huineng, dorastał w ubóstwie, wychowywany przez samotną matkę. Rodzina Shuliang He nie pomagała im po jego śmierci. Niewiele wiadomo o najwcześniejszych latach jego życia, ale według \textit{Zapisków historyka}, młody Kong Qiu wykazywał wielkie zainteresowanie naczyniami rytualnymi z brązu. Według legendy, lubił bawić się nimi, starannie układając je tak, jak przy składaniu ofiar przodkom. W wieku 15 lat Kong Qiu zaczął zgłębiać teksty i rytuały z początków dynasti Zhou\index{Dynastia Zhou}\glosref{Dynastia Zhou}{glosZhouDynasty}, którą uważał za złoty wiek cywilizacji Chin. Pierwszą pracą, którą podjął, było zarządzanie gospodarstwem rolnym lokalnej arystokracji. Przez pewien czas pracował jako urzędnik niskiego szczebla, lecz wkrótce został z tej posady zwolniony, mimo że wykazywał ponadprzeciętne uzdolnienia. Uważał dynastię Zhou za złoty wiek cywilizacji chińskiej, czas zjednoczenia i pokoju, zaś jej władców, a także królów sprzed założenia tej dynastii --- za mędrców, rządzących swoimi państwami sprawiedliwie i moralnie. Twierdził, że powodem, dla którego we współczesnych mu Chinach zapanował chaos, było odejście od ideałów Zhou, i postawił sobie za cel przywrócenie w kraju dawnych wartości i rytuałów. Konfucjusz nie cieszył się szczególną popularnością wśród lokalnej elity państwa Lu, a jego idee nie zostały docenione za jego życia, tak jak idee Huinenga, które spopularyzowała dopiero \textit{Sutra platformy}. (Schuman 2015: 45-50; Huineng, Schlütter i Teiser 2012: 36-37).

Shenhui był wykształcony w zakresie filozofii konfucjańskiej, a nawet został przez Zongmi porównany do Konfucjusza. Zan Ning (贊寧 \nazwisko{Zàn Níng}), autor \textit{Biografii wybitnych mnichów Song} (宋高僧傳 \pinyin{Sòng gāosēng zhuàn}) porównał go do Yan Hui (顏回 \nazwisko{Yán Huí}), ulubionego ucznia Konfucjusza.
Symboliczne znaczenie miało również liczba patriarchów w linii przekazu Chan. Według panującej wówczas interpretacji myśli konfucjańskiej jedynie cesarz mógł w świątyni swojej rodziny urządzić siedem izb. Linia przekazu w wersji Shenhui wiodła od Bodhidharmy, Pierwszego Patriarchy, przez Hongrena, Piątego Patriarchę, do Huinenga, Szóstego Patriarchy. Jako spadkobierca Huinenga, Shenhui stawał się Siódmym Patriarchą i tym samym ustanawiał doskonałą linię Chan.
\if 0
Southern Learning
siedem świątyń lub izb w świątyni przodków było zarezerwowane dla cesarzy
宗
Schlütter 37 \fi
(Huineng, Schlütter i Teiser 2012: 37).

Z perspektywy nauk Konfucjusza Huineng był postacią godną naśladowania. Mimo iż wywodził się z nizin społecznych i wychował się w oddalonym od centrum cywilizacji chińskiej rejonie, to posiadał jedną z najwyższych cnót konfucjańskich --- \textit{nabożność synowską} (子孝 \pinyin{zǐxiào}). Po śmierci ojca ciężko pracował, utrzymywał starą matkę i opiekował się nią. Taki obraz Huinenga stał w opozycji do wykształconych, bogatych elit, z których wywodziła się w owym okresie większość mnichów buddyjskich. Twórcy jego hagiografii mogli również inspirować się historią życia Hongrena, który według niektórych podań medytował za dnia, zaś w nocy zajmował się bydłem. Sam Shenxiu, rzekomy konkurent Huinenga do pozycji patriachy, pochodził ze szlachetnego rodu i był wykształcony zarówno w literaturze buddyjskiej, jak i świeckiej, a niektórzy podejrzewają nawet, że mógł być związany z rodem cesarskim (McRae 2004: 68).

Celem takiego przedstawienia Szóstego Patriarchy było pokazanie, że każdy, nawet osoba świecka, niezależnie od pozycji społecznej, miejsca pochodzenia i wykształcenia mógł rozpoznać naturę swojego umysłu i zostać zwierzchnikiem Chan. Co za tym idzie, jeżeli oświecenie było jedynym warunkiem zostania patriachą, to czytelnik mógł oczekiwać, że wszyscy poprzednicy i następcy Huinenga również byli oświeceni. W ten sposób szkoła Chan zyskiwała autorytet i wiarygodność (McRae 2004: 69).
