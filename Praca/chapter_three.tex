\chapter{Historyczne reperkusje \textit{Sutry platformy}}

Lorem ipsum dolor sit amet.

\if 0
3
Sutra platformy napisała historię Chan od nowa; wskazuje na to pominięcie roli Shenhui
W początkowej fazie istnienia Południowej Szkoły, była ona mało znana; sutra wyjaśnia to długim czasem, jaki Huineng spędził, ukrywając się u myśliwych
Lankavatara => Diamentówka
McRae 1986: 5
Szkoła Południowa twierdziła, że posiada nauki niedualne
Wg SS natura oświecenie, przeszkadzające emocje, cierpienie i iluzje są w istocie tym samym, co oświecenie, ale NS widziała je jako różne; wg 宗密 oznacza to, że wiele lat, lub nawet żywotów praktyki idzie na marne; wszystko, czego potrzebuje praktykujące, to całkowite odcięcie dualistycznego myślenia
SS była lewicowa: uważała, że każdy powinien mieć prawo poznać Dharmę i osiągnąć oświecenie, a nie tylko ci, którzy włożyli w to wysiłek
Zongmi usystematyzował różne interpretacje Chan, Szkoła Północna była najniżej
wykładnia 宗密 NS na początku była popularna, ale potem została niemal całkowicie wyparta przez SS, bo prawowitym spadkobiercą był Huineng, a nie Shenxiu
Nie ma dobrych badań nt. NS

17 Pierwsze wzmianki o Bodhidharmie w 洛陽伽藍記, to był tekst, w którym ludzie z różnych stron świata przyjeżdżali do Luoyangu, by zachwycać się jego architekturą
(Wiki) 曇林 napisał przedmowę do 二入四行, który tradycyjnie przypisuje się Bodhidharmie, wspomina 道育 i 慧可, i że był z południowych Indii Broughton 1999, p. 53

續高僧傳:
南天竺婆羅門種 Brahmin z Południowych Indii
Przybył do Nanyue w okresie Liu Song, (a więc przed rokiem 479), nie wiadomo, gdzie zginął
\fi
