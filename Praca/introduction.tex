\introduction
Buddyzm Chan (禪宗 \pinyin{Chán zōng}) jest, obok Szkoły Czystej Krainy\fnm (淨土宗 \pinyin{Jìngtǔ zōng}) jedną z najważniejszych tradycji buddyzmu w Chinach. Samo słowo \textit{chan} jest chińskim wariantem %odwzorowaniem
sanskryckiego słowa Dhyāna, które oznacza medytację. %Wyjaśnić
Początkowo termin ten tłumaczono jako 禪那 \pinyin{chánnà}, pierwotnie wymawiane \textit{dianna}, co było fonetycznym odwzorowaniem oryginalnego terminu. W późniejszym okresie upowszechniła się skrócona forma \textit{chan}. W Japonii buddyzm Chan znany jest pod nazwą Zen (禅 \textit{zen}), a w Korei Seon ({\Korean 선} \textit{seon}).
%
\footnotetext{Szkoła Czystej Krainy, zwana również Szkołą Czystej Ziemi lub amidyzmem --- tradycja buddyzmu chińskiego, w której za najważniejszą postać przyjmuje się \textit{Buddę Amitabhę} (阿彌陀佛 \pinyin{Āmìtuófó}). Celem praktyki tej tradycji jest odrodzenie po śmierci w Czystej Krainie tego buddy, \textit{Sukhavati} (極樂 \pinyin{Jílè}, 安樂 \pinyin{Ānlè}, 西天 \pinyin{Xītiān}).} % Ā U+100

Za założyciela buddyzmu Chan uznaje się indyjskiego mistrza Bodhidharmę (菩提達摩 Pútídámó, od skt. \textit{bodhi} `oświecenie' i \textit{dharma} `zjawiska, porządek wszechświata, nauki Buddy'), dwudziestego ósmego patriarchę Indii. Dokładne daty jego narodzin i śmierci nie są znane, jednak współcześnie przyjmuje się, że żył w V w. n.e. Przywędrował on z Indii do Chin, by nauczać buddyzmu Mahajany\fnm w tym kraju. Według chińskich przekazów, Bodhidharma był brodatym mężczyzną o niechińskiej fizjonomii. Z powodu bariery językowej nazywano go ,,błękitnookim barbarzyńcą'' (碧眼胡 \pinyin{Bìyǎnhú}). (Huineng i Hsüan Hua 1998: bez nru strony; Soothill i Hodous 2003: 1004; Buswell 2004: 57).
%
\footnotetext{Buddyzm Mahajany (大乘佛教 \pinyin{Dàshèng fójiào} lub \pinyin{Dàchéng fójiào}, `buddyzm Wielkiego Wozu', od skt. \textit{Mahāyāna}, `wielki wóz', nazywany również buddyzmem Wielkiej Drogi) --- jeden z trzech głównych odłamów buddyzmu (dwa pozostałe to Hinajana, tzw. Mała Droga lub Mały Wóz, oraz Wadżrajana, Diamentowa Droga lub Diamentowy Wóz). Filarami Mahajany są wyzwalająca mądrość i współczucie dla wszystkich czujących istot, rozwijane w równowadze. Do buddyzmu Mahajany zalicza się m.in. buddyzm Chan i Zen, Szkołę Czystej Krainy, a także szkołę Gelugpa buddyzmu tybetańskiego.}

%Najstarsza zachowana wersja Sutry Platformy powstała w VIII w. w Chinach. 
Odwołania do tekstu Sutry Platformy w niniejszej pracy odnoszą się do przekładu na język angielski Philipa B. Yampolsky'ego, wydanej drukiem przez Columbia University Press w roku 1967 i wznowionej w roku 2012.
(Huineng, Schlütter i Teiser 2012: vii)
