\introduction
Buddyzm Chan (禪宗 \pinyin{Chán zōng}) jest, obok Szkoły Czystej Krainy\footnote{Szkoła Czystej Krainy, zwana również Szkołą Czystej Ziemi lub amidyzmem --- tradycja buddyzmu chińskiego, w której za najważniejszą postać przyjmuje się \textit{Buddę Amitabhę} (阿彌陀佛 \pinyin{Āmìtuófó}). Celem praktyki tej tradycji jest odrodzenie po śmierci w Czystej Krainie tego buddy, \textit{Sukhavati} (w języku chińskim nazywana 極樂 \pinyin{Jílè}, 安樂 \pinyin{Ānlè} lub 西天 \pinyin{Xītiān}).} % Ā U+100
 jedną z najważniejszych tradycji buddyzmu w Chinach. Samo słowo \textit{chan} jest chińskim wariantem %odwzorowaniem
sanskryckiego słowa \textit{dhyāna}, które oznacza medytację.
Początkowo termin ten zapisywano w języku chińskim jako \pinyin{chánnà} (禪那), pierwotnie wymawiane \textit{dianna}, co było fonetycznym odwzorowaniem oryginalnego terminu. W późniejszym okresie upowszechniła się skrócona forma \textit{chan}.
W Japonii buddyzm Chan znany jest pod nazwą Zen\footnote{Kanji: {\ipaexgothic 禅}, Rōmaji: \textit{zen}.}, a w Korei Seon\footnote{Hangeul: {\Korean 선}, Revised Romanization of Korean: \textit{seon}}.

Za założyciela buddyzmu Chan uznaje się indyjskiego mistrza Bodhidharmę (菩提達摩 Pútídámó, od skt. \textit{bodhi} `oświecenie' i \textit{dharma} `zjawiska; nauki Buddy').
Pierwsza wzmianka o nim pojawia się w tekście pt. \textit{Luoyang qielan ji} (洛陽伽藍記 \pinyin{Luòyáng qiélán jì} `Zapisy o klasztorach w Luoyangu') autorstwa Yang Xuanzhi (楊衒之), pochodzącym z ok. 547 roku.
Dzieło to opisuje świątynie i klasztory miasta Luoyang (洛陽 \toponim{Luòyáng}) w prowincji Henan, a także opowiada o podróżnych, którzy przybywali tam z dalekich strony, by podziwiać jego wyszukaną architekturę.
Bodhidharma został w tym tekście przedstawiony jako stupięćdziesięcioletni \textit{śrama\d{n}a} (święty mąż) z rejonu zwanego \textit{Bosi} w Azji Środkowej.
Przybywszy do Luoyangu, Bodhidharma miał się zachwycać pięknem tamtejszych świątyń, a zwłaszcza jednej, zwanej Yongning Si (永寧寺, \pinyin{Yǒngníng Sì}, `Świątynia Wiecznego Spokoju').
Autor dzieła nie przywiązywał szczególnej uwagi do osobowości ani dokonań życiowych pielgrzymów, o których pisał.
Pełnili oni jedynie rolę świadków piękna i przepychu architektury Luoyangu (McRae 1986: 17).

Dokładne daty narodzin i śmierci Bodhidharmy nie są znane.
% Tu można wspomnieć o zapisach, które mówią, że przybył do Chin w okresie Liu Song
Bodhidharma miał otrzymać przekaz Dharmy, pochodzący w nieprzerwanej linii od samego historycznego Buddy Siakjamuniego, Siddharta Gautamy, i przywędrować z Indii do Chin, by nauczać buddyzmu Mahajany\fnm w tym kraju.
Według chińskich przekazów, Bodhidharma był braminem z południowych Indii. Z powodu bariery językowej nazywano go ,,błękitnookim barbarzyńcą'' (碧眼胡 \pinyin{Bìyǎnhú}).
(Huineng i Hsüan Hua 1998: bez nru strony; Soothill i Hodous 2003: 1004; Buswell 2004: 57).
%
\footnotetext{Buddyzm Mahajany (大乘佛教 \pinyin{Dàshèng fójiào} lub \pinyin{Dàchéng fójiào}, `buddyzm Wielkiego Wozu', od skt. \textit{Mahāyāna}, `wielki wóz', nazywany również buddyzmem Wielkiej Drogi) --- jeden z trzech głównych odłamów buddyzmu (dwa pozostałe to Hinajana, tzw. Mała Droga lub Mały Wóz, oraz Wadżrajana, Diamentowa Droga lub Diamentowy Wóz). Filarami Mahajany są wyzwalająca mądrość i współczucie dla wszystkich czujących istot, rozwijane w równowadze. Do buddyzmu Mahajany zalicza się m.in. buddyzm Chan i Zen, Szkołę Czystej Krainy, a także szkołę Gelugpa buddyzmu tybetańskiego.}

\section{Pochodzenie \textit{Sutry Platformy} oraz jej przekłady na język angielski}
\textit{Sutra Platformy Szóstego Patriarchy} (六祖壇經 \pinyin{Liùzǔ Tánjīng}) jest apokryficznym tekstem buddyzmu Chan, którego najstarsza zachowana wersja powstała w VIII w. w Chinach. Tekst napisany został częściowo w formie monologu, a częściowo w formie dialogu nauczyciela z uczniami. Nauki w niej zawarte miały zostać wygłoszone przez legendarnego patriarchę buddyzmu Chan, Huinenga (大鑒惠能 \nazwisko{Dàjiàn Huìnéng}, zapisywane również jako 大鑒慧能). % źródło
Huineng jest w \textit{Sutrze Platformy} przedstawiony jako niepiśmienny, prosty człowiek z leżącego poza zasięgiem chińskiej cywilizacji południa.

Pełen tytuł \textit{Sutry Platformy} brzmi \textit{Doktryna nagłego oświecenia Szkoły Południowej, Najwyższa Doskonałość Mądrości Mahajany: Sutra Platformy, przekazana przez Szóstego Patriarchę Huineng w świątyni Dafan, w prefekturze Shao} (南宗頓教最上大乘摩訶般若波羅蜜經六祖惠能大師於韶州大梵寺施法壇經 \pinyin{Nánzōng dùnjiào zuìshàng dàshèng móhēbōrě bōluómì jīng liùzǔ Huìnéng Dàshī yú Shāozhōu Dàfán Sì shīfǎ Tánjīng}). W języku chińskim zwykle nazywana jest w skrócie 壇經 \pinyin{Tánjīng} `Sutra platformy', 六祖壇經 \pinyin{Liùzǔ Tánjīng} `Sutra platformy Szóstego Patriarchy', bądź 六祖大師法寶壇經 \pinyin{Liùzǔ Dàshī Fǎbǎo Tánjīng}, `Skarb Dharmy, Sutra platformy Szóstego Patriarchy'.

Sutra Platformy jest uważana za jedno z najważniejszych dzieł buddyzmu Chan, ponieważ wprowadziła nauki o nagłym oświeceniu (頓教 \pinyin{dùnjiào}, `nagła szkoła, subityzm'), stojące w opozycji do nauk tzw. stopniowej szkoły (漸教 \pinyin{jiànjiào}), i wywołała podział szkoły Chan na odłam północny i południowy. % reference
(Buswell 2004: 347-348).%; McRae, 2004: ).

Chociaż tradycyjne chińskie zapisy dotyczące historii tradycji Chan przedstawiają ją jako nieprzerwaną linię przekazu nauk i doświadczenia, z patriarchy (祖師 \pinyin{zǔshī}) na patriarchę, sięgającą aż do historycznego Buddy, to obecnie uważa się, że takie zapisy nie oddają stanu faktycznego. Miały one nadać przekazowi Chan autentyczność i przedstawić ją jako ortodoksyjną tradycję buddyzmu, pod jakimś względem lepszą od pozostałych, głównie w ramach rywalizacji z innymi tradycjami buddyzmu o wsparcie warstwy rządzącej oraz osób świeckich. Pierwsze wzmianki o linii przekazu patriarchów Chan pojawiają się na steli pogrzebowej poświęconej mnichowi Faru (法如 \nazwisko{Fǎrú}), który miał być uczniem Piątego Patriarchy Hongrena (大滿弘忍 \nazwisko{Dàmǎn Hóngrěn}). Według tego zapisu linia przekazu Chan prowadziła od Bodhidharmy, poprzez Huike (大祖慧可 \nazwisko{Dàzǔ Huìkě}), Sengcana (鑑智僧璨 \nazwisko{Jiànzhì Sēngcàn}), Daoxina (大醫道信 \nazwisko{Dàyī Dàoxìn}) i Hongrena, do Faru (Huineng, Schlütter i Teiser 2012: 53-54, 56).

\if 0
Budowa pracy: najpierw rzekoma autobiografia w formie monologu Huinenga
potem jego nauki w formie dialogów z uczniami
a wszystko to zapisał uczeń Huinenga, Fahai
\fi

Ważnym aspektem tekstu są nauki o tym, że każda czująca istota ma naturę buddy, i że zarówno ludzie świeccy, jak i mnisi mogą z powodzeniem praktykować jego nauki. Tekst opisuje również specjalny rytuał przekazywania mnichom i świeckim praktykującym ,,bezforemnych zasad''. % formless precepts
Były to niektóre z powodów, dla których w roku 796 Huineng został oficjalnie obwołany szóstym patriarchą Chan przez cesarską komisję, a jego dzieło stworzyło podwaliny pod dalszy rozwój szkoły Chan (Huineng, Schlütter i Teiser 2012: 2).

\section{Wstęp techniczny}
Terminy chińskie w niniejszej pracy podane są w nawiasach w znakach tradycyjnych, ponieważ wiernie oddają one pierwotną semantyczno-fonetyczną budowę znaków. Domyślną transkrypcją terminów chińskich jest \textit{Hanyu Pinyin} (漢語拼音 \pinyin{Hànyǔ Pīnyīn}) z oznaczonymi tonami, z wyjątkiem chińskich nazwisk, do których nie podano transkrypcji.

Odwołania do tekstu \textit{Sutry Platformy} w niniejszej pracy odnoszą się do przekładu na język angielski Philipa B. Yampolsky'ego, wydanej drukiem przez Columbia University Press w roku 1967 i wznowionej w roku 2012. [[Electronic Tripitaka]]

Słowo ,,Budda'', zapisywane wielką literą, oznacza historycznego Buddę Siakjamuniego, tj. Siddhārta Gautamę. Zapisywane małą literą ,,budda'' oznacza stan umysłu.
%(Huineng, Schlütter i Teiser 2012: vii)
