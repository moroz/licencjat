\introduction
Buddyzm Chan (禪宗 \pinyin{Chán zōng}) jest, obok Szkoły Czystej Krainy\fnm (淨土宗 \pinyin{Jìngtǔ zōng}) jedną z najważniejszych tradycji buddyzmu w Chinach. Samo słowo \textit{chan} jest chińskim wariantem %odwzorowaniem
sanskryckiego słowa Dhyāna, które oznacza medytację. %Wyjaśnić
Początkowo termin ten tłumaczono jako 禪那 \pinyin{chánnà}, pierwotnie wymawiane \textit{dianna}, co było fonetycznym odwzorowaniem oryginalnego terminu. W późniejszym okresie upowszechniła się skrócona forma \textit{chan}. W Japonii buddyzm Chan znany jest pod nazwą Zen (禅 \textit{zen}), a w Korei Seon ({\Korean 선} \textit{seon}).
%
\footnotetext{Szkoła Czystej Krainy, zwana również Szkołą Czystej Ziemi lub amidyzmem --- tradycja buddyzmu chińskiego, w której za najważniejszą postać przyjmuje się \textit{Buddę Amitabhę} (阿彌陀佛 \pinyin{Āmìtuófó}). Celem praktyki tej tradycji jest odrodzenie po śmierci w Czystej Krainie tego buddy, \textit{Sukhavati} (極樂 \pinyin{Jílè}, 安樂 \pinyin{Ānlè}, 西天 \pinyin{Xītiān}).} % Ā U+100

Za założyciela buddyzmu Chan uznaje się indyjskiego mistrza Bodhidharmę (菩提達摩 Pútídámó, od skt. \textit{bodhi} `oświecenie' i \textit{dharma} `zjawiska, porządek wszechświata, nauki Buddy'), dwudziestego ósmego patriarchę Indii. Dokładne daty jego narodzin i śmierci nie są znane, jednak współcześnie przyjmuje się, że żył w V w. n.e. Bodhidharma miał otrzymać przekaz Dharmy, pochodzący w nieprzerwanej linii od samego historycznego Buddy Siakjamuniego, Siddharta Gautamy, i przywędrować z Indii do Chin, by nauczać buddyzmu Mahajany\fnm w tym kraju. Według chińskich przekazów, Bodhidharma był brodatym mężczyzną o niechińskiej fizjonomii. Z powodu bariery językowej nazywano go ,,błękitnookim barbarzyńcą'' (碧眼胡 \pinyin{Bìyǎnhú}). (Huineng i Hsüan Hua 1998: bez nru strony; Soothill i Hodous 2003: 1004; Buswell 2004: 57).
%
\footnotetext{Buddyzm Mahajany (大乘佛教 \pinyin{Dàshèng fójiào} lub \pinyin{Dàchéng fójiào}, `buddyzm Wielkiego Wozu', od skt. \textit{Mahāyāna}, `wielki wóz', nazywany również buddyzmem Wielkiej Drogi) --- jeden z trzech głównych odłamów buddyzmu (dwa pozostałe to Hinajana, tzw. Mała Droga lub Mały Wóz, oraz Wadżrajana, Diamentowa Droga lub Diamentowy Wóz). Filarami Mahajany są wyzwalająca mądrość i współczucie dla wszystkich czujących istot, rozwijane w równowadze. Do buddyzmu Mahajany zalicza się m.in. buddyzm Chan i Zen, Szkołę Czystej Krainy, a także szkołę Gelugpa buddyzmu tybetańskiego.}

\section{Pochodzenie \textit{Sutry Platformy} oraz jej przekłady na język angielski}
\textit{Sutra Szóstego Patriarchy} jest apokryficznym tekstem buddyzmu Chan, którego najstarsza zachowana wersja powstała w VIII w. w Chinach. Tekst napisany został częściowo w formie monologu, a częściowo w formie dialogu nauczyciela z uczniem. Nauki w niej zawarte miały został wygłoszone przez legendarnego patriarchę buddyzmu Chan, Huinenga (惠能 \pinyin{Huìnéng}, zapisywane również jako 慧能). % źródło
Huineng jest w \textit{Sutrze Platformy} przedstawiony jako niepiśmienny, prosty człowiek z leżącego poza zasięgiem chińskiej cywilizacji południa.

Pełen tytuł \textit{Sutry Platformy} brzmi 南宗頓教最上大乘摩訶般若波羅蜜經六祖惠能大師於韶州大梵寺施法壇經 \pinyin{Nánzōng dùnjiào zuìshàng dàshèng móhēbānnuò bōluómì jīng liùzǔ Huìnéng Dàshī yú Shāozhōu Dàfán Sì shīfǎ Tánjīng}, `Doktryna nagłego oświecenia Szkoły Południowej, Najwyższa Doskonałość Mądrości Mahajany: Sutra Platformy, przekazana przez Szóstego Patriarchę Huineng w świątyni Dafan, w prefekturze Shao'. W języku chińskim zwykle nazywana jest w skrócie 壇經 \pinyin{Tánjīng} `Sutra platformy', 六祖壇經 \pinyin{Liùzǔ Tánjīng} `Sutra platformy Szóstego Patriarchy', bądź 六祖大師法寶壇經 \pinyin{Liùzǔ Dàshī Fǎbǎo Tánjīng}.

Sutra Platformy jest uważana za jedno z najważniejszych dzieł buddyzmu Chan, ponieważ wprowadziła ona nauki o nagłym oświeceniu (頓教 \pinyin{dùnjiào}, `nagła szkoła, subityzm'), stojące w opozycji do nauk tzw. stopniowej szkoły, i wywołała podział szkoły Chan na odłam północny i południowy. % reference
(Buswell 2004: 347-348).%; McRae, 2004: ).

Chociaż tradycyjne chińskie zapisy dotyczące historii tradycji Chan przedstawiają ją jako nieprzerwaną linię przekazu nauk i urzeczywistnienia z patriarchy (祖師 \pinyin{zǔshī}) na patriarchę, sięgającą aż do historycznego Buddy, to obecnie uważa się, że takie zapisy nie oddają stanu faktycznego. Miały one poprzeć nadać przekazowi Chan autentyczność i przedstawić ją jako ortodoksyjną tradycję buddyzmu, pod jakimś względem lepszą od pozostałych, głównie w ramach rywalizacji z innymi tradycjami buddyzmu o wsparcie warstwy rządzącej oraz osób świeckich. Pierwsze wzmianki o linii przekazu patriarchów Chan pojawiają się na steli pogrzebowej poświęconej mnichowi Faru (法如 \pinyin{Fǎrú}), który miał być uczniem Piątego Patriarchy Hongrena (弘忍 \pinyin{}). Według tego zapisu linia przekazu Chan prowadziła od Bodhidharmy, poprzez Huike (慧可 \pinyin{Huìkě}), Sengcana (僧璨 \pinyin{Sēngcàn}), Daoxina (道信 \pinyin{Dàoxìn}) i Hongrena, do Faru (Huineng, Schlütter i Teiser 2012: 53-54, 56).

W późniejszym okresie wynikły kontrowersje dotyczące tego, kto był prawowitym Szóstym Patriarchą i prawdziwym spadkobiercą przekazu Chan. Spór ten toczył się w licznych tekstach historycznych i genealogicznych, z których najwcześniejsze pochodzą z początku VIII w.: \textit{Annały przekazu skarbu Dharmy} (傳法寶紀 \pinyin{Chuán fǎbǎo jì}), \textit{Zapisy ludzi i metod w przekazie Sutry Laṅkāvatāra} (楞伽人法誌 \pinyin{Léngqié rén fǎ zhì}) oraz \textit{Zapisy mistrzów i uczniów w przekazie Sutry Laṅkāvatāra} (楞伽師資記 \pinyin{Léngqié shīzī jì}). Każdy z tych tekstów przedstawia nieco odmienną wersję historii linii przekazu Chan. Zapisy te różnią się znacząco zwłaszcza w kwestii przekazu nauk po śmierci Piątego Patriarchy. \textit{Sutra Platformy} wysunęła tezę, jakoby nieznany dotąd Huineng był prawowitym następcą Hongrena. W późniejszym okresie z przyczyn politycznych wersja historii zapisana w \textit{Sutrze Platformy} została uznana za obowiązującą, zaś teksty stojące z nią w sprzeczności zostały zniszczone i dotarły do naszych czasów jedynie dlatego, że zostały znalezione w grotach Mogao (莫高窟 \pinyin{Mògāo kū}) w pobliżu miasta Dunhuang (敦煌 \toponim{Dūnhuáng}) w prowincji Gansu (Huineng, Schlütter i Teiser 2012: 54-55).

Ważnym aspektem tekstu są nauki o tym, że każda czująca istota ma naturę buddy, i że zarówno ludzie świeccy, jak i mnisi mogą z powodzeniem praktykować jego nauki. Tekst opisuje również specjalny rytuał przekazywania mnichom i świeckim praktykującym ,,bezforemnych zasad''. % formless precepts
Były to niektóre z powodów, dla których w roku 796 Huineng został oficjalnie obwołany szóstym patriarchą Chan przez cesarską komisję, a jego dzieło stworzyło podwaliny pod dalszy rozwój szkoły Chan (Huineng, Schlütter i Teiser 2012: 2).

Odwołania do tekstu Sutry Platformy w niniejszej pracy odnoszą się do przekładu na język angielski Philipa B. Yampolsky'ego, wydanej drukiem przez Columbia University Press w roku 1967 i wznowionej w roku 2012.
(Huineng, Schlütter i Teiser 2012: vii)
