\introduction
Buddyzm Chan (禪宗 \pinyin{Chán zōng}) jest, obok Szkoły Czystej Krainy\glosref{Szkoła Czystej Krainy}{glosPureLand} % Ā U+100
 jedną z najważniejszych tradycji buddyzmu w Chinach. Samo słowo \textit{chan} jest chińskim wariantem %odwzorowaniem
sanskryckiego słowa \textit{dhyāna}, które oznacza medytację.
Początkowo termin ten zapisywano w języku chińskim jako \pinyin{chánnà} (禪那), pierwotnie wymawiane \textit{dianna}, co było fonetycznym odwzorowaniem oryginalnego terminu. W późniejszym okresie upowszechniła się skrócona forma \textit{chan}.
W Japonii buddyzm Chan znany jest pod nazwą Zen\footnote{Kanji: {\ipaexgothic 禅}, Rōmaji: \textit{zen}.}, a w Korei Seon\footnote{Hangeul: {\Korean 선}, Revised Romanization of Korean: \textit{seon}}.

Za założyciela buddyzmu Chan uznaje się indyjskiego mistrza Bodhidharmę (菩提達摩 Pútídámó, od skt. \textit{bodhi} `oświecenie' i \textit{dharma} `zjawiska; nauki Buddy').
Pierwsza wzmianka o nim pojawia się w tekście pt. \textit{Luoyang qielan ji} (洛陽伽藍記 \pinyin{Luòyáng qiélán jì} `Zapisy o klasztorach w Luoyangu') autorstwa Yang Xuanzhi (楊衒之), pochodzącym z ok. 547 roku.
Dzieło to opisuje świątynie i klasztory miasta Luoyang (洛陽 \toponim{Luòyáng}) w prowincji Henan, a także opowiada o podróżnych, którzy przybywali tam z dalekich strony, by podziwiać jego wyszukaną architekturę.
Bodhidharma został w tym tekście przedstawiony jako stupięćdziesięcioletni \textit{śrama\d{n}a} (święty mąż) z Persji.
Przybywszy do Luoyangu, Pierwszy Patriarcha miał się zachwycać pięknem tamtejszych świątyń, a zwłaszcza jednej, zwanej Yongning Si (永寧寺, \pinyin{Yǒngníng Sì}, `Świątynia Wiecznego Spokoju').
Budowla ta została wzniesiona w roku 516 i zniszczona w wyniku działań wojennych i katastrof naturalnych w roku 526, a więc Bodhidharma musiał przebywać w mieście w ciągu tych dziesięciu lat
% Autor dzieła nie przywiązywał szczególnej uwagi do osobowości ani dokonań życiowych pielgrzymów, o których pisał.
% Pełnili oni jedynie rolę świadków piękna i przepychu architektury Luoyangu
(McRae 1986: 17).

Kanoniczna biografia Pierwszego Patriarchy oparta jest na przedmowie do przypisywanego mu \textit{Erru sixing lun} (二入四行論 \pinyin{Èrrù sìxíng lùn}, `Traktat o dwóch wejściach i czterech praktykach').
Została ona napisana przez specjalistę od \textit{Śrīmālādevī Si\d{m}hanāda Sūtra}, uczonego imieniem Tanlin (曇林).
Tradycyjnie uważano go za ucznia Bodhidharmy, jednak bardziej prawdopodobne jest, że jego nauczycielem był Huike (大祖慧可 \nazwisko{Dàzǔ Huìkě}), uczeń Pierwszego Patriarchy.
Tanlin podaje, że autor dzieła był trzecim synem pewnego króla z południowych Indii, i że ,,przeszedł przez morza i góry'', by nauczać buddyzmu na północy Chin.
Według tego zapisu jego najważniejszymi uczniami byli Daoyu (道育 \nazwisko{Dàoyù}) i Huike.
W traktacie biograficznym \textit{Xu gaoseng zhuan} (續高僧傳 \pinyin{Xù gāosēng zhuàn}, `Kontynuowane biografie wybitnych mnichów') autorstwa mistrza Daoxuan (道宣 \pinyin{Dàoxuān}) z dynastii Tang\footnote{Dynastia Tang (唐朝 \pinyin{Táng Cháo}) --- dynastia panująca w Chinach w latach 618-907. Okres szybkiego rozwoju buddyzmu chińskiego.} zawarta została natomiast zmodyfikowana wersja biografii z \textit{Erru sixing lun}.
Daoxuan uściślił lakoniczny zapis o podróży Bodhidharmy, podając, że przybył on drogą morską do południowych Chin za czasów dynastii Liu Song\footnote{Dynastia Liu Song (劉宋朝 \pinyin{Liú Sòng Cháo}), zwana też Południową Song (南宋朝 \pinyin{Nán Sòng Cháo}) --- dynastia panująca w południowych Chinach w latach 420-479, pierwsza z czterech Południowych Dynastii (南朝 \pinyin{Nán Cháo})} i przeprawił się przez rzekę Yangzi (揚子 \toponim{Yángzǐ}).
Bodhidharma miał też udzielić Huike przekazu sutry \textit{La\.nkāvatāra}.
Z biografii tej wynika, że Bodhidharma musiał przybyć do Chin przed rokiem 479, kiedy dynastia Liu Song została podbita przez Południową Qi\footnote{Przypis o Nan Qi 南齊}.
(Broughton 1999: 53-56; Buswell 2004: 57).

\textit{Xu gaoseng zhuan} zawiera również biografię praktykującego imieniem Sengfu (僧副), ucznia mistrza dhjāny, nazwanego w tekście imieniem Dharma. Sengfu pochodził z Qixian (祁縣 \toponim{Qíxiàn}) w pobliżu miasta Jinzhong (晉中 \toponim{Jìnzhōng}) w prowincji Shanxi.
Spotkał on swego nauczyciela w jaskini, w której ten mieszkał, a otrzymawszy od niego pouczenia na temat ,,zasad medytacji'' (定學宗 \pinyin{dìngxué zōng}), przyjął ślubowania mnisie. Pomiędzy rokiem 494 a 497 udał się do miasta Jiankang\footnote{Jiankang (建康 \toponim{Jiànkāng}) --- miasto w delcie rzeki Yangzi, stolica sześciu różnych dynastii, m.in. czterech Południowych Dynastii. Ruiny Jiankangu znajdują się w granicach administracyjnych Nankinu (南京 \toponim{Nánjīng}) w prowincji Jiangsu.}, wówczas stolicy południowych Chin, i osiedlił w świątyni (定林下寺 \pinyin{Dìnglín xià sì}) w bezpośrednim sąsiedztwie miasta.
Jeżeli przyjąć, że mistrz Dharma był w istocie Bodhidharmą, tak jak czynią to niektórzy historycy buddyzmu, z tego zapisu wynikałoby, że Pierwszy Patriarcha przewędrował na północ najpóźniej w roku 495, a być może nawet około roku 480.
(McRae 1986: 18-21).

Wedle tradycji Bodhidharma miał otrzymać przekaz Dharmy, pochodzący w nieprzerwanej linii od indyjskiego mistrza Mahakaśjapy\footnote{Skt. \textit{Mahākāśyapa}, chiń. 摩訶迦葉 \nazwisko{Móhējiāshè} lub \nazwisko{Móhējiāyè}}, ucznia historycznego Buddy Siakjamuniego, Siddhārta Gautamy.
(Buswell 2004: 57).
\if 0
\footnotetext{Buddyzm Mahajany (大乘佛教 \pinyin{Dàshèng fójiào} lub \pinyin{Dàchéng fójiào}, `buddyzm Wielkiego Wozu', od skt. \textit{Mahāyāna}, `wielki wóz', nazywany również buddyzmem Wielkiej Drogi) --- jeden z trzech głównych odłamów buddyzmu (dwa pozostałe to Hinajana, tzw. Mała Droga lub Mały Wóz, oraz Wadżrajana, Diamentowa Droga lub Diamentowy Wóz). Filarami Mahajany są wyzwalająca mądrość i współczucie dla wszystkich czujących istot, rozwijane w równowadze. Do buddyzmu Mahajany zalicza się m.in. buddyzm Chan i Zen, Szkołę Czystej Krainy, a także szkołę Gelugpa buddyzmu tybetańskiego.}
Stolica Jiankang 建康 tak jak zdrowie, tylko bez człowieka; obecnie ruiny w granicach administracyjnych Nankinu
\fi % Odniesienie do Mahajany zostało usunięte z tekstu, dlatego tymczasowo to ukrywamy

\section{Pochodzenie \textit{Sutry Platformy} oraz jej przekłady na język angielski}
\textit{Sutra Platformy Szóstego Patriarchy} (六祖壇經 \pinyin{Liùzǔ Tánjīng}) jest apokryficznym tekstem buddyzmu Chan, którego najstarsza zachowana wersja powstała w VIII w. w Chinach. Tekst napisany został częściowo w formie monologu, a częściowo w formie dialogu nauczyciela z uczniami. Nauki w niej zawarte miały zostać wygłoszone przez legendarnego patriarchę buddyzmu Chan, Huinenga (大鑒惠能 \nazwisko{Dàjiàn Huìnéng}, zapisywane również jako 大鑒慧能). % źródło
Huineng jest w \textit{Sutrze Platformy} przedstawiony jako niepiśmienny, prosty człowiek z leżącego poza zasięgiem chińskiej cywilizacji południa.

Pełen tytuł \textit{Sutry Platformy} brzmi \textit{Doktryna nagłego oświecenia Szkoły Południowej, Najwyższa Doskonałość Mądrości Mahajany: Sutra Platformy, przekazana przez Szóstego Patriarchę Huineng w świątyni Dafan, w prefekturze Shao} (南宗頓教最上大乘摩訶般若波羅蜜經六祖惠能大師於韶州大梵寺施法壇經 \pinyin{Nánzōng dùnjiào zuìshàng dàshèng móhēbōrě bōluómì jīng liùzǔ Huìnéng Dàshī yú Shāozhōu Dàfán Sì shīfǎ Tánjīng}). W języku chińskim zwykle nazywana jest w skrócie 壇經 \pinyin{Tánjīng} `Sutra platformy', 六祖壇經 \pinyin{Liùzǔ Tánjīng} `Sutra platformy Szóstego Patriarchy', bądź 六祖大師法寶壇經 \pinyin{Liùzǔ Dàshī Fǎbǎo Tánjīng}, `Skarb Dharmy, Sutra platformy Szóstego Patriarchy'.

Sutra Platformy jest uważana za jedno z najważniejszych dzieł buddyzmu Chan, ponieważ wprowadziła nauki o nagłym oświeceniu (頓教 \pinyin{dùnjiào}, `nagła szkoła, subityzm'), stojące w opozycji do nauk tzw. stopniowej szkoły (漸教 \pinyin{jiànjiào}), i wywołała podział szkoły Chan na odłam północny i południowy. % reference
(Buswell 2004: 347-348).%; McRae, 2004: ).

Chociaż tradycyjne chińskie zapisy dotyczące historii tradycji Chan przedstawiają ją jako nieprzerwaną linię przekazu nauk i doświadczenia, z patriarchy (祖師 \pinyin{zǔshī}) na patriarchę, sięgającą aż do historycznego Buddy, to obecnie uważa się, że takie zapisy nie oddają stanu faktycznego. Miały one nadać przekazowi Chan autentyczność i przedstawić ją jako ortodoksyjną tradycję buddyzmu, pod jakimś względem lepszą od pozostałych, głównie w ramach rywalizacji z innymi tradycjami buddyzmu o wsparcie warstwy rządzącej oraz osób świeckich. Pierwsze wzmianki o linii przekazu patriarchów Chan pojawiają się na steli pogrzebowej poświęconej mnichowi Faru (法如 \nazwisko{Fǎrú}), który miał być uczniem Piątego Patriarchy Hongrena (大滿弘忍 \nazwisko{Dàmǎn Hóngrěn}). Według tego zapisu linia przekazu Chan prowadziła od Bodhidharmy, poprzez Huike (大祖慧可 \nazwisko{Dàzǔ Huìkě}), Sengcana (鑑智僧璨 \nazwisko{Jiànzhì Sēngcàn}), Daoxina (大醫道信 \nazwisko{Dàyī Dàoxìn}) i Hongrena, do Faru (Huineng, Schlütter i Teiser 2012: 53-54, 56).

\if 0
Budowa pracy: najpierw rzekoma autobiografia w formie monologu Huinenga
potem jego nauki w formie dialogów z uczniami
a wszystko to zapisał uczeń Huinenga, Fahai
\fi

Ważnym aspektem tekstu są nauki o tym, że każda czująca istota ma naturę buddy, i że zarówno ludzie świeccy, jak i mnisi mogą z powodzeniem praktykować jego nauki. Tekst opisuje również specjalny rytuał przekazywania mnichom i świeckim praktykującym ,,bezforemnych zasad''. % formless precepts
Były to niektóre z powodów, dla których w roku 796 Huineng został oficjalnie obwołany szóstym patriarchą Chan przez cesarską komisję, a jego dzieło stworzyło podwaliny pod dalszy rozwój szkoły Chan (Huineng, Schlütter i Teiser 2012: 2).

\section{Wstęp techniczny}
Terminy chińskie w niniejszej pracy podane są w nawiasach w znakach tradycyjnych, ponieważ wiernie oddają one pierwotną semantyczno-fonetyczną budowę znaków. Domyślną transkrypcją terminów chińskich jest \textit{Hanyu Pinyin} (漢語拼音 \pinyin{Hànyǔ Pīnyīn}) z oznaczonymi tonami, z wyjątkiem chińskich nazwisk, do których nie podano transkrypcji.

Odwołania do tekstu \textit{Sutry Platformy} w niniejszej pracy odnoszą się do przekładu na język angielski Philipa B. Yampolsky'ego, wydanej drukiem przez Columbia University Press w roku 1967 i wznowionej w roku 2012. [[Electronic Tripitaka]]

Słowo ,,Budda'', zapisywane wielką literą, oznacza historycznego Buddę Siakjamuniego, tj. Siddhārta Gautamę. Zapisywane małą literą ,,budda'' oznacza stan umysłu.
%(Huineng, Schlütter i Teiser 2012: vii)
