\documentclass[12pt]{wzmgr}
\special{papersize=210mm,297mm}
% Wersja XeLaTeX
\usepackage[polish]{babel}
\usepackage[no-math]{fontspec}
\usepackage[usenames,dvipsnames,svgnames,table]{xcolor}
\usepackage{xeCJK,makeidx,sectsty}
\usepackage[compact]{titlesec}
\usepackage{setspace,etoolbox,csquotes,indentfirst,tikz}
\usepackage[top=2.5cm,left=2.5cm,right=2.5cm,bottom=2.5cm]{geometry}
\usepackage{url}
\makeatletter
\g@addto@macro{\UrlBreaks}{\UrlOrds}
\makeatother
\setmainfont[Mapping=tex-text]{Minion Pro}
\setsansfont[Scale=0.88]{IPAexGothic}
\setCJKmainfont{SimSun}
\setCJKsansfont{SimHei}
\setCJKmonofont{SimHei}
\newCJKfontfamily{\ipaexgothic}{IPAexGothic}
\newCJKfontfamily{\korm}{NanumMyeongjo}
\newfontfamily\pyfont{Times New Roman}
\newtoggle{brudnopis}\newtoggle{prowincjepy}
\togglefalse{brudnopis} % change \toggletrue to \togglefalse to disable comments
\toggletrue{prowincjepy}
\newcommand{\prowincja}[1]{\iftoggle{prowincjepy}{ (#1)}{}}
\newcommand{\Korean}{\korm\CJKspace}
\newcommand{\toponim}[1]{{\pyfont #1}}
\newcommand{\nazwisko}[1]{{\pyfont #1}}
\newcommand{\pinyin}[1]{{\pyfont\itshape #1}}
\newcommand{\fnm}{\footnotemark}
\newcommand\glos[2][]{\hspace*{1.5pt}\tikz[overlay]\node[inner sep=1.5pt, anchor=text, rectangle, draw=blue, dotted, thick, #1] {#2};\phantom{#2}\hspace*{1.5pt}}
% \newcommand{\glosref}[2]{\footnote{\textbf{#1} --- patrz: \textit{Glosariusz} na stronie \pageref{#2}.}}
\newcommand{\glosref}[2]{}
\author{Karol Moroz}
\title{Narodziny legendy\\Sutra Platformy Szóstego Patriarchy Huineng}
\nralbumu{386956}
\kierunek{sinologia}
\opiekun{dr Maria Kurpaska}
\email{dmuhafc@gmail.com}
\UniversityName{Uniwersytet im. Adama Mickiewicza --- Wydział Neofilologii}
\nrwersji{0.0}
\miejsce{Poznań}
%\hyphenation{ La\.n-kā-va-tā-ra}
\makeindex

\begin{document}
\onehalfspacing
\maketitle
\titleformat{\chapter}[hang]
  {\normalfont\huge\color[HTML]{000099}}{{\thechapter}}{20pt}{}
\titlespacing{\chapter}{0pt}{0.5em}{0.5em}
\sectionfont{\color[HTML]{990000}}
\introduction
Buddyzm Chan (禪宗 \pinyin{Chán zōng}) jest, obok Szkoły Czystej Krainy\footnote{Szkoła Czystej Krainy, zwana również Szkołą Czystej Ziemi lub amidyzmem --- tradycja buddyzmu chińskiego, w której za najważniejszą postać przyjmuje się \textit{Buddę Amitabhę} (阿彌陀佛 \pinyin{Āmìtuófó}). Celem praktyki tej tradycji jest odrodzenie po śmierci w Czystej Krainie tego buddy, \textit{Sukhavati} (w języku chińskim nazywana 極樂 \pinyin{Jílè}, 安樂 \pinyin{Ānlè} lub 西天 \pinyin{Xītiān}).} % Ā U+100
 jedną z najważniejszych tradycji buddyzmu w Chinach. Samo słowo \textit{chan} jest chińskim wariantem %odwzorowaniem
sanskryckiego słowa \textit{dhyāna}, które oznacza medytację.
Początkowo termin ten zapisywano w języku chińskim jako \pinyin{chánnà} (禪那), pierwotnie wymawiane \textit{dianna}, co było fonetycznym odwzorowaniem oryginalnego terminu. W późniejszym okresie upowszechniła się skrócona forma \textit{chan}.
W Japonii buddyzm Chan znany jest pod nazwą Zen\footnote{Kanji: {\ipaexgothic 禅}, Rōmaji: \textit{zen}.}, a w Korei Seon\footnote{Hangeul: {\Korean 선}, Revised Romanization of Korean: \textit{seon}}.

Za założyciela buddyzmu Chan uznaje się indyjskiego mistrza Bodhidharmę (菩提達摩 Pútídámó, od skt. \textit{bodhi} `oświecenie' i \textit{dharma} `zjawiska; nauki Buddy').
Pierwsza wzmianka o nim pojawia się w tekście pt. \textit{Luoyang qielan ji} (洛陽伽藍記 \pinyin{Luòyáng qiélán jì} `Zapisy o klasztorach w Luoyangu') autorstwa Yang Xuanzhi (楊衒之), pochodzącym z ok. 547 roku.
Dzieło to opisuje świątynie i klasztory miasta Luoyang (洛陽 \toponim{Luòyáng}) w prowincji Henan, a także opowiada o podróżnych, którzy przybywali tam z dalekich strony, by podziwiać jego wyszukaną architekturę.
Bodhidharma został w tym tekście przedstawiony jako stupięćdziesięcioletni \textit{śrama\d{n}a} (święty mąż) z Persji.
Przybywszy do Luoyangu, Pierwszy Patriarcha miał się zachwycać pięknem tamtejszych świątyń, a zwłaszcza jednej, zwanej Yongning Si (永寧寺, \pinyin{Yǒngníng Sì}, `Świątynia Wiecznego Spokoju').
Budowla ta została wzniesiona w roku 516 i zniszczona w wyniku działań wojennych i katastrof naturalnych w roku 526, a więc Bodhidharma musiał przebywać w mieście w ciągu tych dziesięciu lat
% Autor dzieła nie przywiązywał szczególnej uwagi do osobowości ani dokonań życiowych pielgrzymów, o których pisał.
% Pełnili oni jedynie rolę świadków piękna i przepychu architektury Luoyangu
(McRae 1986: 17).

Kanoniczna biografia Pierwszego Patriarchy oparta jest na przedmowie do przypisywanego mu \textit{Erru sixing lun} (二入四行論 \pinyin{Èrrù sìxíng lùn}, `Traktat o dwóch wejściach i czterech praktykach').
Została ona napisana przez specjalistę od \textit{Śrīmālādevī Si\d{m}hanāda Sūtra}, uczonego imieniem Tanlin (曇林).
Tradycyjnie uważano go za ucznia Bodhidharmy, jednak bardziej prawdopodobne jest, że jego nauczycielem był Huike (大祖慧可 \nazwisko{Dàzǔ Huìkě}), uczeń Pierwszego Patriarchy.
Tanlin podaje, że autor dzieła był trzecim synem pewnego króla z południowych Indii, i że ,,przeszedł przez morza i góry'', by nauczać buddyzmu na północy Chin.
Według tego zapisu jego najważniejszymi uczniami byli Daoyu (道育 \nazwisko{Dàoyù}) i Huike.
W traktacie biograficznym \textit{Xu gaoseng zhuan} (續高僧傳 \pinyin{Xù gāosēng zhuàn}, `Kontynuowane biografie wybitnych mnichów') autorstwa mistrza Daoxuan (道宣 \pinyin{Dàoxuān}) z dynastii Tang\footnote{Dynastia Tang (唐朝 \pinyin{Táng Cháo}) --- dynastia panująca w Chinach w latach 618-907. Okres szybkiego rozwoju buddyzmu chińskiego.} zawarta została natomiast zmodyfikowana wersja biografii z \textit{Erru sixing lun}.
Daoxuan uściślił lakoniczny zapis o podróży Bodhidharmy, podając, że przybył on drogą morską do południowych Chin za czasów dynastii Liu Song\footnote{Dynastia Liu Song (劉宋朝 \pinyin{Liú Sòng Cháo}), zwana też Południową Song (南宋朝 \pinyin{Nán Sòng Cháo}) --- dynastia panująca w południowych Chinach w latach 420-479, pierwsza z czterech Południowych Dynastii (南朝 \pinyin{Nán Cháo})} i przeprawił się przez rzekę Yangzi (揚子 \toponim{Yángzǐ}).
Bodhidharma miał też udzielić Huike przekazu sutry \textit{La\.nkāvatāra}.
Z biografii tej wynika, że Bodhidharma musiał przybyć do Chin przed rokiem 479, kiedy dynastia Liu Song została podbita przez Południową Qi\footnote{Przypis o Nan Qi 南齊}.
(Broughton 1999: 53-56; Buswell 2004: 57).

\textit{Xu gaoseng zhuan} zawiera również biografię praktykującego imieniem Sengfu (僧副), ucznia mistrza dhjāny, nazwanego w tekście imieniem Dharma. Sengfu pochodził z Qixian (祁縣 \toponim{Qíxiàn}) w pobliżu miasta Jinzhong (晉中 \toponim{Jìnzhōng}) w prowincji Shanxi.
Spotkał on swego nauczyciela w jaskini, w której ten mieszkał, a otrzymawszy od niego pouczenia na temat ,,zasad medytacji'' (定學宗 \pinyin{dìngxué zōng}), przyjął ślubowania mnisie. Pomiędzy rokiem 494 a 497 udał się do miasta Jiankang\footnote{Jiankang (建康 \toponim{Jiànkāng}) --- miasto w delcie rzeki Yangzi, stolica sześciu różnych dynastii, m.in. czterech Południowych Dynastii. Ruiny Jiankangu znajdują się w granicach administracyjnych Nankinu (南京 \toponim{Nánjīng}) w prowincji Jiangsu.}, wówczas stolicy południowych Chin, i osiedlił w świątyni (定林下寺 \pinyin{Dìnglín xià sì}) w bezpośrednim sąsiedztwie miasta.
Jeżeli przyjąć, że mistrz Dharma był w istocie Bodhidharmą, tak jak czynią to niektórzy historycy buddyzmu, z tego zapisu wynikałoby, że Pierwszy Patriarcha przewędrował na północ najpóźniej w roku 495, a być może nawet około roku 480.
(McRae 1986: 18-21).

Wedle tradycji Bodhidharma miał otrzymać przekaz Dharmy, pochodzący w nieprzerwanej linii od indyjskiego mistrza Mahakaśjapy\footnote{Skt. \textit{Mahākāśyapa}, chiń. 摩訶迦葉 \nazwisko{Móhējiāshè} lub \nazwisko{Móhējiāyè}}, ucznia historycznego Buddy Siakjamuniego, Siddhārta Gautamy.
(Buswell 2004: 57).
\if 0
\footnotetext{Buddyzm Mahajany (大乘佛教 \pinyin{Dàshèng fójiào} lub \pinyin{Dàchéng fójiào}, `buddyzm Wielkiego Wozu', od skt. \textit{Mahāyāna}, `wielki wóz', nazywany również buddyzmem Wielkiej Drogi) --- jeden z trzech głównych odłamów buddyzmu (dwa pozostałe to Hinajana, tzw. Mała Droga lub Mały Wóz, oraz Wadżrajana, Diamentowa Droga lub Diamentowy Wóz). Filarami Mahajany są wyzwalająca mądrość i współczucie dla wszystkich czujących istot, rozwijane w równowadze. Do buddyzmu Mahajany zalicza się m.in. buddyzm Chan i Zen, Szkołę Czystej Krainy, a także szkołę Gelugpa buddyzmu tybetańskiego.}
Stolica Jiankang 建康 tak jak zdrowie, tylko bez człowieka; obecnie ruiny w granicach administracyjnych Nankinu
\fi % Odniesienie do Mahajany zostało usunięte z tekstu, dlatego tymczasowo to ukrywamy

\section{Pochodzenie \textit{Sutry Platformy} oraz jej przekłady na język angielski}
\textit{Sutra Platformy Szóstego Patriarchy} (六祖壇經 \pinyin{Liùzǔ Tánjīng}) jest apokryficznym tekstem buddyzmu Chan, którego najstarsza zachowana wersja powstała w VIII w. w Chinach. Tekst napisany został częściowo w formie monologu, a częściowo w formie dialogu nauczyciela z uczniami. Nauki w niej zawarte miały zostać wygłoszone przez legendarnego patriarchę buddyzmu Chan, Huinenga (大鑒惠能 \nazwisko{Dàjiàn Huìnéng}, zapisywane również jako 大鑒慧能). % źródło
Huineng jest w \textit{Sutrze Platformy} przedstawiony jako niepiśmienny, prosty człowiek z leżącego poza zasięgiem chińskiej cywilizacji południa.

Pełen tytuł \textit{Sutry Platformy} brzmi \textit{Doktryna nagłego oświecenia Szkoły Południowej, Najwyższa Doskonałość Mądrości Mahajany: Sutra Platformy, przekazana przez Szóstego Patriarchę Huineng w świątyni Dafan, w prefekturze Shao} (南宗頓教最上大乘摩訶般若波羅蜜經六祖惠能大師於韶州大梵寺施法壇經 \pinyin{Nánzōng dùnjiào zuìshàng dàshèng móhēbōrě bōluómì jīng liùzǔ Huìnéng Dàshī yú Shāozhōu Dàfán Sì shīfǎ Tánjīng}). W języku chińskim zwykle nazywana jest w skrócie 壇經 \pinyin{Tánjīng} `Sutra platformy', 六祖壇經 \pinyin{Liùzǔ Tánjīng} `Sutra platformy Szóstego Patriarchy', bądź 六祖大師法寶壇經 \pinyin{Liùzǔ Dàshī Fǎbǎo Tánjīng}, `Skarb Dharmy, Sutra platformy Szóstego Patriarchy'.

Sutra Platformy jest uważana za jedno z najważniejszych dzieł buddyzmu Chan, ponieważ wprowadziła nauki o nagłym oświeceniu (頓教 \pinyin{dùnjiào}, `nagła szkoła, subityzm'), stojące w opozycji do nauk tzw. stopniowej szkoły (漸教 \pinyin{jiànjiào}), i wywołała podział szkoły Chan na odłam północny i południowy. % reference
(Buswell 2004: 347-348).%; McRae, 2004: ).

Chociaż tradycyjne chińskie zapisy dotyczące historii tradycji Chan przedstawiają ją jako nieprzerwaną linię przekazu nauk i doświadczenia, z patriarchy (祖師 \pinyin{zǔshī}) na patriarchę, sięgającą aż do historycznego Buddy, to obecnie uważa się, że takie zapisy nie oddają stanu faktycznego. Miały one nadać przekazowi Chan autentyczność i przedstawić ją jako ortodoksyjną tradycję buddyzmu, pod jakimś względem lepszą od pozostałych, głównie w ramach rywalizacji z innymi tradycjami buddyzmu o wsparcie warstwy rządzącej oraz osób świeckich. Pierwsze wzmianki o linii przekazu patriarchów Chan pojawiają się na steli pogrzebowej poświęconej mnichowi Faru (法如 \nazwisko{Fǎrú}), który miał być uczniem Piątego Patriarchy Hongrena (大滿弘忍 \nazwisko{Dàmǎn Hóngrěn}). Według tego zapisu linia przekazu Chan prowadziła od Bodhidharmy, poprzez Huike (大祖慧可 \nazwisko{Dàzǔ Huìkě}), Sengcana (鑑智僧璨 \nazwisko{Jiànzhì Sēngcàn}), Daoxina (大醫道信 \nazwisko{Dàyī Dàoxìn}) i Hongrena, do Faru (Huineng, Schlütter i Teiser 2012: 53-54, 56).

\if 0
Budowa pracy: najpierw rzekoma autobiografia w formie monologu Huinenga
potem jego nauki w formie dialogów z uczniami
a wszystko to zapisał uczeń Huinenga, Fahai
\fi

Ważnym aspektem tekstu są nauki o tym, że każda czująca istota ma naturę buddy, i że zarówno ludzie świeccy, jak i mnisi mogą z powodzeniem praktykować jego nauki. Tekst opisuje również specjalny rytuał przekazywania mnichom i świeckim praktykującym ,,bezforemnych zasad''. % formless precepts
Były to niektóre z powodów, dla których w roku 796 Huineng został oficjalnie obwołany szóstym patriarchą Chan przez cesarską komisję, a jego dzieło stworzyło podwaliny pod dalszy rozwój szkoły Chan (Huineng, Schlütter i Teiser 2012: 2).

\section{Wstęp techniczny}
Terminy chińskie w niniejszej pracy podane są w nawiasach w znakach tradycyjnych, ponieważ wiernie oddają one pierwotną semantyczno-fonetyczną budowę znaków. Domyślną transkrypcją terminów chińskich jest \textit{Hanyu Pinyin} (漢語拼音 \pinyin{Hànyǔ Pīnyīn}) z oznaczonymi tonami, z wyjątkiem chińskich nazwisk, do których nie podano transkrypcji.

Odwołania do tekstu \textit{Sutry Platformy} w niniejszej pracy odnoszą się do przekładu na język angielski Philipa B. Yampolsky'ego, wydanej drukiem przez Columbia University Press w roku 1967 i wznowionej w roku 2012. [[Electronic Tripitaka]]

Słowo ,,Budda'', zapisywane wielką literą, oznacza historycznego Buddę Siakjamuniego, tj. Siddhārta Gautamę. Zapisywane małą literą ,,budda'' oznacza stan umysłu.
%(Huineng, Schlütter i Teiser 2012: vii)

\chapter{Biografie Huinenga}
Życie Huinenga, Szóstego Patriarchy Chan, jest owiane tajemnicą. Jego imię pojawia się w kronice pt. \textit{Księga przekazu lampy z okresu Jingde} (景德傳燈錄 \pinyin{Jǐngdé chuán dēng lù}) jako jednego z dziesięciu głównych uczniów piątego patriarchy Hongren, z owego tekstu nie wynika jednak, by był postacią szczególnie ważną dla rozwoju całej szkoły Chan. W tekście tym wspomniano, że Huineng żył i nauczał w miejscowości Caoqi (曹溪 \toponim{Cáoqī}, również: \toponim{Cáoxī}).

Imię Huineng pojawia się również w pewnym tekście z grot Dunhuang, upamiętniającym Piątego Patriarchę, Hongrena (大滿弘忍 \nazwisko{Dàmǎn Hóngrěn}), jednak tekst ów nie mówi nic o przypisywanych Huinengowi doktrynach. Kanoniczna biografia Huinenga oparta jest na przypisywanej mu \textit{Sutrze Platformy}. Pierwszą osobą, która przedstawiła Huinenga jako świętego, był Shenhui (菏澤神會 \nazwisko{Hézé Shénhuì}, 684-758). Biografia Szóstego Patriarchy w takiej wersji, jak opisana w tekście \textit{Sutry Platformy} z Dunhuang, jest najprawdopodobniej uzupełnioną i zmienioną wersją jego opowieści. (McRae 2004: 68).

\section{Biografia Huinenga według \textit{Sutry Platformy} w wersji z Dunhuang}
Jak podaje tekst \textit{Sutry Platformy}, Huineng urodził się w miejscowości Xinxing w regionie Nanhai (南海新興 \toponim{Nánhǎi Xīnxīng}, obecnie prowincja Guangdong). Za ramy czasowe jego życia przyjmuje się lata 638-713. Szósty Patriarcha jest w tym tekście przedstawiany jako ubogi, niepiśmienny człowiek świecki z południa Chin.

Jak podaje tekst, jego ojciec był urzędnikiem z regionu Fanyang (范陽 \toponim{Fànyáng}), obecnie miasto Zhuozhou (涿州 \toponim{Zhuōzhōu}) w prowincji Hebei), lecz został odwołany ze stanowiska i skazany na banicję. W związku z tym musiał przenieść się z całą rodziną do Xinxing, gdzie niedługo później zmarł. Po jego śmierci, Huineng trudnił się zbieraniem i sprzedażą drewna na opał.

% ===============
% TAJEMNICZY MĘŻCZYZNA I PODRÓŻ
% ===============
Pewnego dnia, gdy dwudziestodwuletni Huineng sprzedawał drewno na targowisku, pewien klient zażyczył sobie, żeby drewno zostało przyniesione do jego sklepu. Huineng dostarczył drewno i dostał za nie pieniądze, a kiedy wyszedł ze sklepu, spotkał mężczyznę, który recytował na ulicy Sutrę Diamentową (金剛經 \pinyin{Jīngāng jīng}, skt. \textit{Vajracchedikā Prajñāpāramitā Sūtra}). Usłyszawszy ów tekst, Huineng uzyskał wgląd w naturę swego umysłu i osiągnął oświecenie. Następnie spytał tajemniczego mężczyznę, skąd przybył. Ten odpowiedział, że przybył z klasztoru Dongshan (東山寺 \pinyin{Dōngshān sì}) na górze Fengmushan (憑墓山 \toponim{Féngmù shān}) w powiecie Huangmei (黃梅懸 \toponim{Huángméi xiàn}) w Qizhou (蘄州 \toponim{Qízhōu}), którego opatem był Piąty Patriarcha, Hongren, i gdzie przebywało około tysiąca mnichów. Patriarcha miał zalecić mnichom, aby recytowali ową sutrę, ponieważ dzięki tej praktyce można szybko osiągnąć oświecenie (Huineng i Hsüan Hua 1977: bez nru strony; Huineng i Yampolsky 2012: 127).

Według tej biografii, niedługo po spotkaniu tajemniczego mężczyzny, Huineng spotkał kogoś, kto poradził mu udać się do klasztoru Dongshan, aby poprosić Patriarchę o nauki, i dał mu pieniądze, aby mógł zaaranżować opiekę dla swej matki.

Kiedy Huineng przybył do klasztoru, Piąty Patriarcha Hongren zapytał go, skąd przybył i w jakiej sprawie przybył do patriarchy. Odparł, że pochodzi z Kantonu i przyszedł oddać cześć patriarsze, i że nie prosi o nic prócz Dharmy. Patriarcha stwierdził wówczas, że Huineng, jako \textit{geliao} (獦獠 \pinyin{géliáo}, `barbarzyńca'), niegodny jest otrzymania nauk. Obszar obecnego Kantonu był wówczas zamieszkany przez niechińskie ludy, mówiące własnymi językami, posiadające własną, niechińską kulturę i nieżyjące zgodnie z naukami Buddy --- mieszkańcy południa polowali bowiem i jedli mięso. Dla wielu ówczesnych buddystów nie do pomyślenia było dla mnichów, by człowiek z południa mógł otrzymać nauki od Patriarchy i osiągnąć oświecenie. Huineng odparł wtedy, że ludzie dzielą się na tych z południa i tych z północy, ale takie podziały nie mają wpływu na ich naturę buddy. Patriarcha uznał, że Huineng dobrze rozumiał nauki Buddy, lecz w obawie, że inni uczniowie mogliby zrobić mu krzywdę, kazał mu iść pracować w stajni. Tam, przez następne osiem miesięcy, Huineng rąbał drewno i młócił zboże (Huineng, Wong i Humphreys 1998: rozdział 1; Huineng, Schlütter i Teiser 2012: 27).

% ===============
% KONKURS POEZJI
% ===============
Według biografii zawartej w \textit{Sutrze Platformy}, pewnego dnia Patriarcha Hongren zwołał zebranie wszystkich uczniów i ogłosił:
,,Dla ludzi w tym świecie narodziny i śmierć są doniosłymi kwestiami. Całymi dniami składacie podarowania i poszukujecie tylko pól błogosławieństw, ale nie staracie się wyzwolić z pełnego goryczy oceanu uwarunkowanej egzystencji\fnm. Wasze własne ego stoi na drodze do błogosławieństw. Jak w takiej sytuacji możecie osiągnąć wyzwolenie? Powróćcie teraz do swoich cel i spójrzcie w swój umysł. Ludzie mądrzy samoistnie pojmą prawdziwą naturę \textit{pradżni}\fnm. Niech każdy z was napisze wiersz i przyniesie mi go. Przeczytam każdy z nich, a jeżeli jest wśród was ktoś, kto rozpoznał swoją prawdziwą naturę, przekażę mu swoją szatę i Dharmę, i uczynię go Szóstym Patriarchą. Spieszcie się!''
(Huineng i Yampolsky 2012: 128).%
%
\footnotetext{Idea ,,pola błogosławieństw'' (福田 \pinyin{fútián}, skt. \textit{pu\d{n}yak\d{s}etra}) jest związana z buddyjską koncepcją karmy jako prawa przyczyny i skutku. Oznacza stan, w którym dana osoba zgromadziła bardzo wiele dobrej karmy w rezultacie praktyki szczodrości (skt. \textit{dānā}), pierwszej z tzw. Sześciu Paramit lub Sześciu Wyzwalających Działań (pozostałe pięć to właściwe działanie, cierpliwość, radosny wysiłek, medytacja i mądrość). Słowo \textit{dānā} (布施 \pinyin{bùshī}) pojawia się m.in. w \textit{Dānādhikāramahāyānasūtra} (佛说布施经 \pinyin{Fóshuō bùshī jīng}): “若求勝妙福報而行施時,慈心不殺離諸嫉妒,正見相應遠於不善,堅持禁戒親近善友,閉惡趣門開生天路,自利利他其心平等。若如是施,是真布施,是大福田。'' Piąty Patriarcha Hongren krytykował tu swoich uczniów, ponieważ praktyka szczodrości jest wprawdzie w buddyzmie postrzegana jako pozytywne działanie, jednak nie wystarcza ona do osiągnięcia ostatecznego wyzwolenia z samsary (Anonim 2007; Nydahl 2010).}%
%
\footnotetext{Pradżnia (skt. \textit{prajñā}, w języku chińskim nazywana 慧 \pinyin{huì}, 智 \pinyin{zhì} lub 智慧 \pinyin{zhìhuì} --- wszystkie trzy terminy oznaczają `mądrość' lub fonetycznie 般若 \pinyin{bōrě}, to, obok współczucia (悲 \pinyin{bēi} `litość', skt. \textit{karuṇā}), jedna z dwóch najważniejszych cnót buddyzmu Mahajany. Termin ten można rozumieć na wiele sposobów, zależnie od tradycji, praktykowanej ścieżki i metody interpretacji. Tu odnosi się do prawidłowego, ponadintelektualnego zrozumienia prawdziwej natury zjawisk.}
% Pradżniaparamita! Tekst należy do tego trendu

Mnisi stwierdzili zgodnie: ,,Nie ma sensu oczyszczać umysłu i zadawać sobie trudu układania wiersza dla patriarchy. Shenxiu (神秀 \nazwisko{Shénxiù}), przewodniczący kongregacji, jest naszym nauczycielem. Kiedy on zostanie patriarchą, możemy liczyć na jego wsparcie. Dlatego nie będziemy układać wierszy.'' Żaden z nich nie podjął się więc tego zadania (Huineng i Yampolsky 2012: 127).

Shenxiu zaś dręczyły wątpliwości. Z jednej strony nie uważał, żeby jego zrozumienie nauk i urzeczywistnienie były wystarczające do przyjęcia stanowiska patriarchy, z drugiej zaś pragnął otrzymać przekaz Dharmy. Przyjęcie Dharmy dla pożytku istot byłoby bowiem pożądane i chwalebne, jednak dążenie do objęcia stanowiska patriarchy byłoby niewłaściwe. Wychodził z założenia, że jeżeli Piąty Patriarcha Hongren uzna, że jego urzeczywistnienie natury umysłu jest niewystarczające, to będzie musiał porzucić dążenia do zostania dzierżawcą linii przekazu. Wreszcie skomponował wiersz i wymknął się w nocy ze swojej celi, by napisać go na ścianie, w miejscu, gdzie miały zostać namalowane sceny z sutry \textit{La\.nkāvatāra} (楞伽經 \pinyin{Léngqié jīng}) (McRae 2004: 62):

\vspace*{6pt}
\begin{minipage}[t]{0.4\textwidth}
\begin{verse}
身是菩提樹\\
心如明鏡臺\\
時時勤佛拭\\
莫使有塵埃
\end{verse}
\end{minipage}
\begin{minipage}[t]{0.6\textwidth}
\itshape
\begin{verse}
Ciało jest drzewem Bodhi,\\
Umysł --- jasną lustrzaną podstawą.\\
Czyść ją stale i gorliwie,\\
Nie pozwalając aby przylgnął kurz.\fnm
\end{verse}
\end{minipage}
\label{ShenxiuVerse}
\vspace*{6pt}
\footnotetext{Niniejszy wiersz, jak również następny, został przytoczony w języku chińskim za \textit{Chinese Electronic Tripitaka} T48, no. 2007, w języku polskim za tekstem \textit{Sutry Szóstego Patriarchy Zen} nieznanego tłumacza.}

O poranku, Piąty Patriarcha Hongren ujrzał wiersz napisany przez Shenxiu na ścianie i uznał, że wiersz ten mógł przynieść ludziom wiele pożytku. Zapłacił wówczas malarzowi, któremu zlecił namalowanie scen z sutry \textit{La\.nkāvatāra}, i odwołał zamówienie. Zwołał całe zgromadzenie i nakazał mnichom recytować wiersz. Patriarcha spytał Shenxiu, czy to on jest jego autorem, oznaczałoby to bowiem, że jest on właściwym spadkobiercą Dharmy i jego następcą na stanowisku patriarchy. Powiedział też jednak, że wiersz nie wskazuje na to, aby Shenxiu rozpoznał już naturę swojego umysłu. Wiersz nadawał się do recytacji przez zwykłych ludzi i dawał gwarancję, że praktykujące nie upadnie do niższych sfer egzystencji\fnm, jednak takie niepełne zrozumienie nie wystarczało do rozpoznania prawdziwej natury umysłu. Powiedział, że przekaże mu nauki i szatę patriarchy, jeżeli w ciągu dwóch dni uda mu się osiągnąć ostateczne urzeczywistnienie. Shenxiu rozmyślał przez wiele dni, ale nie udało mu się skomponować nic lepszego. (Huineng i Yampolsky 2012: 131).
%
\footnotetext{Według kosmologii buddyjskiej istoty krążące w samsarze, tj. uwarunkowanej egzystencji, od niemającego początku czasu odradzają się w jednej z sześciu sfer egzystencji, zależnie od swojej karmy i indywidualnych skłonności. Trzy z nich, sfera niebiańska (天道 \pinyin{tiāndào}, skt. \textit{devaloka}), którą zamieszkują bogowie, sfera półbogów lub asurów (阿修羅 \pinyin{Āxiūluódào}) i sfera ludzi (人道 \pinyin{réndào}), nazywa się trzema wyższymi sferami egzystencji (三善道 \pinyin{sān shàndào}), ponieważ życie w tych sferach jest relatywnie przyjemne. Trzy niższe sfery egzystencji, sfera zwierząt (畜牲道 \pinyin{chùshēngdào}), sfera głodnych duchów lub pretów (餓鬼道 \pinyin{èguǐdào}) oraz sfery piekielne (地獄道 \pinyin{dìyùdào}), w których życie pełne jest cierpienia, nazywane są niższymi sferami egzystencji (三惡道 \pinyin{sān èdào}).}

Pewnego dnia młody mnich-akolita przechodził koło stajni, w której pracował akurat Huineng, recytując wiersz Shenxiu. Huineng zrozumiał, że autor wiersza nie rozpoznał jeszcze natury swojego umysłu. Spytał mnicha, co ten recytował. Mnich odparł, że Patriarcha Hongren kazał wszystkim swoim uczniom skomponować wiersz i przynieść mu go, aby określić, kto otrzyma przekaz Dharmy i zostanie patriarchą. Wiersz o pustości, który właśnie recytował, został skomponowany przez mnicha o imieniu Shenxiu. Piąty Patriarcha nakazał wszystkim swoim uczniom recytować go, mówiąc, że ci, którzy go urzeczywistnią, zobaczą swoją prawdziwą naturę, a ci, którzy praktykują zgodnie z nim, osiągną wyzwolenie.

Huineng poprosił mnicha, aby ten zaprowadził go do miejsca, gdzie na ścianie wymalowany był wiersz Shenxiu. Ponieważ nie umiał czytać, poprosił kogoś, by przeczytał mu ten wiersz na głos. Usłyszawszy go, osiągnął ostateczne urzeczywistnienie. Huineng ułożył własny wiersz i poprosił kogoś o napisanie go na ścianie (Huineng i Yampolsky 2012: 131).

\begin{minipage}[t]{0.4\textwidth}
\begin{verse}
菩提本無樹\\
明鏡亦無臺\\
佛性常清淨\\
何處有塵埃
\end{verse}
\end{minipage}
\begin{minipage}[t]{0.6\textwidth}
\itshape
\begin{verse}
Sama istota Bodhi nie ma drzewa,\\
Nie ma też jasnej lustrzanej podstawy.\\
W rzeczywistości nie ma niczego,\\
Cóż miałoby przyciągać jakikolwiek kurz?
\end{verse}
\end{minipage}
\label{HuinengVerse}
\vspace*{6pt}

Mnisi ze zgromadzenia byli pod wielkim wrażeniem wiersza Huinenga, a Piąty Patriarcha Hongren stwierdził na jego podstawie, że Huineng miał już wówczas ponadprzeciętne zrozumienie natury zjawisk, ale dla bezpieczeństwa Huinenga oznajmił zgromadzeniu, że wciąż nie było to pełne urzeczywistnienie (Huineng i Yampolsky 2012: 132).

% ===============
% PRZEKAZ DHARMY I WYGNANIE
% ===============
Patriarcha przywołał go do siebie w nocy i udzielił mu wyjaśnień do \textit{Sutry Diamentowej}, dzięki której Huineng natychmiast rozpoznał naturę umysłu. Hongren przekazał mu również nauki o spontanicznym oświeceniu oraz szatę, insygnium patriarchatu. Przekazał mu również ostatnie pouczenia: ,,Mianuję cię Szóstym Patriarchą. Szata jest tego dowodem i jest przekazywana z pokolenia na pokolenie. Moja Dharma musi być przekazywana z umysłu na umysł. Spraw, by ludzie rozpoznali swoją prawdziwą naturę. (\ldots) Od czasów starożytnych przekaz Dharmy był równie słaby, jak zwisający sznurek. Jeżeli pozostaniesz tutaj, inni ludzie zrobią ci krzywdę. Musisz więc niezwłocznie odejść.'' (Huineng i Yampolsky 2012: 133).

Huineng udał się na południe. Jego śladem podążyło kilkuset ludzi, pragnących go zabić i siłą odebrać od niego szatę i Dharmę. Po dwóch miesiącach miał dotrzeć do miejsca zwanego \textit{Dayu ling} (大庾嶺 \toponim{Dàyǔ líng}). Jest to pasmo górskie, znajdujące się pomiędzy południowym wschodem prowincji Jiangxi a prowincją Guangzhou. Tam doścignął go mnich imieniem Huiming (惠明 \pinyin{Huìmíng}***) lub Huishun (惠順 \pinyin{Huìshùn}), były generał, człowiek szorstki i porywczy. Huiming groził Huinengowi, który bez wahania oddał mu szatę, lecz Huiming nie chciał jej przyjąć, mówiąc, że przybył wyłącznie po to, by otrzymać przekaz Dharmy. Huineng miał przekazać mu Dharmę na szczycie góry, a gdy Huiming usłyszał nauki, natychmiast osiągnął oświecenie. Następnie Huineng polecił Huimingowi udać się na północ i nauczać tamtejszych ludzi (Huineng i Yampolsky 2012: 134).

W związku z prześladowaniami, Huineng schronił się w miejscu zwanym Caoqi (曹溪, także: Caoxi), gdzie przez piętnaście lat ukrywał się wśród prostego ludu --- myśliwych. Dopiero potem opuścił miejsce odosobnienia i zaczął nauczać Dharmy (Huineng, Wong i Humphreys 1998: rozdział 5).

\textit{Sutra Platformy} podaje, że Huineng przebywał w Caoqi w sumie przez 40 lat, nauczając ludzi z Shaozhou i Kantonu w oparciu o \textit{Sutrę Diamentową}, a jako symbolu przekazu nauk używał \textit{Sutry Platformy}. Huineng miał wiele tysięcy uczniów, z których dziesięciu zostało mistrzami o regionalnym zasięgu działalności. W 712 roku powrócił do Xinzhou, miejsca swych narodzin, a w 713 roku zmarł w wieku 76 lat. Tuż przed jego śmiercią, jego uczeń Fahai, uważany za autora sutry, zapytał, kto będzie jego następcą i co stanie się z szatą patriarchy. Huineng odrzekł, że przekaz szaty dobiegł końca, i sugeruje, że w przyszłości pojawi się uczeń o imieniu Shenhui. W momencie jego śmierci pojawiło się wiele pomyślnych znaków. Huineng miał zostać pochowany w Caoqi, a Wei Ju (韋璩 \pinyin{Wéi Qú}), prefekt, który wysłuchał nauk zawartych w dalszej części tekstu, napisał ku jego czci inskrypcję, którą następnie zniszczyli przedstawiciele Północnej Szkoły (Huineng, Schlütter i Teiser 2012: 31, 34).

\if 0
Schlütter 31
Wang Wei 王維 napisał epitafium dla Huinenga, w którym pojawia się postać mnicha Yinzonga, który miał ostrzyc głowę Huinengowi
nie atakuje Północnego Chan
Matka Wang Wei była wyznawczynią Puji
Huineng nie był członkiem szlachetnego rodu

romans: quest (podróż), konkurs, walka o życie, w której Huineng w końcu wygrywa, bo jest lepszy

\fi

\section{Analiza biografii Huinenga zawartej w tekście \textit{Sutry Platformy} z Dunhuang}
Obecnie uważa się, że biografia Huinenga, opisana w \textit{Sutrze Platformy} w formie monologu, nie jest autentyczną autobiografią, a jedynie tekstem hagiograficznym, mającym przedstawić go jako człowieka świętego i bohatera. Tekst wysuwa twierdzenia, jakoby Huineng miał być prawowitym spadkobiercą Piątego Patriarchy Hongrena, szóstym dzierżawcą przekazu Chan, pochodzącego w prostej, nieprzerwanej linii od samego historycznego Buddy. (Huineng, Schlütter i Teiser 2012: 25-26). % ***

Shenhui w 732 roku zaczął promować Huinenga jako Szóstego Patriarchę linii Chan, i jednocześnie atakował uczniów i spadkobierców Shenxiu, szczególnie Puji (嵩山普寂 \pinyin{Sōngshān Pǔjì}, 651-739), który rościł sobie prawa do tytułu Siódmego Patriarchy. Shenhui twierdził, że Szkoła Północna, której przewodzili Shenxiu i Puji, nie była autentyczna, gdyż propagowała nauki stopniowej ścieżki. Prawdziwe, ponadczasowe nauki buddy, to jest nauki o nagłym oświeceniu, znane również jako subityzm, miały być przekazywane w południowym Chan. Znamienne jest to, że w pismach Shenhui nie było żadnej wzmianki o konkursie poezji ani o dwóch wierszach, co świadczy o tym, że fragmenty te zostały dodane po jego śmierci. O ile w tekście \textit{Sutry platformy} zawarto nauki Shenhui i jego spadkobierców, wkład Shenhui w jej powstanie został w tekście przemilczany (Huineng, Schlütter i Teiser 2012: 32-33; Huineng i Yampolsky 2012: 26, 28; McRae 2004: 63).

Historyczny Huineng był postacią stosunkowo mało znaną i prawdopodobnie dlatego Shenhui wybrał go jako bohatera swoich historii. Ponieważ niewiele było wiadomo o jego prawdziwych naukach, Shenhui mógł przypisać mu dowolne nauki niestojące w sprzeczności z naukami o nagłym oświeceniu. Prawdopodobnie był on mistrzem medytacji nauczającym o nagłym oświeceniu, ale wbrew temu, co możemy przeczytać w \textit{Sutrze platformy}, w jego czasach nie było to w istocie niczym szczególnym. Mimo że pochodził z południa, obszaru oddalonego od serca chińskiej cywilizacji, raczej nie wchodził w konflikty z innymi mistrzami medytacji, a wręcz miał z nimi dobre relacje (Huineng i McRae 2000: xv).

W późniejszym okresie, tj. po śmierci Shenxiu, historia życia Huinenga została uzupełniona o opowieść o konkursie poezji, w którym Huineng miał pokonać Shenxiu, jednoznacznie dowodząc wyższości subityzmu Szkoły Południowej nad stopniową ścieżką Szkoły Północnej. Biorąc pod uwagę, że Huineng został w tekście przedstawiony jako ubogi, niepiśmienny człowiek, jest niezwykle mało prawdopodobne, by był w stanie ułożyć przypisywany mu wiersz w klasycznym, literackim języku chińskim. Z drugiej strony, biorąc pod uwagę, że Huineng wywodził się z rodziny urzędnika, nawet popadłego w niełaskę władz i skazanego na banicję, wydaje się nieprawdopodobne, że Huineng mógłby nie otrzymać żadnego wykształcenia. W dziełach Shenhui pojawiały się również twierdzenia, jakoby Puji wysłał swojego ucznia, niejakiego Zhang Xingchang (張行昌 \nazwisko{Zhāng Xíngchāng}), do Shaozhou, z poleceniem ucięcia głowy zwłokom Huinenga, a także jakoby inny uczeń Puji, imieniem Wu Pingyi (武平一 \nazwisko{Wǔ Píngyī}), wymazał inskrypcję na steli poświęconej Huinengowi i wstawił tam własną, podającą Shenxiu jako prawowitego Szóstego Patriarchę. Ataki Shenhui na Szkołę Północną zostały spisane przez Dugu Pei (獨孤沛 \nazwisko{Dúgū Pèi}) w dziele zwanym \textit{Putidamo Nanzong ding shifei lun} (菩提達摩南宗定是非論 \pinyin{Pútídámó Nánzōng dìng shìfēi lùn}). Chiński pisarz, doktor filozofii Hu Shi (胡適 \pinyin{Hú Shì}, 1891-1962) zebrał odkryte w Dunhuang dzieła Shenhui i jego uczniów i opisał je w pracy pt. \textit{Shenhui heshang yiji} (社會和尚遺集 \pinyin{Shénhuì héshàng yíjí}) % (\textit{Ibidem}).
(Huineng, Schlütter i Teiser 2012: 32-33; Huineng i Yampolsky 2012: 26, 28; Huineng i McRae 2000: xv).

Kim był Shenxiu i czym zasłużył sobie na osobiste ataki Shenhui? [Huineng i Yampolsky 2012] podaje, że na przełomie VII i VIII w. Shenxiu był uważany za jednego z najbardziej znaznych i najwybitniejszych mistrzów Chan, a historia jego życia jest szczogólnie dobrze znana. Jego stosunkowo rzetelna biografia została zapisana w dziele \textit{Chuan fabao ji} w pozbawiony elementów fantastycznych i legend sposób. O ile we wszystkich innych dziełach z tego okresu jest wymieniony jako uczeń Hongrena, jedynie \textit{Chuan fabao ji} podaje, że był uczniem Faru, a ten uczniem Hongrena. Według tej biografii pochodził z miasta Daliang (大梁 \toponim{Dàliáng}), obecnie Kaifeng (開封 \toponim{Kāifēng}) w prowincji Henan, i był członkiem rodu Li (Yampolsky 15-16). % 李

Był ponadprzeciętnie uzdolnionym dzieckiem, a w wieku 13 lat, w związku z zawirowaniami historycznymi i związaną z nimi klęską głodu, postanowił wyrzec się światowego życia i zostać mnichem buddyjskim. Następnie wędrował od jednej świątyni do drugiej, by wreszcie otrzymać pełne ślubowania mnisie w wieku 20 lat. W wieku 46 lat udał się do Hongrena, a ten natychmiast poznał się na jego talencie. Po wielu latach studiowania nauk osiągnął ostateczne oświecenie, a następnie udał się do Jingzhou (荊州 \pinyin{Jīngzhōu}) w prowincji Hubei, ale przez cały czas swojego pobytu tam nie udzielał nauk. Za panowania cesarza Tang Gaozonga (儀鳳 ***) udał się do świątyni Yuquan (玉泉寺 \toponim{Yùquán sì}) w pobliżu obecnego miasta Dangyang (當陽 \toponim{Dāngyáng}) w prowincji Hubei, również nie udzielając nauk. Po śmierci jego mistrza, Faru, zaczęli do niego przybywać uczniowie z dalekich stron, a wówczas Shenxiu zaczął nauczać Dharmy, przynosząc pożytek wielu istotom i prowadząc je do wyzwolenia (\textit{Ibidem}). % (Yampolsky 15-16)

Między rokiem 730 i początkiem lat 50. VIII w. Shenhui stworzył historię życia Huinenga zbliżoną do zawartej w \textit{Sutrze Platformy}, z tą różnicą, że nie było w niej jeszcze wzmianki o konkursie poezji między Shenxiu i Huinengiem. Wedle obecnego stanu wiedzy Shenhui nie posiadał niemal żadnych prawdziwych informacji na temat postaci Huinenga, oprócz tego, że był uczniem Hongrena, żył w Shaozhou i że przez niektórych praktykujących uważany był za nauczyciela o regionalnym zasięgu działalności. W najwcześniejszych pismach przypisywanych Shenhui było napisane jedynie, że Huineng był Szóstym Patriarchą, i że otrzymał szatę, insygnium patriarchatu i przekazu nauk, od Hongrena, Piątego Patriarchy. W \textit{Sutrze Platformy} Shenhui został wymienony jako ostatni z dziesięciu uczniów Huinenga, a także jedyny, który nie płakał, gdy Huineng poinformował ich o zbliżającej się śmierci. Częste odniesienia do jego postaci wskazują, że \textit{Sutra Platformy} powstała najprawdopodobniej wkrótce po jego śmierci
(Huineng, Schlütter i Teiser 2012: 33).

Współcześni badacze uważają, że niemożliwe jest, żeby wydarzenia przedstawione w biografii Huinenga kiedykolwiek miały miejsce, i należy je traktować jedynie jako ciekawą anegdotę o wyraźnym podtekście duchowym. Powodem tego jest fakt, że Shenxiu był uczniem Hongrena jedynie przez kilka lat w początkowym etapie jego działalności, a więc kiedy nie istniał jeszcze problem wyboru jego następcy. Shenxiu i Huineng nie przebywali w klasztorze Hongrena jednocześnie, a więc nie mogli współzawodniczyć w konkursie poezji. Po drugie, w owym okresie nie istniała jeszcze koncepcja jedynego prawowitego patriarchy --- pojawiła się ona dopiero w dziełach Shenhui. Ponadto historia życia Huinenga w zachowanej do dnia dzisiejszego wersji nie pojawia się w pismach Shenhui, a jako propagator Huinenga w roli Szóstego Patriarchy na pewno zapisałby tę historię, gdyby była mu znana. Jednak jeszcze w latach 30. VIII w. dzieje życia Huinenga były mu w dużej mierze obce. (McRae 2004: 67; Huineng i McRae 2000: xv).

Ważnym aspektem nauk Shenhui, przypisywanych Huinengowi, jest odejście od sutry \textit{La\.nkā\-vatāra} na rzecz \textit{Sutry Diamentowej}. Odzwierciedla ono wzrost popularności tej sutry w VIII w. O tym, jak wielką rolę odgrywała ona dla Shenhui, świadczą liczne odniesienia do niej w tekście \textit{Sutry Platformy}. W tekście jest wysunięte również twierdzenie, jakoby to \textit{Sutra Diamentowa}, a nie \textit{La\.nkāvatāra}, była podstawą nauk przekazywanych przez patriarchów, od Bodhidharmy do Huinenga. Inne teksty na temat linii przekazu Chan, takie jak \textit{Kontynuowane biografie wybitnych mnichów} (續高僧傳 \pinyin{Xù gāosēng zhuàn}), \textit{Chuan fabao ji} (傳法寶紀 \pinyin{Chuán fǎbǎo jì}, `Annały przekazu skarbu Dharmy') i \textit{Zapisy mistrzów i uczniów w przekazie Sutry La\.nkāvatāra} (楞伽師資記 \pinyin{Léngqié shīzī jì}) zaprzeczają tym twierdzeniom. Symboliczne odejście od sutry \textit{La\.nkāvatāra} jest też zaznaczone w biografii Huinenga w \textit{Sutrze Platformy}; gdy Piąty Patriarcha Hongren ujrzał wiersz Shenxiu, napisany na ścianie w miejscu, gdzie miały być namalowane sceny z sutry \textit{La\.nkāvatāra}, patriarcha zrezygnował z wykonania malowidła, uznając, że wiersz Shenxiu jest od nich ważniejszy, i zapłacił wezwanego w tym celu malarzowi, chociaż ten nie wykonał swojego zadania (Huineng i Yampolsky 2012: 34; McRae 2004: ***).

W dziełach Shenhui pojawiły się też dwie opowieści, powielone w późniejszych dziełach. Pierwsza z nich, zapisana zarówno w \textit{Sutrze Platformy}, jak i \textit{Putidamo Nanzong ding shifei lun}, dotyczyła Bodhidharmy i cesarza Liang Wudi (梁武帝 \nazwisko{Liáng Wǔdì}). Według tej historii, kiedy Bodhidharma przybył do stolicy Liang (***), przeprowadził dyskusję z cesarzem. Cesarz miał spytać Bodhidharmy, czy budując świątynie, dając ofiary mnichom i ludziom w potrzebie, zgromadził zasługę\fnm. Mistrz odparł: ,,Nie zgromadziłeś zasługi.'' Miał przez to na myśli, że cesarz, nie podążając za właściwą ścieżką, szukał jedynie błogosławieństw, a nie prawdziwej zasługi. Cesarz, nie rozumiejąc tej nauki, był nią rozczarowany i wygnał Bodhidharmę ze swego państwa. Następnie mistrz udał się do państwa Wei (Huineng i Yampolsky 2012: 27, 155-156).
\if 0
T48n2007_p0341a24(10)║朕一生未來造寺布施供養有有功德否。達磨答言。並無功德。
T48n2007_p0341a25(03)║武帝惆悵遂遣。達磨出境。未審此言。請和尚說。
a w Wei spotkał Huike
\fi
\footnotetext{Zasługa (功德 \pinyin{gōngdé}, skt. \textit{puṇya}) w większości tradycji buddyzmu odnosi się do dobrych wrażeń karmicznych, zebranych w rezultacie właściwego postępowania i podążania ścieżką duchową.  (Buswell 2004: 532).}

Inna, nieco drastyczna opowieść, propagowana przez Shenhui, dotyczy Bodhidharmy i jego ucznia i spadkobiercy, Huike. Według tej historii, kiedy mistrz i uczeń spotkali się po raz pierwszy, Huike był zdesperowany, by zostać uczniem Bodhidharmy, lecz ten nie chciał go przyjąć. Mistrz miał ustąpić Huike dopiero wówczas, gdy ten w dowód swej determinacji dobył miecza i demonstracyjnie uciął swoje lewe ramię. Huike został następnie głównym uczniem Bodhidharmy i odziedziczył po nim szatę, symbol przekazu Dharmy. Później tę samą szatę mieli otrzymać kolejni patriarchowie: Sengcan, Daoxin, Hongren, aż do Huinenga. Przy pomocy tej opowieści Shenhui osiągnął dwa cele: nie tylko ustanowił szatę Bodhidharmy jako insygnium prawowitego patriarchy Chan, lecz również podważył uznaną dotychczas linię przekazu, wiodącą od Bodhidharmy do Shenxiu. Pochodzenie tych legend nie jest znane. Hu Shi uważa, że zostały one wymyślone przez Shenhui, ale równie prawdopodobne jest, że krążyły wśród ludu, a Shenhui jedynie zapisał je i wykorzystał do swoich celów (Huineng i Yampolsky 2012: 27).

\if 0
Yampolsky
Nawet jeżeli istniał Bodhidharma, linia przekazu Chan raczej nie istniała w nieprzerwanej formie
Chan jako taki wymyślono po czasach Hongrena
biografia Huinenga ma poprzeć roszczenia
Schlütter 32
\fi

\subsection{Wiersze Shenxiu i Huinenga}
Tradycyjna interpretacja wierszy Shenxiu i Huinenga, zawartych w tekście \textit{Sutry Platformy} (patrz: strony \pageref{ShenxiuVerse} i \pageref{HuinengVerse}), według filozofa Zongmi (圭峰宗密 \pinyin{Guīfēng Zōngmì}, 780-841), jest prosta. Wiersz Shenxiu ma symbolizować stopniową ścieżkę, zaś wiersz Huinenga --- ścieżkę nagłego, ostatecznego oświecenienia, które wydarza się w jednej chwili. Tym samym, w rozumieniu Zongmi wiersze te reprezentują pogląd dwóch konkurencyjnych tradycji Chan, północnej i południowej (McRae 2004: 63).

Takie rozumienie jest jednak nadmiernym uproszczeniem. Wiersz przypisywany Shenxiu odnosi się nie tyle do stopniowej ścieżki, ile do ciągłej, bezustannej praktyki oczyszczania zwierciadła z kurzu. Wiersz Huinenga nie opisuje natomiast poglądu nagłej ścieżki, a jedynie neguje twierdzenia, zawarte w wierszu Shenxiu. Oprócz tego, wiersze przedstawiają dwa punkty widzenia na ten sam temat i nie mogą być interpretowane osobno. Zwłaszcza wiersz Huinenga odwołuje się do wiersza Shenxiu (McRae 2004: 63-64).

Opis konkursu poezji, zawarty w tekście \textit{Sutry platformy}, ma za zadanie udowodnić wyższość nauk Huinenga i jego spadkobierców nad naukami Shenxiu. Chociaż zwycięzcą w starciu był Huineng, to wiersz, przypisywany Shenxiu, wcale nie jest mierny, a wręcz przeciwnie --- jest głęboki i wyrafinowany. W ten sposób wiersz Huinenga tylko zyskuje w oczach odbiorcy. Nie jest bowiem sztuką stworzyć dzieła lepszego od czegoś miernego.

[Jak niepiśmienny Huineng mógł ułożyć wiersz w klasycznym chińskim?]

\if 0
McRae xvi
Shenxiu i Huineng nie przebywali w Huangmei w tym samym czasie

Tekst jest nie tyle biografią, ile hagiografią
Oparta na naukach Shenhui

McRae 62-64:
Wiersz Shenxiu:
Nie ma dowodu, żeby kiedykolwiek go napisał, albo żeby porównał umysł do jasnej lustrzanej podstawy, ale z innych jego dzieł wynika, że mógł napisać coś w tym stylu
constant and perfect teaching, the endless personal manifestation of the
bodhisattva ideal
McRae, Northern School, 235. The English “suchlike'' renders the word 如, as in the Chinese translation of Tathágata, 如來, when used as a modifier.
神秀觀心論
T85n2833_p1271c15(08)║眾生修伽藍鑄形像燒香散花然長明燈。
T85n2833_p1271c16(09)║晝夜六時遶塔行道持齋禮拜。種種功德皆成佛道。
T85n2833_p1271c17(06)║若唯觀心總攝諸行。如是事應妄也 答曰。
T85n2833_p1271c18(06)║佛所說無量方便。一切眾生鈍根狹劣。
T85n2833_p1271c19(08)║甚深所以假有為喻無為。若不內行唯只外求。希望獲福。
T85n2833_p1271c20(03)║無有是處。言伽藍者。西國梵音。
T85n2833_p1271c21(07)║此地翻為清淨處地。若永除三毒常淨六根。
T85n2833_p1271c22(07)║身心湛然內外清淨。是名為修伽藍也。又鑄形像者。
ciało jest drzewem bodhi, oba są fizyczne
ciało jest miejscem, w którym człowiek osiąga oświecenie
lustrzana podstawa

Wiersz Shenxiu był głęboki i wyrafinowany, bo tym sposobem wiersz Huinenga jest lepszy od czegoś wybitnego; być lepszym od czegoś miernego to żadne osiągnięcie
Wiersz Huinenga wcale nie wyraża poglądów szkoły nagłego oświecenia, a jedynie zaprzecza poglądom wiersza Shenxiu
Więc nie jest to wykładnia szkół nagłego i stopniowego oświecenia
Wiersz Huinenga pojawiał się w kilku wersjach w różnych tekstach
Wiersze oparte na pismach Szkoły Północnej (McRae 67)

Przejście od Lankavatara do Diamentowej Sutry, w VIII w. ta właśnie sutra nabierała popularności, na niekorzyść Lankavatary
\fi
\subsection{Analogie do biografii Konfucjusza}
Pisząc historię życia Huinenga, Shenhui w oczywisty sposób czerpał z legendy o Konfucjuszu, opisanej w \textit{Zapiskach historyka} (史記 \pinyin{Shǐjì}) autorstwa Sima Qian (司馬遷 \nazwisko{Sīmǎ Qiān}) zwanego ,,Wielkim Historykiem''. Tekst ten miał wówczas ugruntowaną pozycję wśród chińskich elit, jako że Konfucjusz był twórcą głównego systemu filozoficznego w państwie, i był znany niemal wszystkim, zaś postać Konfucjusza była dla współczesnych uniwersalnym wzorcem cnót (Huineng, Schlütter i Teiser 2012: 36).

Według tej legendy, Konfucjusz urodził się z mezaliansu mężczyzny z wybitnego rodu, pochodzącego z innego rejonu, i lokalnej kobiety. Ojciec Konfucjusza osierocił go, gdy ten był jeszcze dzieckiem, w rezultacie czego Konfucjusz i jego matka wiedli życie w ubóstwie. Konfucjusz nie cieszył się szczególną popularnością wśród lokalnej elity państwa Lu, przez pewien czas pracował jako urzędnik niskiego szczebla, lecz wkrótce został z niej zwolniony, mimo że wykazywał ponadprzeciętne uzdolnienia (Huineng, Schlütter i Teiser 2012: 36-37).

Shenhui był wykształcony w zakresie filozofii konfucjańskiej, a nawet został przez Zongmi porównany do Konfucjusza. Zanning (贊寧 \nazwisko{Zàn Níng}), autor \textit{Biografii wybitnych mnichów Song} (宋高僧傳 \pinyin{Sòng gāosēng zhuàn}) porównał go do Yan Hui (顏回 \nazwisko{Yán Huí}), ulubionego ucznia Konfucjusza.
Symboliczne znaczenie miało również liczba patriarchów w linii przekazu Chan. Według *** Southern Learning, jedynie cesarz mógł w świątyni swojej rodziny urządzić siedem izb. Linia przekazu w wersji Shenhui wiodła od Bodhidharmy, Pierwszego Patriarchy, przez Hongrena, Piątego Patriarchę, do Huinenga, Szóstego Patriarchy. Będąc spadkobiercą Huinenga, Shenhui stawał się Siódmym Patriarchą i tym samym ustanawiał doskonałą, ,,cesarską'' tradycję Chan.
\if 0
Southern Learning
siedem świątyń lub izb w świątyni przodków było zarezerwowane dla cesarzy
宗
southern learning
Schlütter 37 \fi
(Huineng, Schlütter i Teiser 2012: 37).

Z punktu widzenia nauk Konfucjusza, Huineng był postacią godną naśladowania. Mimo iż wywodził się z nizin społecznych i wychował się w rejonie tak bardzo oddalonym od centrum cywilizacji chińskiej, jak to było możliwe, to posiadał jedną z najwyższych cnót konfucjańskich --- \textit{nabożność synowską} (子孝 {zǐxiào}). Po śmierci ojca ciężko pracował, utrzymywał starą matkę i opiekował się nią. Taki obraz postaci Huinenga stał w opozycji do wykształconych, bogatych elit, z których wywodziła się w owym okresie większość mnichów buddyjskich, i był najprawdopodobniej zainspirowany historią życia Hongrena, który według niektórych podań medytował za dnia, zaś w nocy zajmował się bydłem. Sam Shenxiu, rzekomy konkurent Huinenga do pozycji patriachy, pochodził ze szlachetnego rodu i był wykształcony zarówno w literaturze buddyjskiej, jak i świeckiej, a niektórzy podejrzewają nawet, że mógł być związany z rodem cesarskim (McRae 2004: 68).

Celem takiego przedstawienia postaci Huinenga było pokazanie, że każdy, nawet osoba świecka, niezależnie od pozycji społecznej, miejsca pochodzenia i wykształcenia mógł osiągnąć oświecenie i zostać patriarchą. W szerszym rozumieniu, jeżeli osiągnięcie najwyższego oświecenia było jedynym i koniecznym warunkiem zostania mianowanym patriachą szkoły Chan, to można było oczekiwać, że wszyscy jego poprzednicy i następcy również byli oświeceni. W ten sposób szkoła Chan zyskiwała autorytet i wiarygodność (McRae 2004: 69).

\chapter{Analiza tekstu \textit{Sutry Platformy}}

\section{Budowa Sutry}
Tekst Sutry w wersji z grot w Dunhuang w przekładzie Philipa B. Yampolsky'ego został podzielony na 57 sekcji o różnej długości. Pierwsze 37 sekcji zawiera nauki wygłoszone przez Huinenga w świątyni Dafan. Jak podaje tekst, wysłuchało ich zgromadzenie ponad 10000 słuchaczy, w tym mnichów, mniszek i ludzi świeckich, a także prefekt Shaozhou, Wei Qu (韋璩 \nazwisko{Wéi Qú}). Sekcje 2-11 stanowią autobiografię Szóstego Patriarchy, zaś w sekcjach 12-37 przekazane zostały jego ,,bezforemne nauki'' (無相戒 \pinyin{wúxiàng jiè}, ang. \textit{precepts of formlessness} lub \textit{formless precepts}).

W sekcjach 39-44 zamieszczone zostały anegdoty dotyczące najważniejszych spadkobierców Huinenga, takich jak Fahai, Zhicheng, Fada i Shenhui. Sekcje 45-47 to nauki udzielone dziesięciu bliskim uczniom. Sekcje 48-57 opisują okoliczności śmierci Szóstego Patriarchy, a także: nauki, które przekazał swym uczniom w formie wierszy bezpośrednio przed śmiercią; informacje o następcach Huinenga, dalszych losach \textit{Sutry platformy} i przekazie nauk szkoły Chan.

\section{Nauki o medytacji}
W sekcjach 13-19 przekazane są pouczenia o medytacji. Najważniejsza nauka, przedstawiona w tym fragmencie, dotyczy jedności medytacji (惠 \pinyin{huì}) i mądrości (定 \pinyin{dìng}). Według Szóstego Patriarchy medytacja i mądrość są częściami tej samej całości i stwarzają siebie nawzajem. Są od siebie współzależne, i nie można stwierdzić, które z nich pojawiło się jako pierwsze. W tekście zostały porównane do światła i lampy. Takie same nauki pojawiają się w \textit{Shenhui yulu} (Huineng i Yampolsky 2012: 137).

Według tych nauk, główną doktryną subityzmu jest ,,brak myśli'' lub ,,brak idei'' (無念 \pinyin{wúniàn} lub 無心 \pinyin{wúxīn}, `brak umysłu'), jego istotą --- ,,brak formy'' (無相 \pinyin{wúxiàng}), a jego podstawą --- ,,brak przywiązania'' (無住 \pinyin{wúzhù}) (\textit{Ibidem}).
% istotą/substancją; subityzm był dodany; wúzhù --- no abiding

,,Brak idei'' oznacza wolność od rozproszenia --- pilnowanie, by umysł nie podążał za myślami i aby nic, co pojawia się w umyśle, nie prowadziło do powstania negatywnych emocji i błędnych poglądów. Oznacza też brak przywiązania do dualistycznego postrzegania zewnętrznych zjawisk i postrzegającego je umysłu. Według tej doktryny wszystkie myśli powstają w esencji umysłu i są wyrażeniem się jego potencjału, dlatego próby całkowitego wyparcia myśli są błędne. Właściwa praktyka polega na odcięciu pomieszanych, dualistycznych myśli i koncentracji na prawdziwej naturze Takości (Huineng i Yampolsky 2012: 137-138).

,,Brak obiektu'' oznacza unikanie rozproszenia pod wpływem zewnętrznych zjawisk. Nauki te nie oznaczają, że należy fizycznie odciąć się od myśli i form, lecz ,,być oddzielonym od formy nawet wtedy, gdy jest się z nią związanym''. Człowiek praktykujący tę ścieżkę związanym z formą i myślami, nie traktujemy ich jako prawdziwie istniejących, a jedynie jako przejawienie się potencjału przestrzeni. Brak przywiązania oznacza niemyślenie o przeszłości ani przeszłości (\textit{Ibidem}).

Według Huinenga, praktykujący Chan nie powinien koncentrować się ani na umyśle, ani na czystości, nie powinien też mówić o niewzruszoności. Twierdzi, że umysł jako taki jest zwodniczy, jest jedynie iluzją i jako taki nie powinien być obiektem medytacji. Również sama koncentracja na czystości nie przynosi spodzewanych rezultatów. Taka praktyka stwarza jedynie kolejne złudzenia i sztywne koncepcje, jeśli praktykujący nie zrozumie, że natura umysłu jest sama w sobie doskonała i czysta. Jedynym powodem, dla którego istoty nie są w stanie postrzegać jej w ten sposób, są błędne poglądy i zaciemnienia (Huineng i Yampolsky 2012: 139-140).

Prawidłowa praktyka medytacji siedzącej (坐禪 \pinyin{zuòchán}) została przez Huinenga zdefiniowana jako niepodążanie za myślami oraz postrzeganie własnej prawdziwej natury bez rozproszenia. Natomiast ,,medytacja Chan'' (禪定 \pinyin{chándìng}) oznacza według niego niezależność od zewnętrznych zjawisk oraz unikanie pomieszania w umyśle (Huineng i Yampolsky 2012: 140-141).

\if 0
McRae 66
Natura buddy przesłonięta wyłącznie zaciemnieniami umysłu
\fi

\chapter{Historyczne reperkusje \textit{Sutry platformy}}

Lorem ipsum dolor sit amet.

\chapter*{Glosariusz}
\addcontentsline{toc}{chapter}{Glosariusz}

\label{glosMahayana}
\textbf{Mahajana, Buddyzm Mahajany} (大乘佛教 \pinyin{Dàshèng fójiào} lub \pinyin{Dàchéng fójiào}, `buddyzm Wielkiego Wozu', od skt. \textit{Mahāyāna}, `wielki wóz', nazywany również buddyzmem Wielkiej Drogi) --- jeden z trzech głównych odłamów buddyzmu (dwa pozostałe to Hinajana, tzw. Mała Droga lub Mały Wóz, oraz Wadżrajana, Diamentowa Droga lub Diamentowy Wóz). Filarami Mahajany są wyzwalająca mądrość i współczucie dla wszystkich czujących istot, rozwijane w równowadze. Do buddyzmu Mahajany zalicza się m.in. buddyzm Chan i Zen, Szkołę Czystej Krainy, a także szkołę Gelugpa buddyzmu tybetańskiego.
\medskip

\label{glosPureLand}
\textbf{Szkoła Czystej Krainy}, zwana również \textbf{Szkołą Czystej Ziemi} lub \textbf{amidyzmem} --- tradycja buddyzmu chińskiego, w której za najważniejszą postać przyjmuje się \textit{Buddę Amitabhę} (阿彌陀佛 \pinyin{Āmìtuófó}). Celem praktyki tej tradycji jest odrodzenie po śmierci w Czystej Krainie tego buddy, \textit{Sukhavati} (w języku chińskim nazywana 極樂 \pinyin{Jílè}, 安樂 \pinyin{Ānlè} lub 西天 \pinyin{Xītiān}).\medskip

\label{glosMerit}
\textbf{Zasługa} (功德 \pinyin{gōngdé}, skt. \textit{pu\d{n}ya}) w większości tradycji buddyzmu odnosi się do dobrych wrażeń karmicznych, zebranych w rezultacie właściwego postępowania i podążania ścieżką duchową.  (Buswell 2004: 532).\medskip

\label{glosLowerRealms}
\textbf{Niższe Sfery Egzystencji}:
Według kosmologii buddyjskiej, istoty krążące w samsarze, tj. uwarunkowanej egzystencji, od niemającego początku czasu odradzają się w jednej z sześciu sfer egzystencji, zależnie od swojej karmy i indywidualnych skłonności. Trzy z nich, sfera niebiańska (天道 \pinyin{tiāndào}, skt. \textit{devaloka}), którą zamieszkują bogowie, sfera półbogów lub asurów (阿修羅 \pinyin{Āxiūluódào}) i sfera ludzi (人道 \pinyin{réndào}), nazywa się trzema wyższymi sferami egzystencji (三善道 \pinyin{sān shàndào}), ponieważ życie w tych sferach jest relatywnie przyjemne. Trzy niższe sfery egzystencji, sfera zwierząt (畜牲道 \pinyin{chùshēngdào}), sfera głodnych duchów lub pretów (餓鬼道 \pinyin{èguǐdào}) oraz sfery piekielne (地獄道 \pinyin{dìyùdào}), w których życie pełne jest cierpienia, nazywane są niższymi sferami egzystencji (三惡道 \pinyin{sān èdào}).\medskip

\label{glosMahaprajnaparamita}
\textbf{Mahapradżniaparamita} (skt. \textit{Mahāprajñāpāramitā}, chiń. 摩訶般若波羅蜜多 \pinyin{Móhē Bōrě Bōluómìduō}, `Wielka Doskonałość Mądrości').\medskip

\label{glosZhouDynasty}
\textbf{Dynastia Zhou} (周朝 \pinyin{Zhōu Cháo}) --- dynastia rządząca Chinami w latach ok. 1045-256 p.n.e. Dzieli się na tzw. Zachodnią Dynastię Zhou i Wschodnią Dynastię Zhou.\medskip

\makeatletter
\@openrightfalse
\makeatother
\chapter*{Streszczenie}
\addcontentsline{toc}{chapter}{Streszczenia}
\markboth{STRESZCZENIE}{STRESZCZENIE}
\textit{Sutra Platformy Szóstego Patriarchy} jest jednym z najważniejszych tekstów buddyzmu chan.
Najstarsza zachowana wersja tego dzieła powstała w VIII wieku w Chinach.
Jej autorstwo przypisuje się Dajian Huinengowi (638-713), legendarnemu Szóstemu Patriarsze Chan.
Niniejsza praca opisuje tło historyczne i okoliczności powstania tekstu, a także jego wpływ na dalszy rozwój buddyzmu chińskiego.

W drugim rozdziale pracy omówiono biografię Szóstego Patriachy.
W \textit{Sutrze Platformy} został on przedstawiony jako ubogi, niepiśmienny człowiek z południa Chin.
Przebywając w klasztorze Piątego Patriarchy, Daman Hongrena (601-674), pokonał w konkursie poezji Yuquan Shenxiu (606-706), a następnie został mianowany Szóstym Patriarchą.

Trzeci rozdział pracy poświęcony jest szczegółowej analizie \textit{Sutry Platformy} oraz zawartych w niej nauk.
Chociaż \textit{Sutrę Platformy} uważa się za dzieło z kanonu Mahapradżniaparamity, tekst łączy w sobie nauki pochodzące z różnych nurtów filozofii buddyjskiej, szczególnie Tathagathagarbha.

W czwartym rozdziale pracy opisano wpływ sutry na dalszy rozwój buddyzmu chan, ataki Heze Shenhui na Szkołę Północną, oraz schizmę w szkole chan.
% przeniesienie Chan do Japonii

\chapter*{摘要}
\markboth{摘要}{摘要}
《六祖壇經》為中國禪宗佛教最重要經典之一。其最早現存的版本是編於西元八世紀的敦煌本。按照傳統說法,《壇經》為六祖大鑒惠能(638-713)所說,其弟子法海所寫。本論文論述了《壇經》撰寫的情況與它對禪宗後來的發展的影響。

本論文的第二章論述了惠能事跡。根據《壇經》裡面的傳記,惠能為中國南方貧民,不識字。

第三章裡,筆者仔細地分析了《壇經》及其所法門。雖然《壇經》通常被視為摩訶般若波羅密多類的經典,其文本是很多佛教哲學流派,尤其是如來藏。

第四章描敘述了《壇經》對禪宗佛教後來的發展,荷澤神會對北宗的打擊與禪宗內的分裂。

\chapter*{Abstract}
\markboth{ABSTRACT}{ABSTRACT}
\textit{The Platform Sutra of the Sixth Patriarch} is among the most important texts of Chinese Chan Buddhism.
Its earliest extant version was compiled in the 8th century CE in China.
It is traditionally attributed to Dajian Huineng (638-713), the legendary Sixth Patriarch of Chan.
This thesis describes the historical background and circumstances of its creation, as well as the impact it had on the further development of Chinese Buddhism.

The first chapter of this paper presents the origins of Chan Buddhism, which is said to have been brought to China by a \textit{dhyāna} master called Bodhidharma.

The second chapter of this thesis deals with the biography of the Sixth Patriarch.
In the \textit{Platform Sutra}, he is presented as a poor, illiterate man from Southern China.
During his stay at the Huangmei Monastery with the Fifth Patriarch, Daman Hongren (601-674), he is said to have won a verse competiton against Yuquan Shenxiu (606-706), and to have been declared the Sixth Patriarch.

The third chapter is dedicated to a thorough analysis of the \textit{Platform Sutra} and the teachings it conveys.
The most important ideas found in the text are the identification of meditation (\textit{ding}) with wisdom (\textit{hui}), ``seeing into the self-nature and becoming a Buddha'' (\textit{jian xing cheng fo}), as well as the so-called ``formless precepts''.
Although generally considered to belong to the Mahaprajñaparamita canon, the text combines teachings originating from different doctrines of Buddhist philosophy, especially Tathagathagarbha.

The fourth chapter describes the influence of the \textit{Platform Sutra} on the further development of Chan Buddhism, Heze Shenhui's attacks on the Northern School, and the resulting schism in the Chan school.
Only two schools that trace their lineages back to Huineng survived to this day: the Linji school, founded by Linji Yixuan, and the Caodong school, started by the masters Dongshan Liangjie and Caoshan Benji.
These were later transmitted to Japan, where they are known as Rinzai and Sōtō schools, respectively.

% \printindex
\onecolumn
\section*{Bibliografia}

Jing 靜, Yun 筠 i Mo Yansheng 莫言生. \textit{Zutang ji} [Antologia gmachu patriarchów]. \url{http://www.lianhua33.com/c/c47-1.htm}

% WSTAWIĆ SUTRĘ PLATFORMY Z CBETA
Dumoulin, Heinrich. 1963. \textit{A history of Zen Buddhism}. New York: Pantheon Books.

Huineng, Mou-lam Wong i Christmas Humphreys. 1973. \textit{The sutra of Wei Lang (or Hui Neng)}. Westport, Conn: Hyperion Press. \url{http://www.sinc.sunysb.edu/Clubs/buddhism/huineng/content.html}

Huineng, \textit{Sutra Szóstego Patriarchy Zen}, tłumacz nieznany, \url{http://www.zen.ite.pl/teksty/sutra6.html}

McRae, John R. 1986. \textit{The Northern School and the formation of early Ch'an Buddhism}. Honolulu: University of Hawaii Press.

Shi Shengyan 釋聖嚴 1990. ``Liuzu tanjing de sixiang'' 六祖壇經的思想 [Idee Sutry Platformy Szóstego Patriarchy]. \textit{Zhonghua foxue xuebao} 中華佛學學報. [Chung-Hwa Buddhist Journal]. Taipei: The Chung-Hwa Institute of Buddhist Studies.

Bodhidharma, and Jeffrey L. Broughton. 1999. \textit{The Bodhidharma anthology the earliest records of Zen}. Berkeley, Calif: University of California Press.

Huineng i John R. McRae. 2000. \textit{The Platform Sutra of the Sixth Patriarch: translated from the Chinese of Tsung-pao.} Berkeley, CA: Numata Center for Buddhist Translation and Research.

William E. Soothill i Lewis Hodous. 2003. \textit{A Dictionary of Chinese Buddhist Terms.} RoutledgeCurzon.% \url{http://buddhistinformatics.ddbc.edu.tw/glossaries/files/soothill-hodous.ddbc.pdf}

Buswell, Robert E. 2004. \textit{Encyclopedia of Buddhism.} New York: Macmillan Reference.

McRae, John R. 2004. \textit{Seeing Through Zen: Encounter, Transformation, and Genealogy in Chinese Chan Buddhism.} Berkeley, Calif: University of California Press.

Anonim 佚名 2007. ``Fojiao de futian'' 佛教的福田 [Pola błogosławieństw w buddyzmie]. \textit{Zhongguo minzu bao} 中國民族報, za: \url{http://www.wuys.com/news/Article_Show.asp?ArticleID=12791}

Shi Yinshun 釋印順 2008. \textit{Zhongguo Chanzong shi} 中國禪宗史 [Historia buddyzmu chińskiego tradycji Chan]. Yangzhou: Guangling shushe.

Nydahl, Ole. 2010. ``Sześć wyzwalających działań''. \textit{Diamentowa Droga} 34. \url{http://diamentowadroga.pl/dd34/szesc_wyzwalajacych_dzialan}

Huineng, Morten Schlütter i Stephen F. Teiser. 2012. \textit{Readings of the Platform sūtra.} New York: Columbia University Press.% \url{http://site.ebrary.com/id/10538320}.

Huineng i Philip B. Yampolsky. 2012. \textit{The Platform sutra of the Sixth Patriarch. The Text of the Tun-huang manuscript.} New York: Columbia University Press.% \url{http://public.eblib.com/choice/publicfullrecord.aspx?p=909420.}

Schuman, Michael. 2015. \textit{Confucius: and the world he created.} New York: Basic Books.

\end{document}
