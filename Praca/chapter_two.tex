\chapter{Analiza tekstu \textit{Sutry Platformy}}

\section{Budowa Sutry}
Tekst Sutry w wersji z grot Mogao w Dunhuang w przekładzie Philipa B. Yampolsky'ego został podzielony na 57 sekcji o różnej długości. Pierwsze 37 sekcji zawiera nauki wygłoszone przez Huinenga w świątyni Dafan, dla zgromadzenia ponad 10000 słuchaczy, w tym mnichów, mniszek i ludzi świeckich, a także prefekta Shaozhou, Wei Qu (韋璩 \nazwisko{Wéi Qú}) przy czym sekcje 2-11 stanowią autobiografię Huinenga, zaś w sekcjach 12-37 przekazane zostały ,,bezforemne nauki'' (無相戒 \pinyin{wúxiàng jiè}, ang. \textit{precepts of formlessness}).

W sekcjach 39-44 zamieszczone zostały anegdoty dotyczące najważniejszych uczniów Huinenga, takich jak Fahai, Zhicheng, Fada i Shenhui. Sekcje 45-47 to nauki udzielone 10 najważniejszym uczniom. Sekcje 48-57 opisują okoliczności śmierci Huinenga, nauki, które przekazał swym uczniom w formie wierszy bezpośrednio przed śmiercią, a także informacje o jego następcach, dalszych losach \textit{Sutry platformy} i przekazie nauk Huinenga.

\section{Nauki o medytacji}
W sekcjach 13-19 przekazane są nauki o medytacji. Najważniejsze koncepcje dotyczące medytacji, które pojawiają się w tekście, to jedność medytacji (惠 ) i mądrości (定 ). Według tych nauk, medytacja i mądrość są częściami tej samej całości i stwarzają siebie nawzajem. Są od siebie współzależne, i nie można stwierdzić, które z nich pojawiło się jako pierwsze. W tekście zostały porównane do światła i lampy. Takie same nauki pojawiają się w \textit{Shenhui yulu}.

Według tych nauk, ,,brak myśli'' lub ,,brak idei'' (無念 aka 無心) jest główną doktryną, ,,brak formy'' (無相 ) substancją/istotą, ,,brak przywiązania'' (無住 no abiding) podstawą.

,,Brak idei'' oznacza wolność od rozproszenia --- pilnowanie, by umysł nie podążał za myślami i aby nic, co pojawia się w umyśle, nie prowadziło do powstania negatywnych emocji i błędnych poglądów. Oznacza też nieprzywiązywanie się do dualistycznego postrzegania zewnętrznych zjawisk i postrzegającego je umysłu. Wszystkie myśli powstają w esencji umysłu i są wyrażeniem się jego potencjału. Dlatego też ,,brak idei'' nie jest równoznaczny z całkowitym wyparciem myśli z umysłu, a jedynie odcięciem pomieszanych, dualistycznych myśli i koncentracją na prawdziwej naturze Takości.

Nauki te nie oznaczają, że należy fizycznie odciąć się od myśli i form, a jedynie ,,być w tym świecie, ale nie z tego świata''. To oznacza, że nawet będąc związanym z formą i myślami, nie traktujemy ich jako prawdziwie istniejących, a jedynie jako przejawienie się potencjału przestrzeni. Brak przywiązania oznacza, że nie myślimy o przeszłości ani przeszłości***.

Według Huinenga, praktykujący Chan nie powinien koncentrować się ani na umyśle, ani na czystości. Umysł jako taki jest zwodniczy, jest jedynie iluzją i jako taki nie powinien być obiektem medytacji. Koncentracja na czystości zaś prowadzi do fiksacji na koncepcji czystości. Chociaż prawdziwa natura umysłu jest doskonała i czysta sama w sobie, nie jesteśmy w stanie postrzegać jej w ten sposób z powodu zaciemnień. Dlatego właściwa medytacja oznacza urzeczywistnienie niewzruszonej esencji umysłu.

\if 0
McRae 66
Natura buddy przesłonięta wyłącznie zaciemnieniami umysłu
\fi
