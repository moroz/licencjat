\chapter{Analiza tekstu \textit{Sutry Platformy}}

\section{Budowa Sutry}
Tekst Sutry w wersji z grot w Dunhuang w przekładzie Philipa B. Yampolsky'ego został podzielony na 57 sekcji o różnej długości. Pierwsze 37 sekcji zawiera nauki wygłoszone przez Huinenga w świątyni Dafan. Jak podaje tekst, wysłuchało ich zgromadzenie ponad 10000 słuchaczy, w tym mnichów, mniszek i ludzi świeckich, a także prefekt Shaozhou, Wei Qu (韋璩 \nazwisko{Wéi Qú}). Sekcje 2-11 stanowią autobiografię Szóstego Patriarchy, zaś w sekcjach 12-37 przekazane zostały jego ,,bezforemne nauki'' (無相戒 \pinyin{wúxiàng jiè}, ang. \textit{precepts of formlessness} lub \textit{formless precepts}).

W sekcjach 39-44 zamieszczone zostały anegdoty dotyczące najważniejszych spadkobierców Huinenga, takich jak Fahai, Zhicheng, Fada i Shenhui. Sekcje 45-47 to nauki udzielone dziesięciu bliskim uczniom. Sekcje 48-57 opisują okoliczności śmierci Szóstego Patriarchy, a także: nauki, które przekazał swym uczniom w formie wierszy bezpośrednio przed śmiercią; informacje o następcach Huinenga, dalszych losach \textit{Sutry platformy} i przekazie nauk szkoły Chan.

\section{Nauki o medytacji}
W sekcjach 13-19 przekazane są pouczenia o medytacji. Najważniejsza nauka, przedstawiona w tym fragmencie, dotyczy jedności medytacji (惠 \pinyin{huì}) i mądrości (定 \pinyin{dìng}). Według Szóstego Patriarchy medytacja i mądrość są częściami tej samej całości i stwarzają siebie nawzajem. Są od siebie współzależne, i nie można stwierdzić, które z nich pojawiło się jako pierwsze. W tekście zostały porównane do światła i lampy. Takie same nauki pojawiają się w \textit{Shenhui yulu} (Huineng i Yampolsky 2012: 137).

Według tych nauk, główną doktryną subityzmu jest ,,brak myśli'' lub ,,brak idei'' (無念 \pinyin{wúniàn} lub 無心 \pinyin{wúxīn}, `brak umysłu'), jego istotą --- ,,brak formy'' (無相 \pinyin{wúxiàng}), a jego podstawą --- ,,brak przywiązania'' (無住 \pinyin{wúzhù}) (\textit{Ibidem}).
% istotą/substancją; subityzm był dodany; wúzhù --- no abiding

,,Brak idei'' oznacza wolność od rozproszenia --- pilnowanie, by umysł nie podążał za myślami i aby nic, co pojawia się w umyśle, nie prowadziło do powstania negatywnych emocji i błędnych poglądów. Oznacza też brak przywiązania do dualistycznego postrzegania zewnętrznych zjawisk i postrzegającego je umysłu. Według tej doktryny wszystkie myśli powstają w esencji umysłu i są wyrażeniem się jego potencjału, dlatego próby całkowitego wyparcia myśli są błędne. Właściwa praktyka polega na odcięciu pomieszanych, dualistycznych myśli i koncentracji na prawdziwej naturze Takości (Huineng i Yampolsky 2012: 137-138).

,,Brak obiektu'' oznacza unikanie rozproszenia pod wpływem zewnętrznych zjawisk. Nauki te nie oznaczają, że należy fizycznie odciąć się od myśli i form, lecz ,,być oddzielonym od formy nawet wtedy, gdy jest się z nią związanym''. Człowiek praktykujący tę ścieżkę związanym z formą i myślami, nie traktujemy ich jako prawdziwie istniejących, a jedynie jako przejawienie się potencjału przestrzeni. Brak przywiązania oznacza niemyślenie o przeszłości ani przeszłości (\textit{Ibidem}).

Według Huinenga, praktykujący Chan nie powinien koncentrować się ani na umyśle, ani na czystości, nie powinien też mówić o niewzruszoności. Twierdzi, że umysł jako taki jest zwodniczy, jest jedynie iluzją i jako taki nie powinien być obiektem medytacji. Również sama koncentracja na czystości nie przynosi spodzewanych rezultatów. Taka praktyka stwarza jedynie kolejne złudzenia i sztywne koncepcje, jeśli praktykujący nie zrozumie, że natura umysłu jest sama w sobie doskonała i czysta. Jedynym powodem, dla którego istoty nie są w stanie postrzegać jej w ten sposób, są błędne poglądy i zaciemnienia (Huineng i Yampolsky 2012: 139-140).

Prawidłowa praktyka medytacji siedzącej (坐禪 \pinyin{zuòchán}) została przez Huinenga zdefiniowana jako niepodążanie za myślami oraz postrzeganie własnej prawdziwej natury bez rozproszenia. Natomiast ,,medytacja Chan'' (禪定 \pinyin{chándìng}) oznacza według niego niezależność od zewnętrznych zjawisk oraz unikanie pomieszania w umyśle (Huineng i Yampolsky 2012: 140-141).

\section{Bezforemne nauki}
W sekcjach 20-37 przekazane zostały tzw. ,,bezforemne nauki'', jeden z najważniejszych aspektów \textit{Sutry platformy}. W sekcji 20. Szósty Patriarcha naucza o obecności trzech ciał buddy (三身 \pinyin{sānshēn}, skt. \textit{Trikāya}, w języku polskim nazywane również trzema stanami buddy) w fizycznym ciele praktykującego. Według niego, \textit{Dharmakāya} (法身 \pinyin{fǎshēn}), \textit{Sambhogakāya} (報身 \pinyin{bàoshēn}) oraz \textit{Nirmā\d{n}akāya} (應身 \pinyin{yīngshēn}) są nierozerwalnie związane z naturą buddy wszystkich istot. Ponieważ jednak przesłaniają je błędne poglądy, wielu praktykujących poszukuje ich na zewnątrz.
(Huineng i Yampolsky 2012: 141-143).

W sekcji 21. przekazane są cztery ślubowania (四弘大願 \pinyin{sì hóngdà yuàn}), dotyczące kolejno: wyzwolenia wszystkich czujących istot, odcięcia wszelkich negatywnych emocji, zgłębienia wszystkich buddyjskich nauk i urzeczywistnienia nieprzewyższonej ścieżki Buddy. Dalej wyjaśniono, że pierwsze ślubowanie nie powinno być traktowane dosłownie (jest jedynie metodą pracy z altruistyczną motywacją). W istocie wszystkie istoty muszą same osiągnąć wyzwolenie przez rozpoznanie prawdziwej natury własnego umysłu. Odcięcie negatywnych emocji oznacza w tym kontekście odrzucenie w umyśle tego, co nieprawdziwe i błędne. Ostatnie ślubowanie nakazuje praktykującemu zachowywać w każdej sytuacji skromność, szacunek dla wszystkich istot, unikać przywiązania i, ostatecznie, przebudzić się w mądrości pradżni (Huineng i Yampolsky 2012: 143-144).

Sekcja 22. zawiera nauki dotyczące ,,bezforemnej skruchy'' (無相懺悔 \pinyin{wúxiàng chànhuǐ}), które pozwalają praktykującemu uwolnić się od skutków wcześniejszych negatywnych działań. Huineng uważa, że werbalne wyznawanie grzechów buddom nie ma sensu, zamiast tego zaleca ,,w każdej sytuacji praktykować nie-działanie'' (永斷不作 \pinyin{yǒng duàn bù zuò}). Można to zrozumieć jako pozbycie się negatywnych emocji i złudzeń oraz nigdy niewykonywanie nigdy więcej przynoszących cierpienie działań (Huineng i Yampolsky 2012: 144-145).

W sekcji 23. Huineng przekazał ,,bezforemne nauki o trzech schronieniach'' (無相三歸依戒 \pinyin{wúxiàng sān guīyī jiè}).
W tym fragmencie Szósty Patriarcha utożsamia Buddę z oświeceniem (覺~\pinyin{jué}), Dharmę --- z prawdą (正 \pinyin{zhèng}), a Sanghę --- z czystością (淨 \pinyin{jìng}).
Poleca swym uczniom przyjąć schronienie we własnej oświeconej naturze umysłu, trzymając się z daleka od innych buddów i nauk.
Obiecuje, że jeśli tak uczynią, ich umysł nie będzie splamiony złudzeniami i fizycznym pożądaniem.
Przyjęcie schronienia w Dharmie ma chronić przed błędnymi poglądami i przywiązaniem.
Polegając zaś na Sandze, praktykujący nie będzie ulegał przeszkadzającym emocjom i złudzeniom
(Huineng i Yampolsky 2012: 145-146).

W sekcjach 24-26 Szósty Patriarcha udziela wyjaśnień na temat \glos{\textit{Mahapradżniaparamity}}\glosref{Mahapradżniaparamita}{glosMahaprajnaparamita}.
Huineng podkreśla tu, że Dharmę należy praktykować w swoim umyśle, a nie jedynie bezmyślnie powtarzać.
W tradycji Chan recytacja sutr z gatunku \textit{Mahapradżniaparamity}, takich jak \textit{Sutra Diamentowa} lub \glos{\textit{Sutra Serca}}\glosref{Sutra Serca}{glosHeartSutra} jest bowiem jednym z rodzajów praktyki medytacyjnej.
Wyjaśniając znaczenie nazwy \textit{Mahapradżniaparamita}, Huineng interpretuje pierwszy jej człon, sanskryckie słowo \textit{mahā} (`wielki', chińskie odwzorowanie fonetyczne: 摩訶 \pinyin{móhē}), jako odniesienie do nieograniczonej przestrzeni umysłu, która będąc pusta, zawiera w sobie wszystkie zjawiska.
Człowiek, który praktykuje zasadę \textit{mahā} nie powinien ani lgnąć do zjawisk, ani odpychać ich, lecz traktować je dokładnie tak, jak przestrzeń nieba.
\textit{Pradżnia} (skt. \textit{prajñā}, `mądrość', chińskie odwzorowanie fonetyczne: 般若 \pinyin{bōrě}) oznacza, że w umyśle praktykującego, w żadnej myśli nie może pojawić się niewiedza.
\textit{Paramita} (skt. \textit{pāramitā}, chińskie odwzorowanie fonetyczne: 波羅密多 \pinyin{bōluómìduō}) oznacza ,,osiągnięcie drugiego brzegu''.
Huineng mówi, że ten, kto zrozumie znaczenie słowa \textit{paramita}, będzie poza narodzinami i śmiercią
(Huineng i Yampolsky 2012: 146-148; Buswell 2004: 666).

Sekcja 27. mówi o czystej naturze zaciemnień.
Huineng naucza tu, że gdy medytujący pozbędzie się przeszkadzających emocji, zostanie tylko \textit{pradżnia}, która jest zawsze obecna i nieoddzielna od umysłu.
Ponieważ prawdziwą istotą niewiedzy, złudzeń i błędnych myśli jest natura buddy, praktykujący nie powinen próbować się ich pozbyć, a jedynie rozpoznać ich prawdziwą istotę
(Huineng i Yampolsky 2012: 148-149).

W sekcjach 28-29 Szósty Patriarcha mówi o naturalnych predyspozycjach słuchaczy i praktykujących oraz o mądrości pradżni, nieoddzielnej od natury ich umysłu.
W sekcji 28. zachwala on ponadto \textit{Sutrę Diamentową} jako tekst, który sam w sobie wystarcza do osiągnięcia stanu umysłu zwanego \textit{prajñā samādhi} (z skt. ,,wchłonięcie medytacyjne pradżni', chińskie odwzorowanie fonetyczne: 般若三昧 \pinyin{bōrě sānmèi}).
Stosuje przy tym parabolę o królu smoków, który zbiera wodę z oceanu i zrzuca ją na ziemię.
Jeśli wielki deszcz spadnie na stały ląd, wówczas miasta i wioski spłyną jak trawa i liście.
Jeżeli jednak woda spadnie na taflę oceanu, nic się nie wydarza.
Podobnie człowiek nieposiadający odpowiednich zdolności i predyspozycji nie będzie w stanie zrozumieć tych nauk.
Jednak gdy człowiek praktykujący \glos{Mahajanę}\glosref{Mahajana}{glosMahayana} usłyszy tekst sutry, jego umysł otworzy się i będzie mógł się przebudzić.
Mimo że natura umysłu tych dwóch typów ludzi jest taka sama, słuchacze nieposiadający głębokiego zrozumienia nie są w stanie osiągnąć przebudzenia bezpośrednio po usłyszeniu nauk.
Powodem tego są błędne poglądy oraz przeszkadzające emocje.
Ten zaś, kto rozpozna pradżnię w swoim umyśle, nie będzie już musiał polegać na słowach
% a smok symbolizuje pierwotną mądrość pradżni w umysłach wszystkich czujących istot?
(Huineng i Yampolsky 2012: 149-150).

\if 0
McRae 66
Natura buddy przesłonięta wyłącznie zaciemnieniami umysłu
\fi
