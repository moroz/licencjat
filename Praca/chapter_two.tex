\chapter{Analiza tekstu \textit{Sutry Platformy}}

\if 0
13 T48n2007_p0338b07(01)║
Jedność medytacji 定 i mądrości 惠
Te nauki pojawiają się w 神會語錄
Niedualny pogląd na medytację i mądrość
Nie pytajcie, co było pierwsze (雞還是雞蛋)

14 T48n2007_p0338b15(07)║
straightforward mind 真心 przejrzysty umysł
samadhi of oneness 一行三昧
zakaz siedzenia bez ruchu w medytacji

15 T48n2007_p0338b27(03)║
How are the meditation and wisdom alike?
They are like lamp and light
有燈即有光,無燈即無光

16 T48n2007_p0338b29(01)║
Ludzie praktykują ścieżkę natychmiastowego oświecenia albo stopniową, bo takie mają predyspozycje

17 T48n2007_p0338c03(01)║
無念為宗、無相為體、無住為本
One of the most important teachings of the Sixth Patriarch
aka 無心
Non form is to be separated from form even when associated with form
No thought is not to think even when involved in thought
No abiding is the original nature of man
If you give rise to thoughts from your self-nature....you are not stained....and are always free.

18 T48n2007_p0338c23(06)║
We do not talk about steadfastness 不言動/不是不動
Mind and purity
若言看心。心元是妄。妄如幻故無所看也。
若言看淨。。
看心看淨卻是障道因緣

19 T48n2007_p0339a03(02)║
何名座禪
To see the original nature and not become confused
見本性不亂為禪。

善知識 Dear friends

20 T48n2007_p0339a10(01)║
無相戒 precepts of formlessness bezforemne zasady, bezforemne pouczenia?
If people think of all the evil things, then they will practice evil
The purity of the nature of man is like the blue sky
porównanie zaciemnień do chmur

21 T48n2007_p0339b13(01)║
I vow to save all sentient beings
Każdy ma naturę buddy, with correct views you can be saved
Pradżnia
Cztery ślubowania 四弘大願
ślubuję wyzwolić wszystkie istoty
自身自性自度 save themselves within their own nature
odciąć passions / negatywne emocje 煩惱
ślubuję zgłębiać Dharmę 法門
urzeczywistnić ścieżkę Buddy ,佛道

22 T48n2007_p0339b27(07)║
Formless repentance 無相懺悔
Postanawiamy więcej nie czynić negatywnych działań
Porzucamy niewłaściwy sposób myślenia i pozbywamy się przeszkadzających emocji
永斷不作

23 T48n2007_p0339c07(04)║
皈依 albo 歸依
Formless precepts of the three refuges
Przyjęcie schronienia w trzech klejnotach
take Enlightenment as your master, unikaj niewłaściwych nauk
take refuge in the three treasures of your own natures
Jeżeli przyjmiesz Schronienie w Buddzie, twój umysł nie będzie splamiony
w Dharmie --- nie będziesz się przywiązywał i miał błędnych poglądów
polegamy na buddzie we własnym umyśle, a nie na zewnętrznych buddach

24 T48n2007_p0339c20(03)║
Mahapradżniaparamita
This Dharma must be practiced, not recited
The capacity of the mind is broad and huge, like the sky
błędne koncepcje na temat pustości --- pustka nie jest czarną dziurą, tylko brzemienną przestrzenią, w której wszystkie zjawiska się przejawiają
The emtiness of human nature is also like this
星辰 gwiazdy xing1chen2

25-26 wyjaśnienie słowa Mahapradżniaparamita
25 T48n2007_p0339c29(02)║
Traktować wszystkie zjawiska jednakowo --- nie odrzucać, nie lgnąć, nie być nimi splamionym, a jedynie traktować je jak pustą przestrzeń nieba
to jest znaczenie 'maha' w mahapradżniaparamicie
the deluded man merely recited; the wise man practises with his mind

26 T48n2007_p0340a05(00)║
Czym jest pradżnia? 智惠 chociaż pisze się 智慧
Pradżnia has no shape or form
Jedna głupia myśl odcina pradżnię, jedna mądra ją wytwarza
Czym jest paramita? 言彼岸到
apart from birth and destruction/生滅 birth and death
odcięcie od zewnętrznych czynników odcina związek ze narodzinami i śmiercią, jest jak płynięcie z prądem rzeki, jak dotarcie na drugi brzeg
煩惱是菩提
jeżeli w przeszłych myślach jesteś pomieszany, to jesteś zwyczajnym samsarykiem, jeżeli w przyszłych myślach się wyzwolisz, to będziesz buddą
Mahapradżniaparamita jest najlepsza, jest poza przychodzeniem, odchodzeniem i pozostawaniem, ale dała początek buddom trzech czasów

27 T48n2007_p0340a22(12)║
八萬四千智惠 84000 mądrości (bo są 84000 pomieszanych emocji)
何以故?Why is it so?
紅塵 world of mortals
三昧 samadhi
do not depart from deception and errors, for they are of themselves the nature of true reality
neither grasping nor throwing away

28 T48n2007_p0340a26(05)║
Pochwała Sutry Diamentowej !
pradżnia samadhi
król smoków zbiera wodę z oceanu i zrzuca ją na ziemię, a potem i tak cała woda na ziemi łączy się w oceanie, bo ma jednaką naturę
譬如 pi4ru2 for example
mądrość pradżniaparamity nie polega na słowach 不假文字
諸水衆流卻入大海

29 T48n2007_p0340b10(11)║
Mimo że wszyscy mają taką samą pradżnię w sobie, to ludzie o niewielkich zdolnościach nie osiągają wyzwolenia po usłyszeniu nauk, gdyż przesłony błędnych poglądów są zbyt wielkie
ale każdy, kto usłyszy doktrynę Nagłego Oświecenia, jeżeli nie będzie polegać na zewnętrznych praktykach, a jedynie na własnej prawdziwej naturze, natychmiast osiągnie oświecenie
osoba urzeczywistniona odrzuca ten umysł, który lgnie do rzeczy

30 T48n2007_p0340b22(01)║
Wszystkie teksty buddyjskie zostały napisane przez ludzi. Bo ludzie mają w sobie mądrość. Jeżeli głupi człowiek zapyta mądrego i zrozumie jego odpowiedź, to ten głupi człowiek nie będzie się niczym różnił od mądrego.

\fi

Sutra Platformy wprowadziła nauki o nagłym oświeceniu, jednak podział na pojęcia ,,ścieżki nagłego oświecenia'' i ,,stopniowej ścieżki'' są raczej pozorne, gdyż tak naprawdę chodzi tu o indywidualne zdolności uczniów --- inteligentniejsi, z otwartymi umysłami, są w stanie pojąć nauki o pustości i naturze buddy, i osiągnąć oświecenie w jednej chwili, podczas gdy inni muszą ćwiczyć się na owej ścieżce stopniowo.

W czwartym rozdziale sutry, Huineng nauczał, że za obiekt praktyki duchowej należy przyjąć ,,brak idei'', za jej podstawę przyjąć ,,brak obiektu'', zaś jej fundamentalną zasadą należy uczynić ,,brak przywiązania''. Są to trzy zbliżone i nierozerwalnie związane koncepcje.

,,Brak idei'' jest rozumiany jako wolność od rozproszenia --- pilnowanie, by umysł nie podążał za myślami i aby nic, co pojawia się w umyśle nie odwodziło go od praktyki. W przeciwnym razie, jeżeli praktykujący poświęca czas i energię swoim myślom o teraźniejszości, przeszłości i przyszłości, zaczną one pojawiać się, jedna po drugiej, i ograniczać przejrzystość umysłu. Błędem jest również próba całkowitego pozbycia się myśli; taka praktyka nie umożliwia rozpoznania natury umysłu i nie prowadzi do wyzwolenia. Właściwą praktyką jest koncentracja na prawdziwej naturze Takości (真如 \pinyin{zhēnrú}, skt. \textit{tathātā}), gdyż „Takość jest esencją idei, a idea jest wynikiem aktywności Takości”.

,,Brak obiektu'' oznacza tu unikanie rozproszenia pod wpływem zewnętrznych obiektów. ,,Brak przywiązania'' zaś oznacza traktowanie wszystkich istot, zarówno wrogów, jak też przyjaciół, w taki sam sposób. Praktykujący powinien porzucić myślenie o przeszłości i chęć odwetu za dawne krzywdy (Huineng, Wong i Humphreys 1998: rozdział 4).

Rozdział piąty Sutry Platformy traktuje o \textit{dhyāna}, medytacji. W medytacji Chan nie należy koncentrować się ani na umyśle, ani na czystości. Umysł jako taki jest zwodniczy, jest jedynie iluzją i jako taki nie powinien być obiektem medytacji. Koncentracja na czystości zaś prowadzi do fiksacji na koncepcji czystości. Właściwa medytacja oznacza urzeczywistnienie niewzruszonej esencji umysłu.

Taka praktyka powinna być zrównoważona na poziomie ciała, mowy i umysłu. Praktykujący, który pragnie rozwinąć nieporuszoność, powinien być obojętny na wady innych ludzi. Niewzruszony umysł nie działa w dualistycznych kategoriach, takich jak dobro i zło albo słabość i siła. Analogicznie, praktykujący nie powinien mówić krytykować innych ludzi przy pomocy tych kategorii myślowych (Huineng, Wong i Humphreys 1998: rozdział 5).
%dwell on --- skupiać się na

\if 0
McRae 66
Natura buddy przesłonięta wyłącznie zaciemnieniami umysłu
\fi
