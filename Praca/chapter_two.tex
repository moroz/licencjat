\chapter{Analiza tekstu \textit{Sutry Platformy}}

Sutra Platformy wprowadziła nauki o nagłym oświeceniu, jednak podział na pojęcia ,,ścieżki nagłego oświecenia'' i ,,stopniowej ścieżki'' są raczej pozorne, gdyż tak naprawdę chodzi tu o indywidualne zdolności uczniów --- inteligentniejsi, z otwartymi umysłami, są w stanie pojąć nauki o pustości i naturze buddy, i osiągnąć oświecenie w jednej chwili, podczas gdy inni muszą ćwiczyć się na owej ścieżce stopniowo.

W czwartym rozdziale sutry, Huineng nauczał, że za obiekt praktyki duchowej należy przyjąć ,,brak idei'', za jej podstawę przyjąć ,,brak obiektu'', zaś jej fundamentalną zasadą należy uczynić ,,brak przywiązania''. Są to trzy zbliżone i nierozerwalnie związane koncepcje.

,,Brak idei'' jest rozumiany jako wolność od rozproszenia --- pilnowanie, by umysł nie podążał za myślami i aby nic, co pojawia się w umyśle nie odwodziło go od praktyki. W przeciwnym razie, jeżeli praktykujący poświęca czas i energię swoim myślom o teraźniejszości, przeszłości i przyszłości, zaczną one pojawiać się, jedna po drugiej, i ograniczać przejrzystość umysłu. Błędem jest również próba całkowitego pozbycia się myśli; taka praktyka nie umożliwia rozpoznania natury umysłu i nie prowadzi do wyzwolenia. Właściwą praktyką jest koncentracja na prawdziwej naturze Takości (真如 \pinyin{zhēnrú}, skt. \textit{tathātā}), gdyż „Takość jest esencją idei, a idea jest wynikiem aktywności Takości”.

,,Brak obiektu'' oznacza tu unikanie rozproszenia pod wpływem zewnętrznych obiektów. ,,Brak przywiązania'' zaś oznacza traktowanie wszystkich istot, zarówno wrogów, jak też przyjaciół, w taki sam sposób. Praktykujący powinien porzucić myślenie o przeszłości i chęć odwetu za dawne krzywdy (Huineng, Wong i Humphreys 1998: rozdział 4).

Rozdział piąty Sutry Platformy traktuje o \textit{dhyāna}, medytacji. W medytacji Chan nie należy koncentrować się ani na umyśle, ani na czystości. Umysł jako taki jest zwodniczy, jest jedynie iluzją i jako taki nie powinien być obiektem medytacji. Koncentracja na czystości zaś prowadzi do fiksacji na koncepcji czystości. Właściwa medytacja oznacza urzeczywistnienie niewzruszonej esencji umysłu.

Taka praktyka powinna być zrównoważona na poziomie ciała, mowy i umysłu. Praktykujący, który pragnie rozwinąć nieporuszoność, powinien być obojętny na wady innych ludzi. Niewzruszony umysł nie działa w dualistycznych kategoriach, takich jak dobro i zło albo słabość i siła. Analogicznie, praktykujący nie powinien mówić krytykować innych ludzi przy pomocy tych kategorii myślowych (Huineng, Wong i Humphreys 1998: rozdział 5).
%dwell on -- skupiać się na

\if 0
McRae 66
Natura buddy przesłonięta wyłącznie zaciemnieniami umysłu


\fi
