\chapter*{Glosariusz}
\addcontentsline{toc}{chapter}{Glosariusz}

\label{glosMahayana}
\textbf{Mahajana, Buddyzm Mahajany} (大乘佛教 \pinyin{Dàshèng fójiào} lub \pinyin{Dàchéng fójiào}, `buddyzm Wielkiego Wozu', od skt. \textit{Mahāyāna}, `wielki wóz', nazywany również buddyzmem Wielkiej Drogi) --- jeden z trzech głównych odłamów buddyzmu (dwa pozostałe to Hinajana, tzw. Mała Droga lub Mały Wóz, oraz Wadżrajana, Diamentowa Droga lub Diamentowy Wóz). Filarami Mahajany są wyzwalająca mądrość i współczucie dla wszystkich czujących istot, rozwijane w równowadze. Do buddyzmu Mahajany zalicza się m.in. buddyzm Chan i Zen, Szkołę Czystej Krainy, a także szkołę Gelugpa buddyzmu tybetańskiego.
\medskip

\label{glosPureLand}
\textbf{Szkoła Czystej Krainy}, zwana również \textbf{Szkołą Czystej Ziemi} lub \textbf{amidyzmem} --- tradycja buddyzmu chińskiego, w której za najważniejszą postać przyjmuje się \textit{Buddę Amitabhę} (阿彌陀佛 \pinyin{Āmìtuófó}). Celem praktyki tej tradycji jest odrodzenie po śmierci w Czystej Krainie tego buddy, \textit{Sukhavati} (w języku chińskim nazywana 極樂 \pinyin{Jílè}, 安樂 \pinyin{Ānlè} lub 西天 \pinyin{Xītiān}).\medskip

\label{glosMerit}
\textbf{Zasługa} (功德 \pinyin{gōngdé}, skt. \textit{pu\d{n}ya}) w większości tradycji buddyzmu odnosi się do dobrych wrażeń karmicznych, zebranych w rezultacie właściwego postępowania i podążania ścieżką duchową.  (Buswell 2004: 532).\medskip

\label{glosLowerRealms}
\textbf{Niższe Sfery Egzystencji}:
Według kosmologii buddyjskiej, istoty krążące w samsarze, tj. uwarunkowanej egzystencji, od niemającego początku czasu odradzają się w jednej z sześciu sfer egzystencji, zależnie od swojej karmy i indywidualnych skłonności. Trzy z nich, sfera niebiańska (天道 \pinyin{tiāndào}, skt. \textit{devaloka}), którą zamieszkują bogowie, sfera półbogów lub asurów (阿修羅 \pinyin{Āxiūluódào}) i sfera ludzi (人道 \pinyin{réndào}), nazywa się trzema wyższymi sferami egzystencji (三善道 \pinyin{sān shàndào}), ponieważ życie w tych sferach jest relatywnie przyjemne. Trzy niższe sfery egzystencji, sfera zwierząt (畜牲道 \pinyin{chùshēngdào}), sfera głodnych duchów lub pretów (餓鬼道 \pinyin{èguǐdào}) oraz sfery piekielne (地獄道 \pinyin{dìyùdào}), w których życie pełne jest cierpienia, nazywane są niższymi sferami egzystencji (三惡道 \pinyin{sān èdào}).\medskip

\label{glosMahaprajnaparamita}
\textbf{Mahapradżniaparamita} (skt. \textit{Mahāprajñāpāramitā}, chiń. 摩訶般若波羅蜜多 \pinyin{Móhē Bōrě Bōluómìduō}, `Wielka Doskonałość Mądrości').\medskip

\label{glosZhouDynasty}
\textbf{Dynastia Zhou} (周朝 \pinyin{Zhōu Cháo}) --- dynastia rządząca Chinami w latach ok. 1045-256 p.n.e. Dzieli się na tzw. Zachodnią Dynastię Zhou i Wschodnią Dynastię Zhou.\medskip

\label{glosDiamondSutra}
\textbf{Sutra Diamentowa} (金剛經 \pinyin{Jīngāng jīng}, skt. \textit{Vajracchedikā Prajñāpāramitā Sūtra})\medskip

\label{glosHeartSutra}
\textbf{Sutra Serca} (skt. \textit{Prajñāpāramitā H\d{r}dayasūtra}, chiń. 般若波羅密多心經)
