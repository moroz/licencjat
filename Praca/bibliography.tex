\section*{Bibliografia}

Huineng, Mou-lam Wong i Christmas Humphreys. 1973. \textit{The sutra of Wei Lang (or Hui Neng)}. Westport, Conn: Hyperion Press. \url{http://www.sinc.sunysb.edu/Clubs/buddhism/huineng/content.html}

Huineng i Hsüan Hua. 1977. \textit{The Sixth Patriarch's Dharma jewel platform sutra, with the commentary of Tripitaka Master Hua} [translated from the Chinese by the Buddhist Text Translation Society]. San Francisco: Sino-American Buddhist Association. \url{http://www.cttbusa.org/6patriarch/6patriarch_contents.asp}

Huineng, \textit{Sutra Szóstego Patriarchy Zen}, tłumacz nieznany, \url{http://www.zen.ite.pl/teksty/sutra6.html}

William E. Soothill i Lewis Hodous. 2003. \textit{A Dictionary of Chinese Buddhist Terms.} RoutledgeCurzon.% \url{http://buddhistinformatics.ddbc.edu.tw/glossaries/files/soothill-hodous.ddbc.pdf}

Buswell, Robert E. 2004. \textit{Encyclopedia of Buddhism.} New York: Macmillan Reference.

McRae, John R. 2004. \textit{Seeing through Zen encounter, transformation, and genealogy in Chinese Chan Buddhism.} Berkeley, Calif: University of California Press.

Anonim 佚名 2007. ``Fojiao de futian'' 佛教的福田 [Pola błogosławieństw w buddyzmie]. \textit{Zhongguo minzu bao} 中國民族報, za: \url{http://www.wuys.com/news/Article_Show.asp?ArticleID=12791}

Nydahl, Ole. 2010. ``Sześć wyzwalających działań''. \textit{Diamentowa Droga} 34. \url{http://diamentowadroga.pl/dd34/szesc_wyzwalajacych_dzialan}

Huineng, Morten Schlütter i Stephen F. Teiser. 2012. \textit{Readings of the Platform sūtra.} New York: Columbia University Press.% \url{http://site.ebrary.com/id/10538320}.

Huineng, Philip B. Yampolsky i Huineng. 2012. \textit{The Platform sutra of the Sixth Patriarch the text of the Tun-huang manuscript.} New York: Columbia University Press.% \url{http://public.eblib.com/choice/publicfullrecord.aspx?p=909420.}
