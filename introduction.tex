\chapter*{Wprowadzenie}
\addcontentsline{toc}{chapter}{Wprowadzenie}
\markright{WPROWADZENIE}
\setcounter{chapter}{1}
\textit{Sutra Platformy Szóstego Patriarchy} (chiń. 六祖壇經, Pinyin: \pinyin{Liùzǔ Tánjīng}) jest apokryficznym tekstem buddyzmu Chan (patrz: sekcja \textit{Przedstawienie buddyzmu Chan}), przypisywanym legendarnemu Szóstemu Patriarsze Chan, Dajian Huinengowi (chiń. 大鑒惠能, Pinyin: \nazwisko{Dàjiàn Huìnéng}). Najstarsza zachowana wersja tego dzieła powstała w VIII w. w Chinach.

Celem niniejszej pracy jest przedstawienie historycznego tła powstania \textit{Sutry Platformy}, analiza jej treści oraz opisanie wpływu, jaki wywarła na buddyzm chiński.

W pierwszym rozdziale pracy omówiona jest hagiografia Huinenga, głównej postaci sutry.
Rozdział drugi stanowi analiza treści \textit{Sutry Platformy}, trzeci natomiast poświęcony jest dalszemu rozwojowi szkoły Chan.

\section{Wstęp techniczny}
Terminy chińskie w niniejszej pracy podane są w nawiasach w znakach tradycyjnych oraz w transkrypcji \textit{Hanyu Pinyin} (漢語拼音 \pinyin{Hànyǔ Pīnyīn}) z oznaczonymi tonami. Wyjątkiem są chińskie nazwiska, do których nie podano transkrypcji.

Odwołania do tekstu \textit{Sutry Platformy} w niniejszej pracy odnoszą się do jej przekładu pt. \textit{The Platform Sutra of the Sixth Patriarch: The Text of the Tun-huang Manuscript} Philipa B. Yampolsky'ego, wydanego drukiem przez Columbia University Press w roku 1967 i wznowionego w roku 2012.
Fragmenty oryginału podano za wersją z \textit{Chinese Electronic Tripitaka}, opublikowaną w Internecie przez Chinese Buddhist Electronic Text Association.
%
% Słowo ,,Budda'', zapisywane w tekście wielką literą, odnosi się do historycznego Buddy Siakjamuniego, tj. Siddhārta Gautamy. Zapisywane małą literą ,,budda'' oznacza stan umysłu.

\section{Przedstawienie buddyzmu Chan}
Buddyzm Chan (禪宗 \pinyin{Chán zōng}) jest, obok Szkoły Czystej Krainy%
\footnote{Szkoła Czystej Krainy, zwana również Szkołą Czystej Ziemi lub amidyzmem --- tradycja buddyzmu chińskiego, w której za najważniejszą postać przyjmuje się Buddę Amitābhę (阿彌陀佛 \pinyin{Āmìtuófó}). Celem praktyki tej tradycji jest odrodzenie po śmierci w Czystej Krainie tego buddy, \textit{Sukhavati} (w języku chińskim nazywana 極樂 \pinyin{Jílè}, 安樂 \pinyin{Ānlè} lub 西天 \pinyin{Xītiān}).}
jedną z najważniejszych tradycji buddyzmu Mahajany%
\footnote{Buddyzm Mahajany (大乘佛教 \pinyin{Dàshèng fójiào} lub \pinyin{Dàchéng fójiào}, `buddyzm Wielkiego Wozu', od skt. \textit{Mahāyāna}, `wielki wóz', nazywany również buddyzmem Wielkiej Drogi) --- jeden z trzech głównych odłamów buddyzmu (dwa pozostałe to Hinajana, tzw. Mała Droga lub Mały Wóz, oraz Wadżrajana, Diamentowa Droga lub Diamentowy Wóz). Filarami Mahajany są wyzwalająca mądrość i współczucie dla wszystkich czujących istot, rozwijane w równowadze. Do buddyzmu Mahajany zalicza się m.in. buddyzm Chan i Zen, Szkołę Czystej Krainy, a także szkołę Gelug buddyzmu tybetańskiego.}
w Chinach. Samo słowo \textit{chan} jest chińskim wariantem %odwzorowaniem
sanskryckiego słowa \textit{dhyāna}, które oznacza medytację.
Początkowo termin ten zapisywano w języku chińskim jako \textit{channa} (禪那), dawniej wymawiane \textit{dianna}, co było fonetycznym odwzorowaniem oryginalnego terminu.
W późniejszym okresie upowszechniła się skrócona forma \textit{chan}.
W Japonii buddyzm Chan nazywany jest Zen\footnote{Shinjitai Kanji: {\ipaexgothic 禅}, Rōmaji: \textit{zen}.}, i pod taką nazwą znany jest na Zachodzie.
Buddyzm Chan przeniknął również do Korei, gdzie nazywany jest Seon\footnote{Hangeul: {\Korean 선}, Revised Romanization of Korean: \textit{seon}}.

Tradycyjne chińskie zapisy dotyczące historii tradycji Chan przedstawiają ją jako nieprzerwaną linię przekazu nauk i doświadczenia z patriarchy (祖師 \pinyin{zǔshī}) na patriarchę, sięgającą aż do historycznego Buddy.
Obecnie uważa się jednak, że takie zapisy nie oddają stanu faktycznego.
Miały one nadać szkole Chan autentyczność i przedstawić go jako ortodoksyjną, lepszą od innych tradycję buddyzmu, w ramach rywalizacji z innymi odłamami o wsparcie warstwy rządzącej oraz osób świeckich.
Pierwsze wzmianki o linii przekazu patriarchów Chan pojawiają się na steli pogrzebowej poświęconej mnichowi Faru (法如 \nazwisko{Fǎrú}), który miał być uczniem Piątego Patriarchy, Daman Hongrena (大滿弘忍 \nazwisko{Dàmǎn Hóngrěn}).
Według tego zapisu linia przekazu Chan prowadziła od Bodhidharmy (菩提達摩 Pútídámó, od skt. \textit{bodhi} `oświecenie' i \textit{dharma} `zjawiska; nauki Buddy'), poprzez Dazu Huike  (大祖慧可 \nazwisko{Dàzǔ Huìkě}), Jianzhi Sengcana (鑑智僧璨 \nazwisko{Jiànzhì Sēngcàn}), Dayi Daoxina (大醫道信 \nazwisko{Dàyī Dàoxìn}) i Hongrena, do Faru
(Huineng, Schlütter i Teiser 2012: 53-54, 56).

Za założyciela buddyzmu Chan uznaje się przybyłego z Indii mistrza Bodhidharmę.
Pierwsza wzmianka o nim pojawia się w tekście pt. ``Zapisy o klasztorach w Luoyangu'' (洛陽伽藍記 \pinyin{Luòyáng qiélán jì}) autorstwa Yang Xuanzhi (楊衒之), pochodzącym z ok. 547 roku.
Dzieło to opisuje świątynie i klasztory miasta Luoyang (洛陽 \toponim{Luòyáng}) w prowincji Henan (河南 \toponim{Hénán}), a także podróżnych, którzy przybywali tam z dalekich stron, by podziwiać wyszukaną architekturę.
Bodhidharma został w tym tekście przedstawiony jako stupięćdziesięcioletni \textit{śrama\d{n}a} (`święty mąż; mnich' chiń. 沙門 \pinyin{shāmén}) z Persji.
Przybywszy do Luoyangu, Pierwszy Patriarcha miał się zachwycać pięknem tamtejszych świątyń, a zwłaszcza jednej, zwanej Yongning Si (永寧寺 \pinyin{Yǒngníng Sì}).
Budowla ta została wzniesiona w roku 516 i zniszczona w wyniku działań wojennych oraz katastrof naturalnych w roku 526, a więc można wnioskować, że Bodhidharma przebywał w mieście w ciągu tych dziesięciu lat
(McRae 1986: 17).

Kanoniczna biografia Pierwszego Patriarchy oparta jest na przedmowie do przypisywanego mu ``Traktatu o dwóch wejściach i czterech praktykach'' (二入四行論 \pinyin{Èrrù sìxíng lùn}).
Została ona napisana przez uczonego imieniem Tanlin (曇林 \nazwisko{Tánlín}), specjalistę od tekstu zwanego ``Sutrą Ryku Lwa Królowej Śrimala'' (skt. \textit{Śrīmālādevī Si\d{m}hanāda Sūtra}, chiń. 勝鬘師子吼一乘大方便方廣經 \pinyin{Shèng mán shīzi hǒu yī chéng dà fāngbiàn fāng guǎng jīng}).
Tradycyjnie uważano go za ucznia Bodhidharmy, jednak bardziej prawdopodobne jest, że jego nauczycielem był Dazu Huike, uczeń Pierwszego Patriarchy.
Tanlin podaje, że Bodhidharma był trzecim synem pewnego króla z południowych Indii, i że ,,przeszedł przez morza i góry'', by nauczać buddyzmu na północy Chin.
Według tego zapisu jego najważniejszymi uczniami byli Daoyu (道育 \nazwisko{Dàoyù}) i Huike.
Większa część ``Traktatu o dwóch wejściach i czterech praktykach'' jest niemal identyczna z sutrą pt. \textit{Vajrasamādhi-sūtra} (金剛三昧經 \pinyin{Jīngāng sānmèi jīng}).
Tekst ten nie przedstawiał sobą nic nowatorskiego, nie licząc koncepcji \textit{biguan} (壁觀 \pinyin{bìguān}), dosłownie ,,patrzenia na ścianę''.
W traktacie biograficznym ``Kontynuowane biografie wybitnych mnichów'' (續高僧傳 \pinyin{Xù gāosēng zhuàn}) autorstwa mistrza Daoxuan (道宣 \nazwisko{Dàoxuān}) z dynastii Tang\footnote{Dynastia Tang (唐朝 \pinyin{Táng Cháo}) --- dynastia panująca w Chinach w latach 618-907. Okres szybkiego rozwoju buddyzmu chińskiego.} zawarta została natomiast zmodyfikowana wersja biografii z ``Traktatu o dwóch wejściach i czterech praktykach''.
Daoxuan uściślił lakoniczny zapis o podróży Bodhidharmy, podając, że przybył on drogą morską do południowych Chin za czasów dynastii Liu Song\footnote{Dynastia Liu Song (劉宋朝 \pinyin{Liú Sòng Cháo}), zwana też Południową Song (南宋朝 \pinyin{Nán Sòng Cháo}) --- dynastia panująca w południowych Chinach w latach 420-479, pierwsza z czterech Południowych Dynastii (南朝 \pinyin{Nán Cháo}).} i przeprawił się przez rzekę Yangzi (揚子 \toponim{Yángzǐ}).
Bodhidharma miał też udzielić Huike przekazu \textit{Sutry Lankavatara} (skt. \textit{La\.nkāvatāra}, chiń. 楞伽經 \pinyin{Léngqié jīng}).
Z biografii tej wynika, że Bodhidharma musiał przybyć do Chin przed rokiem 479, kiedy dynastia Liu Song została podbita przez Południową Qi%
\footnote{Dynastia Południowa Qi (南齊朝 \pinyin{Nán Qí cháo}) --- dynastia panująca w południowych Chinach w latach 479-502, druga z czterech Południowych Dynastii.}
(Broughton 1999: 53-56; Buswell 2004: 57; Dumoulin 1963: 71).

W ``Kontynuowanych biografiach wybitnych mnichów'' zawarta jest również historia życia praktykującego imieniem Sengfu (僧副).
Był on uczniem mistrza dhjany, nazwanego w tekście imieniem Dharma. Sengfu pochodził z powiatu Qi (祁縣 \toponim{Qíxiàn}) w pobliżu miasta Jinzhong (晉中 \toponim{Jìnzhōng}) w prowincji Shanxi (山西 \toponim{Shānxī}).
Spotkał on swego nauczyciela w jaskini, w której ten mieszkał, a otrzymawszy od niego pouczenia na temat ,,zasad medytacji'' (定學宗 \pinyin{dìngxué zōng}), przyjął ślubowania mnisie. Pomiędzy rokiem 494 a 497 udał się do miasta Jiankang\footnote{Jiankang (建康 \toponim{Jiànkāng}), w różnych okresach znane również pod nazwami Jianye (建鄴 \toponim{Jiànyè}) oraz Jinling (金陵 \toponim{Jīnlíng}) --- miasto w delcie rzeki Yangzi, stolica sześciu różnych dynastii, m.in. czterech Południowych Dynastii. Ruiny Jiankangu znajdują się w granicach administracyjnych Nankinu (南京 \toponim{Nánjīng}) w prowincji Jiangsu.}, wówczas stolicy południowych Chin, i osiedlił w świątyni Dinglin Xia Si (定林下寺 \pinyin{Dìnglín xià sì}) w bezpośrednim sąsiedztwie miasta.
Jeżeli przyjąć, że mistrz Dharma był w istocie Bodhidharmą, tak jak czynią to niektórzy historycy buddyzmu, z tego zapisu wynikałoby, że Pierwszy Patriarcha przewędrował na północ najpóźniej w roku 495, a być może nawet około roku 480.
(McRae 1986: 18-21).

Wedle tradycji Bodhidharma miał otrzymać przekaz Dharmy, pochodzący w nieprzerwanej linii od indyjskiego mistrza Mahakaśjapy (skt. \textit{Mahākāśyapa}, chiń. 摩訶迦葉 \nazwisko{Móhējiāshè} lub \nazwisko{Móhējiāyè}), ucznia historycznego Buddy Siakjamuniego, Siddhārta Gautamy
(Buswell 2004: 57).

\section{Pochodzenie \textit{Sutry Platformy} oraz jej przekłady na język angielski}
\textit{Sutra Platformy Szóstego Patriarchy} (六祖壇經 \pinyin{Liùzǔ Tánjīng}) jest apokryficznym tekstem buddyzmu Chan.
Jego najstarsza zachowana wersja powstała w VIII w. w Chinach. Tekst napisany został częściowo w formie monologu, a częściowo w formie dialogu nauczyciela z uczniami.
Nauki w niej zawarte miały zostać wygłoszone przez legendarnego Szóstego Patriarchę.
Huineng jest w \textit{Sutrze Platformy} przedstawiony jako niepiśmienny, prosty człowiek z leżącego poza zasięgiem chińskiej cywilizacji południa, choć w istocie jego nauki bazują na powszechnie znanych w VIII w. tekstach Mahajany
(Huineng, Schlütter i Teiser 2012: 78).

Pełen tytuł \textit{Sutry Platformy} brzmi ``Doktryna nagłego oświecenia Szkoły Południowej, Najwyższa Doskonałość Mądrości Mahajany: Sutra Platformy, przekazana przez Szóstego Patriarchę Huineng w świątyni Dafan, w prefekturze Shao'' (南宗頓教最上大乘摩訶般若波羅蜜經六祖惠能大師於韶州大梵寺施法壇經 \pinyin{Nánzōng dùnjiào zuìshàng dàshèng móhēbōrě bōluómì jīng liùzǔ Huìnéng Dàshī yú Shāozhōu Dàfán Sì shīfǎ Tánjīng}).
W języku chińskim zwykle nazywana jest w skrócie ``Sutrą platformy'' (壇經 \pinyin{Tánjīng}) ,``Sutra Platformy Szóstego Patriarchy'' (六祖壇經 \pinyin{Liùzǔ Tánjīng}, bądź ``Skarb Dharmy, Sutra Platformy Szóstego Patriarchy'' (六祖大師法寶壇經 \pinyin{Liùzǔ Dàshī Fǎbǎo Tánjīng}).

\textit{Sutrę Platformy} zalicza się do korpusu dzieł \textit{Mahapradżniaparamity}%
\footnote{Mahapradżniaparamita (skt. \textit{Mahāprajñāpāramitā}, chiń. 摩訶般若波羅蜜多 \pinyin{Móhē Bōrě Bōluómìduō}, `Wielka Doskonałość Mądrości') --- zbiór tekstów, nauk i praktyk oraz związana z nimi tradycja filozoficzna Mahajany, podkreślająca rolę mądrości jako najważniejszej z Sześciu Paramit (patrz: przyp. na str. \pageref{Paramitas}).
Według tych nauk, doskonałość tylko w tej jednej paramicie wystarczy do osiągnięcia perfekcji w pozostałych pięciu. Do tego gatunku zalicza się m.in. takie dzieła, jak \textit{Sutrę diamentową} (patrz: przyp. na str. \pageref{DiamondSutra}), \textit{Sutrę Serca} (patrz: przyp. na str. \pageref{HeartSutra}), a także omawianą tu \textit{Sutrę Platformy}.}.
Na przynależność sutry do tego gatunku wskazują, oprócz przekazanych w tekście nauk na temat Doskonałości Mądrości, jej pełen tytuł (patrz: poprzedni akapit), a także słowa, którymi Szósty Patriarcha miał rozpocząć swoją przemowę: ,,Drodzy przyjaciele, oczyśćcie swoje umysły i skoncentrujcie się na Dharmie Wielkiej Doskonałości Mądrości.''
Ponadto w \textit{Sutrze Platformy} pojawiają się nauki z doktryny \textit{Tathāgatagarbha}%
\footnote{\textit{Tathāgatagarbha} (od skt. \textit{Tathāgata} `budda', dosł. `Ten, Który Przyszedł w Ten Sposób' + \textit{garbha} `płód; zarodek', chiń. 如來藏 \pinyin{Rúláizàng}) --- grupa sutr buddyzmu Mahajany.
Ich głównym założeniem jest obecność w każdej czującej istocie zalążka oświecenia, tzw. ,,natury buddy'', dzięki któremu możliwe jest wykroczenie poza krąg uwarunkowanej egzystencji i osiągnięcie stanu buddy.} % powtórzenie
(Huineng i Yampolsky 2012: 127; Huineng, Schlütter i Teiser 2012: 78).

\textit{Sutrę Platformy} uważa się za jedno z najważniejszych dzieł buddyzmu Chan, ponieważ wprowadziła nauki o nagłym oświeceniu (頓教 \pinyin{dùnjiào}, `nagła szkoła, subityzm'), stojące w opozycji do nauk tzw. stopniowej szkoły (漸教 \pinyin{jiànjiào}), i wywołała podział szkoły Chan na odłam północny i południowy. % reference
(Buswell 2004: 347-348).%; McRae, 2004: ).

Najważniejsze z koncepcji, które wprowadziła \textit{Sutra Platformy}, to identyfikacja medytacji z mądrością (omówione szerzej w sekcji \textit{Nauki o medytacji} na str. \pageref{DingHui}), ,,oświecenie przez postrzeganie własnej natury'' (見性成佛 \pinyin{jiàn xìng chéng fó}).
Ważnym aspektem tekstu są nauki o tym, że każda czująca istota ma naturę buddy, i że zarówno ludzie świeccy, jak i mnisi mogą z powodzeniem praktykować jego nauki.
Tekst opisuje również specjalny rytuał przekazywania mnichom i świeckim praktykującym ,,bezforemnych zasad'' (無相戒 \pinyin{wúxiàng jiè}). % formless precepts
Były to niektóre z powodów, dla których w roku 796 Huineng został oficjalnie obwołany szóstym patriarchą Chan przez cesarską komisję, a jego dzieło stworzyło podwaliny pod dalszy rozwój szkoły Chan
(Huineng, Schlütter i Teiser 2012: 2).
