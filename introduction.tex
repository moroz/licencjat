\chapter*{Wstęp}
\addcontentsline{toc}{chapter}{Wstęp}
\markboth{Wstęp}{Wstęp}
Terminy chińskie w niniejszej pracy podane są w nawiasach w znakach tradycyjnych oraz w transkrypcji \textit{Hanyu Pinyin} (漢語拼音 \pinyin{Hànyǔ Pīnyīn}) z oznaczonymi tonami. Wyjątkiem są chińskie nazwiska, do których nie podano transkrypcji.
Wszystkie terminy japońskie w nawiasach zapisano współczesnym pismem japońskim oraz w romanizacji Hepburna (jap. {\ipaexgothic ヘボン式ローマ字} \textit{Hebon-shiki Rōmaji}).

Odwołania do tekstu \textit{Sutry Platformy} w niniejszej pracy odnoszą się do jej przekładu pt. \textit{The Platform Sutra of the Sixth Patriarch: The Text of the Tun-huang Manuscript} Philipa B. Yampolsky'ego, wydanego drukiem przez Columbia University Press w roku 1967 i wznowionego w roku 2012.
Fragmenty oryginału podano za wersją z \textit{Chinese Electronic Tripitaka}, opublikowaną w Internecie przez Chinese Buddhist Electronic Text Association.
