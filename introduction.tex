\chapter*{Wstęp}
\addcontentsline{toc}{chapter}{Wstęp}
\markboth{Wstęp}{Wstęp}
\renewcommand{\headrulewidth}{0.3pt}

\textit{Sutra Platformy Szóstego Patriarchy} (chiń. 六祖壇經, Pinyin: \pinyin{Liùzǔ Tánjīng}) jest apokryficznym tekstem buddyzmu chan (patrz: rozdział 1), przypisywanym legendarnemu Szóstemu Patriarsze Chan, Dajian Huinengowi (chiń. 大鑒惠能, Pinyin: \nazwisko{Dàjiàn Huìnéng}). Najstarsza zachowana wersja tego dzieła powstała w VIII w. w Chinach.

Celem niniejszej pracy jest przedstawienie historycznego tła powstania \textit{Sutry Platformy}, analiza jej treści oraz opisanie wpływu, jaki wywarła na buddyzm chiński.

W pierwszym rozdziale pracy przedstawiono buddyzm chan i opisano jego historię.

W drugim rozdziale pracy omówiona jest hagiografia Huinenga, głównej postaci sutry.
Rozdział trzeci stanowi analiza treści \textit{Sutry Platformy}, czwarty natomiast poświęcony jest dalszemu rozwojowi szkoły chan.

Terminy chińskie w niniejszej pracy podane są w nawiasach w znakach tradycyjnych oraz w transkrypcji \textit{Hanyu Pinyin} (漢語拼音 \pinyin{Hànyǔ Pīnyīn}) z oznaczonymi tonami. Wyjątkiem są chińskie nazwiska, do których nie podano transkrypcji.
Wszystkie terminy japońskie w nawiasach zapisano współczesnym pismem japońskim oraz w romanizacji Hepburna (jap. {\ipaexgothic ヘボン式ローマ字} \textit{Hebon-shiki Rōmaji}).

Odwołania do tekstu \textit{Sutry Platformy} w niniejszej pracy odnoszą się do jej przekładu pt. \textit{The Platform Sutra of the Sixth Patriarch: The Text of the Tun-huang Manuscript} Philipa B. Yampolsky'ego, wydanego drukiem przez Columbia University Press w roku 1967 i wznowionego w roku 2012.
Fragmenty oryginału podano za wersją z \textit{Chinese Electronic Tripitaka}, opublikowaną w Internecie przez Chinese Buddhist Electronic Text Association.
