\chapter{Biografie Huinenga}
Życie Szóstego Patriarchy jest owiane tajemnicą. Jego imię pojawia się w kronice pt. ``Księga przekazu lampy z okresu Jingde'' (景德傳燈錄 \pinyin{Jǐngdé chuándēng lù}) jako jednego z dziesięciu głównych uczniów Piątego Patriarchy Hongren.
Z owego tekstu nie wynika jednak, by był postacią szczególnie ważną dla rozwoju całej szkoły chan, wspomniano w nim natomiast, że Huineng żył i nauczał w miejscowości Caoqi (曹溪 \toponim{Cáoqī}).

Imię Huineng pojawia się również w pewnym tekście z grot Dunhuang%
\footnote{Groty Dunhuang (敦煌石窟 \pinyin{Dūnhuáng shíkū}), --- zbiorcze określenie stanowisk archeologicznych w północno-zachodniej części prowincji Gansu (甘肅 \toponim{Gānsù}), nazwanych od pobliskiego miasta Dunhuang (敦煌 \toponim{Dūnhuáng}).},
upamiętniającym Piątego Patriarchę, Hongrena, jednak tekst ów nie mówi nic o związanych z Huinengiem doktrynach.
Kanoniczna biografia Huinenga oparta jest na przypisywanej mu \textit{Sutrze Platformy}. Pierwszą osobą, która przedstawiła Huinenga jako świętego, był Heze Shenhui (菏澤神會 \nazwisko{Hézé Shénhuì}, 684-758).
Biografia Szóstego Patriarchy w takiej wersji, jak opisana w tekście \textit{Sutry Platformy} z Dunhuang, jest najprawdopodobniej uzupełnioną i zmienioną wersją jego opowieści (McRae 2004: 68).

\section{Biografia Huinenga według \textit{Sutry Platformy} w wersji z Dunhuang}
Według \textit{Sutry Platformy} Huineng urodził się w regionie Nanhai (南海 \toponim{Nánhǎi}, położonym w granicach obecnej prowincji Guangdong (廣東 \toponim{Guǎngdōng}).
Za ramy czasowe jego życia przyjmuje się lata 638-713. Szósty Patriarcha jest w tym tekście przedstawiany jako ubogi, niepiśmienny człowiek świecki
(Huineng i Yampolsky 2012: 126, 174).

Jak podaje tekst, ojciec Huinenga był urzędnikiem z regionu Fanyang (范陽 \toponim{Fànyáng}), obecnie miasto Zhuozhou (涿州 \toponim{Zhuōzhōu}) w prowincji Hebei (河北 \toponim{Héběi}), lecz został odwołany ze stanowiska i skazany na banicję.
W związku z tym musiał przenieść się z całą rodziną do Xinxing, gdzie niedługo później zmarł. Po jego śmierci Huineng trudnił się zbieraniem i sprzedażą drewna na opał
(Huineng i Yampolsky 2012: 126).

Pewnego dnia, gdy dwudziestodwuletni Huineng sprzedawał drewno na targowisku, pewien klient zamówił zapas opału z dostawą do jego sklepu.
Huineng dostarczył towar i otrzymał swoją zapłatę, a kiedy wyszedł na zewnątrz, spotkał człowieka, który recytował na ulicy Sutrę Diamentową%
\footnote{Sutra Diamentowa (金剛經 \pinyin{Jīngāng jīng}, skt. \textit{Vajracchedikā Prajñāpāramitā Sūtra}) --- jeden z najważniejszych tekstów buddyzmu mahajany. Ułożony prawdopodobnie około II lub IV w. n.e. w Indiach, przetłumaczony na język chiński na początku V w.}.\label{DiamondSutra}
Usłyszawszy ów tekst, Huineng uzyskał wgląd w naturę swego umysłu i osiągnął oświecenie. Następnie spytał tajemniczego mężczyznę, skąd pochodził.
Ten odpowiedział, że przybył ze świątyni Dongshan Si (東山寺 \pinyin{Dōngshān sì}) na górze Fengmu (憑墓山 \toponim{Féngmù shān}) w powiecie Huangmei (黃梅懸 \toponim{Huángméi xiàn}) w Qizhou (蘄州 \toponim{Qízhōu}), którego opatem był Piąty Patriarcha, Hongren, i gdzie przebywało około tysiąca mnichów.
Patriarcha zalecił swoim uczniom recytować ową sutrę, mówiąc, że dzięki tej praktyce można szybko osiągnąć oświecenie.
Wkrótce potem Huineng spotkał kogoś, kto poradził mu udać się do klasztoru Dongshan, aby poprosić Patriarchę o nauki, i dał mu pieniądze na zaaranżowanie opieki dla starej matki
(Huineng i Yampolsky 2012: 127).

Kiedy Huineng dotarł do klasztoru, Piąty Patriarcha Hongren zapytał go, skąd i w jakiej sprawie przyszedł do patriarchy.
Huineng odparł, że przybył z Kantonu\footnote{Nazwa ,,Kanton'' w języku polskim odnosi się zarówno do prowincji Guangdong (廣東 \toponim{Guǎngdōng}), jak i do jej stolicy --- miasta Guangzhou (廣州 \toponim{Guǎngzhōu}).}, by oddać cześć patriarsze, oraz że nie prosi o nic prócz Dharmy.
Patriarcha stwierdził wówczas, że Huineng, jako \textit{geliao} (獦獠 \pinyin{géliáo}, `barbarzyńca'), niegodny jest otrzymania nauk.
Obszar obecnego Kantonu był wówczas zamieszkany przez niechińskie ludy, mówiące własnymi językami, posiadające własną kulturę i nieżyjące zgodnie z naukami Buddy --- mieszkańcy południa polowali bowiem i jedli mięso.
Dla wielu ówczesnych buddystów nie do pomyślenia było, by człowiek z południa mógł otrzymać nauki od Patriarchy i osiągnąć oświecenie.
Huineng odparł wtedy, że chociaż w społeczeństwie istnieją takie podziały, nie mają one wpływu na naturę buddy, która jest taka sama we wszystkich ludziach.
Patriarcha uznał, że Huineng dobrze rozumiał nauki Buddy, lecz w obawie, że inni uczniowie mogliby zrobić mu krzywdę, wyznaczył mu prace gospodarcze.
Przez następne osiem miesięcy Huineng rąbał drewno i młócił zboże
(Huineng i Yampolsky: 127-128; Huineng, Schlütter i Teiser 2012: 27).

Pewnego dnia Patriarcha Hongren zwołał zebranie wszystkich uczniów i ogłosił:
\longquote{Dla ludzi w tym świecie narodziny i śmierć są doniosłymi kwestiami. Całymi dniami składacie dary i poszukujecie pola błogosławieństw, ale nie staracie się wyzwolić z pełnego goryczy oceanu uwarunkowanej egzystencji%
\footnote{Idea ,,pola błogosławieństw'' (福田 \pinyin{fútián}, skt. \textit{pu\d{n}yak\d{s}etra}) jest związana z buddyjską koncepcją karmy jako prawa przyczyny i skutku. Oznacza stan, w którym dana osoba zgromadziła bardzo wiele dobrych wrażeń w umyśle, w rezultacie praktyki szczodrości (skt. \textit{dānā}, chiń. 布施 \pinyin{bùshī}), pierwszej z tzw. Sześciu Paramit lub Sześciu Wyzwalających Działań (pozostałe pięć to właściwe działanie, cierpliwość, radosny wysiłek, medytacja i mądrość).\label{Paramitas}
Piąty Patriarcha Hongren krytykował tu swoich uczniów, ponieważ praktyka szczodrości jest wprawdzie w buddyzmie postrzegana jako pozytywne działanie, jednak nie wystarcza ona do osiągnięcia ostatecznego wyzwolenia z samsary (Anonim 2007; Nydahl 2010).}.
Wasze własne ego stoi na drodze do błogosławieństw. Jak w takiej sytuacji możecie osiągnąć wyzwolenie? Powróćcie teraz do swoich cel i spójrzcie w swój umysł. Ludzie mądrzy samoistnie pojmą prawdziwą naturę \textit{pradżni}\fnm. Niech każdy z was napisze wiersz i przyniesie mi go. Przeczytam każdy z nich, a jeżeli jest wśród was ktoś, kto rozpoznał swoją prawdziwą naturę, przekażę mu swoją szatę i Dharmę, a także uczynię go Szóstym Patriarchą. Spieszcie się\footnote{Huineng i Yampolsky 2012: 128.}!}
\footnotetext{Pradżnia (skt. \textit{prajñā}, w języku chińskim nazywana 慧 \pinyin{huì}, 智 \pinyin{zhì} lub 智慧 \pinyin{zhìhuì} --- wszystkie trzy terminy oznaczają `mądrość' --- lub fonetycznie 般若 \pinyin{bōrě}), to, obok współczucia (悲 \pinyin{bēi} `litość', skt. \textit{karu\d{n}ā}), jedna z dwóch najważniejszych cnót buddyzmu mahajany. Termin ten można rozumieć na wiele sposobów, zależnie od tradycji, praktykowanej ścieżki i metody interpretacji. Tu odnosi się do prawidłowego, ponadintelektualnego zrozumienia prawdziwej natury zjawisk.}

Mnisi stwierdzili zgodnie: ,,Nie ma sensu oczyszczać umysłu i zadawać sobie trudu układania wiersza dla patriarchy. Yuquan Shenxiu (玉泉神秀 \nazwisko{Yùquán Shénxiù}), przewodniczący kongregacji, jest naszym nauczycielem.
Kiedy on zostanie patriarchą, możemy liczyć na jego wsparcie. Dlatego nie będziemy układać wierszy.'' Żaden z nich nie podjął się więc tego zadania
(Huineng i Yampolsky 2012: 127).

Shenxiu zaś dręczyły wątpliwości. Z jednej strony nie uważał się za godnego przyjęcia stanowiska patriarchy, z drugiej zaś pragnął otrzymać przekaz nauk.
Przyjęcie przekazu Dharmy dla pożytku istot byłoby bowiem pożądane i chwalebne, nie mógł jednak za wszelką cenę dążyć do zostania patriarchą.
Wychodził z założenia, że jeżeli jego mistrz uzna jego urzeczywistnienie natury umysłu za niewystarczające, to będzie musiał pogodzić się z faktem, iż kto inny zostanie dzierżawcą linii przekazu.
Wreszcie skomponował wiersz i wymknął się w nocy ze swojej celi, by napisać go na ścianie, w miejscu, gdzie miały zostać namalowane sceny z sutry \textit{La\.nkāvatāra} (楞伽經 \pinyin{Léngqié jīng}) (McRae 2004: 62):

\vspace*{6pt}
\begin{minipage}[t]{0.4\textwidth}
\begin{verse}
身是菩提樹\\
心如明鏡臺\\
時時勤佛拭\\
莫使有塵埃
\end{verse}
\end{minipage}
\begin{minipage}[t]{0.6\textwidth}
\itshape
\begin{verse}
Ciało jest drzewem Bodhi,\\
Umysł --- jasną lustrzaną podstawą.\\
Czyść ją stale i gorliwie,\\
Nie pozwalając aby przylgnął kurz.\fnm
\end{verse}
\end{minipage}
\label{ShenxiuVerse}
\vspace*{1em}
\footnotetext{Tekst w języku polskim przytoczony za przekładem \textit{Sutry Szóstego Patriarchy Zen} nieznanego tłumacza, zamieszczonym w serwisie mahajana.net.}

O poranku Piąty Patriarcha Hongren ujrzał wiersz napisany przez Shenxiu na ścianie i uznał, że wiersz ten mógł przynieść ludziom wiele pożytku. Zapłacił wówczas malarzowi, któremu zlecił namalowanie scen z sutry \textit{La\.nkāvatāra}, i odwołał zamówienie.
Zwołał całe zgromadzenie i nakazał mnichom recytować wiersz. Patriarcha spytał Shenxiu, czy to on jest jego autorem, oznaczałoby to bowiem, że jest on właściwym spadkobiercą Dharmy i jego następcą na stanowisku patriarchy.
Przynał jednak, że wiersz nie wskazuje na to, aby Shenxiu rozpoznał już naturę swojego umysłu. Wiersz nadawał się do recytacji przez zwykłych ludzi i dawał gwarancję, że praktykujący nie upadnie do niższych sfer egzystencji%
\footnote{Według kosmologii buddyjskiej, istoty krążące w samsarze, tj. uwarunkowanej egzystencji, od niemającego początku czasu odradzają się w jednej z sześciu sfer egzystencji, zależnie od swojej karmy i indywidualnych skłonności. Trzy z nich, sfera niebiańska (天道 \pinyin{tiāndào}, skt. \textit{devaloka}), którą zamieszkują bogowie, sfera półbogów lub asurów (阿修羅 \pinyin{Āxiūluódào}) i sfera ludzi (人道 \pinyin{réndào}), nazywa się trzema wyższymi sferami egzystencji (三善道 \pinyin{sān shàndào}), ponieważ życie w tych sferach jest relatywnie przyjemne. Trzy niższe sfery egzystencji, sfera zwierząt (畜牲道 \pinyin{chùshēngdào}), sfera głodnych duchów lub pretów (餓鬼道 \pinyin{èguǐdào}) oraz sfery piekielne (地獄道 \pinyin{dìyùdào}), w których życie pełne jest cierpienia, nazywane są niższymi sferami egzystencji (三惡道 \pinyin{sān èdào}).},
jednak takie niepełne zrozumienie nie wystarczało do rozpoznania prawdziwej natury umysłu.
Powiedział, że przekaże mu nauki i szatę patriarchy, jeżeli w ciągu dwóch dni uda mu się osiągnąć ostateczne urzeczywistnienie.
Shenxiu rozmyślał przez wiele dni, ale nie udało mu się skomponować nic lepszego
(Huineng i Yampolsky 2012: 131).

Pewnego dnia młody mnich-akolita recytował wiersz Shenxiu przechodząc koło gumna, gdzie Huineng młócił zboże. Przyszły Szósty Patriarcha zrozumiał, że autor wiersza nie rozpoznał jeszcze natury swojego umysłu. Spytał więc akolity, co właśnie powtarzał. Mnich odparł, że wiersz o pustości, który powtarzał, został skomponowany przez mnicha o imieniu Shenxiu, wspomniał również o zaleceniach Piątego Patriarchy, dotyczących tego wiersza.

Huineng poprosił mnicha, aby ten zaprowadził go do miejsca, gdzie na ścianie wymalowano wiersz Shenxiu. Ponieważ nie umiał czytać, poprosił kogoś, by przeczytał mu te słowa na głos. Usłyszawszy je, Szósty Patriarcha osiągnął ostateczne urzeczywistnienie. Następnie ułożył własny wiersz i poprosił kogoś o napisanie go na ścianie (Huineng i Yampolsky 2012: 131).
\vspace*{1em}

\begin{minipage}[t]{0.38\textwidth}
\begin{verse}
菩提本無樹\\
明鏡亦無臺\\
佛性常清淨\\
何處有塵埃
\end{verse}
\end{minipage}
\begin{minipage}[t]{0.62\textwidth}
\itshape
\begin{verse}
Sama istota Bodhi nie ma drzewa,\\
Nie ma też jasnej lustrzanej podstawy.\\
W rzeczywistości nie ma niczego,\\
Cóż miałoby przyciągać jakikolwiek kurz?
\end{verse}
\end{minipage}
\label{HuinengVerse}
\vspace*{1em}

Mnisi ze zgromadzenia byli pod wielkim wrażeniem tego wiersza, a Piąty Patriarcha Hongren stwierdził na jego podstawie, że Huineng miał już wówczas ponadprzeciętne zrozumienie natury zjawisk, ale dla jego bezpieczeństwa oznajmił zgromadzeniu, że wciąż nie było to pełne urzeczywistnienie
(Huineng i Yampolsky 2012: 132).

Patriarcha przywołał go do siebie w nocy i udzielił mu wyjaśnień do \textit{Sutry Diamentowej}, dzięki którym Huineng natychmiast rozpoznał naturę umysłu.
Hongren przekazał mu również nauki o spontanicznym oświeceniu oraz szatę, insygnium patriarchatu. Udzielił mu również ostatnich pouczeń:
\longquote{Mianuję cię Szóstym Patriarchą. Szata jest tego dowodem, przechodzącym z pokolenia na pokolenie. Moja Dharma musi być przekazywana z umysłu na umysł. Spraw, by ludzie przebudzili się w swej prawdziwej naturze. (\ldots) Od czasów starożytnych przekaz Dharmy był równie słaby, jak zwisający sznurek. Jeżeli pozostaniesz tutaj, inni ludzie zrobią ci krzywdę. Musisz więc niezwłocznie odejść\footnote{Huineng i Yampolsky 2012: 133.}.}

Huineng udał się na południe. Jego śladem podążyło kilkuset ludzi, pragnących go zabić i siłą odebrać szatę oraz Dharmę. Po dwóch miesiącach miał dotrzeć do miejsca zwanego \textit{Dayu ling} (大庾嶺 \toponim{Dàyǔ líng}). Jest to pasmo górskie, znajdujące się pomiędzy południowym wschodem prowincji Jiangxi (江西 \toponim{Jiāngxī}) a prowincją Guangdong (廣東 \toponim{Guǎngdōng}). Tam doścignął go mnich imieniem Huiming (惠明 \pinyin{Huìmíng}) lub Huishun (惠順 \pinyin{Huìshùn}), były generał, człowiek szorstki i porywczy. Huiming groził Huinengowi, który bez wahania oddał mu szatę, lecz Huiming nie chciał jej przyjąć, wyjaśniając, że przybył wyłącznie po to, by otrzymać przekaz Dharmy. Huineng miał przekazać mu Dharmę na szczycie góry, a gdy Huiming usłyszał nauki, natychmiast osiągnął oświecenie.
Następnie Huineng polecił Huimingowi udać się na północ i nauczać tamtejszych ludzi, a sam udał się do Kantonu (Huineng i Yampolsky 2012: 134; Huineng, Schlütter i Teiser: 31).

Dalszy ciąg \textit{Sutry Platformy} podaje, że Huineng przebywał w Caoqi w sumie przez 40 lat.
Nauczał ludzi z Shaozhou (韶州 \toponim{Shāozhōu}) i Kantonu w oparciu o \textit{Sutrę Diamentową}, a jako symbolu przekazu Dharmy używał \textit{Sutry Platformy}. Huineng miał wiele tysięcy uczniów, z których dziesięciu zostało mistrzami, mało znanymi poza obszarem ich działalności.
W 712 roku powrócił do Xinzhou, miejsca swych narodzin, a w 713 roku zmarł, przeżywszy 76 lat.
Tuż przed śmiercią Huinenga, jego uczeń Fahai (法海 \nazwisko{Fǎhǎi}), uważany za autora sutry, zapytał, kto będzie jego następcą i co stanie się z szatą patriarchy.
Huineng odrzekł, że przekaz szaty dobiegł końca, i zasugerował, że w przyszłości pojawi się uczeń o imieniu Shenhui.
Gdy umarł, pojawiło się wiele pomyślnych znaków.
Huineng miał zostać pochowany w Caoqi, a Wei Qu (韋璩), prefekt, który wysłuchał nauk zawartych w dalszej części tekstu, napisał ku jego czci inskrypcję. Mieli ją następnie zniszczyć przedstawiciele Północnej Szkoły
(Huineng, Schlütter i Teiser 2012: 31, 34).

\section{Analiza biografii Huinenga zawartej w tekście \textit{Sutry Platformy} z Dunhuang}
Obecnie uważa się, że autobiograficzny monolog Huinenga, przytoczony w \textit{Sutrze Platformy}, nie jest autentyczną historią jego życia, a jedynie hagiografią.
Tekst wysuwa twierdzenia, jakoby Huineng był prawowitym spadkobiercą Piątego Patriarchy Hongrena, a co za tym idzie, szóstym dzierżawcą przekazu chan, pochodzącego w prostej, nieprzerwanej linii od Buddy Siakjamuniego
(Huineng, Schlütter i Teiser 2012: 25-26).

McRae (2004:67), uważa, że niemożliwe jest, by wydarzenia przedstawione w biografii Huinenga faktycznie miały miejsce, i należy je traktować jako ciekawą anegdotę o wyraźnym podtekście duchowym.
Za tą tezą przemawia fakt, że Shenxiu był uczniem Hongrena jedynie przez kilka lat w początkowym etapie działalności Piątego Patriarchy, a więc kiedy nie istniał jeszcze problem wyboru jego następcy. Shenxiu i Huineng nie przebywali w klasztorze Hongrena w tym samym czasie, a więc nie mogli współzawodniczyć w konkursie poezji.
Po drugie, w owym okresie nie istniała jeszcze koncepcja jedynego prawowitego patriarchy --- pojawiła się ona dopiero w dziełach Shenhui. Ponadto historia życia Huinenga w zachowanej do dnia dzisiejszego wersji nie pojawia się w pismach Shenhui, a jako propagator Huinenga w roli Szóstego Patriarchy na pewno zapisałby tę historię, gdyby była mu znana
(Huineng i McRae 2000: xv).

Jednym z najważniejszych dzieł Shenhui, które zachowały się do dzisiejszych czasów, jest odnaleziony w Dunhuang tekst, zatytułowany \pinyin{Nanyang heshang wenda za zheng yi} (南陽和尚問答雜徵義 \textit{Nányáng héshang wèndá zá zhēngyì}), obecnie znany lepiej pod nazwą ``Cytaty Shenhui'' (神會語錄 \pinyin{Shénhuì yǔlù}).
Chiński pisarz, doktor filozofii Hu Shi (胡適, 1891-1962) zebrał odkryte w Dunhuang dzieła Shenhui i jego uczniów i opisał je w pracy pt. ``Zebrane mowy Shenhui'' (神會和尚遺集 \pinyin{Shénhuì héshàng yíjí}).

Shenhui zaczął promować Huinenga jako Szóstego Patriarchę linii chan w roku 732.
Jednocześnie atakował uczniów i spadkobierców Shenxiu, szczególnie Songshan Puji (嵩山普寂 \nazwisko{Sōngshān Pǔjì}, 651-739), roszczącego sobie prawa do tytułu Siódmego Patriarchy (kwestia ta została opisana szerzej w rozdziale 4. na str. \pageref{ch:chapter_four}).
Między rokiem 730 a 750 Shenhui spisał biografię Huinenga, zbliżoną do zawartej w \textit{Sutrze Platformy}.
Najwcześniejsza jej wersja podaje jedynie, że Huineng był Szóstym Patriarchą i otrzymał od Hongrena, swojego mistrza, szatę --- insygnium patriarchatu --- oraz przekaz Dharmy.
W późniejszych pismach Shenhui uzupełniono tę opowieść informacją, że ojciec Huinenga miał na nazwisko Lu i był popadłym w niełaskę władz urzędnikiem, wysłanym do Xinzhou.
Historia zawierała również zapis o pobycie przyszłego Szóstego Patriarchy w klasztorze Hongrena, gdzie przez osiem miesięcy miał młócić zboże.
Piąty Patriarcha miał go potajemnie odwiedzić w nocy, uznać za godnego spadkobiercę linii chan i polecić mu ucieczkę na południe.
Znamienne jest to, że w twórczości Shenhui nie pojawiła się żadna wzmianka o konkursie poezji, co świadczy o tym, iż fragment ten zostały dodany po jego śmierci
(Huineng, Schlütter i Teiser 2012: 32-34).

Około roku 750, wybitny poeta Wang Wei (王維) wykonał na prośbę Shenhui stelę poświęconą Huinengowi.
Uwieczniono na niej wersję biografii Szóstego Patriarchy propagowaną wówczas przez Shenhui i jego uczniów.
Zawarto w niej trzy ważne informacje, powielone w późniejszych źródłach.
Według steli Piąty Patriarcha przekazał Huinengowi szatę patriarchów i Dharmę dopiero w ostatnim roku swojego życia, a więc 675. % 上元二年 臨終密授
Dalej podano, że Huineng przebywał w ukryciu przez 16 lat, a następnie przyjął ślubowania mnisie od mistrza o imieniu Yinzong (印宗 \nazwisko{Yìnzōng}).
Oznaczałaby to, iż Szósty Patriarcha zaczął udzielać nauk dopiero po roku 690, a więc mógłby to robić jedynie przez 22 lata.
Stoi to w sprzeczności z innymi zapisami Południowej Szkoły, wedle których Huineng nauczał przez 40 lat
(Shi 2008: 116-117).

W późniejszym okresie hagiografia Huinenga została uzupełniona opowieścią o konkursie poezji, w którym Huineng miał pokonać Shenxiu, jednoznacznie dowodząc wyższości subityzmu Szkoły Południowej nad stopniową ścieżką Szkoły Północnej.
Biorąc pod uwagę, że Huineng został w tekście przedstawiony jako ubogi, niepiśmienny człowiek, jest niezwykle mało prawdopodobne, by był w stanie ułożyć przypisywany mu wiersz w klasycznym, literackim języku chińskim.
Z drugiej strony, ponieważ wywodził się z rodziny urzędnika (nawet popadłego w niełaskę władz i skazanego na banicję), wydaje się nieprawdopodobne, że mógłby nie otrzymać żadnego wykształcenia.
Jak podaje Dumoulin (1963:89), te same źródła, które piszą o analfabetyzmie Szóstego Patriarchy, jednocześnie rozwodzą się nad jego szeroką znajomością sutr buddyzmu mahajany.
Już przy pierwszym spotkaniu z Hongrenem mówi o naturze buddy w sposób, przypominający \textit{Nirvā\d{n}a Sūtra}%
\footnote{\textit{Nirvā\d{n}a Sūtra} (skt. \textit{Mahāparinirvā\d{n}a Sūtra}, chiń. 大般涅槃經 \pinyin{Dà bān nièpán jīng}) --- sutra buddyzmu mahajany, której przewodnią myślą jest, że fizyczna śmierć (zwana w sanskrycie \textit{parinirvā\d{n}a}) Buddy Siakjamuniego nie oznacza końca jego istnienia, lecz dowodzi, że umysł %buddhakaya
jest niestworzony i niezniszczalny. Tekst naucza również, że wszystkie czujące istoty mają taki sam potencjał do osiągnięcia stanu buddy.}.
W \textit{Sutrze Platformy} w wielu miejscach pojawia się imię Shenhui, co wskazuje, że najprawdopodobniej została ona spisana wkrótce po jego śmierci.
Został w sutrze wymienony jako jeden z dziesięciu najbliższych uczniów Huinenga, a także jedyny, który nie płakał, gdy Szósty Patriarcha oznajmił, że jego życie ma się ku końcowi.
W tekście przemilczano jednak wkład jego i jego spadkobierców w powstanie sutry
(Huineng i McRae 2000: xv-xvi; McRae 2004: 63).

Historyczny Huineng był postacią stosunkowo mało znaną i prawdopodobnie dlatego Shenhui wybrał go na bohatera swoich historii.
Wedle obecnego stanu wiedzy Shenhui nie posiadał niemal żadnych wiarygodnych informacji na temat swojego rzekomego nauczyciela.
Ponieważ niewiele było wiadomo o jego prawdziwych naukach, Shenhui mógł przypisać mu dowolne pouczenia niestojące w sprzeczności z doktryną subityzmu.
Wiedział o nim jedynie, że był uczniem Hongrena, nauczycielem Dharmy o regionalnym zasięgu działalności, oraz że zamieszkiwał w Shaozhou, na ziemiach oddalonych od serca chińskiej cywilizacji.
Prawdopodobnie był mistrzem medytacji nauczającym o nagłym oświeceniu, ale wbrew pełnym wyolbrzymień opowieściom, zawartym w \textit{Sutrze Platformy}, już za jego życia idee te nie uchodziły za nic nadzwyczajnego.
Raczej nie wchodził w konflikty z innymi nauczycielami Dharmy, a wręcz utrzymywał z nimi dobre relacje
(Huineng i McRae 2000: xv).

Ważnym aspektem nauk Shenhui, przypisywanych Huinengowi, jest odejście od sutry \textit{La\.nkā\-vatāra} na rzecz \textit{Sutry Diamentowej}, która zyskiwała na popularności w VIII w.
O tym, jak wielką rolę odgrywała ona dla Shenhui, świadczą liczne odniesienia do niej w \textit{Sutrze Platformy}.
W tekście jest wysunięta również teza, jakoby to \textit{Sutra Diamentowa}, a nie \textit{La\.nkāvatāra}, była podstawą nauk przekazywanych przez patriarchów, od Bodhidharmy do Huinenga.
Twierdzeniom tym zaprzeczają jednak inne teksty na temat linii przekazu chan, takie jak ``Kontynuowane biografie wybitnych mnichów'', ``Annały przekazu skarbu Dharmy'' (傳法寶紀 \pinyin{Chuán fǎbǎo jì}) i ``Zapisy mistrzów i uczniów w przekazie Sutry La\.nkāvatāra'' (楞伽師資記 \pinyin{Léngqié shīzī jì}).
Symboliczne odejście od sutry \textit{La\.nkāvatāra} jest też zaznaczone w biografii Huinenga w \textit{Sutrze Platformy}, w miejscu, w którym Piąty Patriarcha Hongren zrezygnował z wykonania malowideł ze scenami z sutry \textit{La\.nkāvatāra} na rzecz wiersza Shenxiu (Huineng i Yampolsky 2012: 34; McRae 2004: 62).

Interesujące jest, jak w \textit{Sutrze Platformy} połączono nauki, pochodzące z różnych systemów filozofii buddyjskiej.
Według filozofii Madhjamaki oraz związanego z nią kanonu sutr \textit{Pradżniaparamity}, takich jak \textit{Sutra Diamentowa} czy \textit{Sutra Serca}, rozwinięcie mądrości \textit{pradżni} jest ostatecznym celem praktyki duchowej.
Osiągnięcie go równa się zrozumieniu pustej natury wszystkich zjawisk, czyniąc zbędnymi nauki o naturze buddy.
Natomiast w systemie \textit{Tathagathagarbha}, z którym związana jest sutra \textit{La\.nkā\-vatāra}, \textit{pradżnia} jest wyłącznie zdolnością umysłu, zaś głównym obiektem zainteresowania jest natura buddy, obecna w umyśle wszystkich istot.
W \textit{Sutrze Platformy} obecne są idee \textit{Mahapradżniaparamity}, ale fundamentem jej filozofii jest ewidentnie system \textit{Tathagathagarbha}.
Chociaż w sutrze mówi się o odrzuceniu sutry \textit{La\.nkā\-vatāra} na rzecz \textit{Sutry Diamentowej}, w rzeczywistości w tekście wygłoszono wiele nauk, bliższych pierwszemu z tekstów.
(Shi 1990: 150-152).

W dziełach Shenhui pojawiły się też dwie opowieści, powielone w późniejszych zapisach. % źródłach
Pierwsza z nich, zapisana zarówno w \textit{Sutrze Platformy}, jak i \textit{Putidamo Nanzong ding shifei lun}%
\footnote{\textit{Putidamo Nanzong ding shifei lun} (菩提達摩南宗定是非論 \pinyin{Pútídámó Nánzōng dìng shìfēi lùn}) --- tekst, w którym Dugu Pei (獨孤沛), uczeń Heze Shenhui, opisał ataki swojego nauczyciela na Yuquan Shenxiu. Kwestia walki o wpływy między szkołami została szczegółowo opisana w rozdziale 4.}, dotyczyła Bodhidharmy i cesarza Liang Wudi (梁武帝 \nazwisko{Liáng Wǔdì}).
Według tej historii, kiedy Bodhidharma przybył do stolicy Liang, miasta Jiangling (江陵 \toponim{Jiānglíng}), przeprowadził dyskusję z cesarzem. Cesarz miał spytać Bodhidharmy, czy budując świątynie, dając ofiary mnichom i ludziom w potrzebie, zgromadził zasługę%
\footnote{Zasługa (功德 \pinyin{gōngdé}, skt. \textit{pu\d{n}ya}) w większości tradycji buddyzmu odnosi się do dobrych wrażeń karmicznych, zebranych w rezultacie właściwego postępowania i podążania ścieżką duchową. (Buswell 2004: 532).}.
Mistrz odparł: ,,Nie zgromadziłeś zasługi.'' Miał przez to na myśli, że cesarz, nie podążając za właściwą ścieżką, szukał jedynie błogosławieństw, a nie prawdziwej zasługi. Cesarz, nie rozumiejąc tej nauki, był nią rozczarowany i wygnał Bodhidharmę ze swego państwa. Następnie mistrz udał się do\label{LiangWuDi} państwa Wei % jeśli będzie trzeba, dodamy przypis o państwie Wei w okresie 南北朝
(Huineng i Yampolsky 2012: 27, 155-156).

% T48n2007_p0341a24(10)║朕一生未來造寺布施供養有有功德否。達磨答言。並無功德。
% T48n2007_p0341a25(03)║武帝惆悵遂遣。達磨出境。未審此言。請和尚說。
% a w Wei spotkał Huike

Inna opowieść propagowana przez Shenhui dotyczy Bodhidharmy oraz jego ucznia i spadkobiercy, Huike. Według tej historii, kiedy mistrz i uczeń spotkali się po raz pierwszy, Huike był zdeterminowany zostać uczniem Bodhidharmy, lecz ten nie chciał go przyjąć. Mistrz miał ustąpić Huike dopiero wówczas, gdy ten w dowód swej determinacji dobył miecza i demonstracyjnie uciął swoje lewe ramię. Huike został następnie głównym uczniem Bodhidharmy i odziedziczył po nim szatę. Później tę samą szatę mieli otrzymać kolejni patriarchowie: Sengcan, Daoxin, Hongren, aż do Huinenga. Przy pomocy tej opowieści Shenhui osiągnął dwa cele: nie tylko ustanowił szatę Bodhidharmy insygnium prawowitego patriarchy chan, lecz również podważył uznaną dotychczas linię przekazu, wiodącą od Bodhidharmy do Shenxiu. Pochodzenie tych legend nie jest znane. Hu Shi uważa, że zostały one wymyślone przez Shenhui, ale równie prawdopodobne jest, że krążyły wśród ludu, a Shenhui jedynie zapisał je i wykorzystał do swoich celów (Huineng i Yampolsky 2012: 27).

\subsection{Wiersze Shenxiu i Huinenga}
Tradycyjna interpretacja wierszy Shenxiu i Huinenga, zawartych w tekście \textit{Sutry Platformy} (patrz: strony \pageref{ShenxiuVerse} i \pageref{HuinengVerse}), jest prosta.
Według filozofa Guifeng Zongmi (圭峰宗密 \pinyin{Guīfēng Zōngmì}, 780-841) wiersz Shenxiu ma symbolizować stopniową ścieżkę, zaś wiersz Huinenga --- ścieżkę nagłego, ostatecznego oświecenienia, które dokonuje się w jednej chwili.
Tym samym, w rozumieniu Zongmi, wiersze te reprezentują pogląd dwóch konkurencyjnych tradycji chan, północnej i południowej.
W późniejszym okresie była to dominująca interpretacja szkoły chan
(McRae 2004: 63).

Takie rozumienie jest jednak nadmiernym uproszczeniem. Wiersz przypisywany Shenxiu odnosi się nie tyle do stopniowej ścieżki, ile do ciągłej, bezustannej praktyki oczyszczania zwierciadła z kurzu.
Wiersz Huinenga nie opisuje natomiast poglądu nagłej ścieżki, a jedynie neguje twierdzenia, zawarte w wierszu Shenxiu.
Oprócz tego, wiersze przedstawiają dwa punkty widzenia na ten sam temat i nie mogą być interpretowane osobno
(McRae 2004: 63-64).

Opis konkursu poezji, zawarty w tekście \textit{Sutry Platformy}, ma za zadanie udowodnić wyższość nauk Huinenga i jego spadkobierców nad naukami Shenxiu.
Jak zaznacza McRae (2004:65), chociaż zwycięzcą w starciu był Huineng, to wiersz, przypisywany Shenxiu, wcale nie jest mierny, a wręcz przeciwnie --- głęboki i wyrafinowany.
W ten sposób wiersz Huinenga tylko zyskuje w oczach odbiorcy. Nie jest bowiem sztuką stworzyć dzieło lepsze od czegoś przeciętnego.

\subsection{Analogie do biografii Konfucjusza}
Pisząc historię życia Huinenga, Shenhui w oczywisty sposób czerpał z legendy o Konfucjuszu, opisanej w \textit{Zapiskach historyka} (史記 \pinyin{Shǐjì}) autorstwa Sima Qian (司馬遷 \nazwisko{Sīmǎ Qiān}).
Tekst ten miał wówczas ugruntowaną pozycję wśród chińskich elit, jako że Konfucjusz był twórcą głównego systemu filozoficznego w państwie i był znany niemal wszystkim jako uniwersalny wzorzec cnót
(Huineng, Schlütter i Teiser 2012: 36).

Według wspomnianej legendy Konfucjusz urodził się jako Kong Qiu (孔丘 \nazwisko{Kǒng Qiū}), owoc mezaliansu Shuliang He (叔梁紇), oficera wojsk państwa Lu, i lokalnej kobiety, Yan Zhengzai (顏徵在).
Gdy Konfucjusz był jeszcze dzieckiem, zmarł jego ojciec.
Po śmierci Shuliang He, Kong Qiu, tak jak Huineng, dorastał w ubóstwie, wychowywany przez samotną matkę, bez jakiegokolwiek wsparcia ze strony rodziny ojca.
Niewiele wiadomo o najwcześniejszych latach jego życia, ale według \textit{Zapisków historyka}, już w dzieciństwie wykazywał wielkie zainteresowanie naczyniami rytualnymi z brązu.
Według legendy, Konfucjusz lubił bawić się nimi, starannie układając je tak, jak przy składaniu ofiar przodkom.
W wieku 15 lat Kong Qiu zaczął zgłębiać teksty i rytuały z początków dynastii Zhou%
\footnote{Dynastia Zhou (周朝 \pinyin{Zhōu Cháo}) --- dynastia rządząca Chinami w latach ok. 1045-256 p.n.e. Dzieli się na tzw. Zachodnią Dynastię Zhou i Wschodnią Dynastię Zhou.}.
Uważał ów okres za złoty wiek cywilizacji chińskiej, czas zjednoczenia i pokoju, zaś jej władców, a także królów sprzed założenia tej dynastii --- za mędrców, rządzących swoimi państwami sprawiedliwie i moralnie.
Twierdził, że powodem, dla którego we współczesnych mu Chinach zapanował chaos, było odejście od ideałów Zhou i postawił sobie za cel przywrócenie w kraju dawnych wartości i rytuałów.
Konfucjusz nie cieszył się szczególną popularnością wśród lokalnej elity państwa Lu.
Jego idee nie zostały docenione za jego życia, tak jak idee Huinenga, które spopularyzowała dopiero \textit{Sutra Platformy}.
Pierwszą pracą, którą podjął, było zarządzanie gospodarstwem rolnym lokalnej arystokracji.
Przez pewien czas pracował jako urzędnik niskiego szczebla, lecz wkrótce został z tej posady zwolniony, mimo że wykazywał ponadprzeciętne uzdolnienia
(Schuman 2015: 45-50; Huineng, Schlütter i Teiser 2012: 36-37).

Shenhui był wykształcony w zakresie filozofii konfucjańskiej, a nawet został przez Zongmi porównany do Konfucjusza. Zan Ning (贊寧), autor ``Biografii wybitnych mnichów Song'' (宋高僧傳 \pinyin{Sòng gāosēng zhuàn}) porównał go do Yan Hui (顏回), ulubionego ucznia Konfucjusza.
Symboliczne znaczenie miała również liczba patriarchów w linii przekazu chan. Według panującej wówczas interpretacji myśli konfucjańskiej jedynie cesarz mógł w świątyni swojej rodziny urządzić siedem izb.
Linia przekazu w wersji Shenhui wiodła od Bodhidharmy, przez Hongrena, do Huinenga.
Jako spadkobierca Szóstego Patriarchy, Shenhui stawał się Siódmym Patriarchą i tym samym ustanawiał doskonałą linię chan
(Huineng, Schlütter i Teiser 2012: 37).

Z perspektywy nauk Konfucjusza Huineng był postacią godną naśladowania.
Mimo iż wywodził się z nizin społecznych i wychował się w oddalonym od centrum cywilizacji chińskiej rejonie, to posiadał jedną z najwyższych cnót konfucjańskich --- \textit{nabożność synowską} (子孝 \pinyin{zǐxiào}).
Po śmierci ojca ciężko pracował, utrzymywał starą matkę i opiekował się nią.
Taki obraz Huinenga stał w opozycji do wykształconych, bogatych elit, z których wywodziła się w owym okresie większość mnichów buddyjskich.
Twórcy jego hagiografii mogli również inspirować się historią życia Hongrena, który według niektórych podań medytował za dnia, zaś w nocy zajmował się bydłem.
Sam Shenxiu, rzekomy konkurent Huinenga do pozycji patriachy, pochodził ze szlachetnego rodu i był wykształcony zarówno w literaturze buddyjskiej, jak i świeckiej, a niektórzy podejrzewają nawet, że mógł być związany z rodem cesarskim
(McRae 2004: 68).

Celem takiego przedstawienia Szóstego Patriarchy było pokazanie, że każdy, nawet osoba świecka, niezależnie od pozycji społecznej, miejsca pochodzenia i wykształcenia mógł rozpoznać naturę swojego umysłu i zostać zwierzchnikiem chan.
Co za tym idzie, jeżeli oświecenie było jedynym warunkiem zostania patriachą, to czytelnik mógł oczekiwać, że wszyscy poprzednicy i następcy Huinenga również byli oświeceni.
W ten sposób szkoła chan zyskiwała autorytet i wiarygodność
(McRae 2004: 69).
