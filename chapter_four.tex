\chapter{Historyczne reperkusje \textit{Sutry Platformy}}
\label{ch:chapter_four}
\textit{Sutra Platformy} pozostaje jednym z najbardziej popularnych i wpływowych tekstów buddyzmu Chan do dnia dzisiejszego. Jej publikacja stanowi punkt kulminacyjny w historii buddyzmu Chan, a analizie zawartych w niej nauk i opowieści poświęcono wiele prac naukowych.

\section{Podział Chan na Szkołę Północną i Południową}
Shenhui zyskał wpływy dopiero po pojawieniu się negatywnego sentymentu do jego rywali, w następstwie jego publicznych ataków na Szkołę Północną.
Zostały one opisane przez Dugu Pei (獨孤沛) w dziele zwanym \textit{Putidamo Nanzong ding shifei lun} (菩提達摩南宗定是非論 \pinyin{Pútídámó Nánzōng dìng shìfēi lùn}).
Jak podaje ów tekst, w roku 732 Shenhui zorganizował otwartą konferencję buddyjską w świątyni Dayun Si w miejscu zwanym Huatai w obecnej prowincji Henan.\label{Huatai}
Wysunął wówczas twierdzenia, jakoby Szkoła Północna, której przewodzili Shenxiu i Puji, nie była autentyczna, gdyż propagowała nauki stopniowej ścieżki.
Prawdziwe, ponadczasowe nauki buddy, tzn. nauki o nagłym oświeceniu, znane również jako subityzm, miały być przekazywane w południowym Chan.
W tekście opisano dyskusje Shenhui z mistrzem Chongyuan (崇遠 \nazwisko{Chóngyuǎn}).
Chongyuan był znanym specjalistą od sutr buddyjskich, wygrywającym wszystkie debaty filozoficzne, jednak na konferencji w Huatai poniósł porażkę.
W późniejszym czasie Szkoła Południowa używała tego faktu na dowód swej wyższości nad Szkołą Północną, była to jednak manipulacja, gdyż mistrz Chongyuan nie reprezentował żadnej ze szkół.
W owym czasie szkoła Shenxiu cieszyła się wsparciem dworu cesarskiego, miała swoje wpływy w Luoyangu, w mieście Chang'an (長安 \toponim{Cháng'an}, obecnie Xi'an 西安 \toponim{Xī'ān} w prowincji Shaanxi) oraz dorzeczu Rzeki Żółtej (黃河 \toponim{Huáng Hé}).
Shenhui zorganizował konferencję w Huatai nie tylko po to, by pogrążyć swoich przeciwników, lecz również by przeciwdziałać podziałom w szkole Chan.
W owym czasie istniało bowiem wielu samozwańczych mistrzów Chan, którzy nauczali Dharmy i zakładali własne szkoły, przeinaczając prawdziwą istotę nauk Buddy
(Shi 2008: 191, 200).

W dziełach Shenhui pojawiły się twierdzenia, jakoby Puji wysłał swojego ucznia, niejakiego Zhang Xingchang (張行昌), do Shaozhou, z poleceniem ucięcia głowy zwłokom Huinenga, a także, że inny uczeń Puji, imieniem Wu Pingyi (武平一), wymazał inskrypcję na steli poświęconej Huinengowi i wstawił tam własną, podającą Shenxiu jako prawowitego Szóstego Patriarchę.
Zarzucił Puji, że w dziele ``Annały przekazu skarbu Dharmy'' (傳法寶紀 \pinyin{Chuán fǎbǎo jì}) spisał historię linii przekazu Chan z pominięciem Huinenga.
Stwierdzenie to było o tyle bezpodstawne, że Puji nie był autorem tego dzieła
(Huineng i Yampolsky 2012: 28).

Na konferencji w Huatai Shenhui zarzucał swoim przeciwnikom, że uzurpowali sobie prawo do linii przekazu patriarchatu, oraz że przeinaczyli istotę buddyzmu, nauczając błędnej, stopniowej ścieżki.
W rzeczywistości, Shenxiu opierał swoje nauki na tych samych podstawowych ideach buddyzmu Mahajany, co Shenhui.
W obu szkołach ostatecznym celem praktyki duchowej było osiągnięcie oświecenia poprzez rozpoznanie prawdziwej natury umysłu; w obydwu przebudzenie było postrzegane jako krótki przebłysk wglądu.
Jednak w doktrynie Szkoły Północnej, opartej głównie na sutrze \textit{La\.nkā\-vatāra}, kładziono nacisk na praktyki przygotowawcze.
Ich zadaniem było oczyszczenie umysłu z zaciemnień, postrzeganych w tej tradycji jako istniejące i rzeczywiste.
Tymczasem Szkoła Południowa głosiła pustość wszystkich myśli i zjawisk.
Ideą praktyki w tej doktrynie było nie tyle pozbyć się trucizn umysłu, ile zrozumieć, że w rzeczywistości nigdy nie istniały.
W \textit{Sutrze Platformy} dychotomię tę ilustrują ostatnie dwa wersy wiersza Huinenga: ,,W rzeczywistości nie ma niczego, / Cóż miałoby przyciągać jakikolwiek kurz?''
Nauki Shenxiu obejmowały o wiele większą część tradycji medytacyjnej Mahajany, podczas gdy Shenhui bardziej radykalnie promował doktrynę nagłego oświecenia, traktując ją jako jedyne kryterium decydujące o tym, która ze szkół reprezentowała ortodoksyjny buddyzm.
(Dumoulin 1963: 81, 84-87). % sprawdzić

Kim był Shenxiu i czym zasłużył sobie na osobiste ataki Shenhui? Yampolsky (2012, str. 15-16) podaje, że na przełomie VII i VIII w. Shenxiu był uważany za jednego z najbardziej znaczących i najwybitniejszych mistrzów Chan.
Jego biografia jest szczególnie dobrze zachowana. Jej stosunkowo rzetelna wersja została zapisana w dziele ``Annały przekazu skarbu Dharmy'' (傳法寶紀 \pinyin{Chuán fǎbǎo jì}) w pozbawiony elementów fantastycznych sposób. O ile we wszystkich innych dziełach z tego okresu jest wymieniony jako uczeń Hongrena, ``Annały'' podają, że był uczniem Faru, a ten --- Hongrena. Według tej biografii pochodził z miasta Daliang (大梁 \toponim{Dàliáng}), obecnie Kaifeng (開封 \toponim{Kāifēng}) w prowincji Henan (河南 \toponim{Hénán}) i był członkiem rodu Li.

Shenxiu już od najmłodszych lat wykazywał się ponadprzeciętnymi uzdolnieniami.
W wieku 13 lat, w związku z zawirowaniami historycznymi i związaną z nimi klęską głodu, postanowił porzucić dotychczasowe życie i zostać mnichem buddyjskim.
Później wędrował od świątyni do świątyni, by wreszcie jako dwudziestolatek otrzymać pełne ślubowania.
W wieku 46 lat udał się do Hongrena, a ten natychmiast poznał się na jego talencie.
Po wielu latach studiowania nauk osiągnął ostateczne oświecenie, a następnie udał się do Jingzhou (荊州 \toponim{Jīngzhōu}) w prowincji Hubei (湖北 \toponim{Húběi}).
Za panowania cesarza Tang Gaozonga (唐高宗 \nazwisko{Táng Gāozōng}), podczas ery Yifeng (儀鳳 \pinyin{Yífèng}) udał się do świątyni Yuquan Si (玉泉寺 \toponim{Yùquán sì}) w pobliżu obecnego miasta Dangyang (當陽 \toponim{Dāngyáng}) w prowincji Hubei.
Dopiero po śmierci swojego mistrza zaczął gromadzić wokół siebie uczniów, nauczając ich Dharmy. Przynosił pożytek wielu istotom, prowadząc je do wyzwolenia
(Yampolsky 15-16).

\if 0
Huineng i Yampolsky 2012: 22
Songshan Puji pochodził z Hedong 河東, obecnie 永濟縣 w 山西
W roku 735 został zaproszony na dwór przez cesarzową Wu Zetian. W stolicy cieszył się znaczącą popularnością, zarówno we dworze, jak i wśród prostego ludu.
Zmarł w roku 739, a na jego pogrzeb przybyło wielu ludzi, w tym urzędnicy cesarscy.

Yifu pochodził z Tongdi 銅鞮, obecnie 沁縣 w 山西.
Za młodu studiował daoizm
\fi

\section{Dalszy podział Szkoły Południowej}
Spośród uczniów Huinenga, najbardziej wpływowi byli Qingyuan Xingsi (青原行思 \nazwisko{Qīngyuán Xíngsī}) oraz Nanyue Huairang (南嶽懷讓 \nazwisko{Nányuè Huáiràng}).
Biorąc pod uwagę, że żaden z nich nie został wspomniany w \textit{Sutrze platformy} w wersji z Dunhuang, ich spotkania z Huinengiem prawdopodobnie nigdy nie miały miejsca.
Ich spadkobiercy podzielili się na szkoły: Caodong (曹洞宗 \pinyin{Cáodòng zōng}), Yunmen (雲門宗 \pinyin{Yúnmén zōng}) i (法眼宗 \pinyin{Fǎyǎn zōng}) od Xingsi, oraz Linji (臨濟宗 \pinyin{Línjì zōng}) i Guiyang (溈仰宗 \pinyin{Guīyǎng zōng}) od Huairanga.
Następnie od Linji oddzieliły się linie Huanglong (黃龍派 Huánglóng pài) oraz Yangqi (楊岐派 Yángqí pài). Do chwili obecnej zachowały się Linji i Caodong
(佛學課本; McRae 2004: 82).

\subsection{Nanyue Huairang i jego uczniowie}
Nanyue Huairang narodził się w drugim roku ery Yifeng (677) w miejscowości Ankang (安康 \toponim{Ānkāng}) w obecnej prowincji Shaanxi (陝西 \toponim{Shǎnxī}).
Jak podaje Ferguson (2011), Huairang w wieku piętnastu lat opuścił rodzinny dom, a następnie uczył się Vinaya\footnote{Vinaya --- wyjaśnienie. (stub)} u mistrza Yuquan Hongjing (玉泉弘景 \nazwisko{Yùquán Hóngjǐng}).
Nieusatysfakcjonowany jego naukami, wkrótce udał się na górę Songshan (嵩山 \toponim{Sōngshān}), gdzie uczył się u Songshan Hui'an (嵩山惠安 \nazwisko{Sōngshān Huì'ān}).
W roku 699 udał się do Caoqi, gdzie przez dwanaście lat uczył się Dharmy od Szóstego Patriarchy.
Jak podaje kronika \textit{Wudeng huiyuan} (五燈會元 \pinyin{Wǔdēng huìyuán}), zredagowana w roku 1252, Huinenga miał go wówczas poinformować o przepowiedni indyjskiej mistrzyni dhjany o imieniu Prajñātārā (skt., chiń. 般若多羅 \nazwisko{Bōrěduōluó}), 27. patriarchy Indii oraz nauczycielki Bodhidharmy.
Proroctwo głosiło, że ,,spod stóp Huairanga wyjdzie koń, który zadepcze na śmierć wszystkich ludzi na tym świecie''.
Słowa te odnosiły się do Mazu Daoyi (馬祖道一 \nazwisko{Mǎzǔ Dàoyī}), ucznia Huairanga, który znacząco wpłynął na dalszy rozwój buddyzmu Chan.
Daoyi miał bowiem na nazwisko Ma (馬 \nazwisko{Mǎ}), co oznacza konia.
W roku 713 Huairang przybył do świątyni Bore Si (般若寺 \pinyin{Bōrě Sì}), na górze Hengshan (衡山 \toponim{Héngshān}) w prowincji Hunan (湖南 \toponim{Húnán}).
McRae (2004, str. 82) uważa, że Huairang prawdopodobnie nigdy nie spotkał Huinenga.
Jego głównymi uczniami byli Daojun (道峻 \nazwisko{Dàojùn}), Shenzhao (神照 \nazwisko{Shénzhào}) oraz Mazu Daoyi (馬祖道一 \nazwisko{Mǎzǔ Dàoyī}).

Mazu Daoyi (707-786) przybył do Hengshan w roku 735.
W ``Antologii gmachu patriarchów'' zawarto anegdotę dotyczacą jego rozmowy z mistrzem Huairangiem.
Mazu siedział w medytacji, kiedy przyszedł do niego mistrz i zapytał go, co robi.
Uczeń odparł, że medytuje, by osiągnąć stan buddy.
W odpowiedzi Huairang wziął cegłę i zaczął ją energicznie pocierać, mówiąc, że poleruje cegłę, by zrobić z niej lustro.
Kiedy zdziwiony Mazu zwrócił nauczycielowi uwagę, że to niemożliwe, ten skwitował to słowami: ,,Polerując cegłę, nie zrobisz z niej lustra; jak możesz osiągnąć oświecenie, siedząc w medytacji?'' % (磨磚既不成鏡,坐禪豈得成佛)
Według Dumoulina (1963, str. 98), Huairang chciał w ten sposób zaobrazować uczniowi, że praktyka medytacji jest zbędna, gdyż oświecenie jest zawsze obecne w umyśle.

% Tu będzie o jego doktrynie
Mazu Daoyi słynął z niekonwencjonalnych metod nauczania.
Był pierwszym mistrzem Chan, który zastosował technikę zwaną \textit{katsu} (chiń., jap. 喝, Pinyin: \pinyin{hè}, Rōmaji: \textit{katsu}).
Polega ona na krzyczeniu na ucznia w celu przełamania jego sztywnych koncepcji.
W późniejszym okresie stosował ją mistrz Linji Yixuan (臨濟義玄 \nazwisko{Línjì Yìxuán}), założyciel szkoły Linji (臨濟宗 \pinyin{Línjì zōng}). % tutaj pierdolniemy odnośnik
Podczas ery Dali (大曆 \pinyin{Dàlì}), Mazu Daoyi zamieszkiwał w świątyni Baohua Si (寶華寺 \toponim{Bǎohuá Sì}) na górze Gonggong Shan (龔公山 \toponim{Gōnggōng Shān}).
W czwartym roku tego okresu przeniósł się do świątyni Kaiyuan Si (開元寺 \toponim{Kāiyuán Sì}) w rejonie Hongzhou (洪州 \toponim{Hóngzhōu}), obecnie w granicach administracyjnych miasta Nanchang (南昌 \toponim{Nánchāng}) w prowincji Jiangxi, gdzie nauczał aż do śmierci w roku 788.
Nazwa wywodzącej się od niego szkoły Hongzhou (洪州宗 \pinyin{Hóngzhōu zōng}) pochodzi właśnie od tego miejsca.
Miał bardzo wielu uczniów, z których najważniejsi dla historii Chanu byli Baizhang Huaihai (百丈懷海 \nazwisko{Bǎizhàng Huáihǎi}) oraz Nanquan Puyuan (南泉普願 \nazwisko{Nánquán Pǔyuàn})
(Dumoulin 1963: 97-98; Chang 1971: 148).

Szkoła Shenxiu była popularna na północy Chin, podupadła po śmierci głównych uczniów Shenxiu: Songshan Puji i Dazhi Yifu (大智義福), by w końcu wygasnąć. % sprawdź kiedy
\if 0
Doczytać o rebelii An Lushana!

McRae 1986: 3
W początkowej fazie istnienia Południowej Szkoły, była ona mało znana; sutra wyjaśnia to długim czasem, jaki Huineng spędził, ukrywając się u myśliwych
McRae 1986: 5
Szkoła Południowa twierdziła, że posiada nauki niedualne
Wg SS natura oświecenie, przeszkadzające emocje, cierpienie i iluzje są w istocie tym samym, co oświecenie, ale NS widziała je jako różne; wg 宗密 oznacza to, że wiele lat, lub nawet żywotów praktyki idzie na marne; wszystko, czego potrzebuje praktykujące, to całkowite odcięcie dualistyczn, ego myślenia
SS była lewicowa: uważała, że każdy powinien mieć prawo poznać Dharmę i osiągnąć oświecenie, a nie tylko ci, którzy włożyli w to wysiłek
Zongmi usystematyzował różne interpretacje Chan, Szkoła Północna była najniżej
wykładnia 宗密 NS na początku była popularna, ale potem została niemal całkowicie wyparta przez SS, bo prawowitym spadkobiercą był Huineng, a nie Shenxiu
Nie ma dobrych badań nt. NS

Dumoulin 1963: 70
W czasie, kiedy Bodhidharma przyniósł Chan do Chin, na północy Chin był Buddhabhadra, a na południu szkoła 三论
Sanlun interesowała się Pradżniaparamitą Nagardżuny

79: 弘忍 rozwijał własne metody medytacji w oparciu o sutry Avatamsaka (華嚴)

Dumoulin 81
Pierwsza schizma w tradycji Chan
SS wygrała walkę o dominację, bo NS nie rozwijało się po śmierci uczniów Shenxiu, a SS publikowało wiele koanów i kronik
w 700 Shenxiu polecił cesarzowi zaprosić Huinenga do stolicy
(89: w 705 Huineng to zaproszenie odrzucił)
Shenxiu cieszył się szacunkiem dworu, główni uczniowie 嵩山普寂、嶗山義福
83
Według niektórych podań Shenhui był przez kilka lat uczniem Shenxiu, ale jest to mało prawdopodobne
Shenhui na pewno przebywał razem z Huinengiem
91 Dualizm rozpuszcza się w pustości -- myśl z Diamentowej Sutry
91/92 Samadhi w którym nie ma myśli
Umysł ma skłonność do konceptualizacji absolutu, np. przywiązuje się do koncepcji nirwany albo pustki
Jeżeli zamiast tego pozbędzie się wszelkich koncepcji, to pozostanie samo lustro


中國禪宗史 115:
W sutrze jest napisane, że Huineng 先天二年八月三日滅度 oraz 春秋七十有六, ale są stele, które mówią co innego i przesuwają datę śmierci o 3 lata
《曹溪大師別傳》
117: Z legend 神會, zapisanych przez 王維 wynika, że Huineng ukrywał się przez 16 lat, dostał 衣法 od 弘忍 na jego łożu śmierci, i że po 16 latach ukrycia spotkał się z 印宗 i zaczął nauczać
神會語錄:能禪師過嶺至韶州,居曹溪山,來往四十年。 nie ma 印宗,隱遁
歷代法寶記:ukrycie, 印宗
Teoria ukrycia wzięła się z niespójności lat

182:713-815 神會向中原傳播南宗頓教
W latach 713-815 nauki Shenhui rozprzestrzeniły się w rejonie Niziny Chińskiej (a właściwie 中原) w dorzeczu Rzeki Żółtej (黃河)

188: 宋高僧传 podaje, że w roku 開元八年 zamieszkał w świątyni 龍興寺 w 南陽
potem nauczał Dharmy w Luoyangu, a Shenxiu i jego uczniowie aż zielenieli z zazdrości
Mieszkał przez dłuższy czas w Nanyang, więc nazywano go 南陽和上
神會語錄 autor 劉澄

189:南宗定是非論 opisuje wydarzenia z 滑臺寺, gdzie Shenhui urządził otwartą konferencję
圓覺經大疏釋義鈔 podaje, że w 20 lat po śmierci Huinenga (czyli do Huatai - 732), stopniowa ścieżka Shenxiu była rozpowszechniona w 荊吳 - środkowym i dolnym biegu Changjiang i bodajże w dwóch stolicach (秦洛 czyli Chang'an i Luoyang?)
Shenhui twierdził, że Shenxiu nie miał prawdziwego przekazu, a jedynie poboczny; prawowitym spadkobiercą Bodhidharmy miał być Huineng
Shenhui musiał mieć wsparcie społeczności z Dayun Si w Huatai, inaczej w żadnym wypadku nie umożliwiliby mu organizacji takiej konferencji
190 musieli skołować na konferencję 屏風 parawany, które w czasie jej trwania ktoś wyniósł, żeby przyjąć jakiegoś urzędnika
191 Shenhui urządził tę konferencję nie tylko dlatego, że chciał pogrążyć swoich przeciwników, ale również dlatego, że uważał, że istniało zbyt wielu samozwańczych mistrzów Chan, którzy przeinaczali istotę nauk Buddy.
\fi
