\chapter{Historyczne reperkusje \textit{Sutry Platformy}}
\label{ch:chapter_four}
\textit{Sutra Platformy} pozostaje jednym z najbardziej popularnych i wpływowych tekstów buddyzmu Chan do dnia dzisiejszego. Jej publikacja stanowi punkt kulminacyjny w historii szkoły, a analizie zawartych w niej nauk i opowieści poświęcono wiele prac naukowych.

\section{Podział Chan na Szkołę Północną i Południową}
Podział buddyzmu Chan na Szkołę Południową i Północną związany jest z Heze Shenhui.
Początkowo nieznany, zyskał wpływy dopiero po pojawieniu się negatywnego sentymentu do jego rywali, w następstwie publicznych ataków na Yuquan Shenxiu oraz jego uczniów, Songshan Puji i Dazhi Yifu.
Zostały one szczegółowo opisane przez ucznia Shenhui, Dugu Pei w dziele zwanym \textit{Putidamo Nanzong ding shifei lun}. % (菩提達摩南宗定是非論 \pinyin{Pútídámó Nánzōng dìng shìfēi lùn}).
Jak podaje ów tekst, w roku 732 Shenhui zorganizował otwartą konferencję buddyjską w świątyni Dayun Si, w miejscu zwanym Huatai, w obecnej prowincji Henan.\label{Huatai}
Wysunął wówczas twierdzenia, jakoby Szkoła Północna, której przewodzili Shenxiu i Puji, nie była autentyczna, gdyż propagowała nauki stopniowej ścieżki.
Prawdziwe, ponadczasowe nauki buddy, tzn. nauki o nagłym oświeceniu, znane również jako subityzm, miały być przekazywane w południowym Chan.
W tekście opisano dyskusję Shenhui z mistrzem Chongyuan (崇遠 \nazwisko{Chóngyuǎn}).
Był on znanym specjalistą od sutr buddyjskich, wygrywającym wszystkie debaty filozoficzne, jednak na konferencji w Huatai poniósł porażkę.
W późniejszym czasie Szkoła Południowa używała tego faktu na dowód swej wyższości nad rywalami.
Była to jednak manipulacja, gdyż w rzeczywistości Chongyuan nie był reprezentantem Szkoły Północnej.
W owym czasie Shenxiu i jego szkoła cieszyli się wsparciem dworu cesarskiego, cieszyli się poparciem dworów w obu stolicach, w Luoyangu i w Chang'an (長安 \toponim{Cháng'an}, obecnie Xi'an 西安 \toponim{Xī'ān} w prowincji Shaanxi) oraz w dorzeczu Rzeki Żółtej (黃河 \toponim{Huáng Hé}).
Shenhui zorganizował konferencję w Huatai nie tylko po to, by pogrążyć swoich przeciwników, lecz również by przeciwdziałać podziałom w szkole Chan.
W owym czasie istniało bowiem wielu samozwańczych mistrzów medytacji, którzy nauczali Dharmy i zakładali własne szkoły, przeinaczając prawdziwą istotę nauk Buddy
(Shi 2008: 191, 200).

W dziełach Shenhui pojawiły się twierdzenia, jakoby Puji wysłał swojego ucznia, niejakiego Zhang Xingchang (張行昌), do Shaozhou, z poleceniem ucięcia głowy zwłokom Huinenga, a także, że inny uczeń Puji, imieniem Wu Pingyi (武平一), wymazał inskrypcję na steli poświęconej Huinengowi i wstawił tam własną, podającą Shenxiu jako prawowitego Szóstego Patriarchę.
Zarzucił Puji, że w dziele ``Annały przekazu skarbu Dharmy'' (傳法寶紀 \pinyin{Chuán fǎbǎo jì}) spisał historię linii przekazu Chan z pominięciem Huinenga.
Stwierdzenie to było o tyle bezpodstawne, że Puji nie był autorem tego dzieła
(Huineng i Yampolsky 2012: 28).

Na konferencji w Huatai Shenhui zarzucał swoim przeciwnikom, że uzurpowali sobie prawo do linii przekazu patriarchatu, oraz że przeinaczyli istotę buddyzmu, nauczając błędnej, stopniowej ścieżki.
W rzeczywistości, Shenxiu opierał swoje nauki na tych samych podstawowych ideach buddyzmu Mahajany, co Shenhui.
W obu szkołach ostatecznym celem praktyki duchowej było osiągnięcie oświecenia poprzez rozpoznanie prawdziwej natury umysłu; w obydwu przebudzenie było postrzegane jako krótki przebłysk wglądu.
Jednak w doktrynie Szkoły Północnej, opartej głównie na sutrze \textit{La\.nkā\-vatāra}, kładziono większy nacisk na praktyki przygotowawcze.
Ich zadaniem było oczyszczenie umysłu z zaciemnień, postrzeganych w tej tradycji jako istniejące i rzeczywiste.
Szkoła Południowa głosiła pustość wszystkich myśli i zjawisk.
Ideą praktyki w tej doktrynie było nie tyle pozbyć się trucizn umysłu, ile zrozumieć, że w rzeczywistości nigdy nie istniały.
W \textit{Sutrze Platformy} dychotomię tę ilustrują ostatnie dwa wersy wiersza Huinenga: ,,W rzeczywistości nie ma niczego, / Cóż miałoby przyciągać jakikolwiek kurz?'' % NIE DYCHOTOMIĘ
Nauki Shenxiu obejmowały o wiele większą część tradycji medytacyjnej Mahajany, podczas gdy Shenhui bardziej radykalnie promował doktrynę nagłego oświecenia, traktując ją jako jedyne kryterium decydujące o tym, która ze szkół reprezentowała ortodoksyjny buddyzm.
(Dumoulin 1963: 81, 84-87). % sprawdzić

Kim był Shenxiu i czym zasłużył sobie na osobiste ataki Shenhui?
Yampolsky (2012, str. 15-16) podaje, że na przełomie VII i VIII w. Shenxiu był uważany za jednego z najbardziej znaczących i najwybitniejszych mistrzów Chan.
Stosunkowo rzetelna wersja jego biografii została zapisana w dziele ``Annały przekazu skarbu Dharmy'' w pozbawiony elementów fantastycznych sposób.
O ile we wszystkich innych dziełach z tego okresu jest wymieniony jako uczeń Hongrena, ``Annały'' podają, że był uczniem Faru, a ten --- Hongrena.
Według tej biografii pochodził z miasta Daliang (大梁 \toponim{Dàliáng}), obecnie Kaifeng (開封 \toponim{Kāifēng}) w prowincji Henan (河南 \toponim{Hénán}) i był członkiem rodu Li.

Shenxiu już od najmłodszych lat wykazywał się ponadprzeciętnymi uzdolnieniami.
W wieku 13 lat, w związku z zawirowaniami historycznymi i związaną z nimi klęską głodu, opuścił rodzinny dom z zamiarem zostania mnichem buddyjskim.
Później wędrował od świątyni do świątyni, by wreszcie jako dwudziestolatek otrzymać pełne ślubowania.
W wieku 46 lat udał się do Hongrena, a ten natychmiast poznał się na jego talencie.
Po wielu latach studiowania nauk osiągnął ostateczne oświecenie, a następnie udał się do Jingzhou (荊州 \toponim{Jīngzhōu}) w prowincji Hubei (湖北 \toponim{Húběi}).
Za panowania cesarza Tang Gaozonga (唐高宗 \nazwisko{Táng Gāozōng}), podczas ery Yifeng (儀鳳 \pinyin{Yífèng}) udał się do świątyni Yuquan Si (玉泉寺 \pinyin{Yùquán sì}) w pobliżu obecnego miasta Dangyang (當陽 \toponim{Dāngyáng}) w prowincji Hubei.
Dopiero po śmierci swojego mistrza zaczął gromadzić wokół siebie uczniów, nauczając ich Dharmy
(Yampolsky 15-16).

Główny uczeń Shenxiu, Songshan Puji, pochodził z miejscowości Hedong (河東 \toponim{Hédōng}), obecnie położonej w powiecie Yongji (永濟縣 \toponim{Yǒngjì xiàn}) w prowincji Shanxi (山西 \toponim{Shānxī}).
Studia buddyzmu rozpoczął w mieście Daliang, kładąc szczególny nacisk na nauki Yogācāra\footnote{Yogācāra --- i tu coś napiszemy.}, Sutrę Lotosu oraz tekst pt. ``Przebudzenie wiary w Mahajanie'' (大乘起信論 \pinyin{Dàchéng qǐ xìn lùn}).
W 688 roku otrzymał pełne święcenia mnisie.
Po latach pobierania nauk od różnych mistrzów, udał się do świątyni Shaolin Si (少林寺 \pinyin{Shǎolín Sì}) w poszukiwaniu mistrza Faru, jednak kiedy tam przybył, okazało się, że Faru już nie żył
Następnie wyruszył do świątyni Yuquan Si, gdzie pod okiem Shenxiu zgłębiał przez pięć lat takie teksty, jak ``Sutra Siyi'' (思益梵天所問經 \pinyin{Sīyì fántiān suǒwèn jīng}, w skrócie 思益經 \pinyin{Sīyì jīng}) oraz sutra \textit{La\.nkā\-vatāra}.
Około roku 696 przeniósł się na górę Song.
Po śmierci Shenxiu w roku 706 otrzymał propozycję *** tu będzie coś więcej% 67
(Huineng i Yampolsky 2012: 22; McRae 1986: 65-67).

Tradycyjne stanowisko historyków Chan na temat rozwoju Szkoły Północnej głosi, iż w następstwie ataków Shenhui oraz śmierci głównych spadkobierców Shenxiu, tj. Songshan Puji oraz Dazhi Yifu, liczba jej uczniów drastycznie spadła, natomiast Szkoła Północna rosła w siłę.
McRae (1986, str. 61) zaprzecza jednak tym twierdzeniom.

Jak podaje Fauré (1997, str. 91), jeszcze w roku 758, a więc 26 lat po konferencji w Huatai i 19 lat po śmierci Puji, poeta Wang Wei --- ten sam, który na prośbę Shenhui wykonał stelę upamiętniającą Huinenga --- wystosował do cesarza pismo, dziękując w imieniu mistrza zwanego \textit{ācārya} Shun (舜闍黎 \nazwisko{Shùn shélí}) za wykonanie inskrypcji na stupie mistrzów Shenxiu i Puji w świątyni Songyue Si (嵩嶽寺 \pinyin{Sōngyuè sì}) na górze Song.

Shenxiu zmarł w roku 739, a w następstwie jego śmierci wielu ludzi odłączało się od niego i dołączało do nowej szkoły, nazwanej Heze od klasztoru, w którym rezydował wówczas Shenhui.
W roku 753, urzędnik cesarski Lu Yi (盧弈), który popierał Szkołę Północną, oskarżył Shenhui o wichrzycielstwo i wnioskował o jego banicję.
Wygnanie lidera zagrażało istnieniu szkoły Heze, jednak sytuacja zmieniła przybrała korzystniejszy obrót już w roku 755, w związku z wybuchem rebelii An Lushana%
\footnote{Rebelia An Lushana (安史之亂 \pinyin{Ān shǐ zhī luàn}).}, w wyniku której Lu Yi stracił życie.
W 757 roku Shenhui został wezwany do Luoyangu, aby sprzedając świadectwa święceń mnisich, pomagał zbierać fundusze na finansowanie armii cesarskiej.
W tym samym czasie uczniowie Puji wzbraniali się przed tego typu praktykami czysto politycznej natury, co stało w kontraście do oportunizmu Shenhui.
W nagrodę za zasługi dla państwa, nowy cesarz Tang Suzong	(唐肅宗 \nazwisko{Táng Sùzōng}) zatrudnił go do odprawiania rytuałów w kaplicy pałacu cesarskiego.
W 792 roku zgromadzenie mistrzów Chan, zwołane przez cesarza Tang Dezonga (唐德宗 \nazwisko{Táng Dézōng}), pośmiertnie mianowało Shenhui Siódmym Patriarchą Chan
% podżeganie, podjudzanie
(Fauré 1997: 89-90).

\if 0
Huineng i Yampolsky 2012: 22
W roku 735 został zaproszony na dwór przez cesarzową Wu Zetian. W stolicy cieszył się znaczącą popularnością, zarówno we dworze, jak i wśród prostego ludu.
Zmarł w roku 739, a na jego pogrzeb przybyło wielu ludzi, w tym urzędnicy cesarscy.
Szkoła Shenxiu była popularna na północy Chin, podupadła po śmierci głównych uczniów Shenxiu: Songshan Puji i Dazhi Yifu (大智義福), by w końcu wygasnąć. % sprawdź kiedy

McRae 1986: 70
Szkoła Północna nie zaczęła tracić na sile w bezpośrednim następstwie ataków Shenhui, ale

Yifu pochodził z Tongdi 銅鞮, obecnie 沁縣 w 山西.
Za młodu studiował daoizm
\fi

\section{Dalszy podział Szkoły Południowej}
Spośród mistrzów uważanych za uczniów Huinenga, największy wpływ na dalszy rozwój szkoły Chan mieli Qingyuan Xingsi (青原行思 \nazwisko{Qīngyuán Xíngsī}) oraz Nanyue Huairang (南嶽懷讓 \nazwisko{Nányuè Huáiràng}).
Biorąc pod uwagę, że żaden z nich nie został wspomniany w \textit{Sutrze platformy} w wersji z Dunhuang, ich spotkania z Huinengiem prawdopodobnie nigdy nie miały miejsca i zostały dodane do kronik jedynie w celu nadania ich liniom przekazu autentyczności.
Chociaż Xingsi i Huairang uważani byli przez współczesnych za wybitnych mistrzów, okres rozwoju ich szkół rozpoczął się dopiero za czasów ich uczniów, Mazu Daoyi oraz Shitou Xiqian % Jeżeli ta linia zostaje, to należy przenieść nawiasy
(McRae 2004: 82).

\section{Nanyue Huairang}
Nanyue Huairang narodził się w drugim roku ery Yifeng (677) w miejscowości Ankang (安康 \toponim{Ānkāng}) w obecnej prowincji Shaanxi (陝西 \toponim{Shǎnxī}).
W wieku piętnastu lat opuścił rodzinny dom, a następnie uczył się Vinaya\footnote{Vinaya --- wyjaśnienie. (stub)} u mistrza Yuquan Hongjing (玉泉弘景 \nazwisko{Yùquán Hóngjǐng}).
Nieusatysfakcjonowany jego naukami, wkrótce udał się na górę Song (嵩山 \toponim{Sōngshān}), gdzie znalazł swojego następnego nauczyciela, Songshan Hui'an (嵩山惠安 \nazwisko{Sōngshān Huì'ān}).
W roku 699 udał się do Caoqi, gdzie przez kolejne dwanaście lat zgłębiał Dharmę pod okiem Szóstego Patriarchy.
Jak podaje kronika ``Kompendium Pięciu Lamp'' (五燈會元 \pinyin{Wǔdēng huìyuán}), zredagowana w roku 1252, Huineng miał go wówczas poinformować o przepowiedni indyjskiej mistrzyni dhjany o imieniu Prajñātārā (skt., chiń. 般若多羅 \nazwisko{Bōrěduōluó}), 27. patriarchy Indii oraz nauczycielki Bodhidharmy.
Proroctwo głosiło, że ,,spod stóp Huairanga wyjdzie koń, który zadepcze na śmierć wszystkich ludzi na tym świecie''.
Słowa te odnosiły się do Mazu Daoyi (馬祖道一 \nazwisko{Mǎzǔ Dàoyī}), ucznia Huairanga, który znacząco wpłynął na dalszy rozwój buddyzmu Chan.
Daoyi miał bowiem na nazwisko Ma (馬 \nazwisko{Mǎ}), co oznacza konia.
W roku 713 Huairang przybył do świątyni Bore Si (般若寺 \pinyin{Bōrě Sì}), na górze Heng (衡山 \toponim{Héngshān}) w prowincji Hunan (湖南 \toponim{Húnán}).
McRae (2004, str. 82) uważa, że Huairang prawdopodobnie nigdy nie spotkał Huinenga.
Jego głównymi uczniami byli Daojun (道峻 \nazwisko{Dàojùn}), Shenzhao (神照 \nazwisko{Shénzhào}) oraz Mazu Daoyi (馬祖道一 \nazwisko{Mǎzǔ Dàoyī}).
Nanyue Huairang zmarł na górze Heng w roku 744
(Ferguson 2011: 53-56).

\subsection{Mazu Daoyi i szkoła Hongzhou}
Mazu Daoyi (707-786) przybył do Hengshan w roku 735.
Słynął z niekonwencjonalnych metod nauczania.
Był pierwszym mistrzem Chan, który zastosował technikę zwaną \textit{katsu} (chiń., jap. 喝, Pinyin: \pinyin{hè}, Rōmaji: \textit{katsu}).
Polega ona na krzyczeniu na ucznia w celu przełamania jego sztywnych koncepcji.
W późniejszym okresie stosował ją mistrz Linji Yixuan (臨濟義玄 \nazwisko{Línjì Yìxuán}), założyciel szkoły Linji (臨濟宗 \pinyin{Línjì zōng}). % tutaj walniemy odnośnik
Podczas ery Dali (大曆 \pinyin{Dàlì}), Mazu Daoyi zamieszkiwał w świątyni Baohua Si (寶華寺 \pinyin{Bǎohuá Sì}) na górze Gonggong (龔公山 \toponim{Gōnggōng Shān}).
W czwartym roku tego okresu przeniósł się do świątyni Kaiyuan Si (開元寺 \pinyin{Kāiyuán Sì}) w rejonie Hongzhou\footnote{洪州 \toponim{Hóngzhōu}, obecnie położone w granicach administracyjnych miasta Nanchang (南昌 \toponim{Nánchāng}) w prowincji Jiangxi.}, gdzie nauczał aż do śmierci w roku 788.
Nazwa wywodzącej się od niego szkoły Hongzhou (洪州宗 \pinyin{Hóngzhōu zōng}) pochodzi właśnie od tego miejsca.
Miał bardzo wielu uczniów, z których najważniejsi dla historii Chanu byli Baizhang Huaihai (百丈懷海 \nazwisko{Bǎizhàng Huáihǎi}) oraz Nanquan Puyuan (南泉普願 \nazwisko{Nánquán Pǔyuàn})
(Dumoulin 1963: 97-98; Chang 1971: 148).

\subsection{Powstanie szkół Linji i Guiyang}
Baizhang Huaihai miał ucznia imieniem Huangbo Xiyun, który z kolei udzielił przekazu Dharmy Linji Yixuanowi.
% Założycielem szkoły Linji był mistrz Linji Yixuan, uczeń Huangbo Xiyun, który z kolei

\section{Qingyuan Xingsi}
Qingyuan Xingsi urodził się w roku 660 w miejscowości Ancheng (安城 \toponim{Ānchéng}), położonej w granicach obecnego powiatu Ji'an (吉安縣 \toponim{Jí'ān xiàn}), w prowincji Jiangxi.
Jak podaje Ferguson (2011, str. 56), Xingsi opuścił dom rodzinny w młodym wieku.
% Jego cechą charakterystyczną była małomówność; zawsze milczał w trakcie debat filozoficznych.
Przez pewien czas przebywał w Caoqi, gdzie pobierał nauki od Szóstego Patriarchy, który miał go wówczas uczynić dzierżawcą przekazu Chan.
Później zamieszkał w świątyni Jingju Si (淨居寺 \pinyin{Jìngjū Sì}) na górze Qingyuan (青原山 \toponim{Qīngyuán shān}).
Kronika ``Biografie wybitnych mnichów Song'' podaje, że uczniowie z czterech stron świata przybywali tłumnie do jego świątyni i pobierali jego nauki, jednak w chwili obecnej jedynym jego uczniem znanym z imienia jest Shitou Xiqian (石頭希遷 \nazwisko{Shítóu Xīqiān}).
Zmarł w 28. roku ery Kaiyuan (740).
(Shi 2008: 207-208)

\subsection{Shitou Xiqian i szkoła Shitou}
Historia życia Shitou Xiqian została spisana w ``Księdze przekazu lampy z okresu Jingde''.
Według tego zapisu, Xiqian zmarł w 6. roku ery Zhenyuan (貞元 \pinyin{Zhēnyuán}), a więc 790, przeżywszy 91 lat.
W związku z tym Shi (2008, str. 208) przyjmuje, że urodził się w 1. roku ery Jiushi (久視 \pinyin{Jiǔshì}), czyli 700.
Pochodził z powiatu Gaoyao (高要縣 \toponim{Gāoyào xiàn}), położonego w granicach obecnej prowincji Guangdong.
W wieku 29 lat przyjął pełne święcenia mnisie.
W 742 roku przybył na górę Heng, gdzie założył świątynię na półce skalnej.
Stąd wziął się jego przydomek, Shitou (石頭 \pinyin{shítóu} `kamień'), będący zarazem nazwą jego szkoły.
Shitou Xiqian nauczał Dharmy przez 48 lat, a w ``Księdze przekazu lampy z okresu Jingde'' wymieniono 21 jego uczniów, z których najważniejsi byli Tianhuang Daowu (天皇道悟 \nazwisko{Tiānhuáng Dàowù}), Yaoshan Weiyan (藥山惟儼 \nazwisko{Yàoshān Wéiyǎn}) oraz Xishan Dadian (西山大顛 \nazwisko{Xīshān Dàdiān}).
% Szkoła Shitou stanowiła najważniejszy, obok szkoły Hongzhou, odłam buddyzmu Chan.
% W początkowym etapie swego rozwoju liczyła znacznie mniej wyznawców niż konkurencyjne tradycje, i
W ``Inskrypcji upamiętniającej mistrza Chan, Dade Dayi, w świątyni Xingfu Si'' (興福寺內道場供奉大德大義禪師碑銘 \pinyin{Xīngfú Sì nèi dàochǎng gòngfèng Dàdé Dàyì Chán shī bēimíng}), powstałej pomiędzy rokiem 818 a 828, urzędnik cesarski Wei Chuhou (韋處厚) opisał ówczesny rozkład tradycji buddyzmu w różnych rejonach Chin.
W mieście Chang'an aktywna była wówczas Szkoła Północna; w Luoyangu szkoła Heze (荷澤宗 \pinyin{Hézé zōng}), a więc bezpośredni kontynuatorzy myśli Shenhui; w delcie Yangzi szkoła Niutou (牛頭宗 \pinyin{Niútóu zōng}), a na obszarze odpowiadającym obecnej prowincji Hunan szkoła Hongzhou.
% 应身无数,天竺降其一;禅祖有六,圣唐得其三。在高祖时,有道信叶昌运;在太宗时,有宏忍示元珠;在高宗时,有惠能筌月指。自此脉散丝分,或遁秦,或居洛,或之吴,或在楚。秦者曰秀,以方便显,普寂其允也。洛者曰会,得总持之印,独曜莹珠,习徒迷真,橘枳变体,竟成《檀经》传宗,优劣详矣。吴者曰融,以牛头闻,径山其裔也。楚者曰道一,以大乘摄,大师其党也。
Tekst ten nie wspomniał o istnieniu szkoły Shitou, co wskazuje, że w owym czasie miała dla historyków niewielkie znaczenie, oraz że nie była uważana za ortodoksyjny odłam Chanu.
Sytuacja ta zmieniła się po prześladowaniach buddyzmu pod panowaniem cesarza Wuzong z Tang (唐武宗 \nazwisko{Táng Wǔzōng}), których szczytowy okres przypadł na rok 845.
Wedle tradycji, trzy spośród pięciu głównych tradycji Chan: Yunmen (雲門宗 \pinyin{Yúnmén zōng}), Fayan (法眼宗 \pinyin{Fǎyǎn zōng}) oraz Caodong (曹洞宗 \pinyin{Cáodòng zōng}) wywodzą się od uczniów Shitou.
(Shi 2008: 209).

\if 0
Doczytać o rebelii An Lushana!

McRae 1986: 3
W początkowej fazie istnienia Południowej Szkoły, była ona mało znana; sutra wyjaśnia to długim czasem, jaki Huineng spędził, ukrywając się u myśliwych
McRae 1986: 5
Szkoła Południowa twierdziła, że posiada nauki niedualne
Wg SS natura oświecenie, przeszkadzające emocje, cierpienie i iluzje są w istocie tym samym, co oświecenie, ale NS widziała je jako różne; wg 宗密 oznacza to, że wiele lat, lub nawet żywotów praktyki idzie na marne; wszystko, czego potrzebuje praktykujące, to całkowite odcięcie dualistyczn, ego myślenia
SS była lewicowa: uważała, że każdy powinien mieć prawo poznać Dharmę i osiągnąć oświecenie, a nie tylko ci, którzy włożyli w to wysiłek
Zongmi usystematyzował różne interpretacje Chan, Szkoła Północna była najniżej
wykładnia 宗密 NS na początku była popularna, ale potem została niemal całkowicie wyparta przez SS, bo prawowitym spadkobiercą był Huineng, a nie Shenxiu
Nie ma dobrych badań nt. NS

Dumoulin 1963: 70
W czasie, kiedy Bodhidharma przyniósł Chan do Chin, na północy Chin był Buddhabhadra, a na południu szkoła 三论
Sanlun interesowała się Pradżniaparamitą Nagardżuny

79: 弘忍 rozwijał własne metody medytacji w oparciu o sutry Avatamsaka (華嚴)

Dumoulin 81
Pierwsza schizma w tradycji Chan
SS wygrała walkę o dominację, bo NS nie rozwijało się po śmierci uczniów Shenxiu, a SS publikowało wiele koanów i kronik
w 700 Shenxiu polecił cesarzowi zaprosić Huinenga do stolicy
(89: w 705 Huineng to zaproszenie odrzucił)
Shenxiu cieszył się szacunkiem dworu, główni uczniowie 嵩山普寂、嶗山義福
83
Według niektórych podań Shenhui był przez kilka lat uczniem Shenxiu, ale jest to mało prawdopodobne
Shenhui na pewno przebywał razem z Huinengiem
91 Dualizm rozpuszcza się w pustości -- myśl z Diamentowej Sutry
91/92 Samadhi w którym nie ma myśli
Umysł ma skłonność do konceptualizacji absolutu, np. przywiązuje się do koncepcji nirwany albo pustki
Jeżeli zamiast tego pozbędzie się wszelkich koncepcji, to pozostanie samo lustro


中國禪宗史 115:
W sutrze jest napisane, że Huineng 先天二年八月三日滅度 oraz 春秋七十有六, ale są stele, które mówią co innego i przesuwają datę śmierci o 3 lata
《曹溪大師別傳》
117: Z legend 神會, zapisanych przez 王維 wynika, że Huineng ukrywał się przez 16 lat, dostał 衣法 od 弘忍 na jego łożu śmierci, i że po 16 latach ukrycia spotkał się z 印宗 i zaczął nauczać
神會語錄:能禪師過嶺至韶州,居曹溪山,來往四十年。 nie ma 印宗,隱遁
歷代法寶記:ukrycie, 印宗
Teoria ukrycia wzięła się z niespójności lat

182:713-815 神會向中原傳播南宗頓教
W latach 713-815 nauki Shenhui rozprzestrzeniły się w rejonie Niziny Chińskiej (a właściwie 中原) w dorzeczu Rzeki Żółtej (黃河)

188: 宋高僧传 podaje, że w roku 開元八年 zamieszkał w świątyni 龍興寺 w 南陽
potem nauczał Dharmy w Luoyangu, a Shenxiu i jego uczniowie aż zielenieli z zazdrości
Mieszkał przez dłuższy czas w Nanyang, więc nazywano go 南陽和上
神會語錄 autor 劉澄

189:南宗定是非論 opisuje wydarzenia z 滑臺寺, gdzie Shenhui urządził otwartą konferencję
圓覺經大疏釋義鈔 podaje, że w 20 lat po śmierci Huinenga (czyli do Huatai - 732), stopniowa ścieżka Shenxiu była rozpowszechniona w 荊吳 - środkowym i dolnym biegu Changjiang i bodajże w dwóch stolicach (秦洛 czyli Chang'an i Luoyang?)
Shenhui twierdził, że Shenxiu nie miał prawdziwego przekazu, a jedynie poboczny; prawowitym spadkobiercą Bodhidharmy miał być Huineng
Shenhui musiał mieć wsparcie społeczności z Dayun Si w Huatai, inaczej w żadnym wypadku nie umożliwiliby mu organizacji takiej konferencji
190 musieli skołować na konferencję 屏風 parawany, które w czasie jej trwania ktoś wyniósł, żeby przyjąć jakiegoś urzędnika
191 Shenhui urządził tę konferencję nie tylko dlatego, że chciał pogrążyć swoich przeciwników, ale również dlatego, że uważał, że istniało zbyt wielu samozwańczych mistrzów Chan, którzy przeinaczali istotę nauk Buddy.
\fi
