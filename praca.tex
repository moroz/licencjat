\documentclass[12pt]{wzmgr}
\special{papersize=210mm,297mm}
\usepackage[polish]{babel}
\usepackage[no-math]{fontspec}
\usepackage[usenames,dvipsnames,svgnames,table]{xcolor}
\usepackage{xeCJK,makeidx,sectsty}
\usepackage[compact]{titlesec}
\usepackage{setspace}
\usepackage{indentfirst,url,hyphenat}
\usepackage[top=2.5cm,left=2.5cm,right=2.5cm,bottom=2.5cm]{geometry}
\makeatletter
\g@addto@macro{\UrlBreaks}{\UrlOrds}
\makeatother
\setmainfont[Mapping=tex-text]{Minion Pro}
\setsansfont[Scale=0.88]{IPAexGothic}
\setCJKmainfont{SimSun}
\setCJKsansfont{SimSun}
\setCJKmonofont{SimSun}
\newCJKfontfamily{\ipaexgothic}{IPAexGothic}
\newCJKfontfamily{\korm}{NanumMyeongjo}
\newfontfamily\pyfont{Times New Roman}
\newcommand{\Korean}{\korm\CJKspace}
\newcommand{\toponim}[1]{\nohyphens{\pyfont #1}}
\newcommand{\nazwisko}[1]{\nohyphens{\pyfont #1}}
\newcommand{\pinyin}[1]{\nohyphens{\pyfont\itshape #1}}
\newcommand{\fnm}{\footnotemark}
\newcommand{\ibid}{ (\textit{Ibidem}).}
\author{Karol Moroz}
\title{Narodziny legendy.\\Sutra Platformy Szóstego Patriarchy Chan}
\nralbumu{386956}
\kierunek{sinologia}
\opiekun{dr Maria Kurpaska}
\email{dmuhafc@gmail.com}
\UniversityName{Uniwersytet im. Adama Mickiewicza --- Wydział Neofilologii}
\nrwersji{0.0}
\miejsce{Poznań}
%\hyphenation{ La\.n-kā-va-tā-ra}

\begin{document}
\spacing{1.25}
\maketitle
\titleformat{\chapter}[hang]
  {\normalfont\huge\color[HTML]{000099}}{{\thechapter}}{20pt}{}
\titlespacing{\chapter}{0pt}{0.5em}{0.5em}
\sectionfont{\color[HTML]{990000}}
\chapter*{Wstęp}
\addcontentsline{toc}{chapter}{Wstęp}
\markboth{Wstęp}{Wstęp}
\renewcommand{\headrulewidth}{0.3pt}

\textit{Sutra Platformy Szóstego Patriarchy} (chiń. 六祖壇經, Pinyin: \pinyin{Liùzǔ Tánjīng}) jest apokryficznym tekstem buddyzmu chan (patrz: rozdział 1), przypisywanym legendarnemu Szóstemu Patriarsze Chan, Dajian Huinengowi (chiń. 大鑒惠能, Pinyin: \nazwisko{Dàjiàn Huìnéng}). Najstarsza zachowana wersja tego dzieła powstała w VIII w. w Chinach.

Celem niniejszej pracy jest przedstawienie historycznego tła powstania \textit{Sutry Platformy}, analiza jej treści oraz opisanie wpływu, jaki wywarła na buddyzm chiński.

W pierwszym rozdziale pracy przedstawiono buddyzm chan i opisano jego historię.

W drugim rozdziale pracy omówiona jest hagiografia Huinenga, głównej postaci sutry.
Rozdział trzeci stanowi analiza treści \textit{Sutry Platformy}, czwarty natomiast poświęcony jest dalszemu rozwojowi szkoły chan.

Terminy chińskie w niniejszej pracy podane są w nawiasach w znakach tradycyjnych oraz w transkrypcji \textit{Hanyu Pinyin} (漢語拼音 \pinyin{Hànyǔ Pīnyīn}) z oznaczonymi tonami. Wyjątkiem są chińskie nazwiska, do których nie podano transkrypcji.
Wszystkie terminy japońskie w nawiasach zapisano współczesnym pismem japońskim oraz w romanizacji Hepburna (jap. {\ipaexgothic ヘボン式ローマ字} \textit{Hebon-shiki Rōmaji}).

Odwołania do tekstu \textit{Sutry Platformy} w niniejszej pracy odnoszą się do jej przekładu pt. \textit{The Platform Sutra of the Sixth Patriarch: The Text of the Tun-huang Manuscript} Philipa B. Yampolsky'ego, wydanego drukiem przez Columbia University Press w roku 1967 i wznowionego w roku 2012.
Fragmenty oryginału podano za wersją z \textit{Chinese Electronic Tripitaka}, opublikowaną w Internecie przez Chinese Buddhist Electronic Text Association.

\chapter{Biografie Huinenga}
Życie Szóstego Patriarchy Huineng jest owiane tajemnicą. Jego imię pojawia się bowiem w kronice pt. \textit{Księga przekazu lampy z okresu Jingde} (景德傳燈錄 \pinyin{Jǐngdé chuán dēng lù}) jako jednego z dziesięciu głównych uczniów piątego patriarchy Hongren, z tekstu nie wynika jednak, by był postacią szczególnie ważną dla rozwoju całej szkoły Chan. W tekście tym wspomniano, że Huineng żył i nauczał w miejscowości Caoqi (曹溪 \toponim{Cáoqī}, również: \toponim{Cáoxī}). Imię Huineng pojawia się również w pewnym tekście z grot Dunhuang, upamiętniającym Piątego Patriarchę Hongren, jednak tekst ów nie mówi nic o przypisywanych Huinengowi doktrynach. Kanoniczna biografia Huinenga oparta jest na przypisywanej mu \textit{Sutrze Platformy} (McRae 2004: 68).

\section{Biografia Huinenga według \textit{Sutry Platformy} w wersji z Dunhuang}
Jak podaje tekst \textit{Sutry Platformy}, Huineng urodził się w miejscowości Xinxing w regionie Nanhai (南海新興 \toponim{Nánhǎi Xīnxīng}, obecnie prowincja Guangdong). Za ramy czasowe jego życia przyjmuje się lata 638-713. Szósty Patriarcha jest w tym tekście przedstawiany jako ubogi, niepiśmienny człowiek świecki z południa Chin.

Jak podaje tekst, jego ojciec był urzędnikiem z regionu Fanyang (范陽 \toponim{Fànyáng}), obecnie miasto Zhuozhou (涿州 \toponim{Zhuōzhōu}) w prowincji Hebei), lecz został odwołany ze stanowiska i skazany na banicję. W związku z tym musiał przenieść się z całą rodziną do Xinxing, gdzie niedługo później zmarł. Po jego śmierci, Huineng trudnił się zbieraniem i sprzedażą drewna na opał.

% ===============
% TAJEMNICZY MĘŻCZYZNA I PODRÓŻ
% ===============
Pewnego dnia, gdy dwudziestodwuletni Huineng sprzedawał drewno na targowisku, pewien klient zażyczył sobie, żeby drewno zostało przyniesione do jego sklepu. Huineng dostarczył drewno i dostał za nie pieniądze, a kiedy wyszedł ze sklepu, spotkał mężczyznę, który recytował na ulicy Sutrę Diamentową (金剛經 \pinyin{Jīngāng jīng}, skt. \textit{Vajracchedikā Prajñāpāramitā Sūtra}). Usłyszawszy ów tekst, Huineng uzyskał wgląd w naturę swego umysłu i osiągnął oświecenie. Następnie spytał tajemniczego mężczyznę, skąd przybył. Ten odpowiedział, że przybył z klasztoru Dongshan (東山寺 \pinyin{Dōngshān sì}) na górze Fengmushan (憑墓山 \toponim{Féngmù shān}) w powiecie Huangmei (黃梅懸 \toponim{Huángméi xiàn}) w Qizhou (蘄州 \toponim{Qízhōu}), którego opatem był Piąty Patriarcha, Hong Ren (弘忍 \pinyin{Hóngrěn}), i gdzie przebywało około tysiąca mnichów. Patriarcha miał zalecić mnichom, aby recytowali ową sutrę, ponieważ dzięki tej praktyce można szybko osiągnąć oświecenie (Huineng i Hsüan Hua 1977: bez nru strony; Huineng i Yampolsky 2012: 127).

Według tej biografii, niedługo po spotkaniu tajemniczego mężczyzny, Huineng spotkał kogoś, kto poradził mu udać się do klasztoru Dongshan, aby poprosić Patriarchę o nauki, i dał mu pieniądze, aby mógł zaaranżować opiekę dla swej matki.

Kiedy Huineng przybył do klasztoru, spytano go, skąd przybył i w jakiej sprawie przybył do patriarchy. Odparł, że pochodzi z Kantonu i przyszedł oddać cześć patriarsze, i że nie prosi o nic prócz Dharmy. Mnisi stwierdzili wówczas, że Huineng, jako \textit{geliao} (獦獠 \pinyin{géliáo}, `barbarzyńca'), niegodny jest otrzymania nauk. Obszar obecnego Kantonu był wówczas zamieszkany przez niechińskie ludy, mówiące własnymi językami, posiadające własną, niechińską kulturę i nieżyjące zgodnie z naukami Buddy --- mieszkańcy południa polowali bowiem i jedli mięso. Nie do pomyślenia było dla mnichów, by człowiek z południa mógł otrzymać nauki od Patriarchy i osiągnąć oświecenie. Huineng odparł wtedy, że ludzie dzielą się na tych z południa i tych z północy, ale takie podziały nie mają wpływu na ich naturę buddy. Patriarcha uznał, że Huineng dobrze rozumiał nauki Buddy, lecz w obawie, że inni uczniowie mogliby zrobić mu krzywdę, kazał mu iść pracować w stajni. Tam, przez następne osiem miesięcy, Huineng rąbał drewno i młócił zboże (Huineng, Wong i Humphreys 1998: rozdział 1; Huineng, Schlütter i Teiser 2012: 27).

% ===============
% KONKURS POEZJI
% ===============
Według biografii zawartej w \textit{Sutrze Platformy}, pewnego dnia Patriarcha Hongren zwołał zebranie wszystkich uczniów i ogłosił: ,,Uwarunkowana egzystencja jest kwestią doniosłą. Dzień po dniu zasiewacie tylko nasiona ponownego odrodzenia, zamiast starać się wyzwolić z oceanu samsary\fnm. Te działania w niczym wam nie pomogą, jeżeli esencja waszego umysłu jest przysłoniona. Szukajcie pradżni (mądrości) w swoim umyśle i napiszcie wiersz na ten temat. Ten z was, który rozpozna esencję umysłu, otrzyma ode mnie szatę Patriarchy i przekaz nauk. (\ldots) Człowiek, który urzeczywistnił esencję umysłu, potrafi mówić o niej od razu, kiedy tylko zostanie o nią zapytany; nigdy też nie jest w stanie jej utracić, nawet podczas bitwy.'' (Huineng, Wong i Humphreys 1998: rozdział 1).
\footnotetext{[Huineng i Yampolsky 2012] podaje w tym miejscu: ,,Całymi dniami składacie ofiary i szukacie tylko pola błogosławieństw, (\ldots)''. Idea ,,pola błogosławieństw'' (福田 \pinyin{fútián}) jest związana z buddyjską koncepcją karmy jako prawa przyczyny i skutku. Oznacza stan, w którym dana osoba zgromadziła bardzo wiele dobrej karmy w rezultacie praktyki szczodrości (skt. \textit{dānā}), pierwszej z tzw. Sześciu Paramit lub Sześciu Wyzwalających Działań (pozostałe pięć to właściwe działanie, cierpliwość, radosny wysiłek, medytacja i mądrość). Słowo \textit{dānā} (布施 \pinyin{bùshī}) pojawia się m.in. w \textit{Dānādhikāramahāyānasūtra} (佛说布施经 \pinyin{Fóshuō bùshī jīng}): “若求勝妙福報而行施時,慈心不殺離諸嫉妒,正見相應遠於不善,堅持禁戒親近善友,閉惡趣門開生天路,自利利他其心平等。若如是施,是真布施,是大福田。'' Piąty Patriarcha Hongren krytykował tu swoich uczniów, ponieważ praktyka szczodrości jest wprawdzie w buddyzmie postrzegana jako pozytywne działanie, jednak nie wystarcza ona do osiągnięcia ostatecznego wyzwolenia z samsary. (Anonim 2007; Nydahl 2010)}

Mnisi stwierdzili zgodnie: ,,Nie ma sensu oczyszczać umysłu i zadawać sobie trudu układania wiersza dla patriarchy. Shenxiu (神秀 \pinyin{Shénxiù}), przewodniczący kongregacji, jest naszym nauczycielem. Kiedy on zostanie patriarchą, możemy liczyć na jego wsparcie. Dlatego nie będziemy układać wierszy.'' Żaden z nich nie podjął się więc tego zadania (Huineng i Yampolsky 2012: 127).

Shenxiu zaś dręczyły wątpliwości. Z jednej strony nie uważał, żeby jego zrozumienie nauk i urzeczywistnienie były wystarczające do przyjęcia stanowiska patriarchy, z drugiej zaś pragnął otrzymać przekaz Dharmy. Przyjęcie Dharmy dla pożytku istot byłoby bowiem pożądane i chwalebne, jednak dążenie do objęcia stanowiska patriarchy byłoby niewłaściwe. Wychodził z założenia, że jeżeli Piąty Patriarcha Hongren uzna, że jego urzeczywistnienie natury umysłu jest niewystarczające, to będzie musiał porzucić dążenia do zostania dzierżawcą linii przekazu. Wreszcie skomponował wiersz i wymknął się w nocy ze swojej celi, by napisać go na ścianie, w miejscu, gdzie miały zostać namalowane sceny z sutry \textit{La\.nkāvatāra} (楞伽經 \pinyin{Léngqié jīng}) (McRae 2004: 62):

\vspace*{6pt}
\begin{minipage}[t]{0.4\textwidth}
\begin{verse}
身是菩提樹\\
心如明鏡臺\\
時時勤佛拭\\
莫使有塵埃
\end{verse}
\end{minipage}
\begin{minipage}[t]{0.6\textwidth}
\itshape
\begin{verse}
Ciało jest drzewem Bodhi,\\
Umysł --- jasną lustrzaną podstawą.\\
Czyść ją stale i gorliwie,\\
Nie pozwalając aby przylgnął kurz.\fnm
\end{verse}
\end{minipage}
\label{ShenxiuVerse}
\vspace*{6pt}
\footnotetext{Niniejszy wiersz, jak również następny, został przytoczony w języku chińskim za \textit{Chinese Electronic Tripitaka} T48, no. 2007, w języku polskim za tekstem \textit{Sutry Szóstego Patriarchy Zen} nieznanego tłumacza.}

O poranku, Piąty Patriarcha Hongren ujrzał wiersz napisany przez Shenxiu na ścianie i uznał, że wiersz ten mógł przynieść ludziom wiele pożytku. Zapłacił wówczas malarzowi, któremu zlecił namalowanie scen z sutry \textit{La\.nkāvatāra}, i odwołał zamówienie. Zwołał całe zgromadzenie i nakazał mnichom recytować wiersz. Patriarcha spytał Shenxiu, czy to on jest jego autorem, oznaczałoby to bowiem, że jest on właściwym spadkobiercą Dharmy i jego następcą na stanowisku patriarchy. Powiedział też jednak, że wiersz nie wskazuje na to, aby Shenxiu rozpoznał już naturę swojego umysłu. Wiersz nadawał się do recytacji przez zwykłych ludzi, jednak takie niepełne zrozumienie nie wystarczało do rozpoznania prawdziwej natury umysłu. Powiedział, że przekaże mu nauki i szatę patriarchy, jeżeli temu w ciągu dwóch dni uda się osiągnąć ostateczne urzeczywistnienie. Shenxiu rozmyślał przez wiele dni, ale nie udało mu się skomponować nic lepszego. (Huineng i Yampolsky 2012: 131).

Pewnego dnia młody mnich-akolita przechodził koło stajni, w której pracował akurat Huineng, recytując wiersz Shenxiu. Huineng zrozumiał, że autor wiersza nie rozpoznał jeszcze natury swojego umysłu. Spytał mnicha, co ten recytował. Mnich odparł, że Patriarcha Hongren kazał wszystkim swoim uczniom skomponować wiersz i przynieść mu go, aby określić, kto otrzyma przekaz Dharmy i zostanie patriarchą. Wiersz o pustości, który właśnie recytował, został skomponowany przez mnicha o imieniu Shenxiu. Piąty Patriarcha nakazał wszystkim swoim uczniom recytować go, mówiąc, że ci, którzy go urzeczywistnią, zobaczą swoją prawdziwą naturę, a ci, którzy praktykują zgodnie z nim, osiągną wyzwolenie.

Huineng poprosił mnicha, aby ten zaprowadził go do miejsca, gdzie na ścianie wymalowany był wiersz Shenxiu. Ponieważ nie umiał czytać, poprosił kogoś, by przeczytał mu ten wiersz na głos. Usłyszawszy go, osiągnął ostateczne urzeczywistnienie. Huineng ułożył własny wiersz i poprosił kogoś o napisanie go na ścianie (Huineng i Yampolsky 2012: 131).

\begin{minipage}[t]{0.4\textwidth}
\begin{verse}
菩提本無樹\\
明鏡亦無臺\\
佛性常清淨\\
何處有塵埃
\end{verse}
\end{minipage}
\begin{minipage}[t]{0.6\textwidth}
\itshape
\begin{verse}
Sama istota Bodhi nie ma drzewa,\\
Nie ma też jasnej lustrzanej podstawy.\\
W rzeczywistości nie ma niczego,\\
Cóż miałoby przyciągać jakikolwiek kurz?
\end{verse}
\end{minipage}
\label{HuinengVerse}
\vspace*{6pt}

Mnisi ze zgromadzenia byli pod wielkim wrażeniem wiersza Huinenga, a Piąty Patriarcha Hongren stwierdził na jego podstawie, że Huineng miał już wówczas ponadprzeciętne zrozumienie natury zjawisk, ale dla bezpieczeństwa Huinenga oznajmił zgromadzeniu, że wciąż nie było to pełne urzeczywistnienie (Huineng i Yampolsky 2012: 132).

% ===============
% PRZEKAZ DHARMY I WYGNANIE
% ===============
Patriarcha przywołał go do siebie w nocy i udzielił mu wyjaśnień do \textit{Sutry Diamentowej}, dzięki której Huineng natychmiast rozpoznał naturę umysłu. Hongren przekazał mu również nauki o spontanicznym oświeceniu oraz szatę, insygnium patriarchatu. Przekazał mu również ostatnie pouczenia: ,,Mianuję cię Szóstym Patriarchą. Szata jest tego dowodem i jest przekazywana z pokolenia na pokolenie. Moja Dharma musi być przekazywana z umysłu na umysł. Spraw, by ludzie rozpoznali swoją prawdziwą naturę. (\ldots) Od czasów starożytnych przekaz Dharmy był równie słaby, jak zwisający sznurek. Jeżeli pozostaniesz tutaj, inni ludzie zrobią ci krzywdę. Musisz więc niezwłocznie odejść.'' (Huineng i Yampolsky 2012: 133).

Huineng udał się na południe. Jego śladem podążyło kilkuset ludzi, pragnących go zabić i siłą odebrać od niego szatę i Dharmę. Po dwóch miesiącach miał dotrzeć do miejsca zwanego \textit{Dayu ling}\fnm (大庾嶺 \toponim{Dàyǔ líng}). Jest to pasmo górskie, znajdujące się pomiędzy południowym wschodem prowincji Jiangxi a prowincją Guangzhou. Tam doścignął go mnich imieniem Huiming (惠明 \pinyin{Huìmíng}***) lub Huishun (惠順 \pinyin{Huìshùn}), były generał, człowiek szorstki i porywczy. Huiming groził Huinengowi, który bez wahania oddał mu szatę, lecz Huiming nie chciał jej przyjąć, mówiąc, że przybył wyłącznie po to, by otrzymać przekaz Dharmy. Huineng miał przekazać mu Dharmę na szczycie góry, a gdy Huiming usłyszał nauki, natychmiast osiągnął oświecenie. Następnie Huineng polecił Huimingowi udać się na północ i nauczać tamtejszych ludzi (Huineng i Yampolsky 2012: 134).
%
\footnotetext{W tekście sutry w \textit{Chinese Electronic Tripitaka} w tym miejscu podana jest nazwa 大庚嶺 \pinyin{Dageng ling}. Jest to najprawdopodobniej błąd w tekście. ***}

W związku z prześladowaniami, Huineng schronił się w miejscu zwanym Caoqi (曹溪, także: Caoxi), gdzie przez piętnaście lat ukrywał się wśród prostego ludu --- myśliwych. Dopiero potem opuścił miejsce odosobnienia i zaczął nauczać Dharmy (Huineng, Wong i Humphreys 1998: rozdział 5).

\textit{Sutra Platformy} podaje, że Huineng przebywał w Caoqi w sumie przez 40 lat, nauczając ludzi z Shaozhou i Kantonu w oparciu o \textit{Sutrę Diamentową}, a jako symbolu przekazu nauk używał \textit{Sutry Platformy}. Huineng miał wiele tysięcy uczniów, z których dziesięciu zostało mistrzami o regionalnym zasięgu działalności. W 712 roku powrócił do Xinzhou, miejsca swych narodzin, a w 713 roku zmarł w wieku 76 lat. Tuż przed jego śmiercią, jego uczeń Fahai, uważany za autora sutry, zapytał, kto będzie jego następcą i co stanie się z szatą patriarchy. Huineng odrzekł, że przekaz szaty dobiegł końca, i sugeruje, że w przyszłości pojawi się uczeń o imieniu Shenhui. W momencie jego śmierci pojawiło się wiele pomyślnych znaków. Huineng miał zostać pochowany w Caoqi, a Wei Ju (韋據 \pinyin{Wéi Jù}), prefekt, który wysłuchał nauk zawartych w dalszej części tekstu, napisał ku jego czci inskrypcję, którą następnie zniszczyli przedstawiciele Północnej Szkoły (Huineng, Schlütter i Teiser 2012: 31, 34).

\if 0
Schlütter 31
Wang Wei 王維 napisał epitafium dla Huinenga, w którym pojawia się postać mnicha Yinzonga, który miał ostrzyc głowę Huinengowi
nie atakuje Północnego Chan
Matka Wang Wei była wyznawczynią Puji
Huineng nie był członkiem szlachetnego rodu

romans: quest (podróż), konkurs, walka o życie, w której Huineng w końcu wygrywa, bo jest lepszy

\fi

\section{Analiza biografii Huinenga zawartej w tekście \textit{Sutry Platformy} z Dunhuang}
Obecnie uważa się, że biografia Huinenga, opisana w \textit{Sutrze Platformy} w formie monologu, nie jest autentyczną autobiografią, a jedynie tekstem hagiograficznym, mającym przedstawić go jako człowieka świętego i bohatera. Tekst wysuwa twierdzenia, jakoby Huineng miał być prawowitym spadkobiercą Piątego Patriarchy Hongrena, szóstym dzierżawcą przekazu Chan, pochodzącego w prostej, nieprzerwanej linii od samego historycznego Buddy. Pierwszą osobą, która przedstawiła Huinenga jako świętego, był Shenhui (菏澤神會 \pinyin{Hézé Shénhuì}, 684-758). Biografia Szóstego Patriarchy w takiej wersji, jak opisana w tekście \textit{Sutry Platformy} z Dunhuang, jest najprawdopodobniej uzupełnioną i zmienioną wersją historii. Znamienne jest to, że w tekście sutry nie pojawia się imię Shenhui, zaś w pismach Shenhui nie było żadnej wzmianki o konkursie poezji, ani o dwóch wierszach, co świadczy o tym, że zostały one napisane po jego śmierci (Huineng, Schlütter i Teiser 2012: 25-26; McRae 2004: 63). % ***

Shenhui w 732 roku zaczął promować Huinenga jako Szóstego Patriarchę linii Chan, i jednocześnie atakował uczniów i spadkobierców Shenxiu, szczególnie Puji (嵩山普寂 \pinyin{Sōngshān Pǔjì}, 651-739), który rościł sobie prawa do tytułu Siódmego Patriarchy. Shenhui twierdził, że Szkoła Północna, której przewodzili Shenxiu i Puji, nie była autentyczna, gdyż propagowała nauki stopniowej ścieżki. Prawdziwe, ponadczasowe nauki buddy, to jest nauki o nagłym oświeceniu, znane również jako subityzm, miały być przekazywane w południowym Chan. (Huineng, Schlütter i Teiser 2012: 32-33; Huineng i Yampolsky 2012: 26, 28).

Historyczny Huineng był postacią stosunkowo mało znaną i prawdopodobnie dlatego Shenhui wybrał go jako bohatera swoich historii. Ponieważ niewiele było wiadomo o jego prawdziwych naukach, Shenhui mógł przypisać mu dowolne nauki niestojące w sprzeczności z naukami o nagłym oświeceniu. Prawdopodobnie był on mistrzem medytacji nauczającym o nagłym oświeceniu, ale wbrew temu, co możemy przeczytać w \textit{Sutrze platformy}, w jego czasach nie było to w istocie niczym szczególnym. Mimo że pochodził z południa, obszaru oddalonego od serca chińskiej cywilizacji, raczej nie wchodził w konflikty z innymi mistrzami medytacji, a wręcz miał z nimi dobre relacje (Huineng i McRae 2000: xv).

W późniejszym okresie historia życia Huinenga została uzupełniona o opowieść o konkursie poezji, w którym Huineng miał pokonać Shenxiu, jednoznacznie dowodząc wyższości subityzmu Szkoły Południowej nad stopniową ścieżką Szkoły Północnej. Biorąc pod uwagę, że Huineng został w tekście przedstawiony jako ubogi, niepiśmienny człowiek, jest niezwykle mało prawdopodobne, by był w stanie ułożyć przypisywany mu wiersz w klasycznym, literackim języku chińskim. Z drugiej strony, biorąc pod uwagę, że Huineng wywodził się z rodziny urzędnika, nawet popadłego w niełaskę władz i skazanego na banicję, wydaje się nieprawdopodobne, że Huineng mógłby nie otrzymać żadnego wykształcenia. W dziełach Shenhui pojawiały się również twierdzenia, jakoby Puji wysłał swojego ucznia, niejakiego Zhang Xingchang (張行昌 \pinyin{Zhāng Xíngchāng}), do Shaozhou, z poleceniem ucięcia głowy zwłokom Huinenga, a także jakoby inny uczeń Puji, imieniem Wu Pingyi (武平一 \pinyin{Wǔ Píngyī}), wymazał inskrypcję na steli poświęconej Huinengowi i wstawił tam własną, podającą Shenxiu jako prawowitego Szóstego Patriarchę. Ataki Shenhui na Szkołę Północną zostały spisane przez Dugu Pei (獨孤沛 \pinyin{Dúgū Pèi}) w dziele zwanym \textit{Putidamo Nanzong ding shifei lun} (菩提達摩南宗定是非論 \pinyin{Pútídámó Nánzōng dìng shìfēi lùn}). Doktor Hu Shi (胡適 \pinyin{Hú Shì}, 1891-1962) zebrał odkryte w Dunhuang dzieła Shenhui i jego uczniów i opisał je w pracy pt. \textit{Shenhui heshang yiji} (社會和尚遺集 \pinyin{Shénhuì héshàng yíjí}) % (\textit{Ibidem}). 
(Huineng, Schlütter i Teiser 2012: 32-33; Huineng i Yampolsky 2012: 26, 28; Huineng i McRae 2000: xv).

Kim był Shenxiu i czym zasłużył sobie na osobiste ataki Shenhui? [Huineng i Yampolsky 2012] podaje, że na przełomie VII i VIII w. Shenxiu był uważany za jednego z najbardziej znanych i najwybitniejszych mistrzów Chan, a historia jego życia jest szczogólnie dobrze znana. Jego stosunkowo rzetelna biografia została zapisana w dziele \textit{Chuan fabao ji} w pozbawiony elementów fantastycznych i legend sposób. O ile we wszystkich innych dziełach z tego okresu jest wymieniony jako uczeń Hongrena, jedynie \textit{Chuan fabao ji} podaje, że był uczniem Faru, a ten uczniem Hongrena. Według tej biografii pochodził z miasta Daliang (大梁 \toponim{Dàliáng}), obecnie Kaifeng (開封 \toponim{Kāifēng}) w prowincji Henan, i był członkiem rodu Li (Yampolsky 15-16). % 李

Był ponadprzeciętnie uzdolnionym dzieckiem, a w wieku 13 lat, w związku z zawirowaniami historycznymi i związaną z nimi klęską głodu, postanowił wyrzec się światowego życia i zostać mnichem buddyjskim. Następnie wędrował od jednej świątyni do drugiej, by wreszcie otrzymać pełne ślubowania mnisie w wieku 20 lat. W wieku 46 lat udał się do Hongrena, a ten natychmiast poznał się na jego talencie. Po wielu latach studiowania nauk osiągnął ostateczne oświecenie, a następnie udał się do Jingzhou (荊州 \pinyin{Jīngzhōu}) w prowincji Hubei, ale przez cały czas swojego pobytu tam nie udzielał nauk. Za panowania cesarza Tang Gaozonga (儀鳳 ***) udał się do świątyni Yuquan (玉泉寺 \toponim{Yùquán sì}) w pobliżu obecnego miasta Dangyang (當陽 \toponim{Dāngyáng}) w prowincji Hubei, również nie udzielając nauk. Po śmierci jego mistrza, Faru, zaczęli do niego przybywać uczniowie z dalekich stron, a wówczas Shenxiu zaczął nauczać Dharmy, przynosząc pożytek wielu istotom i prowadząc je do wyzwolenia (\textit{Ibidem}). % (Yampolsky 15-16)

Między rokiem 730 i początkiem lat 50. VIII w. Shenhui stworzył historię życia Huinenga zbliżoną do zawartej w \textit{Sutrze Platformy}, z tą różnicą, że nie było w niej jeszcze wzmianki o konkursie poezji między Shenxiu i Huinengiem. Wedle obecnego stanu wiedzy Shenhui nie posiadał niemal żadnych prawdziwych informacji na temat postaci Huinenga, oprócz tego, że był uczniem Hongrena, żył w Shaozhou i że przez niektórych praktykujących uważany był za nauczyciela o regionalnym zasięgu działalności. W najwcześniejszych pismach przypisywanych Shenhui było napisane jedynie, że Huineng był Szóstym Patriarchą, i że otrzymał szatę, insygnium patriarchatu i przekazu nauk, od Hongrena, Piątego Patriarchy. W \textit{Sutrze Platformy} Shenhui został wymienony jako ostatni z dziesięciu uczniów Huinenga, a także jedyny, który nie płakał, gdy Huineng poinformował ich o zbliżającej się śmierci.
%Częste odniesienia do jego postaci wskazują, że \textit{Sutra Platformy} powstała najprawdopodobniej wkrótce po jego śmierci 
(Huineng, Schlütter i Teiser 2012: 33).

Współcześni badacze uważają, że niemożliwe jest, żeby wydarzenia przedstawione w biografii Huinenga kiedykolwiek miały miejsce, i należy je traktować jedynie jako ciekawą anegdotę o wyraźnym podtekście duchowym. Powodem tego jest fakt, że Shenxiu był uczniem Hongrena jedynie przez kilka lat w początkowym etapie jego działalności, a więc kiedy nie istniał jeszcze problem wyboru jego następcy. Shenxiu i Huineng nie przebywali w klasztorze Hongrena jednocześnie, a więc nie mogli współzawodniczyć w konkursie poezji. Po drugie, w owym okresie nie istniała jeszcze koncepcja jedynego prawowitego patriarchy --- pojawiła się ona dopiero w dziełach Shenhui. Ponadto historia życia Huinenga w zachowanej do dnia dzisiejszego wersji nie pojawia się w pismach Shenhui, a jako propagator Huinenga w roli Szóstego Patriarchy na pewno zapisałby tę historię, gdyby była mu znana. Jednak jeszcze w latach 30. VIII w. dzieje życia Huinenga były mu w dużej mierze obce. (McRae 2004: 67; Huineng i McRae 2000: xv).

Ważnym aspektem nauk Shenhui, przypisywanych Huinengowi, jest odejście od sutry \textit{La\.nkā\-vatāra} na rzecz \textit{Sutry Diamentowej}. Odzwierciedla ono wzrost popularności tej sutry w VIII w. O tym, jak wielką rolę odgrywała ona dla Shenhui, świadczą liczne odniesienia do niej w tekście \textit{Sutry Platformy}. W tekście jest wysunięte również twierdzenie, jakoby to \textit{Sutra Diamentowa}, a nie \textit{La\.nkāvatāra}, była podstawą nauk przekazywanych przez patriarchów, od Bodhidharmy do Huinenga. Inne teksty na temat linii przekazu Chan, takie jak \textit{Kontynuowane biografie wybitnych mnichów} (續高僧傳 \pinyin{Xù gāosēng zhuàn}), \textit{Chuan fabao ji} (傳法寶紀 \pinyin{Chuán fǎbǎo jì}, `Annały przekazu skarbu Dharmy') i \textit{Zapisy mistrzów i uczniów w przekazie Sutry La\.nkāvatāra} (楞伽師資記 \pinyin{Léngqié shīzī jì}) zaprzeczają tym twierdzeniom. Symboliczne odejście od sutry \textit{La\.nkāvatāra} jest też zaznaczone w biografii Huinenga w \textit{Sutrze Platformy}; gdy Piąty Patriarcha Hongren ujrzał wiersz Shenxiu, napisany na ścianie w miejscu, gdzie miały być namalowane sceny z sutry \textit{La\.nkāvatāra}, patriarcha zrezygnował z wykonania malowidła, uznając, że wiersz Shenxiu jest od nich ważniejszy, i zapłacił wezwanego w tym celu malarzowi, chociaż ten nie wykonał swojego zadania (Huineng i Yampolsky 2012: 34; McRae 2004: ***).

W dziełach Shenhui pojawiły się też dwie opowieści, powielone w późniejszych dziełach. Pierwsza z nich, zapisana zarówno w \textit{Sutrze Platformy}, jak i \textit{Putidamo Nanzong ding shifei lun}, dotyczyła Bodhidharmy i cesarza Liang Wudi (梁武帝 \pinyin{Liáng Wǔdì}). Według tej historii, kiedy Bodhidharma przybył do stolicy Liang (***), przeprowadził dyskusję z cesarzem. Cesarz miał spytać Bodhidharmy, czy budując świątynie, dając ofiary mnichom i ludziom w potrzebie, zebrał zasługę\fnm. Mistrz odparł: ,,Nie zebrałeś żadnej zasługi.'' Miał przez to na myśli, że cesarz, nie podążając za właściwą ścieżką, szukał jedynie błogosławieństw, a nie prawdziwej zasługi. Cesarz, nie rozumiejąc tej nauki, był nią rozczarowany i wygnał Bodhidharmę ze swego państwa. Następnie mistrz udał się do państwa Wei (Huineng i Yampolsky 2012: 27, 155-156).
\if 0
T48n2007_p0341a24(10)║朕一生未來造寺布施供養有有功德否。達磨答言。並無功德。
T48n2007_p0341a25(03)║武帝惆悵遂遣。達磨出境。未審此言。請和尚說。
a w Wei spotkał Huike
\fi
\footnotetext{Zasługa 功德}

Inna, nieco drastyczna opowieść, propagowana przez Shenhui, dotyczy Bodhidharmy i jego ucznia i spadkobiercy, Huike. Według tej historii, kiedy mistrz i uczeń spotkali się po raz pierwszy, Huike był zdesperowany, by zostać uczniem Bodhidharmy, lecz ten nie chciał go przyjąć. Mistrz miał ustąpić Huike dopiero wówczas, gdy ten w dowód swej determinacji dobył miecza i demonstracyjnie uciął swoje lewe ramię. Huike został następnie głównym uczniem Bodhidharmy i odziedziczył po nim szatę, symbol przekazu Dharmy. Później tę samą szatę mieli otrzymać kolejni patriarchowie: Sengcan, Daoxin, Hongren, aż do Huinenga. Przy pomocy tej opowieści Shenhui osiągnął dwa cele: nie tylko ustanowił szatę Bodhidharmy jako insygnium prawowitego patriarchy Chan, lecz również podważył uznaną dotychczas linię przekazu, wiodącą od Bodhidharmy do Shenxiu. Pochodzenie tych legend nie jest znane. Hu Shi uważa, że zostały one wymyślone przez Shenhui, ale równie prawdopodobne jest, że krążyły wśród ludu, a Shenhui jedynie zapisał je i wykorzystał do swoich celów (Huineng i Yampolsky 2012: 27).

\if 0
Yampolsky
Nawet jeżeli istniał Bodhidharma, linia przekazu Chan raczej nie istniała w nieprzerwanej formie
Chan jako taki wymyślono po czasach Hongrena
biografia Huinenga ma poprzeć roszczenia
Schlütter 32
\fi

\subsection{Wiersze Shenxiu i Huinenga}
Tradycyjna interpretacja wierszy Shenxiu i Huinenga, zawartych w tekście \textit{Sutry Platformy} (patrz: strony \pageref{ShenxiuVerse} i \pageref{HuinengVerse}), według filozofa Zongmi (圭峰宗密 \pinyin{Guīfēng Zōngmì}, 780-841), jest prosta. Wiersz Shenxiu ma symbolizować stopniową ścieżkę, zaś wiersz Huinenga --- ścieżkę nagłego, ostatecznego oświecenienia, które wydarza się w jednej chwili. Tym samym, w rozumieniu Zongmi wiersze te reprezentują pogląd dwóch konkurencyjnych tradycji Chan, północnej i południowej (McRae 2004: 63).

Takie rozumienie jest jednak nadmiernym uproszczeniem. Wiersz przypisywany Shenxiu odnosi się nie tyle do stopniowej ścieżki, ile do ciągłej, bezustannej praktyki oczyszczania zwierciadła z kurzu. Wiersz Huinenga nie opisuje natomiast poglądu nagłej ścieżki, a jedynie neguje twierdzenia, zawarte w wierszu Shenxiu. Oprócz tego, wiersze przedstawiają dwa punkty widzenia na ten sam temat i nie mogą być interpretowane osobno. Zwłaszcza wiersz Huinenga odwołuje się do wiersza Shenxiu (McRae 2004: 63-64).

[Jak niepiśmienny Huineng mógł ułożyć wiersz w klasycznym chińskim?]

\if 0
McRae xvi
Shenxiu i Huineng nie przebywali w Huangmei w tym samym czasie

Tekst jest nie tyle biografią, ile hagiografią
Oparta na naukach Shenhui

McRae 62-64:
Wiersz Shenxiu:
Nie ma dowodu, żeby kiedykolwiek go napisał, albo żeby porównał umysł do jasnej lustrzanej podstawy, ale z innych jego dzieł wynika, że mógł napisać coś w tym stylu
constant and perfect teaching, the endless personal manifestation of the
bodhisattva ideal
McRae, Northern School, 235. The English “suchlike'' renders the word 如, as in the Chinese translation of Tathágata, 如來, when used as a modifier.
神秀觀心論
T85n2833_p1271c15(08)║眾生修伽藍鑄形像燒香散花然長明燈。
T85n2833_p1271c16(09)║晝夜六時遶塔行道持齋禮拜。種種功德皆成佛道。
T85n2833_p1271c17(06)║若唯觀心總攝諸行。如是事應妄也 答曰。
T85n2833_p1271c18(06)║佛所說無量方便。一切眾生鈍根狹劣。
T85n2833_p1271c19(08)║甚深所以假有為喻無為。若不內行唯只外求。希望獲福。
T85n2833_p1271c20(03)║無有是處。言伽藍者。西國梵音。
T85n2833_p1271c21(07)║此地翻為清淨處地。若永除三毒常淨六根。
T85n2833_p1271c22(07)║身心湛然內外清淨。是名為修伽藍也。又鑄形像者。
ciało jest drzewem bodhi, oba są fizyczne
ciało jest miejscem, w którym człowiek osiąga oświecenie
lustrzana podstawa

Wiersz Shenxiu był głęboki i wyrafinowany, bo tym sposobem wiersz Huinenga jest lepszy od czegoś wybitnego; być lepszym od czegoś miernego to żadne osiągnięcie
Wiersz Huinenga wcale nie wyraża poglądów szkoły nagłego oświecenia, a jedynie zaprzecza poglądom wiersza Shenxiu
Więc nie jest to wykładnia szkół nagłego i stopniowego oświecenia
Wiersz Huinenga pojawiał się w kilku wersjach w różnych tekstach
Wiersze oparte na pismach Szkoły Północnej (McRae 67)

Przejście od Lankavatara do Diamentowej Sutry, w VIII w. ta właśnie sutra nabierała popularności, na niekorzyść Lankavatary
\fi
\subsection{Analogie do biografii Konfucjusza}
Pisząc historię życia Huinenga, Shenhui w oczywisty sposób czerpał z legendy o Konfucjuszu, opisanej w \textit{Zapiskach historyka} (史記 \pinyin{Shǐjì}) autorstwa Sima Qian (司馬遷 \pinyin{Sīmǎ Qiān}) zwanego ,,Wielkim Historykiem''. Tekst ten miał wówczas ugruntowaną pozycję wśród chińskich elit, jako że Konfucjusz był twórcą głównego systemu filozoficznego w państwie, i był znany niemal wszystkim, zaś postać Konfucjusza była dla współczesnych uniwersalnym wzorcem cnót (Huineng, Schlütter i Teiser 2012: 36).

Według tej legendy, Konfucjusz urodził się z mezaliansu mężczyzny z wybitnego rodu, pochodzącego z innego rejonu, i lokalnej kobiety. Ojciec Konfucjusza osierocił go, gdy ten był jeszcze dzieckiem, w rezultacie czego Konfucjusz i jego matka wiedli życie w ubóstwie. Konfucjusz nie cieszył się szczególną popularnością wśród lokalnej elity państwa Lu, przez pewien czas pracował jako urzędnik niskiego szczebla, lecz wkrótce został z niej zwolniony, mimo że wykazywał ponadprzeciętne uzdolnienia (Huineng, Schlütter i Teiser 2012: 36-37).

Shenhui był wykształcony w zakresie filozofii konfucjańskiej, a nawet został przez Zongmi porównany do Konfucjusza. Zanning (贊寧 \pinyin{Zàn Níng}), autor \textit{Biografii wybitnych mnichów Song} (宋高僧傳 \pinyin{Sòng gāosēng zhuàn}) porównał go do Yan Hui (顏回 \pinyin{Yán Huí}), ulubionego ucznia Konfucjusza.
Symboliczne znaczenie miało również liczba patriarchów w linii przekazu Chan. Według 
\if 0
Southern Learning
siedem świątyń lub izb w świątyni przodków było zarezerwowane dla cesarzy
宗
southern learning
Schlütter 37 \fi
(Huineng, Schlütter i Teiser 2012: 37).

Z punktu widzenia nauk Konfucjusza, Huineng był postacią pozytywną. Mimo iż wywodził się z nizin społecznych i wychował się w rejonie tak bardzo oddalonym od centrum cywilizacji chińskiej, jak to było możliwe, to posiadał jedną z najwyższych cnót konfucjańskich --- \textit{nabożność synowską} (子孝 {zǐxiào}). Po śmierci ojca ciężko pracował, utrzymywał starą matkę i opiekował się nią. Taki obraz postaci Huinenga stał w opozycji do wykształconych, bogatych elit, z których wywodziła się w owym okresie większość mnichów buddyjskich, i był najprawdopodobniej zainspirowany historią życia Hongrena, który według niektórych podań medytował za dnia, zaś w nocy zajmował się bydłem. Sam Shenxiu, rzekomy konkurent Huinenga do pozycji patriachy, pochodził ze szlachetnego rodu i był wykształcony zarówno w literaturze buddyjskiej, jak i świeckiej, a niektórzy podejrzewają nawet, że mógł być związany z rodem cesarskim (McRae 2004: 68).

Celem takiego przedstawienia postaci Huinenga było pokazanie, że każdy, nawet osoba świecka, niezależnie od pozycji społecznej, miejsca pochodzenia i wykształcenia mógł osiągnąć oświecenie i zostać patriarchą. W szerszym rozumieniu, jeżeli osiągnięcie najwyższego oświecenia było jedynym i koniecznym warunkiem zostania mianowanym patriachą szkoły Chan, to można było oczekiwać, że wszyscy jego poprzednicy i następcy również byli oświeceni. W ten sposób szkoła Chan zyskiwała autorytet i wiarygodność (McRae 2004: 69).


\subsection{Analogia do dzieł szkoły Niutou}
McRae 65


\chapter{Analiza tekstu \textit{Sutry Platformy}}

Sutra Platformy wprowadziła nauki o nagłym oświeceniu, jednak podział na pojęcia ,,ścieżki nagłego oświecenia'' i ,,stopniowej ścieżki'' są raczej pozorne, gdyż tak naprawdę chodzi tu o indywidualne zdolności uczniów --- inteligentniejsi, z otwartymi umysłami, są w stanie pojąć nauki o pustości i naturze buddy, i osiągnąć oświecenie w jednej chwili, podczas gdy inni muszą ćwiczyć się na owej ścieżce stopniowo.

W czwartym rozdziale sutry, Huineng nauczał, że za obiekt praktyki duchowej należy przyjąć ,,brak idei'', za jej podstawę przyjąć ,,brak obiektu'', zaś jej fundamentalną zasadą należy uczynić ,,brak przywiązania''. Są to trzy zbliżone i nierozerwalnie związane koncepcje.

,,Brak idei'' jest rozumiany jako wolność od rozproszenia --- pilnowanie, by umysł nie podążał za myślami i aby nic, co pojawia się w umyśle nie odwodziło go od praktyki. W przeciwnym razie, jeżeli praktykujący poświęca czas i energię swoim myślom o teraźniejszości, przeszłości i przyszłości, zaczną one pojawiać się, jedna po drugiej, i ograniczać przejrzystość umysłu. Błędem jest również próba całkowitego pozbycia się myśli; taka praktyka nie umożliwia rozpoznania natury umysłu i nie prowadzi do wyzwolenia. Właściwą praktyką jest koncentracja na prawdziwej naturze Takości (真如 \pinyin{zhēnrú}, skt. \textit{tathātā}), gdyż „Takość jest esencją idei, a idea jest wynikiem aktywności Takości”.

,,Brak obiektu'' oznacza tu unikanie rozproszenia pod wpływem zewnętrznych obiektów. ,,Brak przywiązania'' zaś oznacza traktowanie wszystkich istot, zarówno wrogów, jak też przyjaciół, w taki sam sposób. Praktykujący powinien porzucić myślenie o przeszłości i chęć odwetu za dawne krzywdy (Huineng, Wong i Humphreys 1998: rozdział 4).

Rozdział piąty Sutry Platformy traktuje o \textit{dhyāna}, medytacji. W medytacji Chan nie należy koncentrować się ani na umyśle, ani na czystości. Umysł jako taki jest zwodniczy, jest jedynie iluzją i jako taki nie powinien być obiektem medytacji. Koncentracja na czystości zaś prowadzi do fiksacji na koncepcji czystości. Właściwa medytacja oznacza urzeczywistnienie niewzruszonej esencji umysłu.

Taka praktyka powinna być zrównoważona na poziomie ciała, mowy i umysłu. Praktykujący, który pragnie rozwinąć nieporuszoność, powinien być obojętny na wady innych ludzi. Niewzruszony umysł nie działa w dualistycznych kategoriach, takich jak dobro i zło albo słabość i siła. Analogicznie, praktykujący nie powinien mówić krytykować innych ludzi przy pomocy tych kategorii myślowych (Huineng, Wong i Humphreys 1998: rozdział 5).
%dwell on --- skupiać się na

\if 0
McRae 66
Natura buddy przesłonięta wyłącznie zaciemnieniami umysłu


\fi

\section*{Bibliografia}

Huineng, Mou-lam Wong i Christmas Humphreys. 1973. \textit{The sutra of Wei Lang (or Hui Neng)}. Westport, Conn: Hyperion Press. \url{http://www.sinc.sunysb.edu/Clubs/buddhism/huineng/content.html}

Huineng i Hsüan Hua. 1977. \textit{The Sixth Patriarch's Dharma jewel platform sutra, with the commentary of Tripitaka Master Hua} [translated from the Chinese by the Buddhist Text Translation Society]. San Francisco: Sino-American Buddhist Association. \url{http://www.cttbusa.org/6patriarch/6patriarch_contents.asp}

Huineng, \textit{Sutra Szóstego Patriarchy Zen}, tłumacz nieznany, \url{http://www.zen.ite.pl/teksty/sutra6.html}

William E. Soothill i Lewis Hodous. 2003. \textit{A Dictionary of Chinese Buddhist Terms.} RoutledgeCurzon.% \url{http://buddhistinformatics.ddbc.edu.tw/glossaries/files/soothill-hodous.ddbc.pdf}

Buswell, Robert E. 2004. \textit{Encyclopedia of Buddhism.} New York: Macmillan Reference.

McRae, John R. 2004. \textit{Seeing through Zen encounter, transformation, and genealogy in Chinese Chan Buddhism.} Berkeley, Calif: University of California Press.

Anonim 佚名 2007. ``Fojiao de futian'' 佛教的福田 [Pola błogosławieństw w buddyzmie]. \textit{Zhongguo minzu bao} 中國民族報, za: \url{http://www.wuys.com/news/Article_Show.asp?ArticleID=12791}

Nydahl, Ole. 2010. ``Sześć wyzwalających działań''. \textit{Diamentowa Droga} 34. \url{http://diamentowadroga.pl/dd34/szesc_wyzwalajacych_dzialan}

Huineng, Morten Schlütter i Stephen F. Teiser. 2012. \textit{Readings of the Platform sūtra.} New York: Columbia University Press.% \url{http://site.ebrary.com/id/10538320}.

Huineng, Philip B. Yampolsky i Huineng. 2012. \textit{The Platform sutra of the Sixth Patriarch the text of the Tun-huang manuscript.} New York: Columbia University Press.% \url{http://public.eblib.com/choice/publicfullrecord.aspx?p=909420.}

\end{document}
