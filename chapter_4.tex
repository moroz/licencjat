\chapter{Historyczne reperkusje \textit{Sutry Platformy}}
\label{ch:chapter_four}
\textit{Sutra Platformy} pozostaje jednym z najbardziej popularnych i wpływowych tekstów buddyzmu chan do dnia dzisiejszego. Jej ukończenie stanowi punkt kulminacyjny w historii szkoły, a analizie zawartych w niej nauk i opowieści poświęcono wiele prac naukowych.

\section{Podział chan na Szkołę Północną i Południową}
Podział buddyzmu chan na Szkołę Południową i Północną związany jest z Heze Shenhui.
Początkowo nieznany, zaczął pozyskiwać wpływy, gdy jego rywale stracili na popularności w następstwie publicznych ataków na Yuquan Shenxiu oraz jego uczniów, Songshan Puji i Dazhi Yifu.
Zostały one szczegółowo opisane przez ucznia Shenhui, Dugu Pei, w dziele zwanym \textit{Putidamo Nanzong ding shifei lun}. % (菩提達摩南宗定是非論 \pinyin{Pútídámó Nánzōng dìng shìfēi lùn}).
Jak podaje ów tekst, w roku 732 Shenhui zorganizował otwartą konferencję buddyjską w świątyni Dayun Si, w miejscu zwanym Huatai, w obecnej prowincji Henan.\label{Huatai}
Wysunął wówczas twierdzenia, jakoby Szkoła Północna, której przewodzili Shenxiu i Puji, nie była autentyczna, gdyż propagowała nauki stopniowej ścieżki.
% Prawdziwe, ponadczasowe nauki buddy, tzn. nauki o nagłym oświeceniu, znane również jako subityzm, miały być przekazywane w południowym chan.
W tekście opisano dyskusję Shenhui z mistrzem Chongyuan (崇遠 \nazwisko{Chóngyuǎn}), znanym specjalistą od sutr buddyjskich i zwycięzcą wielu debat filozoficznych.
Na konferencji w Huatai zwyciężył Shenhui.
W późniejszym czasie Szkoła Południowa używała tego faktu na dowód swej wyższości nad rywalami.
Była to jednak manipulacja, gdyż w rzeczywistości Chongyuan nie był reprezentantem Szkoły Północnej.
W owym czasie Shenxiu i jego szkoła cieszyli się poparciem dworów w obu stolicach, w Luoyangu i w Chang'an\footnote{長安 \toponim{Cháng'an}, obecnie miasto Xi'an 西安 \toponim{Xī'ān} w prowincji Shaanxi 陝西 \toponim{Shǎnxī}.} oraz w dorzeczu Rzeki Żółtej (黃河 \toponim{Huáng Hé}).
Shenhui zorganizował konferencję w Huatai nie tylko po to, by pogrążyć swoich przeciwników, lecz również by przeciwdziałać podziałom w szkole chan.
W owym czasie istniało bowiem wielu samozwańczych mistrzów medytacji, którzy nauczali Dharmy i zakładali własne szkoły, przeinaczając prawdziwą istotę nauk Buddy
(Shi 2008: 191, 200).

W dziełach Shenhui pojawiły się twierdzenia, jakoby Puji wysłał swojego ucznia, niejakiego Zhang Xingchanga (張行昌), do Shaozhou, z poleceniem ucięcia głowy zwłokom Huinenga.
Oskarżył innego ucznia Puji, imieniem Wu Pingyi (武平一), o wymazanie inskrypcji na steli poświęconej Huinengowi i wstawienie na jej miejsce własnej, podającej Shenxiu jako prawowitego Szóstego Patriarchę.
Zarzucił Puji, że w dziele ``Annały przekazu skarbu Dharmy'' spisał historię linii przekazu chan z pominięciem Huinenga.
Stwierdzenie to było o tyle bezpodstawne, że Puji nie był autorem wspomnianego dzieła
(Huineng i Yampolsky 2012: 28).

Na konferencji w Huatai Shenhui zarzucał przeciwnikom, że uzurpowali sobie prawo do linii przekazu patriarchatu, oraz że przeinaczyli istotę buddyzmu, nauczając błędnej, stopniowej ścieżki.
W rzeczywistości, Shenxiu opierał swoje nauki na tych samych podstawowych ideach buddyzmu mahajany, co Shenhui.
W obu szkołach ostatecznym celem praktyki duchowej było osiągnięcie oświecenia poprzez rozpoznanie prawdziwej natury umysłu.
W obydwu przebudzenie było postrzegane jako krótki przebłysk wglądu.
W doktrynie Szkoły Północnej, opartej głównie na sutrze \textit{La\.nkā\-vatāra}, kładziono szczególny nacisk na praktyki przygotowawcze.
Ich zadaniem było oczyszczenie umysłu z zaciemnień, postrzeganych w tej tradycji jako istniejące i rzeczywiste.
Natomiast Szkoła Południowa głosiła pustość wszystkich myśli i zjawisk.
Ideą praktyki w tej doktrynie było nie tyle pozbyć się trucizn umysłu, ile zrozumieć, że w rzeczywistości nigdy nie istniały.
W \textit{Sutrze Platformy} dychotomię tę ilustrują ostatnie dwa wersy wiersza Huinenga: ,,W rzeczywistości nie ma niczego, / Cóż miałoby przyciągać jakikolwiek kurz?'' % NIE DYCHOTOMIĘ
Nauki Shenxiu obejmowały o wiele większą część tradycji medytacyjnej mahajany, podczas gdy Shenhui bardziej radykalnie promował doktrynę nagłego oświecenia, traktując ją jako jedyne kryterium decydujące o tym, która ze szkół reprezentowała ortodoksyjny buddyzm.
(Dumoulin 1963: 81, 84-87). % sprawdzić

Kim był Shenxiu i czym zasłużył sobie na osobiste ataki Shenhui?
Yampolsky (2012: 15-16) podaje, że na przełomie VII i VIII w. Shenxiu uważano za jednego z najbardziej znaczących i najwybitniejszych mistrzów chan.
Stosunkowo rzetelna wersja jego biografii została zapisana w dziele ``Annały przekazu skarbu Dharmy'' w pozbawiony elementów fantastycznych sposób.
O ile we wszystkich innych dziełach z tego okresu jest wymieniony jako uczeń Hongrena, ``Annały'' podają, że był uczniem Faru, a ten --- Hongrena.
Według tej biografii urodził się w mieście Daliang (大梁 \toponim{Dàliáng}), obecnie Kaifeng (開封 \toponim{Kāifēng}) w prowincji Henan (河南 \toponim{Hénán}) i pochodził z rodu Li.

Shenxiu już od najmłodszych lat wykazywał się ponadprzeciętnymi uzdolnieniami.
W wieku 13 lat, w związku z niekorzystną sytuacją polityczną i panującą klęską głodu, opuścił rodzinny dom z zamiarem zostania mnichem buddyjskim.
Później wędrował od świątyni do świątyni, by wreszcie jako dwudziestolatek otrzymać pełne ślubowania.
W wieku 46 lat udał się do Hongrena, a ten natychmiast poznał się na jego talencie.
Po wielu latach studiowania nauk osiągnął ostateczne oświecenie, a następnie przewędrował do Jingzhou (荊州 \toponim{Jīngzhōu}) w prowincji Hubei (湖北 \toponim{Húběi}).
Poczynając od ery Yifeng (儀鳳 \pinyin{Yífèng}), za panowania cesarza Tang Gaozonga (唐高宗 \nazwisko{Táng Gāozōng}), przebywał w świątyni Yuquan Si (玉泉寺 \pinyin{Yùquán sì}) w pobliżu obecnego miasta Dangyang (當陽 \toponim{Dāngyáng}) w prowincji Hubei.
Po śmierci swojego mistrza zaczął gromadzić wokół siebie uczniów, nauczając ich Dharmy
(Yampolsky 2012: 15-16).

Najważniejszy uczeń Shenxiu, Songshan Puji, pochodził z miejscowości Hedong (河東 \toponim{Hédōng}), obecnie położonej w powiecie Yongji (永濟縣 \toponim{Yǒngjì xiàn}) w prowincji Shanxi (山西 \toponim{Shānxī}).
Studia buddyzmu rozpoczął w mieście Daliang, kładąc szczególny nacisk na nauki Yogācāra%
\footnote{Yogācāra --- jedna z dwóch głównych szkół buddyzmu indyjskiego. Za jej założycieli uważa się żyjących na przełomie IV i V w. mistrzów Asa\.ngę\index{Asa\.nga} oraz Vasubandhu\index{Vasubandhu}. Nazwa szkoły pochodzi od tekstu pt. ``Stopnie Praktyki Jogicznej'' (skt. \textit{Yogācārabhūmi}, chiń. 瑜伽師地論 \pinyin{Yújiā shīdì lùn}). Jedną z najważniejszych idei tej tradycji jest ,,tylko umysł'' (skt. \textit{vijñaptimātra}, chiń. 唯識 \pinyin{wéishí}). Oznacza ona, że wszystko to, co jest postrzegane przez umysł, nie jest rzeczywistością, a jedynie wytworem świadomości i koncepcji (Buswell 2004: 914, 917).},
\textit{Sutrę Lotosu} oraz tekst pt. ``Przebudzenie wiary w mahajanie'' (大乘起信論 \pinyin{Dàchéng qǐ xìn lùn}).
W 688 roku otrzymał pełne święcenia mnisie.
Po latach pobierania nauk od różnych mistrzów, udał się do świątyni Shaolin Si (少林寺 \pinyin{Shǎolín Sì}) w poszukiwaniu mistrza Faru, jednak kiedy tam przybył, okazało się, że Faru już nie żył.
Następnie wyruszył do świątyni Yuquan Si, gdzie pod okiem Shenxiu zgłębiał przez pięć lat takie teksty, jak ``Sutra Siyi'' (思益梵天所問經 \pinyin{Sīyì fántiān suǒwèn jīng}, w skrócie 思益經 \pinyin{Sīyì jīng}) oraz sutra \textit{La\.nkā\-vatāra}.
Około roku 696 przeniósł się na górę Song (嵩山 \toponim{Sōngshān}).
Po śmierci Shenxiu zaproponowano mu objęcie stanowiska przywódcy jego szkoły.
Puji formalnie odrzucił tę propozycję, co nie przeszkadzało mu zgromadzić wokół siebie wielu nowych uczniów.
W roku 735 otrzymał zaproszenie do Chang'an na dwór cesarza Tang Xuanzonga, gdzie nauczał miejscowych arystokratów.
Puji zmarł w roku 739, a po jego śmierci wielu ludzi dołączało do nowej szkoły, zwanej Heze (荷澤宗 \pinyin{Hézé zōng}) od klasztoru, w którym rezydował wówczas Shenhui % Faure
(Huineng i Yampolsky 2012: 22; McRae 1986: 65-67; Fauré 1997: 89).

Tradycyjne stanowisko historyków chan na temat rozwoju Szkoły Północnej głosi, iż w następstwie ataków Shenhui oraz śmierci głównych spadkobierców Shenxiu, tj. Songshan Puji oraz Dazhi Yifu, liczba jej uczniów drastycznie spadła, natomiast Szkoła Północna rosła w siłę.
McRae (1986: 61) zaprzecza jednak tym twierdzeniom.

Fauré (1997: 91) podaje, że jeszcze w roku 758, a więc 26 lat po konferencji w Huatai i 19 lat po śmierci Puji, poeta Wang Wei --- ten sam, który na prośbę Shenhui wykonał stelę upamiętniającą Huinenga --- wystosował do cesarza pismo.
Dziękował w nim w imieniu mistrza zwanego \textit{ācārya} Shun (舜闍黎 \nazwisko{Shùn shélí}) za wykonanie inskrypcji na stupie mistrzów Shenxiu i Puji w świątyni Songyue Si (嵩嶽寺 \pinyin{Sōngyuè sì}) na górze Song.

W roku 753, urzędnik cesarski Lu Yi (盧弈), zwolennik Szkoły Północnej, oskarżył Shenhui o wichrzycielstwo i wnioskował o jego banicję.
Wygnanie lidera poważnie zagrażało dalszemu istnieniu szkoły Heze.
Sytuacja przybrała jednak korzystniejszy obrót już w roku 755, w związku z wybuchem rebelii An Lushana%
\footnote{Rebelia An Lushana (安史之亂 \pinyin{Ān shǐ zhī luàn}) --- krwawa rewolta przeciwko dynastii Tang, wywołana przez generała An Lushana (安祿山), który w roku 755 zajął obie stolice (Chang'an i Luoyang), a w następnym proklamował nową dynastię Yan (燕 \pinyin{Yān}). Chociaż An Lushan już w roku 757 zginął z rąk własnego syna, An Qingxu (安慶緒), polityczny chaos i działania zbrojne trwały aż do roku 763. Rebelia ta znacząco osłabiła pozycję dworu Tang, któremu przez następne 150 lat nie udało się odzyskać pełnej kontroli nad państwem, a zwłaszcza na rubieżach (Fairbank i Goldman 2006: 82-83).}, w wyniku której Lu Yi stracił życie.
W 757 roku Shenhui wezwano do Luoyangu, aby sprzedając święcenia mnisie, pomagał zbierać fundusze na finansowanie armii cesarskiej.
W tym samym czasie uczniowie Puji wzbraniali się przed tego typu praktykami o czysto politycznym charakterze.
Stało to w wyraźnym kontraście z oportunizmem Shenhui, a w ostatecznym rozrachunku przyczyniło się do późniejszego upadku Szkoły Północnej.
W nagrodę za zasługi dla państwa, nowy cesarz Tang Suzong	(唐肅宗 \nazwisko{Táng Sùzōng}) zatrudnił go do odprawiania rytuałów w kaplicy pałacu cesarskiego.
W 792 roku zgromadzenie mistrzów chan, zwołane przez cesarza Tang Dezonga (唐德宗 \nazwisko{Táng Dézōng}), pośmiertnie mianowało Shenhui Siódmym Patriarchą chan
% podżeganie, podjudzanie
(Fauré 1997: 89-90).

\if 0
Huineng i Yampolsky 2012: 22
W roku 735 został zaproszony na dwór przez cesarzową Wu Zetian. W stolicy cieszył się znaczącą popularnością, zarówno we dworze, jak i wśród prostego ludu.
Zmarł w roku 739, a na jego pogrzeb przybyło wielu ludzi, w tym urzędnicy cesarscy.

McRae 1986: 70
Szkoła Północna nie zaczęła tracić na sile w bezpośrednim następstwie ataków Shenhui, ale
\fi

\section{Dalszy podział Szkoły Południowej}
Spośród mistrzów uważanych za uczniów Huinenga, największy wpływ na dalszy rozwój szkoły chan mieli Qingyuan Xingsi (青原行思 \nazwisko{Qīngyuán Xíngsī}) oraz Nanyue Huairang. % (南嶽懷讓 \nazwisko{Nányuè Huáiràng}).
Biorąc pod uwagę, że żaden z nich nie został wspomniany w \textit{Sutrze platformy} w wersji z Dunhuang, spotkania tych mistrzów z Huinengiem prawdopodobnie nigdy nie miały miejsca i zostały dodane do kronik jedynie w celu nadania autentyczności nowym liniom przekazu.
Chociaż Xingsi i Huairang uważani byli przez współczesnych za wybitnych mistrzów, okres rozwoju ich szkół rozpoczął się dopiero w następnym pokoleniu, za czasów Mazu Daoyi (馬祖道一 \nazwisko{Mǎzǔ Dàoyī}) oraz Shitou Xiqian  (石頭希遷 \nazwisko{Shítóu Xīqiān}) % Jeżeli ta linia zostaje, to należy przenieść nawiasy
(McRae 2004: 82).

\section{Nanyue Huairang i szkoła Hongzhou}
Nanyue Huairang narodził się w drugim roku ery Yifeng (677) w miejscowości Ankang (安康 \toponim{Ānkāng}) w obecnej prowincji Shaanxi.
W wieku piętnastu lat opuścił rodzinny dom, a następnie uczył się Vinaya%
\footnote{Vinaya --- zbiór tekstów i nauk Buddy, dotyczących reguł postępowania dla mnichów buddyjskich (Buswell 2004: 885-886).}
u mistrza Yuquan Hongjing (玉泉弘景 \nazwisko{Yùquán Hóngjǐng}).
Nieusatysfakcjonowany jego naukami, niedługo później udał się na górę Song, gdzie znalazł swojego następnego mentora, Songshan Hui'an (嵩山惠安 \nazwisko{Sōngshān Huì'ān}).
W roku 699 udał się do Caoqi, gdzie przez kolejne dwanaście lat zgłębiał Dharmę pod okiem Szóstego Patriarchy.
Jak podaje zredagowana w 1252 roku kronika ``Kompendium Pięciu Lamp'' (五燈會元 \pinyin{Wǔdēng huìyuán}), Huineng miał go wówczas poinformować o przepowiedni indyjskiej mistrzyni \textit{dhjany} imieniem Prajñātārā (skt., chiń. 般若多羅 \nazwisko{Bōrěduōluó}), 27. patriarchy Indii oraz nauczycielki Bodhidharmy.
Proroctwo głosiło, że ,,spod stóp Huairanga wyjdzie koń, który zadepcze na śmierć wszystkich ludzi na tym świecie''.
Słowa te odnosiły się do Mazu Daoyi, ucznia Huairanga, który znacząco wpłynął na dalszy rozwój buddyzmu chan.
Daoyi miał bowiem na nazwisko Ma (馬 \nazwisko{Mǎ} `koń').
W roku 713 Huairang przybył do świątyni Bore Si (般若寺 \pinyin{Bōrě Sì}), na górze Heng (衡山 \toponim{Héngshān}) w prowincji Hunan (湖南 \toponim{Húnán}).
McRae (2004, str. 82) uważa, że Huairang prawdopodobnie nigdy nie spotkał Huinenga.
Jego głównymi uczniami byli Daojun (道峻 \nazwisko{Dàojùn}), Shenzhao (神照 \nazwisko{Shénzhào}) oraz Mazu Daoyi (馬祖道一 \nazwisko{Mǎzǔ Dàoyī}).
Nanyue Huairang zmarł na górze Heng w roku 744
(Ferguson 2011: 53-56).

Mazu Daoyi (707-786) przybył na górę Heng w roku 735.
Zasłynął z niekonwencjonalnych metod nauczania.
Był pierwszym mistrzem chan, który zastosował technikę zwaną \textit{katsu} (chiń., jap. 喝, Pinyin: \pinyin{hè}, Rōmaji: \textit{katsu}).
Polega ona na krzyczeniu na ucznia w celu przełamania jego sztywnych koncepcji.
W późniejszym okresie stosował ją mistrz Linji Yixuan (臨濟義玄 \nazwisko{Línjì Yìxuán}), założyciel szkoły Linji (臨濟宗 \pinyin{Línjì zōng}). % tutaj walniemy odnośnik
Podczas ery Dali (大曆 \pinyin{Dàlì}) Mazu Daoyi zamieszkiwał w świątyni Baohua Si (寶華寺 \pinyin{Bǎohuá Sì}) na górze Gonggong (龔公山 \toponim{Gōnggōng Shān}).
W czwartym roku tego okresu przeniósł się do świątyni Kaiyuan Si (開元寺 \pinyin{Kāiyuán Sì}) w rejonie Hongzhou\footnote{洪州 \toponim{Hóngzhōu}, obecnie położone w granicach administracyjnych miasta Nanchang (南昌 \toponim{Nánchāng}) w prowincji Jiangxi.}, gdzie nauczał aż do śmierci w roku 788.
Nazwa wywodzącej się od niego szkoły Hongzhou (洪州宗 \pinyin{Hóngzhōu zōng}) pochodzi właśnie od tego miejsca.
Mazu Daoyi miał bardzo wielu uczniów, z których najważniejsi dla historii Chanu byli Baizhang Huaihai (百丈懷海 \nazwisko{Bǎizhàng Huáihǎi}) oraz Nanquan Puyuan (南泉普願 \nazwisko{Nánquán Pǔyuàn})
(Dumoulin 1963: 97-98; Chang 1971: 148).

Baizhang Huaihai przekazał Dharmę Huangbo Xiyunowi (黃檗希運 \nazwisko{Huángbò Xīyùn}), a ten z kolei Linji Yixuanowi.
Yixuan zamieszkiwał świątynię Linji Si (臨濟寺 \pinyin{Línjì Sì}), obecnie położoną w powiecie Zhengding (正定縣 \toponim{Zhèngdìng xiàn}) w prowincji Hebei.
Zapoczątkowana przez niego szkoła Linji istnieje do dnia dzisiejszego.
W XII w. oddzieliły się od niej linie Yangqi (楊岐派 \pinyin{Yángqí pài}) oraz Huanglong (黃龍派 \pinyin{Huánglóng pài}).
Ich założycielami byli uczniowie żyjącego na przełomie IX i X wieku mistrza Shishuang Chuyuan (石霜楚圓 \nazwisko{Shíshuāng Chǔyuán}), Yangqi Fanghui (楊岐方會 \nazwisko{Yángqí Fānghuì}) i Huanglong Huinan (黃龍慧南 \nazwisko{Huánglóng Huìnán}).
W końcu XII w. japoński mnich Myōan Esai (jap. 明菴栄西 \nazwisko{Myōan Eisai}) studiował nauki linii Huanglong, a następnie przeniósł je do Japonii, dając początek buddyzmowi zen.
Przybywszy do miasta Hakata (jap. {\ipaexgothic 博多} \toponim{Hakata}), obecnie w granicach administracyjnych miasta Fukuoka (jap. {\ipaexgothic 福岡} \toponim{Fukuoka}), założył pierwszą świątynię zen w Japonii, Shōfuku-ji (jap. {\ipaexgothic 聖福寺} Rōmaji: \textit{Shōfuku-ji}).
Budowla ta zachowała się do dziś.
Linia przekazu pochodząca od Esai nazywana jest szkołą Rinzai%
\footnote{Jap. {\ipaexgothic 臨済宗} \textit{Rinzai-shū}; jest to japońskie odczytanie chińskiej nazwy szkoły Linji.}
(Ferguson 2011: 382; 401).

\section{Qingyuan Xingsi i szkoła Shitou}
Qingyuan Xingsi urodził się w roku 660 w miejscowości Ancheng (安城 \toponim{Ānchéng}), położonej w granicach obecnego powiatu Ji'an (吉安縣 \toponim{Jí'ān xiàn}) w prowincji Jiangxi.
Jak podaje Ferguson (2011, str. 56), Xingsi opuścił dom rodzinny w młodym wieku.
Przez pewien czas przebywał w Caoqi, gdzie pobierał nauki od Szóstego Patriarchy, który miał go wówczas uczynić dzierżawcą przekazu chan.
Później zamieszkał w świątyni Jingju Si (淨居寺 \pinyin{Jìngjū Sì}) na górze Qingyuan (青原山 \toponim{Qīngyuán shān}).
Kronika ``Biografie wybitnych mnichów Song'' podaje, że uczniowie z czterech stron świata przybywali tłumnie do jego świątyni pobierać nauki, jednak w chwili obecnej jedynym uczniem znanym z imienia jest Shitou Xiqian.
Zmarł w 28. roku ery Kaiyuan (開元 \pinyin{Kāiyuán}, 740)
(Shi 2008: 207-208).

Historia życia Shitou Xiqian została spisana w ``Księdze przekazu lampy z okresu Jingde''.
Według tego zapisu, Xiqian zmarł w 6. roku ery Zhenyuan (貞元 \pinyin{Zhēnyuán}, 790), przeżywszy 91 lat.
W związku z tym Shi (2008, str. 208) przyjmuje, że urodził się w 1. roku ery Jiushi (久視 \pinyin{Jiǔshì}, 700).
Pochodził z powiatu Gaoyao (高要縣 \toponim{Gāoyào xiàn}), położonego w granicach obecnej prowincji Guangdong.
W wieku 29 lat przyjął pełne święcenia mnisie.
W 742 roku przybył na górę Heng, gdzie założył świątynię na półce skalnej.
Stąd wziął się jego przydomek, Shitou (石頭 \pinyin{shítóu} `kamień'), będący zarazem nazwą jego szkoły\index{szkoła Shitou 石頭宗} (石頭宗 \pinyin{Shítóu zōng}).

Shitou Xiqian nauczał Dharmy przez 48 lat, a w ``Księdze przekazu lampy z okresu Jingde'' wymieniono 21 jego uczniów, z których najważniejsi byli Tianhuang Daowu (天皇道悟 \nazwisko{Tiānhuáng Dàowù}), Yaoshan Weiyan (藥山惟儼 \nazwisko{Yàoshān Wéiyǎn}) oraz Xishan Dadian (西山大顛 \nazwisko{Xīshān Dàdiān}).
W latach 821-824, na prośbę wyznawców szkoły Shitou, Liu Ke (劉軻) wykonał inskrypcję na steli poświęconej Shitou Xiqian.
Chociaż nie zachowała się do dnia dzisiejszego, w ``Biografiach wybitnych mnichów Song'' cytowano jej najbardziej znany fragment: ,,Mazu był mistrzem w Jiangxi, a Shitou w Hunan. Ci, którzy wahali się i nie odwiedzili tych dwóch mistrzów byli uważani za ignorantów.''.
Wedle tradycji, trzy spośród pięciu głównych tradycji chan: Yunmen (雲門宗 \pinyin{Yúnmén zōng}), Fayan (法眼宗 \pinyin{Fǎyǎn zōng}) oraz Caodong (曹洞宗 \pinyin{Cáodòng zōng}) wywodzą się od uczniów Shitou.
Chociaż późniejsze pokolenia buddystów czciły Shitou jako założyciela jednego z dwóch najważniejszych odłamów Szkoły Południowej, źródła historyczne wskazują, że jeszcze długo po jego śmierci szkoła Shitou miała marginalne znaczenie.
W ``Inskrypcji upamiętniającej mistrza chan, Dade Dayi, w świątyni Xingfu Si'' (興福寺內道場供奉大德大義禪師碑銘 \pinyin{Xīngfú Sì nèi dàochǎng gòngfèng Dàdé Dàyì Chán shī bēimíng}), powstałej pomiędzy rokiem 818 a 828, urzędnik cesarski Wei Chuhou (韋處厚) opisał ówczesny rozkład tradycji buddyzmu w różnych rejonach Chin.
W mieście Chang'an aktywna była wówczas Szkoła Północna; w Luoyangu szkoła Heze, a więc bezpośredni kontynuatorzy myśli Shenhui; w delcie Yangzi szkoła Niutou (牛頭宗 \pinyin{Niútóu zōng}), a na obszarze odpowiadającym obecnej prowincji Hunan szkoła Hongzhou.
% 应身无数,天竺降其一;禅祖有六,圣唐得其三。在高祖时,有道信叶昌运;在太宗时,有宏忍示元珠;在高宗时,有惠能筌月指。自此脉散丝分,或遁秦,或居洛,或之吴,或在楚。秦者曰秀,以方便显,普寂其允也。洛者曰会,得总持之印,独曜莹珠,习徒迷真,橘枳变体,竟成《檀经》传宗,优劣详矣。吴者曰融,以牛头闻,径山其裔也。楚者曰道一,以大乘摄,大师其党也。
Tekst ten nie wspomniał jednak o istnieniu szkoły Shitou.
Sytuacja zmieniła się w następstwie prześladowań buddyzmu pod panowaniem cesarza Tang Wuzonga (唐武宗 \nazwisko{Táng Wǔzōng}), których szczytowy okres przypadł na rok 845
(Shi 2008: 209; Poceski 2007: 98).

Do rozpowszechnienia nauk szkoły Shitou znacząco przyczynił się mistrz Dongshan Liangjie (洞山良价 \pinyin{Dòngshān Liángjiè}), który żył w IX wieku.
Jego najsłynniejszym uczniem był Caoshan Benji (曹山本寂 \nazwisko{Cáoshān Běnjì}), który spopularyzował nauki swojego mistrza.
Wywodząca się od nich szkoła chan znana jest jako Caodong.
Tradycyjnie przyjmuje się, że nazwa ta jest połączeniem ich przydomków, Caoshan i Dongshan.
Żyjący w XI i XII w. dzierżawca tej linii przekazu, Hongzhi Zhengjue (宏智正覺 \nazwisko{Hóngzhì Zhèngjué}) nauczał techniki medytacji zwanej \textit{mozhao chan} (默照禪 \pinyin{mòzhào Chán} `cicha iluminacja').
W XIII w. mnich zwany Dōgen Zenji (jap. {\ipaexgothic 道元禅師} Dōgen Zenji) lub Dōgen Kigen (jap. {\ipaexgothic 道元希玄} Dōgen Kigen) przeniósł nauki szkoły Caodong do Japonii, gdzie do dnia dzisiejszego istnieje pod nazwą Sōtō ({\ipaexgothic 曹洞宗} \textit{Sōtō-shū})
(Li 2001: 1).

\section{Zakończenie}
\textit{Sutra Platformy} powstała w oparciu o nauki Heze Shenhui, który proklamował południową szkołę buddyzmu chan, dając początek pierwszej wielkiej schizmie w tej tradycji.
Shenhui głosił, że prawowitym spadkobiercą przekazu chan był jego rzekomy mistrz, Dajian Huineng.
Podjął szeroko zakrojoną akcję, skierowaną przeciwko swoim rywalom --- Północnej Szkole Yuquan Shenxiu oraz jego uczniów, Songshan Puji i Dazhi Yifu.
W rezultacie jego działań przekaz Szkoły Północnej ostatecznie wygasł.
Wedle obecnego stanu wiedzy nie sposób stwierdzić, kiedy dokładnie to nastąpiło, wiadomo jednak, że stopniowy upadek szkoły Shenxiu trwał dłużej, niż podawali historycy Szkoły Południowej, tacy jak Guifeng Zongmi.
% nie wiadomo, kiedy dokładnie wygasła Szkoła Północna

Z biegiem lat Dajian Huineng stał się postacią półlegendarną, zaś do \textit{Sutry Platformy} dodano liczne elementy fantastyczne oraz nauki pochodzące z wielu różnych tradycji buddyzmu, szczególnie \textit{Mahapradżniaparamita} i \textit{Tathāgathagarbha}.
Tekst stanowił filozoficzny i ideologiczny manifest Szkoły Południowej.
Najważniejszymi założeniami tekstu były: obecność w każdej czującej istocie natury buddy oraz możliwość osiągnięcia ostatecznego oświecenia poprzez krótkotrwały wgląd w naturę własnego umysłu.
Stanowiły one podstawę nauk buddyzmu chan i zen w późniejszych stuleciach.
W późniejszych stuleciach wiele szkół chan, w tym Caodong i Linji, nazywało siebie jego spadkobiercami.
Chociaż nie ma dowodów na spotkania założycieli tych tradycji z Huinengiem, ich nauki przetrwały do dnia dzisiejszego, a wyznawcy do tej pory czczą Szóstego Patriarchę niemal na równi z Bodhidharmą.

W końcu XII w. japoński mnich Myōan Esai przeniósł nauki szkoły Huanglong do Japonii, dając początek szkole Rinzai buddyzmu zen.
W XIII w. inny japoński mnich, imieniem Dōgen Zenji, odbył podróż do Chin, skąd powrócił z naukami szkoły Caodong.
Pochodząca od niego linia przekazu znana jest pod nazwą Sōtō.
W ten sposób \textit{Sutra Platformy} i kult Huinenga jako Szóstego Patriarchy Chan rozpowszechniły się poza terytorium Chin.

\if 0
McRae 1986: 5
Szkoła Południowa twierdziła, że posiada nauki niedualne
Wg SS natura oświecenie, przeszkadzające emocje, cierpienie i iluzje są w istocie tym samym, co oświecenie, ale NS widziała je jako różne; wg 宗密 oznacza to, że wiele lat, lub nawet żywotów praktyki idzie na marne; wszystko, czego potrzebuje praktykujący, to całkowite odcięcie dualistycznego myślenia
SS była lewicowa: uważała, że każdy powinien mieć prawo poznać Dharmę i osiągnąć oświecenie, a nie tylko ci, którzy włożyli w to wysiłek
Zongmi usystematyzował różne interpretacje chan, Szkoła Północna była najniżej
wykładnia 宗密 NS na początku była popularna, ale potem została niemal całkowicie wyparta przez SS, bo prawowitym spadkobiercą był Huineng, a nie Shenxiu

Dumoulin 81
Pierwsza schizma w tradycji chan
SS wygrała walkę o dominację, bo NS nie rozwijało się po śmierci uczniów Shenxiu, a SS publikowało wiele koanów i kronik
w 700 Shenxiu polecił cesarzowi zaprosić Huinenga do stolicy
(89: w 705 Huineng to zaproszenie odrzucił)
Shenxiu cieszył się szacunkiem dworu, główni uczniowie 嵩山普寂、嶗山義福
83
Według niektórych podań Shenhui był przez kilka lat uczniem Shenxiu, ale jest to mało prawdopodobne
Shenhui na pewno przebywał razem z Huinengiem
91 Dualizm rozpuszcza się w pustości -- myśl z Diamentowej Sutry
91/92 Samadhi w którym nie ma myśli
Umysł ma skłonność do konceptualizacji absolutu, np. przywiązuje się do koncepcji nirwany albo pustki
Jeżeli zamiast tego pozbędzie się wszelkich koncepcji, to pozostanie samo lustro

中國禪宗史 115:
W sutrze jest napisane, że Huineng 先天二年八月三日滅度 oraz 春秋七十有六, ale są stele, które mówią co innego i przesuwają datę śmierci o 3 lata
《曹溪大師別傳》
117: Z legend 神會, zapisanych przez 王維 wynika, że Huineng ukrywał się przez 16 lat, dostał 衣法 od 弘忍 na jego łożu śmierci, i że po 16 latach ukrycia spotkał się z 印宗 i zaczął nauczać
神會語錄:能禪師過嶺至韶州,居曹溪山,來往四十年。 nie ma 印宗,隱遁
歷代法寶記:ukrycie, 印宗
Teoria ukrycia wzięła się z niespójności lat

182:713-815 神會向中原傳播南宗頓教
W latach 713-815 nauki Shenhui rozprzestrzeniły się w rejonie Niziny Chińskiej (a właściwie 中原) w dorzeczu Rzeki Żółtej (黃河)

188: 宋高僧传 podaje, że w roku 開元八年 zamieszkał w świątyni 龍興寺 w 南陽
potem nauczał Dharmy w Luoyangu, a Shenxiu i jego uczniowie aż zielenieli z zazdrości
Mieszkał przez dłuższy czas w Nanyang, więc nazywano go 南陽和上
神會語錄 autor 劉澄

189:南宗定是非論 opisuje wydarzenia z 滑臺寺, gdzie Shenhui urządził otwartą konferencję
圓覺經大疏釋義鈔 podaje, że w 20 lat po śmierci Huinenga (czyli do Huatai - 732), stopniowa ścieżka Shenxiu była rozpowszechniona w 荊吳 - środkowym i dolnym biegu Changjiang i bodajże w dwóch stolicach (秦洛 czyli Chang'an i Luoyang?)
Shenhui twierdził, że Shenxiu nie miał prawdziwego przekazu, a jedynie poboczny; prawowitym spadkobiercą Bodhidharmy miał być Huineng
Shenhui musiał mieć wsparcie społeczności z Dayun Si w Huatai, inaczej w żadnym wypadku nie umożliwiliby mu organizacji takiej konferencji
190 musieli skołować na konferencję 屏風 parawany, które w czasie jej trwania ktoś wyniósł, żeby przyjąć jakiegoś urzędnika
191 Shenhui urządził tę konferencję nie tylko dlatego, że chciał pogrążyć swoich przeciwników, ale również dlatego, że uważał, że istniało zbyt wielu samozwańczych mistrzów chan, którzy przeinaczali istotę nauk Buddy.
\fi
